
\subsection{Lustro i~rewers. Węzły skrętne i zwierciadlane}
\begin{definition}[lustro]
% DICTIONARY;mirror;lustro/lustrzany;węzeł
\index{lustro}%
\index{węzeł!lustrzany}% TODO: to się może mylić ze zwieciadlanym
    Niech $L$ będzie zorientowanym splotem.
    Splot $\operatorname{m} L$ powstały przez odbicie splotu $L$ względem dowolnej płaszczyzny nazywamy lustrem.
\end{definition}

\begin{definition}[rewers]
% DICTIONARY;reverse;rewers/odwrotny;węzeł
\index{rewers}%
\index{węzeł!odwrotny}%
    Niech $L$ będzie zorientowanym splotem.
    Splot $\operatorname{r} L$ powstały przez odwrócenie orientacji wszystkich ogniw splotu $L$ nazywamy rewersem.
\end{definition}

\begin{comment}
\begin{figure}[H]
    \begin{minipage}[b]{.32\linewidth}
        \centering
        \includegraphics[height=2.2cm]{../data/links/9_3_2_mirror.png}
        \subcaption{lustro $mL$}
    \end{minipage}
    \begin{minipage}[b]{.32\linewidth}
        \centering
        \includegraphics[height=2.2cm]{../data/links/9_3_2_base.png}
        \subcaption{przykładowy splot $L$}
    \end{minipage}
    \begin{minipage}[b]{.32\linewidth}
        \centering
        \includegraphics[height=2.2cm]{../data/links/9_3_2_reverse.png}
        \subcaption{rewers $rL$}
    \end{minipage}
\end{figure}
\end{comment}

Na lewym obrazku odbiliśmy diagram względem pionowej płaszczyzny, ale ten sam splot dostalibyśmy odwracając wszystkie nad- i podskrzyżowania (czyli odbijając go względem płaszczyzny papieru, ale programy graficzne nie pozwalają na wykonanie tej operacji zbyt łatwo, a~my jesteśmy trochę leniwi).

Niektórzy mówią o odbiciach lustrzanych i odwrotnościach.
Zaletą naszych oznaczeń jest to, że trudniej jest pomylić lustro z odwrotnością; ale w literaturze dominuje $L^*$ jako symbol lustra oraz $-L$ jako symbol odwrotności splotu $L$; używają ich na przykład Burde, Zieschang, Heusener \cite{burde2014} czy Murasugi \cite[s. 14, 24]{murasugi1996}.
Lickorish \cite[s. 4]{lickorish1997} używa $r L$ dla odwrotności oraz $\overline{L}$ dla lustra; zapewne ktoś gdzieś używa jeszcze innych znaczków.

Zauważmy, że wzięcie lustra i/lub rewersu węzła nie musi prowadzić do nowych obiektów.
Na przykład trójlistnik jest własnym rewersem: $3_1 = \operatorname{r} 3_1$, ale nie lustrem.

Dlatego wyróżniamy pięć typów symetrii splotów:

\begin{definition}[sploty zwierciadlane, odwracalne, chiralne]
% DICTIONARY;chiral;skrętny/chiralny;węzeł
% DICTIONARY;reversible;odwracalny;węzeł
% DICTIONARY;achiral/amphicheiral;zwierciadlany;węzeł
    Niech $L$ będzie splotem.
    Wtedy $L$ jest równoważny ze swoim rewersem, lustrem, rewersem lustra, wszystkimi albo żadnym z trzech wymienionych splotów;
    splot $L$ nazywamy odpowiednio:
    \begin{itemize}
        \item odwracalnym ($L = rL$),
        \item dodatnio zwierciadlanym ($L = mL$),
        \item ujemnie zwierciadlanym ($L = mrL$),
        \item całkowicie zwierciadlanym ($mL = L = rL$),
        \item całkowicie chiralnym ($mL \neq L \neq rL$).
    \end{itemize}
\index{węzeł!zwierciadlany}%
\index{węzeł!odwracalny}%
\index{węzeł!chiralny}%
\index{węzeł!skrętny|see {węzeł chiralny}}%
\end{definition}

Węzły całkowicie chiralne nazywa się czasem skrętnymi.

%    Węzły $K$, $rK$, $mK$ są parami nierównoważne. % chiral 9_32
%    Węzły $K \cong rK$ są równoważne. % reversible 3_1
%    Węzły $K \cong mrK$ są równoważne. % negative amphicheiral 8_17
%    Węzły $K \cong mK$ są równoważne. % positive amphicheiral 12a_427
%    Węzły $K, rK, mK$ są parami równoważne. % fully amphicheiral 4_1

\begin{example}
    Węzeł $9_{32}$ jest całkowicie skrętny.
\end{example}

% Całkowicie skrętne są też między innymi wszystkie węzły torusowe.
% TODO: wiki pisze Each nontrivial torus knot is prime[4] and chiral.[2]

\begin{example}
    \label{exm:trefoil_is_chiral}
    Trójlistnik jest odwracalny, ale nie zwierciadlany.
\end{example}

Przypuszczał to już Listing \cite{listing1847} w 1847 roku, ale pierwszy dowód podał dużo później, bo w 1914 roku Dehn \cite{dehn1914}. 
Oto, jak tego dokonał.
% równik = equator
% równoleżnik = parallel (of latitude), najdłuższy równoleżnik to równik
% południk = meridian (of longitude)
% TODO: naprostować bałagan z meridian/longitude w całej książce
% https://math.stackexchange.com/questions/2511364/how-did-dehn-prove-that-the-trefoil-is-chiral myli te pojęcia: pisze o meridian i longitude, kiedy oryginalna praca Dehna operowała na longitude i latitude
Iloraz grafu Cayleya dla grupy podstawowej trójlistnika, $G = \pi_1(S^3 - K)$, zanurza się w~produkt $\mathbb H^2 \times \R$, co pozwala wyznaczyć grupę zewnętrznych automorfizmów grupy $G$, $\Z/2\Z$.
\index{grupa!podstawowa}
% DICTIONARY;latitude;szerokość geograficzna;geografia
% DICTIONARY;longitude;długość geograficzna;geografia
% DICTIONARY;meridian (of longitude);południk;geografia
% DICTIONARY;parallel (of latitude);równoleżnik;geografia
% DICTIONARY;---;geografia;-
Korzystając z~południków i~równoleżników pokazał następnie, że nietrywialny automorfizm zewnętrzny odwraca orientację przestrzeni otaczającej.

My przekonamy się o~tym po wyznaczeniu wielomianu Jonesa trójlistnika, patrz wniosek \ref{cor:jones_of_amphicheiral}.

\begin{example}
    Węzeł $8_{17}$ jest zwierciadlany ujemnie, ale nie odwracalny.
\end{example}

Sześćdziesiąt lat temu matematycy nie byli pewni, czy węzły nieodwracalne w~ogóle istnieją \cite[problem 10]{fox1962};
obecnie wiadomo, że nieodwracalne są prawie wszystkie węzły (Murasugi \cite[s.~46]{murasugi1996}).
\index[persons]{Murasugi, Kunio}%
W~roku 1962 Ralph Fox wskazał kilku kandydatów do tego tytułu.
\index[persons]{Fox, Ralph}%
Hale Trotter odkrył rok później nieskończoną rodzinę nieodwracalnych precli, patrz \ref{prp:pretzel_not_invertible}.
\index[persons]{Trotter, Hale}%

% MAKOTO SAKUMA - A SURVEY OF THE IMPACT OF THURSTON’S WORK ON KNOT THEORY
% Hartley [129] realized that one can apply this method to the problem of identifying noninvertible knots, as follows. Suppose no automorphism of Γ maps γ to γ−1. Then the set R(G(K), Γ, γ) is possibly different from the set R(G(K), Γ, γ−1), and there is a chance to show noninvertibility of K by comparing the homology invariants associated with φ ∈ R(G(K), Γ, γ) with those associated with φ′ ∈ R(G(K), Γ, γ−1). Hartley showed that this method is quite effective: he completely determined the 36 non-invertible knots up to 10 crossings claimed by Conway to be noninvertible.

\begin{example}
    Węzeł $12a_{427}$ jest zwierciadlany dodatnio, ale nie odwracalny.
\end{example}

Żaden inny węzeł pierwszy o mniej niż 13 skrzyżowaniach nie ma tej cechy.

\begin{example}
\label{property_of_eight_knot}%
    Ósemka $4_1$ jest całkowicie zwierciadlana.
\end{example}

To najprostszy typ symetrii, wystarczy jawnie wskazać przekształcenie między diagramem węzła, jego lustra oraz odwrotności.

\label{con:tait_fourth}%
Tait odnosił wrażenie, że zwierciadlane węzły mają parzysty indeks skrzyżowań, ale Hoste, Thistlethwaite znaleźli w~1998 kontrprzykład o~piętnastu skrzyżowaniach, $15_{700}$. % wg https://mathworld.wolfram.com/AmphichiralKnot.html
(Czwarta) hipoteza Taita jest prawdziwa dla węzłów pierwszych, alternujących.
\index{hipoteza Taita}%

Poniższa tabela oparta jest (kolejno) o~ciągi
\href{https://oeis.org/A051766}{51766},
\href{https://oeis.org/A051769}{51769},
\href{https://oeis.org/A051768}{51768},
\href{https://oeis.org/A051767}{51767},
\href{https://oeis.org/A052400}{52400},
z bazy danych ``The On-Line Encyclopedia of Integer Sequences'' (OEIS).

\begin{table}[h]
    \centering
    \begin{tabular}{@{}*{20}l@{}} \toprule
        skrzyżowania & 3 & 4 & 5 & 6 & 7 & 8 & 9 & 10 & 11 & 12 & 13 & 14 \\ \midrule
        całkowicie skrętne & 0 & 0 & 0 & 0 & 0 & 0 & 2 & 27 & 187 & 1103 & 6919 & 37885 \\
        odwracalne & 1 & 0 & 2 & 2 & 7 & 16 & 47 & 125 & 365 & 1015 & 3069 & 8813 \\
        $-$ zwierciadlane & 0 & 0 & 0 & 0 & 0 & 1 & 0 & 6 & 0 & 40 & 0 & 227 \\
        $+$ zwierciadlane & 0 & 0 & 0 & 0 & 0 & 0 & 0 & 0 & 0 & 1 & 0 & 6 \\
        zwierciadlane & 0 & 1 & 0 & 1 & 0 & 4 & 0 & 7 & 0 & 17 & 0 & 41 \\
        \bottomrule
        \hline
    \end{tabular}
    \caption{Liczba węzłów pierwszych o~poszczególnych typach symetrii}
\end{table}

\begin{definition}
    Niech $K \subseteq S^3$ będzie węzłem.
    Jeśli istnieje inwolucja pary $(S^3, K)$, która zachowuje orientację sfery, ale odwraca orientację węzła, to węzeł $K$ nazywamy silnie odwracalnym.
\end{definition}

To jest definicja 10.3.2 z monografii Kawauchiego \cite{kawauchi1996}.

\begin{proposition}
    Jeśli węzeł jest silnie odwracalny, to jest też odwracalny.
\end{proposition}

Hipotezę, że każdy odwracalny węzeł jest też silnie odwracalny, postawił Montesinos \cite[problem 1.6]{kirby1978}, on też zdefiniował klasę silnie odwracalnych węzłów \cite{montesinos1975}.
\index[persons]{Montesinos, José}%
Jednakże...

\begin{proposition}
    Istnieją odwracalne węzły, które nie są silnie odwracalne.
\end{proposition}

\begin{proof}
\index[persons]{Hartley, Richard}%
\index[persons]{Whitten, Wilbur}%
    Stosowne przykłady podali niezależnie od siebie Hartley \cite[s. 183]{hartley1980} oraz Whitten \cite{whitten1981} (węzeł $K$ jest silnie odwracalny wtedy i tylko wtedy, gdy każdy jego dubel jest silnie nieodwracalny, wynika stąd, że pewien dubel $8_{17}$ nie jest silnie odwracalny; jednocześnie Schubert \cite[s. 235]{schubert1953} pokazał, że duble są odwracalne).
\end{proof}

Ale hiperboliczny węzeł odwracalny jest silnie odwracalny, wspomina o tym bez dowodu Kawauchi.

\begin{proposition}
    Każdy wielomian Alexandera jest realizowany przez pewien silnie odwracalny węzeł.
\end{proposition}

\begin{proof}
\index[persons]{Sakai, Tsuyoshi}%
    Sakai konstruuje w \cite{sakai1983} silnie odwracalny węzeł o dowolnie wybranym cyklicznym module Alexandera.
\end{proof}

% Koniec podsekcji Lustro i rewers

