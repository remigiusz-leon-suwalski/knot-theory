
% strona piąta

\chapter*{Przedmowa}
Książka wprowadza we współczesną teorię węzłów, najważniejsze metody i obszary tej wciąż niedocenianej, choć żywo rozwijającej się dyscypliny matematycznej.
Tor wykładu wzorowany jest na seminarium z teorii węzłów jakie odbyło się na Wydziale Matematyki Uniwersytetu Wrocławskiego w semestrze zimowym 2013/2014 i dlatego podręcznik nadaje się szczególnie do pierwszego czytania dla studentów wyższych lat studiów matematycznych, natomiast dla młodych pracowników naukowych może okazać się pożytecznym źródłem odsyłaczy do prac, które zainspirują do dalszych badań.
Zwięźle i na tyle, na ile było to możliwe opisuję rozmaite narzędzia stosowane do badania węzłów, splotów, supłów i innych: klasyczne niezmienniki numeryczne i ruchy Reidemeistera, niezmienniki kolorujące i spokrewnione z nimi kwandle, potem diagramatyczny wielomian Jonesa, homologiczny wielomian Alexandera oraz dość świeże niezmienniki typu skończonego.
Ze względu na niepełne zrozumienie maszynerii topologi algebraicznej, rozdział czwarty został napisany trochę niechlujnie, jednakże żywię nadzieję poprawić się w następnym wydaniu trzymanej przez Ciebie książki.
W ostatnim, piątym rozdziale przedstawiam krótko charakterystyczne rodziny: warkocze, dwumostowe sploty, mutanty, precle, sploty torusowe, satelitarne i hiperboliczne, wreszcie węzły plastrowe i taśmowe, obiekt zainteresowań czterowymiarowej topologii.
Dołączam również tablice węzłów pierwszych o małej liczbie skrzyżowań.
Z takim zakresem i ujęciem materiału pozycja jest unikalna nie tylko we francuskiej, ale i w światowej literaturze matematycznej.

Serdecznie dziękuję Ś. G. oraz J. Ś. bez których ten tekst nigdy by nie powstał.

\begin{flushright}
Casimir Allard,\\Marsylia, 4 lutego 2019
\end{flushright}

\newpage
\section*{Przedmowa do wydania drugiego}
Od poprzedniego wydania minął ponad rok i~trochę się w~tym czasie zmieniło.

Dodałem informację o~różnych notacjach dla węzłów, poprawiłem informację na temat płci niektórych osób, napisałem nową sekcję o~macierzy Seiferta, wymieniłem niezmienniki nieodróżniające mutantów.
Pojawiła się klasyfikacja kwandli, nierówność Mortona, homologia Floera i kilka mniejszych matematycznych obiektów.
Niektóre istniejące sekcje przeniosłem w~lepsze mam nadzieję miejsca, niektóre trudne dowody otrzymały swój zarys.
Tam, gdzie to możliwe, podałem rys historyczny, na przykład w~sekcji o~kolorowaniach wspomniałem o~Foxie, który chciał uczynić teorię węzłów prostszą dla swoich studentów.
Na końcu książki umieściłem krótki słowniczek angielsko-polski.
Wreszcie usunąłem znalezione literówki i~uzupełniłem luki, zarówno w~bibliografii jak i~rysunkach.
Poprawiłem drobnie typografię, dla przykładu od teraz wszystkie równania powinny być numerowane.

Żywię nadzieję, że korzystanie z~książki będzie równie (jeśli nie bardziej) przyjemne, co w~przypadku pierwszego jej wydania.

\begin{flushright}
Casimir Allard,\\Marsylia, 19 sierpnia 2020
\end{flushright}

\section*{Przedmowa do wydania trzeciego}
Lata mijały, wiele wody upłynęło w Loarze, a~wydania trzeciego wciąż nie było widać na horyzoncie.
Głównym winowajcą jest tutaj Amazon, który z~niejasnych dla mnie powodów przestał wspierać język francuski.
Pomimo przeciwności losu nie zapomniałem jednak o miłośnikach teorii węzłów i~nie przestawałem pracować w domowym zaciszu nad książką.
Nieznacznie rozrosła się ze 114 stron w 2020 roku do 1XX stron teraz, kiedy oddaję ją do druku.
Cytowań już wcześniej było bez liku (dokładniej: 240), a~teraz jest ich jeszcze więcej (aż 437).
Autor był sam jak paluch, a teraz jest nas dwoje!
% TODO: to jest licząc ostatnią stronę 5. rozdziału
Staraliśmy się nie zmieniać znacząco układu, w~myśl dewizy, że lepsze jest wrogiem dobrego.
Zupełnych nowości jest tyle, co kot napłakał.
W rozdziale drugim na osobną sekcję zasłużyły sobie etykietowania.
Natomiast w całej książce długie bloki tekstu zostały podzielone na sekcji i podsekcje tak, że odnajdywanie potrzebnych informacji w pośpiechu powinno sprawiać mniej trudu.
Z czystej przyzwoitości wymieniamy też, czego nie ma (i raczej szybko nie będzie):
\begin{enumerate}
    \item \textbf{chirurgii Dehna}.
    \item \textbf{niezmienników Reszetichina-Turajewa}.
    (Wspominamy o nich raz, za definicją \ref{def:jones_polynomial}.)
    \item \textbf{rachunku Kirby'ego}.
    Każdą zamkniętą, orientowalną 3-rozmaitość można otrzymać wykonując chirurgię Dehna na sferze wzdłuż pewnego splotu. Robion Kirby udowodnił około 1978 roku, że dwie takie rozmaitości są homeomorficzne wtedy i tylko wtedy, gdy między odpowiadającymi im splotami można przejść ruchami Kirby'ego.
    (Porównaj to ze sformułowaniem twierdzenia Reidemeistera).
    % Kirby, Robion (1978). "A calculus for framed links in S3". Inventiones Mathematicae. 45 (1): 35–56. Bibcode:1978InMat..45...35K. doi:10.1007/BF01406222. MR 0467753. S2CID 120770295. ?
    % TODO: Yanga-Baxtera?
\end{enumerate}
I to by było na tyle.
\begin{flushright}
    Casimir Allard i Adélaïde Gauthier,\\Bordeaux, \today
\end{flushright}

\section*{Przedmowa do wydania czwartego?}

Wydanie czwarte prawdopodobnie nigdy nie powstanie.
Dlaczego nam nie wyszło? 
Myślimy, że powodów da się wymienić wiele.
Opuściliśmy środowisko akademickie, zapał już teraz jest trochę słomiany i wypala się z każdym nowym pracodawcą szybciej i szybciej.
My nie do końca sprawdziliśmy się jako matematycy, a was nie można nazwać wzorowymi czytelnikami -- ale było warto!
Stworzyliśmy kilka cudownych akapitów godynch zapamiętania.
I nie wiem jak wy, ale my się naprawdę dobrze bawiliśmy.
Dziękujemy wszystkim, którzy zgłosili choć jedną uwagę i ożywili jakoś treść, ale w szczególności wdzięczni jesteśmy A. W. W., która nigdy tego nie przeczyta i której nikt tu nie zna, ale bez której nie powstałaby tak dziwna publikacja.

\begin{flushright}
Casimir Allard i Adélaïde Gauthier,\\gdzieś, kiedyś
\end{flushright}

% koniec strony piątej

