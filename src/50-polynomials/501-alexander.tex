
\section{Wielomian Alexandera}
\index{wielomian!Alexandera|(}%
Wielomian Alexandera to najstarszy niezmiennik tego typu, odkryty w 1923 roku \cite{alexander1923}.
Jego pierwsza definicja była czysto topologiczna algebraicznie:

\begin{definition}
    Niech $K$ będzie węzłem w~3-sferze $S^3$, zaś $X$ nieskończonym nakryciem cyklicznym jego dopełnienia otrzymanym przez rozcięcie dopełnienia wzdłuż powierzchni Seiferta.
\index{powierzchnia!Seiferta}%
    Na przestrzeni $X$ oraz grupie homologii $H_1(X)$, działa automorfizm $t$, który czyni z~niej moduł nad pierścieniem $\Z[t, t^{-1}]$, i~to skończenie prezentowalny.
    Załóżmy, że wiemy mniej niż mało; w szczególności nie wiemy, czy nasz moduł posiada przedstawienie z~$r$ generatorami i~$s$ relacjami, gdzie $r \le s$ (jeśli tak jest, rozpatrzmy ideał generowany przez minory $r \times r$ macierzy prezentacji; jeśli nie, weźmy ideał zerowy).
    Alexander pokazał, że ideał, o którym mowa, jest zawsze niezerowy, główny i generowany przez coś, co teraz nazywamy wielomianem Alexandera.
\end{definition}

Nie jest to definicja, z którą praca stanowi przyjemność.
Dlatego podamy prostszy opis, oparty o równania kolorujące.
Później sprawdzimy, jaki wpływ na wielomian mają suma spójna, lustro i~rewers oraz jak Conway odkrył na nowo wielomian Alexandera, jednocześnie przyspieszając jego liczenie.
Na koniec pokażemy, co łączy go ze zdefiniowanymi wcześniej i~później numerycznymi niezmiennikami.

\input{50-polynomials/501a-colouring}

\input{50-polynomials/501b-operations}

\input{50-polynomials/501c-skein}

\input{50-polynomials/501d-numerical}

\input{50-polynomials/501e-fox}

\input{50-polynomials/501f-miscellaneous}

\index{wielomian!Alexandera|)}%

