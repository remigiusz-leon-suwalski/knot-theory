
\subsection{Relacja kłębiasta}
Dotychczas wyznaczyliśmy wielomian Jonesa jedynie dla splotów trywialnych (fakt \ref{prp:jones_trivial_link}).
Dlaczego?
Chociaż klamra Kauffmana to użyteczne narzędzie podczas dowodzenia różnych teoretycznych własności, niezbyt nadaje się do obliczeń, szczególnie ręcznych.
Na szczęście wtedy z pomocą przychodzi:

\begin{definition}
\index{relacja kłębiasta}%
    Niech $L$ będzie zorientowanym splotem.
    Wtedy wielomian Laurenta $\jones_L(t) \in \Z[t^{\pm 1/2}]$, który spełnia relację kłębiastą
    \begin{equation}
        \label{eqn:jones_skein}
        t^{-1} \jones(L_+) - t\jones(L_-) + (t^{-1/2} - t^{1/2}) \jones(L_0) = 0
    \end{equation}
    z warunkiem brzegowym $\jones(\SmallUnknot) = 1$, nazywamy wielomianem Jonesa.
\end{definition}

Symbole $L_+, L_-, L_0$ objaśnione są przy definicji \ref{skein_symbols}.

\begin{proof}
% TODO: zdefiniować je raz, a dobrze.
% ack -l 'L_\+'
% src/50-polynomials/alexander.tex
% src/50-polynomials/jones-kauffman.tex
% src/50-polynomials/blmho.tex
% src/50-polynomials/homfly.tex
% src/00-meta-latex/diagrams.tex
% TODO: odwrócić kolejność kroków, wtedy dowód zacznie być z-czapy
Wyraźmy wielomian Jonesa przez klamrę Kauffmana i~spin.
Chcemy pokazać, że
\begin{align}
    & A^{4}(-A)^{-3w(L_+)}\bracket{L_+} \\
    - & A^{-4}(-A)^{-3w(L_-)}\bracket{L_0} \\
    + & (A^2-A^{-2})(-A)^{-3w(L_0)}\bracket{L_0} = 0.
\end{align}
\noindent
Ale $w(L_\pm) = w(L_0)\pm 1$, zatem to jest równoważne z
\begin{equation}
    -A \bracket{L_+} +
    A^{-1} \bracket{L_-} +
    (A^2-A^{-2}) \bracket{L_0} =0.
\end{equation}
\noindent
Z definicji klamry Kauffmana wnioskujemy, że
\begin{align}
    \bracket{L_+} & = A\bracket{L_0} + A^{-1}\bracket{L_\infty}, \\
    \bracket{L_-} & = A\bracket{L_\infty} + A^{-1}\bracket{L_0}.
\end{align}
\noindent
Pierwsze równanie przemnóżmy przez $A$, drugie przez $A^{-1}$, a~następnie dodajmy obydwa do siebie.
Wtedy otrzymamy
\begin{equation}
    A\bracket{L_+} - A^{-1}\bracket{L_-} =
    A^2 (\bracket{L_0} - \bracket{L_\infty}),
\end{equation}
quod erat demonstrandum.
\end{proof}

