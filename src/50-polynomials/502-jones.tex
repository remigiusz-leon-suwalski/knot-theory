
\section{Wielomian Jonesa}
\index{wielomian!Jonesa|(}%
Drugi wielomianowy niezmiennik, jaki poznamy, spełnia bardzo podobną relację kłębiastą (porównaj: \ref{eqn:alexander_skein} versus \ref{eqn:jones_skein}), co ten z poprzedniej sekcji.
Zaczniemy od bardzo ogólnikowego opisu algebry Temperleya-Lieba i śladu Markowa; dwóch składników w~konstrukcji Jonesa.
Szczegółowo zajmiemy się późniejszym odkryciem Kauffmana, bo wykorzystał lubiane przez nas diagramy i~ruchy Reidemeistera.
Opiszemy krótko, jak zmienia się wielomian podczas odbijania, odwracania i dodawania.
Na koniec podamy dowód I hipotezy Taita, wspomnimy też, jaką rolę wielomian Jonesa odegrał w dowodzie pozostałych dwóch hipotez.

\input{50-polynomials/502b-temperley}

\input{50-polynomials/502a-kauffman}

\input{50-polynomials/502d-distinguish}

\input{50-polynomials/502e-skein}

\input{50-polynomials/502f-mirror_reverse}

\input{50-polynomials/502g-roots_of_unity}

\index{wielomian!Jonesa|)}%

\input{50-polynomials/502c-span}

