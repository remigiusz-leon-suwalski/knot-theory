\subsection{Dopełnienia węzłów i splotów}
Jeśli dwa węzły są równoważne, to ich dopełnienia są oczywiście homeomorficzne.
Pytanie o~prawdziwość implikacji odwrotnej jako pierwszy zadał najprawdopodobniej Tietze \cite{tietze1908} w~1908 roku.
\index[persons]{Tietze, Heinrich}%
O~jego trudności niech świadczy fakt, że dopiero w~roku 1987 pokazano, że istnieją co najwyżej dwa węzły o~zadanym dopełnieniu (Culler, Gordon, Luecke, Shalen \cite{culler1987}).
\index[persons]{Culler, Marc}%
\index[persons]{Shalen, Peter}%
\index[persons]{Gordon, Cameron}%
\index[persons]{Luecke, John}%
Dwa lata później poznaliśmy pozytywną odpowiedź na pytanie Tietzego: każdy węzeł jest wyznaczony jednoznacznie przez swoje dopełnienie.
Natomiast analogiczne stwierdzenie o~splotach jest fałszywe i wiedziano o tym od bardzo dawna: w~1937 roku Whitehead \cite{whitehead1937} podał nieskończenie wiele splotów, których dopełnienia wyglądają jak dopełnienia splotu Whiteheada.

\begin{theorem}[Gordon, Luecke, 1989]
\index[persons]{Gordon, Cameron}%
\index[persons]{Luecke, John}%
\index{twierdzenie!Gordona-Lueckego}%
    Niech $f \colon (\mathbb R^3 \setminus K_1) \to (\mathbb R^3 \setminus K_2)$ będzie zachowującym orientację homeomorfizmem dopełnień poskromionych węzłów $K_1, K_2$.
    Wtedy węzły $K_1 \cong K_2$ są izotopijne.
\end{theorem}

\begin{proof}[Niedowód]
    Wynika to z teorii Cerfa, kombinatorycznych technik w stylu Litherlanda, cienkich pozycji, cykli Scharlemanna i~ogólniejszego stwierdzenia: nietrywialna chirurgia Dehna na węźle w~3-sferze nigdy nie daje 3-sfery.
\index{chirurgia Dehna}%
\index{cykle Scharlemanna}%
\index{teoria Cerfa}%
    Pełny dowód zawiera praca \cite{gordon1989}.
\end{proof}

Twierdzenie to zamienia problem lokalny (czy dwa węzły w kuli $S^3$ są równoważne?) na problem globalny (czy dwie przestrzenie topologiczne są homeomorficzne?).
\index[persons]{Whitehead, John}%