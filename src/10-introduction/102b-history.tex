
\subsection{Historia tablic węzłów}
% DICTIONARY;knot table;tablica węzłów;-
Pierwszą osobą, która podjęła się szukania węzłów, był Peter Guthrie Tait, szkocki fizyk.
\index[persons]{Tait, Peter}%
Razem z~Thomsonem (lordem Kelvinem) wierzyli, że węzły są kluczem do zrozumienia widma spektroskopowego różnych pierwiastków: na przykład atom sodu mógł być splotem Hopfa ze względu na dwie linie emisyjne.
\index[persons]{Thomson, William (lord Kelvin)}%
Eksperyment Michelsona-Morleya z~1887 roku zabił ich ,,wirową teorię atomu'', ale nie miało to znaczenia dla teorii węzłów jako działu matematyki.

Używana po dziś dzień strategia, którą przyjął Tait, jest stosunkowa prosta: narysować wszystkie możliwe diagramy o~zadanym indeksie skrzyżowaniowym, po czym połączyć ze sobą te, które przedstawiają jeden węzeł.
Na potrzeby pierwszego etapu Tait wymyślił schemat kodowania diagramów.
Opiszemy później jego ulepszenie, kod Dowkera-Thistlethwaite'a.


\subsubsection{Siedem i mniej skrzyżowań}
Tait wykorzystując swoją notację podał w~1876 pierwszą tablicę piętnastu węzłów o~mniej niż ośmiu skrzyżowaniach.
Nie należy traktować tego jako skromny wynik: nie miał on do dyspozycji żadnych twierdzeń topologicznych do odróżniania węzłów.
Onieśmielony przez liczbę możliwych kodów dla kolejnych indeksów skrzyżowaniowych, powstrzymał się przed rozszerzaniem swojej tablicy.
To właśnie grupowanie diagramów przedstawiających ten sam węzeł, a~nie samo szukanie wszystkich możliwych diagramów, sprawia trudność.

Aby sobie pomóc, Tait znalazł lokalną modyfikację diagramu, która nie zmienia indeksu skrzyżowaniowego, znaną obecnie jako flype.
\index{flype}%
Dla Taita ,,flype'' było innym ruchem, prostą transformacją związaną ze zmianą wyboru nieskończonego obszaru, ale mało kto teraz o tym pamięta.
Dowiedzieliśmy się o tym z pracy \cite{menasco1993}; Menasco i~Thistlethwaite dowiedzieli się o~tym od Claude'a Webera.
\index[persons]{Menasco, William}%
\index[persons]{Thistlethwaite, Morwen}%
\index[persons]{Weber, Claude}%
Flype to stary szkocki czasownik oznaczający ,,wykręcać na drugą stronę''.

\begin{comment}
\[
\begin{tikzpicture}[baseline=-0.65ex, scale=0.1]
\begin{knot}[clip width=5, end tolerance=1pt, flip crossing/.list={1}]
    \strand[thick] (-21, -5) [in=180, out=0] to (-7, 5);
    \strand[thick] (-21, 5) [in=180, out=0] to (-7, -5);
    \draw (-7, -7) rectangle (7, 7);
    \node at (0, 0) {\Huge {$T$}};
    \draw[thick] (7, -5) to (21, -5);
    \draw[thick] (7, 5) to (21, 5);
\end{knot}
\end{tikzpicture}
\quad \cong_{\mathrm{flype}} \quad
\begin{tikzpicture}[baseline=-0.65ex, scale=0.1]
\begin{knot}[clip width=5, end tolerance=1pt]
    \strand[thick] (21, -5) [in=0, out=180] to (7, 5);
    \strand[thick] (21, 5) [in=0, out=180] to (7, -5);
    \draw (-7, -7) rectangle (7, 7);
    \node at (0, 0) {\rotatebox[origin=c]{-180}{\Huge $T$}};
    \draw[thick] (-7, -5) to (-21, -5);
    \draw[thick] (-7, 5) to (-21, 5);
\end{knot}
\end{tikzpicture}
\]
\end{comment}

Inną taktykę szukania węzłów przyjał wielebny Thomas Kirkman\footnote{Oto jak Kirkman definiował węzeł w stu słowach: ,,\emph{By a Knot of $n$ crossings, I understand a reticulation of any number of meshes of two or more edges, whose summits, all tessaraces, are each a single crossing, as when you cross your forefingers straight or slightly curved, so as not to link them, and such meshes that every thread is either seen, when the projection of the Knot with its $n$ crossings and no more is drawn in double lines, or conceived by the reader of its course when drawn in single line, to pass alternately under and over the threads to which it comes at successive crossings.}''}: zaczynał od małego zbioru "nieredukowalnych" rzutów, do których systematycznie dokładał skrzyżowania.
\index[persons]{Kirkman, Thomas}%
Kirkman wydrukował diagramy 634 węzłów o~dziesięciu skrzyżowaniach w~numerze ,,Transactions of the Royal Society of Edinburgh'' z~1885 roku.
% wielebny => Adams, s. 31
Tait przeczytał to wydanie i~na jego podstawie opracował prawie kompletną listę węzłów alternujących o~mniej niż 11 skrzyżowaniach.
% Kirkman miał wtedy 78 lat!
Tuż przed oddaniem jej do druku odkrył inny spis węzłów stworzony przez amerykańskiego naukowca Charlesa Little'a.
\index[persons]{Little, Charles}%
Znalazł wtedy jeden duplikat u~siebie, natomiast u Little'a jeden duplikat i~jedno pominięcie.




\subsubsection{Dziesięć skrzyżowań}
Zachęcon przez Taita, Little zabrał się za alternujące węzły o~11 skrzyżowaniach i~za trudniejsze zadanie, stablicowanie węzłów niealternujących, czyli takich, które nie posiadają alternującego diagramu.
Jak wynika z~pierwszej pracy Taita, początkowo nie wierzono, że takie w~ogóle istnieją.
Dowód znaleziono wiele lat później, niealternujące są $8_{19}$, $8_{20}$, $8_{21}$, ale nie pierwsze węzły o mniejszej liczbie skrzyżowań.
Patrz twierdzenie \ref{prp:bankwitz}.
Little pracował przez sześć lat (1893 -- 1899) i~znalazł 43 niealternujące węzły o~10 skrzyżowaniach.
Żadnego nie pominął, ale trafił mu się jeden duplikat.
\index[persons]{Little, Charles}%

W kolejnych dziesięcioleciach nie nastąpił znaczący postęp, zarówno w~rozszerzaniu tablic, jak i~sprawdzaniu tych już istniejących.
Haseman \cite{haseman1918} w~1918 roku znalazła achiralne węzły o~12 (takich jest 54, praca Haseman podaje 61, ponieważ zawiera 7 duplikatów) i~14 skrzyżowaniach.
\index[persons]{Haseman, Mary}%
% AMPHICHEIRALS ACCORDING TO TAIT AND HASEMAN
W 1927 roku Alexander z~Briggsem \cite{alexander1927} przy użyciu pierwszej grupy homologii rozgałęzionego nakrycia cyklicznego (!) potrafili odróżnić od siebie dowolne dwa węzły (z~pominięciem 3 par) o~co najwyżej 9 skrzyżowaniach.
\index[persons]{Briggs, Garland}%
\index[persons]{Alexander, James}%
Reidemeister \cite{reidemeister1932} poradził sobie z~tymi wyjątkami w~1932 roku, korzystając z~indeksu zaczepienia i~homomorfizmów z~grupy węzła na grupy diedralne.
\index[persons]{Reidemeister, Kurt}%
% branch curves in irregular covers associated to homomorphisms of the knot group onto dihedral groups




\subsubsection{Jedenaście skrzyżowań}
\color{white}
Na dalszy postęp musieliśmy czekać do lat sześćdziesiątych.
Wtedy to John Conway\footnote{TODO: biografia Conwaya...} \cite{conway1970} znalazł prawie wszystkie pierwsze węzły o~mniej niż 12 skrzyżowaniach oraz sploty o~mniej niż 11 skrzyżowaniach.
\index[persons]{Conway, John}%
Odkrył jeden duplikat oraz jedenaście pominięć w~starych tablicach Little'a (do 11 skrzyżowań), ale sam zgubił cztery węzły.
Naprawienie błędu zajęło chwilę: dwa brakujące węzły zawierała praca magisterska Lombardero z~1968 roku, dwa odkrył Caudron około 1980 roku \cite{caudron1982}.
\index[persons]{Caudron, Alain}%
\index[persons]{Lombardero, ?}%
Conway też popełnił dwa duplikaty: współcześnie bardziej znany to para Perko, węzeł reprezentowany przez dwa diagramy aż do tablic Rolfsena.
\index{para Perko}%
Nazywamy ją tak, ponieważ została dostrzeżona przez Perko \cite{perko1974}, który bezskutecznie próbował odróżnić składniki pary diedralnym indeksem zaczepienia: dla obydwu diagramów wynosi on $32/11$.
\index[persons]{Perko, Kenneth}%
% https://www.researchgate.net/profile/Ken-Perko/publication/299560799_The_History_of_the_Perko_Pair/links/56ff0ba508aea6b77468d550/The-History-of-the-Perko-Pair.pdf both knots yielded a 5-fold dihedral linking number of 32/11
Jako przyczynę tak długiego niezauważenia pary Perko podaje się hipotezę Little'a, że spin minimalnego diagramu jest niezmiennikiem, gdyż błędnie założył, że 2-przejścia oraz flype wystarczają do zmiany dowolnego minimalnego diagramu w~inny.
% DICTIONARY;2-pass move;2-przejście;-
\index{spin}%
\index[persons]{Little, Charles}%
(Diagramy pary Perko mają różny spin).
Zachęcamy w~tym miejscu do nieprzeskakiwania strony \pageref{rolfsens_mistake}, gdzie para Perko pojawia się raz jeszcze.
Mniej znany duplikat wystąpił w~tablicy splotów do 10 skrzyżowań, gdzie \texttt{2.-2.-20.20} jest lustrem \texttt{8*-20:-20}.

Pomimo opisanych wyżej drobnych niepowodzeń, metodę Conwaya (mającą fundamenty w~pomysłach Kirkmana) uznaje się za bardzo dobrą.
Conway potrzebował zaledwie kilku godzin na przeprowadzenie swojej klasyfikacji; a my używamy jej po dziś dzień, na przykład Tuzun, Sikora zweryfikowali dzięki niej hipotezę \ref{con:jones} do 24 skrzyżowań.
\index[persons]{Tuzun, Robert}%
\index[persons]{Sikora, Adam}%

Rękopis \cite{siebenmann1979} (albo \cite{bonahon1989}?) Bonahona, Siebenmanna klasyfikuje węzły algebraiczne.
\index[persons]{Bonahon, Francis}%
\index[persons]{Siebenmann, Laurent}%
Kres ery ręcznych obliczeń nastąpił, gdy z~nielicznymi niealgebraicznymi węzłami do 11 skrzyżowań poradził sobie Perko \cite{perko1980}, \cite{perko1982}.
\index[persons]{Perko, Kenneth}%

% MAKOTO SAKUMA - A SURVEY OF THE IMPACT OF THURSTON’S WORK ON KNOT THEORY
% through hand calculation of homological invariants (in particular linking invariants) of finite branched coverings for those knots that are not covered by Bonahon and Siebenmann’s result described in Subsection 4.1. See [268] for an interesting historical note.




\subsubsection{Trzynaście skrzyżowań}
Na początku lat osiemdziesiątych ubiegłego wieku Dowker i~Thistlethwaite \cite{dowker83} z~pomocą komputera stablicowali węzły do 13 skrzyżowań.
\index[persons]{Dowker, Clifford}%
\index[persons]{Thistlethwaite, Morwen}%
Przez blisko dekadę nic się nie działo, aż wreszcie grupa studentów (Arnold, Au, Candy, Erdener, Fan, Flynn, Muir, Wu \cite{cray94}) wygrała dostęp do superkomputera Cray.
Razem z~Hoste znaleźli alternujące węzły do 14 skrzyżowań, jednocześnie sprawdzając istniejące tabele Thistlethwaite'a.
\index[persons]{Arnold, Brian}%
\index[persons]{Au, Michael}%
\index[persons]{Candy, Christoper}%
\index[persons]{Erdener, Kaan}%
\index[persons]{Fan, James}%
\index[persons]{Flynn, Richard}%
\index[persons]{Hoste, Jim}%
\index[persons]{Muir, Robs}%
\index[persons]{Wu, Danny}%


\subsubsection{Szesnaście skrzyżowań}
Około roku 1998 Hoste z~Weeksem oraz niezależnie Thistlethwaite \cite{thistlethwaite1998} znaleźli 1 701 936 pierwszych węzłów do 16 skrzyżowań.
\index[persons]{Hoste, Jim}%
\index[persons]{Thistlethwaite, Morwen}%
\index[persons]{Weeks, Jeff}%
Używali przy tym różnych podejść: Hoste z Weeksem wykorzystywali niezmienniki hiperboliczne oraz program SnapPea; Thistlethwaite natomiast wzbogacił zestaw ruchów Reidemeistera o flype, przejście, 2-przejście, ruch Perko oraz kilka innych ezoterycznych przekształceń tak, żeby poradzić sobie z upartymi parami diagramów.
Wśród pierwszych węzłów do 16 skrzyżowań na 32 wyjątkowe (niehiperboliczne) węzły składa się 12 węzłów torusowych oraz 20 satelitów trójlistnika.
To samo jeszcze raz napiszemy na stronie \pageref{page:nonhyperbolic_below_16}.




\subsubsection{Dziewiętnaście skrzyżowań}
Artykuł \cite{thistlethwaite98} zawiera informację, że jego autorzy szukają węzłów o~17 skrzyżowaniach, ale my nie doszukaliśmy się żadnej późniejszej publikacji na ten temat.
\index[persons]{Hoste, Jim}%
\index[persons]{Thistlethwaite, Morwen}%
\index[persons]{Weeks, Jeff}%
W 2004 Flint, Rankin oraz Schermann \cite{rankin04} znaleźli alternujące węzły do 22 skrzyżowań (obliczenia na stacji roboczej z procesorem Xeon oraz 3 gigabajtami pamięci zajęły około 45 godzin), po czym długo nie działo się nic.
\index[persons]{Flint, Ortho}%
\index[persons]{Rankin, Stuart}%
\index[persons]{Schermann, John}%

Dopiero w 2020 Burton \cite{burton20} stablicował węzły pierwsze do 19 skrzyżowań: \emph{,,Here we extend the tables from 16 to 19 crossings, with a total of 352 152 252 distinct non-trivial prime knots.''}
\index[persons]{Burton, Benjamin}%

Thistlethwaite opublikował na swojej stronie internetowej, że znalazł węzły pierwsze do 20 skrzyżowań; czekamy na publikację w renomowanym czasopiśmie.
Jeżeli nie popełnił żadnego błędu, to mamy 199 631 989 alternujących i 1 647 687 439 niealternujących węzłów pierwszych o~dwudziestu skrzyżowaniach.
\index[persons]{Thistlethwaite, Morwen}%
% https://web.math.utk.edu/~morwen/k20v4.pdf
Wśród nich mamy 915 satelitów trójlistnika, 5 satelitów ósemki oraz węzeł (3,10)-torusowy; pozostałe węzły są hiperboliczne.



\subsubsection{Sploty}
Cerf \cite{cerf98} pisze, że Conway znalazł wcześniej sploty do 10 skrzyżowań \cite{conway70}, zaś Caudron \cite{caudron82} poprawił wynik do 11 skrzyżowań, ale wszystkie te sploty są niezorientowane, a~naukowcy potrzebują zorientowanych.
\index[persons]{Cerf, Corinne}%
\index[persons]{Conway, John}%
\index[persons]{Caudron, Alain}%
Problem został zaadresowany najpierw przez Dolla i Hoste'a \cite{doll91}, którzy wydali na mikrofilmie tablicę splotów zorientowanych do 9 skrzyżowań, ale ich diagramy nie zawsze pasowały do tych narysowanych w~książce Rolfsena.
\index[persons]{Doll, Helmut}%
\index[persons]{Hoste, Jim}%

Cerf obiecuje pogodzić punkty widzenia Rolfsena oraz Dolla/Hoste'a i tworzy własną tablicę zorientowanych splotów do 11 skrzyżowań.
Sprawdziła jednocześnie poprawność starszych tablic Conwaya -- i nie znalazła żadnych błędów.

