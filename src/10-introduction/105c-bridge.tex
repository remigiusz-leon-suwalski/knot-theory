
% DICTIONARY;bridge number;liczba mostowa;-
\subsection{Liczba mostowa}
\index{liczba mostowa|(}%

Wprowadzona w~1954 przez Schuberta \cite{schubert54}.
\index[persons]{Schubert, Horst}%

\begin{definition}
    Niech $D$ będzie diagramem węzła.
    Liczbę mostów, czyli długich łuków, które biegną tylko przez nadskrzyżowania, nazywamy liczbą mostową diagramu $D$.
    Minimalną liczbę mostową wśród wszystkich diagramów $D$ węzła $K$, $\bridge K$, nazywamy liczbą mostową węzła $K$.
\end{definition}

% źródło: https://knotinfo.math.indiana.edu/descriptions/bridge_index.html
W 2012 roku Musick \cite{musick12} znalazł liczbę mostową węzłów pierwszych o 11 skrzyżowaniach: węzły, które nie są ani wymierne, ani Montesinosa, są trzymostowe.
\index[persons]{Musick, Chad}%
Potem pchnięto granicę wiedzy do 12 skrzyżowań, dokonał tego międzynarodowy zespół: Blair, Kjuczukowa, Velazquez, Villanueva \cite{blair20}.
\index[persons]{Blair, Ryan}%
\index[persons]{Kjuchukova, Alexandra}%
\index[persons]{Velazquez, Roman}%
\index[persons]{Villanueva, Paul}%

Jedynym węzłem jednomostowym jest niewęzeł.
Kolejne w~hierarchii skomplikowania, czyli dwumostowe, to domknięcia wymiernych supłów (piszemy o~nich w podsekcji \ref{sub:twobridge}).
Fukuhama, Ozawa, Teragaito \cite{fukuhama99} pokazali, że trzymostowe węzły genusu jeden są preclami.
\index[persons]{Fukuhama, Satoshi}%
\index[persons]{Ozawa, Makoto}%
\index[persons]{Teragaito, Masakazu}%
\index{genus}%
\index{precel}%
\index{węzeł!trzymostowy}%
Hilden, Montesinos, Tejada, Toro \cite{hilden12} klasyfikują wszystkie węzły trzymostowe przy użyciu tak zwanej reprezentacji motylkowej, podobną do wyniku Schuberta z~sekcji \ref{sub:twobridge}.
\index[persons]{Hilden, Hugh}%
\index[persons]{Montesinos, José}%
\index[persons]{Tejada, Débora}%
\index[persons]{Toro, Margarita}%
\index{liczba mostowa!węzeł trzymostowy}%
\index{węzeł!trzymostowy!see {liczba mostowa}}%
\index{reprezentacja!motylkowa}%

Węzły $n$-mostowe rozkładają się na sumę dwóch wymiernych $n$-supłów.
% źródło: Wiki Bridge_number, TODO: znaleźć lepsze źródło?
\index{supeł!wymierny}%

\begin{proposition}
\label{prp:bridge_additive}%
    Niech $K_1, K_2$ będą węzłami.
    Wtedy $\bridge (K_1) + \bridge(K_2) = \bridge(K_1 \# K_2) + 1$.
\end{proposition}

\begin{proof}[Niedowód]
\index[persons]{Schubert, Horst}%
\index[persons]{Schultens, Jennifer}%
    Schubert pokazał to blisko pół wieku temu w~\cite{schubert54}[s. 279].
    Nowszy dowód pochodzi od Schultens, w~artykule \cite{schultens03} skorzystała z~foliacji na brzegu węzła towarzyszącego satelitarnemu.
    Dokładniejszy opis powyższych prac wykraczałby poza zakres tego opracowania, zostanie więc pominięty.
\end{proof}

Murasugi wspomina w rozdziale 4.3 podręcznika \cite{murasugi96} następującą hipotezę, nie podaje jednak wcale, skąd się wzięła:
\index[persons]{Murasugi, Kunio}%

\begin{conjecture}[mostowo-skrzyźowaniowa]
\index{hipoteza!mostowo-skrzyżowaniowa}%
    Niech $K$ będzie węzłem.
    Wtedy $\crossing K \ge 3 \bridge K - 3$, przy czym równość zachodzi dokładnie dla niewęzła, trójlistnika i~sumy spójnej trójlistników.
\end{conjecture}

Należy więc uzupełnić brakujące informacje.
Murasugi w pracy \cite{murasugi88} przypuszcza, że dla splotów o $\mu$ ogniwach zachodzi nierównosć $\crossing L + \mu - 1 \ge 3 \bridge L - 3$, przedstawia jednocześnie dowód jej szczególnego przypadku, dla alternujących splotów algebraicznych.
\index[persons]{Murasugi, Kunio}%
Hipoteza Murasugiego stanowi uogólnienie dużo starszego problemu pochodzącego od Foxa \cite{fox50}, który zapytał, czy nierówność jest prawdziwa dla węzłów, gdy $\mu = 1$.
\index[persons]{Fox, Ralph}%

\begin{proposition}
\index{liczba gordyjska}%
\label{no_relation_bridge_unknotting}%
    Nie istnieje bezpośredni związek między liczbą mostową oraz gordyjską.
\end{proposition}

Wiemy o tym z książki Livingstona \cite[s. 146]{livingston93}.

\begin{proof}
    Niech $K_n$ będzie węzłem $(2, 2n+1)$-torusowym.
    Wtedy $K_n$ jest dwumostowy i~jego liczba gordyjska wynosi $n$, to jest dokładnie treść hipotezy Milnora (patrz fakt \ref{prp:torus_unknotting_number}).

    Livingston pisze, że Schubert udowodnił następujący wynik.
    Jeśli węzeł $K'$ jest dublem węzła $K$, to $\bridge K' = 2 \bridge K$ chyba, że $K$ jest niewęzłem (Livingston nie mówi o niewęźle, tylko zostawia znalezienie tego wyjątku jako ćwiczenie).
    Zatem wielokrotne dublowanie prowadzi do węzłów o~dowolnie dużej liczbie mostowej.
    Duble są 1-gordyjskie: rysunek 7.12 w książce Livingstona \cite[s. 146]{livingston93} świetnie to pokazuje.
\end{proof}

% Podobnie nie ma zależności między liczbą mostową oraz genusem.
% \index{genus}%
% TODO: zamiast tego, z każdego miejsca dolinkować znowu do \subsection{Podsumowanie}

\index{liczba mostowa|)}%

% Koniec podsekcji Liczba mostowa

