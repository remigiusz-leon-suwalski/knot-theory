
% DICTIONARY;unknotting number;liczba gordyjska;-
\subsection{Liczba gordyjska}
\index{liczba gordyjska|(}%

\begin{definition}
    Niech $L$ będzie splotem.
    Minimalną liczbę skrzyżowań, które trzeba odwrócić na pewnym jego diagramie, by dostać niewęzeł, nazywamy liczbą gordyjską i~oznaczamy $\unknotting L$.
\end{definition}

Zgodnie z ,,klasyczną'' definicją, między odwracaniem kolejnych skrzyżowań mamy prawo wykonać izotopie otaczające; natomiast zgodnie ze ,,standardową'' definicją, takie izotopie są zabronione.
Obie definicje są równoważne: tłumaczy to książka Adamsa \cite[s. 58]{adams94}.

\begin{lemma}
    \label{lem:unknotting_well_defined}
    W dowolnym rzucie splotu można odwrócić pewne skrzyżowania tak, by uzyskać diagram niesplotu.
\end{lemma}

\begin{proof}
    Bez straty ogólności założę, że diagram przedstawia węzeł.
    Ustalmy zatem diagram węzła i~wybierzmy jakiś początkowy punkt na nim, różny od skrzyżowania wraz z~kierunkiem, wzdłuż którego będziemy przemierzać węzeł.
    Za każdym razem, kiedy odwiedzamy nowe skrzyżowanie, zmieniamy je w~razie potrzeby na takie, przez które przemieszczamy się wzdłuż górnego łuku.
    Skrzyżowań już odwiedzonych nie zmieniamy wcale.

    Teraz wyobraźmy sobie nasz nowy węzeł w~trójwymiarowej przestrzeni $\mathbb R^3$, przy czym oś $z$ skierowana jest z~płaszczyzny, w~której leży diagram, w~naszą stronę.
    Umieśćmy początkowy punkt tak, by jego trzecią współrzędną była $z = 1$.

    Przemierzając węzeł, zmniejszamy stopniowo tę współrzędną, aż osiągniemy wartość $0$ tuż przed punktem, z~którego wyruszyliśmy.
    Połączmy obydwa punkty (początkowy oraz ten, w~którym osiągamy współrzędną $z = 0$) pionowym odcinkiem.
    Zauważmy, że kiedy patrzymy na węzeł w~kierunku osi $z$, nie widzimy żadnych skrzyżowań.

    Oznacza to, że nasza procedura przekształciła początkowy diagram w~diagram niewęzła, co należało okazać.
\end{proof}

W pracy \cite{shimizu14} Shimizu rozpatruje różne operacje, które rozwiązują węzły lub sploty.
\index[persons]{Shimizu, ?}%
Nie będziemy się nimi zajmować, podamy tylko przykład: zamiana pod- i nadskrzyżowań wokół obszaru na diagramie rozwiązuje węzły, ale nie sploty; kontrprzykładem jest splot Hopfa.
\index{splot!Hopfa}%
Patrz też \cite[s. 141-154]{kawauchi96}.

Dla każdego nietrywialnego splotu istnieje diagram wymagający odwrócenia dowolnie wielu skrzyżowań.
Wcześniej Nakanishi znalazł 2-gordyjski diagram 1-gordyjskiego węzła $6_2$ (\cite{nakanishi83}) oraz udowodnił, że każdy nietrywialny węzeł ma diagram, który nie jest 1-gordyjski (\cite{nakanishi96}).
\index[persons]{Nakanishi, Yasutaka}%
Dowód zawiera\footnote{Czy to oznacza, że praca Nakanishiego jednak nie zawiera dowodu?} praca \cite{taniyama09} Taniyamy.
\index[persons]{Taniyama, Kouki}%
Pokazany jest tam jeszcze jeden godny uwagi fakt.
Jeśli liczba gordyjska diagramu $D$ wynosi $\frac 12 (\crossing D - 1)$, co jest maksymalną możliwą wartością zgodnie z~naszym prostym ograniczeniem,
to węzeł jest $(2,p)$-torusowy albo wygląda jak diagram niewęzła po pierwszym ruchu Reidemeistera.


\subsubsection{Sploty 1-gordyjskie}
Sploty o liczbie gordyjskiej 1 zasługują na szczególną uwagę.

\begin{proposition}
\index{węzeł!wymierny}%
    Niech $L$ będzie wymiernym splotem 1-gordyjskim.
    Wtedy na minimalnym diagramie $L$ jedno ze skrzyżowań jest rozwiązujące.
\end{proposition}

\begin{proof}
\index[persons]{Kanenobu, Taizo}%
\index[persons]{Murakami, Hitoshi}%
\index[persons]{Kohn, Peter}%
    Kanenobu, Murakami dla węzłów \cite{kanenobumurakami86}, wkrótce po tym Kohn dla splotów \cite{kohn91}.
\end{proof}

\begin{proposition}
\label{prp:unknotting_one_prime}%
    Węzły $1$-gordyjskie są pierwsze.
\end{proposition}

Podejrzewał to Hilmar Wendt w~1937 roku, kiedy policzył liczbę gordyjską węzła babskiego używając homologii rozgałęzionego nakrycia cyklicznego \cite{wendt37}.
\index[persons]{Wendt, Hilmar}%

\begin{proof}[Niedowód]
\index[persons]{Scharlemann, Martin}%
\index[persons]{Lackenby, Marc}%
\index[persons]{Zhang, Xingru}%
    W pracy \cite{scharlemann85} z~1985 roku Scharlemann podał dość zawiłe uzasadnienie, w~które zamieszane były grafy planarne.
    Obecnie znamy prostsze dowody, patrz \cite{lackenby97} (Lackenby) albo \cite{zhang91} (Zhang).
\end{proof}

Scharlemann pokazał w \cite[wniosek 1.6]{scharlemann98}, że liczba gordyjska jest podaddytywna, to znaczy zachodzi $\unknotting(K_1 \shrap K_2) \le \unknotting(K_1) + \unknotting(K_2)$.
Stąd oraz z faktu \ref{prp:unknotting_one_prime} wynika, że suma dwóch $1$-gordyjskich węzłów jest $2$-gordyjska, ale od początku teorii węzłów podejrzewano dużo więcej:

\begin{conjecture}
\index{hipoteza!o liczbie gordyjskiej}%
    Niech $K_1, K_2$ będą węzłami.
    Wtedy $\unknotting (K_1 \shrap K_2) = \unknotting(K_1) + \unknotting(K_2)$, czyli liczba gordyjska jest addytywna.
\end{conjecture}

Niech $K$ będzie 1-gordyjskim węzłem o genusie 1.
Wtedy $K$ jest dublem pewnego węzła (Scharlemann, Thompson \cite{thompson88}, Kobayashi \cite{kobayashitsuyoshi89}).
\index[persons]{Kobayashi, Tsuyoshi}%
\index[persons]{Scharlemann, Martin}%
\index[persons]{Thompson, Abigail}%
Dużo później Coward, Lackenby dowiedli w~\cite{coward11}, że z~dokładnością do pewnej relacji równoważności, tylko jedna zmiana skrzyżowania rozwiązuje węzeł $K$; chyba że ten jest ósemką -- wtedy takie zmiany są dwie.
\index[persons]{Coward, Alexander}%
\index[persons]{Lackenby, Marc}%




\subsubsection{Dolne ograniczenia liczby gordyjskiej}
Dokładna wartość liczby gordyjskiej jest znana tylko dla niektórych klas węzłów, na przykład torusowych (fakt \ref{prp:torus_unknotting_number}) albo skręconych.
\index{węzeł!torusowy}%
\index{węzeł!skręcony}%

Jeśli odwrócenie pewnych skrzyżowań daje niewęzeł, to odwrócenie pozostałych także.
To daje proste liczby gordyjskiej: $2 \unknotting K \le \crossing K$.
Nie jest zbyt pomocne, daje rozstrzygnięcie pięć razy dla pierwszych węzłów do 12 skrzyżowań: $3_{1}$, $5_{1}$, $7_{1}$, $9_{1}$, $11a_{367}$.

Borodzik oraz Friedl podali niedawno całkiem mocne ograniczenia na liczbę gordyjską w~pracach \cite{borodzik14} i~\cite{borodzik15}.
\index[persons]{Borodzik, Maciej}%
\index[persons]{Friedl, Stefan}%
Ich narzędziem jest parowanie Blanchfielda.
\index{parowanie Blanchfielda}%
Poprawiają tam starsze estymaty wynikające z~sygnatury Levine'a-Tristrama, indeksu Nakanishiego oraz przeszkody Lickorisha.
\index{indeks Nakanishiego}%
\index{przeszkoda Lickorisha}%
\index{sygnatura!Levine'a-Tristrama}%
Wśród pierwszych węzłów o~co najwyżej 12 skrzyżowaniach dwadzieścia pięć ma liczbę gordyjską równą co najmniej trzy, trudno było uzasadnić to innymi metodami.




\subsubsection{Znane wartości}
Dotychczas wyznaczono liczbę gordyjską prawie wszystkich węzłów pierwszych o~co najwyżej dziesięciu skrzyżowaniach.
Cha, Livingston \cite{cha18} podają następującą listę wyjątków:
\index[persons]{Cha, Jae}%
\index[persons]{Livingston, Charles}%
$10_{11}$, $10_{47}$, $10_{51}$, $10_{54}$, $10_{61}$, $10_{76}$, $10_{77}$, $10_{79}$, $10_{100}$ (stan na rok 2018).
Poniżej podajemy za stroną internetową KnotInfo\footnote{Patrz \url{https://knotinfo.math.indiana.edu/descriptions/unknotting_number.html}. Pomijając węzły torusowe, skopiowane przez nas na liście oraz 1-gordyjskie, do dziesięciu skrzyżowań zostawia to: 2 węzły o~siedmiu skrzyżowaniach, 3 o~ośmiu, 15 o~dziewięciu i~68 o~dziesięciu.} listę odkrywców liczb gordyjskich węzłów do 10 skrzyżowań.
KnotInfo wymienia więcej, bo węzły do 12 skrzyżowań.

\begin{compactitem}
\item Lickorish \cite{lickorish85}: $7_{4}$.
\index[persons]{Lickorish, William}%
\item Kanenobu, Murakami \cite{kanenobumurakami86}: $8_{3}$, $8_{4}$, $8_{6}$, $8_{8}$, $8_{12}$, $9_{5}$, $9_{8}$, $9_{15}$, $9_{17}$, $9_{31}$.
\index[persons]{Kanenobu, ?}%
\index[persons]{Murakami, ?}%
\item Szabó \cite{szabo05}: $8_{10}$, $10_{48}$, $10_{52}$, $10_{54}$ ($\unknotting \neq 1$), $10_{57}$, $10_{58}$, $10_{64}$, $10_{68}$, $10_{70}$, $10_{77}$ ($\unknotting \neq 1$), $10_{110}$, $10_{112}$, $10_{116}$, $10_{117}$, $10_{125}$, $10_{126}$, $10_{130}$, $10_{135}$, $10_{138}$, $10_{158}$, $10_{162}$.
\index[persons]{Szabó, ?}%
\item Murakami, Yasuhara \cite{yasuhara00}, Stojmenow \cite{stoimenow04}: $8_{16}$.
\index[persons]{Yasuhara, Akira}%
\index[persons]{Stoimenow, Alexander}%
\item Stojmenow \cite{stoimenow04}: $8_{18}$, $9_{37}$, $9_{40}$, $9_{46}$, $9_{48}$, $9_{49}$, $10_{103}$.
\item Owens \cite{owens08}: $9_{10}$, $9_{13}$, $9_{35}$, $9_{38}$, $10_{53}$, $10_{101}$, $10_{120}$.
\index[persons]{Owens, Brendan}%
\item Kanenobu, Murakami \cite{kanenobumurakami86}, Stojmenow \cite{stoimenow04}: $9_{15}$, $9_{17}$.
\item Kobayashi \cite{kobayashi89}: $9_{25}$.
\index[persons]{Kobayashi, ?}%
\item Gordon, Luecke \cite{gordon06}, Szabó \cite{szabo05}: $9_{29}$, $10_{81}$, $10_{87}$, $10_{90}$, $10_{93}$, $10_{94}$, $10_{96}$.
\index[persons]{Gordon, Cameron}%
\index[persons]{Luecke, ?}%
\item Adams? \cite[s. 62]{adams94}: $10_{8}$.
\index[persons]{Adams, ?}%
\item Miyazawa \cite{miyazawa98}: $10_{65}$, $10_{69}$, $10_{89}$, $10_{108}$, $10_{163}$, $10_{165}$.
\index[persons]{Miyazawa, ?}%
\item Traczyk \cite{traczyk99}, Szabó \cite{szabo05}: $10_{67}$.
\index[persons]{Traczyk, Paweł}%
\item Szabó \cite{szabo05}, ($\unknotting \neq 1$ Gordon, Luecke \cite{gordon06}): $10_{79}$.
\item Gordon, Luecke \cite{gordon06}, Szabó \cite{szabo05}, Nakanishi \cite{nakanishi05}: $10_{83}$.
\index[persons]{Nakanishi, Yasutaka}%
\item Stojmenow \cite{stoimenow04}, Szabó \cite{szabo05}, Gordon, Luecke \cite{gordon06}: $10_{86}$.
\item Miyazawa \cite{miyazawa98}, Nakanishi \cite{nakanishi05}: $10_{97}$.
\item Szabó \cite{szabo05}, Stojmenow \cite{stoimenow04}, Nakanishi \cite{nakanishi05}: $10_{105}$, $10_{106}$, $10_{109}$, $10_{121}$.
\item Stojmenow \cite{stoimenow04}, Szabó \cite{szabo05}: $10_{131}$ (jedyny 1-gordyjski na tej liście!).
\item Gibson, Ishikawa \cite{ishikawa02}: $10_{139}$, $10_{145}$, $10_{152}$.
\index[persons]{Gibson, William}%
\index[persons]{Ishikawa, Masaharu}%
\item Gordon, Luecke \cite{gordon06}, Szabó \cite{szabo05}: $10_{148}$, $10_{151}$.
\item Gordon, Luecke \cite{gordon06}: $10_{153}$.
\item Stojmenow \cite{stoimenow03}, Gibson, Ishikawa \cite{ishikawa02}: $10_{154}$.
\item Gibson, Ishikawa \cite{ishikawa02}: $10_{161}$.
\end{compactitem}




\subsubsection{Przykład Bleilera. Hipoteza Bernharda-Jablana}
% TODO: Bleilera? Niezależnie odkrył to Nakanishi - MR0749196, jak wiem z MR1265448
Bleiler odkrył w~\cite{bleiler84} fascynujący przykład wymiernego węzła $10_8$, który jest $2$-gordyjski, ale świadkiem tego nie może być żaden diagram mininalny, ponieważ, co jeszcze bardziej fascynujące, węzeł ten posiada tylko jeden diagram o~dziesięciu skrzyżowaniach oraz liczbie gordyjskiej 3.
\index[persons]{Bleiler, Steven}%
Wynika stąd, że liczba $\unknotting$ nie musi być osiągana przez diagram minimalny, wbrew powszechnym przypuszczeniom obecnym jeszcze w latach 70.
Praca \cite{bernhard94} opisuje nieskończoną rodzinę węzłów $C_k$, gdzie $C_2 = 10_8$ jest węzłem Bleilera.

Przykład Bleilera pokazuje, że do szukania liczby gordyjskiej potrzeba wyrafinowanego algorytmu.
Ponieważ odwrócenie jednego ze skrzyżowań na minimalnym diagramie węzła $10_8$ daje $1$-gordyjski węzeł $4_1, 5_1, 6_1$ lub $6_2$, możemy liczyć, że każdy diagram minimalny ma skrzyżowanie, którego odwrócenie zmniejsza liczbę gordyjską.
Dlatego jeszcze w~latach 90. postawiono (\cite{bernhard94}, \cite{jablan98}) hipotezę:
\begin{conjecture}[Bernharda-Jablana]
\index{hipoteza!Bernharda-Jablana}%
\index[persons]{Jablan, Slavik}%
\index[persons]{Bernhard, James}%
\label{con:bernhard_jablan}%
    Każdy węzeł $K$ posiada diagram $D$ realizujący liczbę gordyjską oraz skrzyżowanie, którego odwrócenie daje nowy węzeł $K'$ z diagramem $D'$ o mniejszej liczbie gordyjskiej: $u(D') < u(D)$.
\end{conjecture}

Zakładając prawdziwość hipotezy \ref{con:bernhard_jablan}, mamy prosty sposób na wyznaczenie liczby $u(K)$: weźmy skończenie wiele diagramów minimalnych dla węzła $K$, na każdym z~nich odwracajmy skrzyżowania i rekursywnie szukajmy liczb gordyjskich prostszych węzłów.
Najmniejsza spośród nich różni się wtedy o~jeden od liczby $u(K)$.

Brittenham, Hermiller w artykule \cite{brittenham17} twierdzą, że hipoteza jest fałszywa, ten nie został jednak jeszcze zrecenzowany.
\index[persons]{Brittenham, Mark}%
\index[persons]{Hermiller, Susan}%
Prawdziwość sprawdzono natomiast dla węzłów do jedenastu skrzyżowań oraz splotów o dwóch ogniwach do dziewięciu skrzyżowań (Kohn w \cite{kohn93}?).
\index[persons]{Kohn, Peter}%

\begin{example}[Brittenham, Hermiller]
    Hipoteza Bernharda-Jablana jest fałszywa dla co najmniej jednego spośród czterech węzłów: $12n_{288}$, $12n_{491}$, $12n_{501}$, $13n_{3370}$.
\end{example}

Bleiler postawił w~\cite{bleiler84} problem: czy jeden węzeł może mieć kilka diagramów minimalnych, z~których tylko niektóre są świadkiem $1$-gordyjskości?
Rozwiązanie przyszło wkrótce z Japonii: według \cite{kanenobumurakami86} dzieje się tak m.in. dla węzła $8_{14}$.
\index[persons]{Kanenobu, Taizo}%
\index[persons]{Murakami, Hitoshi}%
Stojmenow w~pracy \cite{stoimenow01} pełnej różnych przykładów wskazał dodatkowo węzły $14_{36750}$ oraz $14_{36760}$.
\index[persons]{Stoimenow, Alexander}%




\subsubsection{Liczba gordyjska jako metryka}
Mając dane dwa węzły $K_0, K_1$, rozpatrzmy wszystkie homotopie
\begin{equation}
    f : [0,1] \times S^1 \to \R^3
\end{equation}
takie, że wszystkie funkcje $f_t$ są zanurzeniami z co najwyżej jednym punktem podwójnym.
Zażądajmy dodatkowo, by styczne do krótkich łuków, które przecinają się w tym punkcie, były od siebie różne.
Odległością gordyjską między węzłami $K_0, K_1$ jest minimalna liczba podwójnych punktów, jakie posiada homotopia $f$.
Twierdzenie C~z~pracy Gambaudo, Ghysa \cite{gambaudo05} głosi, że przestrzeń wszystkich węzłów wyposażona w taką metrykę zawiera prawie idealną kopię przestrzeni euklidesowej dowolnego wymiaru.
\index[persons]{Gambaudo, Jean-Marc}%
\index[persons]{Ghys, Étienne}%
Dokładniej:

\begin{proposition}
    Dla każdej liczby całkowitej $n \ge 1$ istnieje funkcja $\xi: \Z^n \to \mathcal{K}$, dodatnie stałe $A, B, C$ oraz norma $\|\cdot\|$ na przestrzeni $\R^n$ takie, że spełniona jest podwójna nierówność
    \begin{equation}
        A\|x-y\|  - B \le d(\xi(x), \xi(y)) \le C\|x-y\|.
    \end{equation}
\end{proposition}

To nie jest główne twierdzenie tamże, a~raczej efekt uboczny pracy nad głównym wynikiem: autorzy definiują $\omega$-sygnaturę domknięcia warkocza, a~że sklejenie dwóch 4-rozmaitości z~narożnikami\footnote{Z angielskiego corners.} nie odpowiada dodaniu ich sygnatur, to ich funkcja nie jest homomorfizmem.
Wspomniany jest wzór Novikowa-Walla, który wyraża różnicę pewnych defektów jako indeks Masłowa i (to jest główne twierdzenie) różnica ta pokrywa się z kocyklem Meyera reprezentacji Burau-Squiera, cokolwiek to znaczy.
Pojawia się również jakaś funkcja Rademachera.

Grupy warkoczowe poznamy, ale znacznie później: w sekcji \ref{sec:braid}.



\index{liczba gordyjska|)}

% Koniec podsekcji Liczba gordyjska

