
\section{Węzły i~sploty}
Wprowadzamy pojęcia węzła i splotu, fundamentalnych obiektów teorii, o której napisana została ta książka.
Oprócz tego podajemy definicję, kiedy dwa węzły lub sploty uznajemy za tożsame oraz uzasadniamy, dlaczego akurat ta definicja jest właściwa.

Istnieją odnogi teorii węzłów, które badają inne, pokrewne obiekty.
Mamy na przykład węzły obramowane:

% DICTIONARY;framed;obramowany;węzeł
% GLOSSAIRE;encadré;obramowany;węzeł
% DICTIONARY;framing;obramowanie;-
% GLOSSAIRE;encadrement;obramowanie;-
\begin{definition}[obramowanie]
\index{obramowanie|see {węzeł obramowany}}%
\index{węzeł!obramowany}%
    Każde nieznikające normalne pole wektorowe $V$ na splocie $L$ nazywamy jego obramowaniem.
    Szczególnie interesujący jest przypadek, gdy wszystkie wektory tego pola są równoległe do płaszczyzny, na której leży diagram tego splotu.
\end{definition}

% DICTIONARY;virtual;wirtualny;węzeł
% GLOSSAIRE;virtuel;wirtualny;węzeł
% DICTIONARY;welded;zespawany;węzeł
% GLOSSAIRE;soudé;zespawany;węzeł
% DICTIONARY;long;długi;węzeł
% GLOSSAIRE;long;długi;węzeł
Są jeszcze węzły wirtualne, zespawane (iloraz węzłów wirtualnych przez ruch znany jako ,,nadskrzyżowania komutują''), długie (gdzie końce nie są ze sobą zszyte, ale umieszczone tak daleko, że są nieosiągalne) i inne.
% http://katlas.math.toronto.edu/ester/weldedknots/explanations.html => because the "overcrossings commute" move is not symmetric
\index{węzeł!wirtualny}%
\index{węzeł!zespawany}%
\index{węzeł!długi}%
Ta książka nie zawiera zbyt wiele informacji o wspomnianych bytach.


\subsection{Węzły}
Matematyczne węzły można traktować jako model elastycznej oraz pozbawionej grubości liny, której luźne końce zostały ze sobą połączone.
Sugeruje to przyjęcie naiwnej definicji:

\begin{definition}[węzeł (prawie)]
\index{węzeł}%
    Ciągłe oraz różnowartościowe odwzorowanie $S^1 \to \R^3$ nazywamy węzłem.
\end{definition}

Takie rozwiązanie nie jest doskonałe, ponieważ oprócz pożądanych (cokolwiek to znaczy) węzłów, obejmuje wiele innych, patologicznych obiektów takich jak ten z~rysunku \ref{fig_wild_knot}.
Milnor \cite{milnor1964} udowodnił, że ,,większość'' węzłów jest dzika.

\begin{figure}[H]
    \centering
\begin{comment}
    \includegraphics[height=0.14\linewidth]{wild_knot.png}
\end{comment}
    \caption[caption-wild-knot]{Węzeł dziki, źródło: Wikimedia{\footnotemark}}
\index{węzeł!dziki}%
\label{fig_wild_knot}%
\end{figure}
\footnotetext{\url{https://upload.wikimedia.org/wikipedia/commons/2/2f/Wildknot.svg}}

Zamiast wyjaśnić, jakie są jego niepożądane właściwości, podamy od razu dobrą definicję.

\begin{definition}[węzeł]
    Różnowartościowe włożenie $S^1 \to \R^3$, którego pochodna istnieje wszędzie i~nie znika nigdzie, nazywamy węzłem.
\end{definition}

\begin{example}[niewęzeł]
    Węzeł zadany w przestrzeni $\R^3$ parametrycznie $(\sin \theta, \cos \theta, 0)$ dla $\theta \in [0, 2\pi]$ nazywamy niewęzłem i oznaczamy $\SmallUnknot$.
\end{example}

Potrzeba jeszcze matematycznego opisu manipulacji, jakim możemy poddawać sznur trzymany w~ręce.
Izotopia jest niewłaściwym narzędziem do tego celu: powiedzielibyśmy, że dwa węzły $K_1, K_2$ są izotopijne, jeśli istnieje ciągła funkcja
\begin{equation}
    F \colon S^1 \times [0, 1] \to \R^3
\end{equation}
taka, że $K_1 = F(-, 0)$ jest pierwszym, zaś $K_2 = F(-,1)$ drugim węzłem (funkcję $F$ nazywa się izotopią).
Tym razem źródło problemów można wskazać jawnie.
Dowolny zaplątany fragment z węzła można usunąć wykonując sztuczkę Alexandera (ponieważ jak powiedziałyby mądre głowy, ,,przestrzeń homeomorfizmów dysku w siebie $D^{n+1} \to D^{n+1}$, które zgadzają się z odwzorowaniem tożsamościowym na brzegu dysku -- sferze $S^n$, jest spójna''):
\index[persons]{Alexander, James}%
\index{sztuczka Alexandera}%

\begin{comment}
\begin{figure}[H]
    \centering
    \fbox{\begin{minipage}[b]{.12\linewidth}
        \centering
        \includegraphics[width=\linewidth]{../data/alexander-trick/0.png}
        \subcaption{$t = 0$}
    \end{minipage}}\,\,
    \fbox{\begin{minipage}[b]{.12\linewidth}
        \centering
        \includegraphics[width=\linewidth]{../data/alexander-trick/1.png}
        \subcaption{$t = 1/4$}
    \end{minipage}}\,\,
    \fbox{\begin{minipage}[b]{.12\linewidth}
        \centering
        \includegraphics[width=\linewidth]{../data/alexander-trick/2.png}
        \subcaption{$t=1/2$}
    \end{minipage}}\,\,
    \fbox{\begin{minipage}[b]{.12\linewidth}
        \centering
        \includegraphics[width=\linewidth]{../data/alexander-trick/3.png}
        \subcaption{$t=3/4$}
    \end{minipage}}\,\,
    \fbox{\begin{minipage}[b]{.12\linewidth}
        \centering
        \includegraphics[width=\linewidth]{../data/alexander-trick/4.png}
        \subcaption{$t = 1$}
    \end{minipage}}
    \caption[caption-alexander-trick]{Sztuczka Alexandera \cite[s. 2]{burde2014}}
\end{figure}
\end{comment}

W podobny sposób moglibyśmy przekształcić dowolny węzeł w~niewęzeł.
Teoria, w~której wszystkie obiekty są takie same, nie jest zbyt ciekawa.
Od izotopii należy wymagać gładkości albo lokalnej płaskości,
% https://math.stackexchange.com/questions/1311865/equivalence-of-knots-ambient-isotopy-vs-homeomorphism
co zdaje się prowadzić do pojęcia izotopii otaczającej, która uwzględnia nie tylko sam węzły, ale też to, jak leżą w otaczającej je przestrzeni.

% DICTIONARY;isotopy;izotopia;-
% GLOSSAIRE;isotopie;izotopia;-
% DICTIONARY;ambient;otaczająca;izotopia
% GLOSSAIRE;ambiante;otaczająca;izotopia
\begin{definition}[izotopia otaczająca]
    \index{izotopia otaczająca}%
    Niech $K_1, K_2 \colon N \hookrightarrow M$ będą włożeniami dwóch rozmaitości $N, M$.
    Ciągłe odwzorowanie $F \colon M \times [0,1] \to M$ spełniające następujące warunki:
    \begin{enumerate}
        \item funkcja $F(-, 0)$ jest odwzorowaniem tożsamościowym,
        \item każda z~funkcji $F(-, t)$ jest homeomorfizmem,
        \item złożenie $F(-, 1)$ z~pierwszym włożeniem $K_1$ daje drugie włożenie $K_2$
    \end{enumerate}
    nazywamy izotopią otaczającą przenoszącą włożenie $K_1$ na $K_2$.
\end{definition}

W~topologii rozważa się włożenia dowolnych rozmaitości, nam wystarczy jeden szczególny przypadek $N = S^1$ oraz $M = \R^3$.
Intuicyjnie, funkcja $F$ zniekształca przestrzeń $\R^3$ tak, że w~chwili początkowej $t = 0$ widzimy pierwszy, zaś w~chwili końcowej $t = 1$ drugi węzeł.
Izotopia otaczająca nie pozwala na ściąganie zaplątanych fragmentów do punktu.

\begin{definition}
    Dwa węzły są równoważne wtedy i tylko wtedy, gdy istnieje pomiędzy nimi izotopia otaczająca.
\end{definition}

Znając już izotopię otaczającą, można podać alternatywny opis węzłów:

% DICTIONARY;knot;węzeł;-
% GLOSSAIRE;nœud;węzeł;-
% DICTIONARY;tame;poskromiony;węzeł
% GLOSSAIRE;lisse/régulier;poskromiony;węzeł
% DICTIONARY;wild;dziki;węzeł
% GLOSSAIRE;sauvage;dziki;węzeł
\begin{definition}[węzeł]
\index{węzeł!poskromiony}%
\label{def:knot}%
    Gładkie włożenie $S^1 \hookrightarrow \R^3$ otaczająco izotopijne z~zamkniętą łamaną bez samoprzecięć nazywamy węzłem poskromionym.
\end{definition}

Czasami wygodniej jest rozpatrywać węzeł jako włożenie $S^1 \hookrightarrow S^3$ albo dopuścić do myśli węzły nieposkromione.
Ale jeśli nie zaznaczono inaczej, nie robimy tego: pisząc węzeł mamy na myśli poskromione włożenie w przestrzeń $\R^3$, nie $S^3$.

Formalnie węzły to pewne odwzorowania, więc prawidłowym sposobem na zapisanie, że są izotopijne (czyli dla nas: równe), jest $K_1 \cong K_2$.
Ponieważ nie prowadzi to do problemów, będziemy jednak stosować zapis $K_1 = K_2$.
Jednocześnie często węzeł jako odwzorowanie nie będzie odróżniany od obrazu tego odwzorowania.

Istnieje jeszcze jedna, konkurencyjna definicja węzłów równoważnych:

\begin{proposition}
\label{def:equivalent_knots_2}%
    Dwa węzły są równoważne wtedy i~tylko wtedy, gdy jeden z~nich jest obrazem drugiego przez zachowujący orientację homeomorfizm $\R^3 \to \R^3$.
\end{proposition}

\begin{proof}
    Podany niżej dowód pochodzi z~książki ,,Topology from the differentiable viewpoint'' Johna Milnora.
\index[persons]{Milnor, John}%
    Musimy pokazać, że dyfeomorfizm $f \colon \R^m \to \R^m$ jest gładko izotopijny z~identycznością.
    Translacje są izotopiami, więc bez straty ogólności zakładamy, że $f(0) = 0$.
    Pochodna $f$ w~zerze jest dana wzorem $\mathrm{d}f_0(x) = \lim_{t \to 0} f(tx) /t$, więc
    \begin{equation}
        F(x, t) = \begin{cases}
            \mathrm{d}f_0(x) & t = 0 \\
            f(tx) / t & 0 < t \le 1
        \end{cases} .
    \end{equation}
    stanowi naturalną definicję izotopii $F \colon \R^m \times [0, 1] \to \R^m$.
    Funkcja $f$ zapisuje się na mocy lematu Hadamarda jako suma $x_1 g_1(x) + \ldots + x_mg_m(x)$, gdzie funkcje $g_i$ są gładkie, więc funkcja $F$ też jest gładka, co jakoś kończy dowód.
\index{lemat Hadamarda}%    
\end{proof}

Milnor zauważa, że istnieje dyfeomorfizm $S^6 \to S^6$ stopnia $+1$, który nie jest gładko izotopijny z~identycznością!
\index[persons]{Milnor, John}%

\begin{remark}[John Willard Milnor]
    Matematyk amerykański urodzon w 1931 roku w Orange, New Jersey.
    Odkrył egzotyczną 7-wymiarową sferę (czyli zwykłą sferę z niezwykłą strukturą różniczkową) w 1956 roku, za co został później odznaczon medalem Fieldsa.
    Obalił Hauptvermutung pięć lat później: hipotezę Steinitza i Tietzego z 1908 roku, że każde dwie triangulacje przestrzeni mają kombinatorycznie równoważne podpodziały.
    Do jego zainteresowań należą topologia różniczkowa, algebraiczna K-teorią, ale też algebry Hopfa, grupy Liego i holomorficzne układy dynamiczne.
    W~świecie węzłów jest znany przez wprowadzenie niezmienników $\mu$ Milnora, które uogólniają grupę podstawową dopełnienia oraz pewne wyniki dotyczące hipotezy plastrowo-taśmowej.
\end{remark}
\index{niezmiennik!$\mu$ Milnora}%
\index{hipoteza!plastrowo-taśmowa}%
\index[persons]{Milnor, John}%




\subsection{Sploty}
% DICTIONARY;link;splot;-
% GLOSSAIRE;un entrelacs;splot;-
% DICTIONARY;component;ogniwo splotu;-
% GLOSSAIRE;composante/boucle;ogniwo splotu;-
\begin{definition}[splot, ogniwo]
\index{splot}%
    Sumę parami rozłącznych węzłów
    \begin{equation}
        L = K_1 \sqcup K_2 \sqcup \ldots \sqcup K_n
    \end{equation}
    nazywamy splotem, a~składniki $K_i$ -- ogniwami splotu.
\end{definition}

Przez analogię do węzłów mówimy, że dwa sploty są takie same, jeśli jeden jest obrazem drugiego przez zachowujący orientację homeomorfizm $\R^3 \to \R^3$.
To oczywiste, że liczba ogniw jest niezmiennikiem splotów.
Później podamy mniej oczywiste niezmienniki.

\begin{example}[niesplot]
\index{niesplot}%
    Splot zadany w przestrzeni $\R^3$ parametrycznie
    \begin{equation}
        \bigcup_{k=1}^n \{(\sin \theta, \cos \theta, k) : \theta \in [0, 2\pi]\}
    \end{equation}
    nazywamy niesplotem i oznaczamy $U_n$.
\end{example}
    
\begin{example}
\index{splot!Hopfa}%
\index[persons]{Hopf, Heinz}%
    Splot Hopfa (rys. \ref{small_links_diagram}a), najprostszy nietrywialny splot. Ma dwa ogniwa.
\end{example}

\begin{remark}[Heinz Hopf]
    Matematyk niemiecki urodzon w 1894 roku w Gräbschen (obecnie część Wrocławia); zmarł w 1971 roku w Zurychu, Szwajcarii.
    Głównym wynikiem jego pracy doktorskiej z~1925 roku było, że każda jednospójna zupełna 3-rozmaitość Riemanna o~stałej krzywiźnie sekcyjnej jest globalnie izometryczna do przestrzeni euklidesowej, sferycznej lub hiperbolicznej.
    Był pionierem topologii algebraicznej.
    W~1931 roku prowadził badania nad tzw. rozwłóknieniem: odwzorowaniem $S^3 \to S^2$ takim, że przeciwobrazy punktów są okręgami wielkimi na 3-sferze.
    Podczas tych badań zajmował się splotem nazywanym teraz splotem Hopfa.
\end{remark}

\begin{example}
\index{splot!Whiteheada}%
\index[persons]{Whitehead, John}%
    Splot Whiteheada (rys. \ref{small_links_diagram}b).
\end{example}

\begin{remark}[John Henry Constantine Whitehead]
    Matematyk brytyjski urodzon w 1903 roku w~Madrasie, Indiach; zmarł w 1960 roku w Princeton, New Jersey.
    Był jednym z założycieli teorii homotopii, podał definicję CW-kompleksów.
\index{CW kompleks}%
    Próbował udowodnić hipotezę Poincarégo, ale popełnił błąd twierdząc, że nie istnieje ściągalna otwarta 3-rozmaitość, która nie jest homeomorficzna z $\R^3$.
\index{hipoteza!Poincarégo}%
\index{rozmaitość!ściągalna}%
    W 1935 roku sam wskazał taką rozmaitość, do jej konstrukcji wykorzystując splot Whiteheada.
\end{remark}

\begin{comment}
    {\setlength{\intextsep}{4pt plus 2pt minus 2pt}
    \begin{figure}[H]
        \centering
        \begin{minipage}[b]{.3\linewidth}
            \centering
            \includegraphics[height=0.6\linewidth]{../data/links/2_2_1.png}
            \subcaption{splot Hopfa}
        \end{minipage}\,\,
        \begin{minipage}[b]{.3\linewidth}
            \centering
            \includegraphics[height=0.6\linewidth]{../data/links/5_2_1.png}
            \subcaption{splot Whiteheada}
        \end{minipage}\,\,
        \begin{minipage}[b]{.3\linewidth}
            \centering
            \includegraphics[height=0.6\linewidth]{../data/links/6_3_2.png}
            \subcaption{pierścienie Boromeuszy}
            \index{pierścienie Boromeuszy}%
        \end{minipage}
        \caption[small-links]{Sploty o małej liczbie skrzyżowań}
        \label{small_links_diagram}
    \end{figure}
    }
\end{comment}

\subsubsection{Sploty rozszczepialne}
Aby wytłumaczyć, czemu trzeci splot z rysunku \ref{small_links_diagram} jest interesujący, potrzebujemy zdefiniować sploty rozszczepialne.

% DICTIONARY;splittable;rozszczepialny;splot
\begin{definition}[rozszczepialność]
\index{splot!rozszczepialny}%
    Jeżeli splot $L$ można zanurzyć w przestrzeni $\R^3$ tak, że niektóre jego ogniwa będą leżeć nad pewną rozłączną ze splotem płaszczyzną, zaś pozostałe pod nią, to powiemy, że splot $L$ jest rozszczepialny.
\end{definition}

Liczbę nierozszczepialnych splotów pierwszych kopiujemy z bazy danych LinkInfo \cite{linkinfo24}:
\renewcommand*{\arraystretch}{1.4}
\footnotesize
\begin{longtable}{lcccccccccccc}
    \hline
    \textbf{skrzyżowania} & 0 & 1 & 2 & 3 & 4 & 5 &  6 &  7 &  8 & 9 & 10 & 11 \\ \hline \endhead
    sploty pierwsze, nierozszczepialne & 0 & 0 & 1 & 0 & 1 & 1 & 6 & 9 & 29 & 83 & 287 & 1007 \\
    (w tym) alternujące & 0 & 0 & 1 & 0 & 1 & 1 & 5 & 7 & 21 & 55 & 174 & 548 \\
    (w tym) niealternujące & 0 & 0 & 0 & 0 & 0 & 0 & 1 & 2 & 8 & 28 & 113 & 459 \\
    \hline
\end{longtable}
\normalsize

W bazie liczb OEIS trafiliśmy tylko na ciąg \href{https://oeis.org/A086826}{A086826} opisujący liczbę nierozszczepialnych pierwszych i złożonych węzłów i splotów, na przykład $a_5 = 4$, bo mamy dwa węzły pierwsze, splot Whiteheada oraz trójlistnik spleciony z~niewęzłem.
Słowa ,,skrzyżowanie'' , ,,alternujący'' oraz ,,pierwszy'' definiujemy w~przyszłości, będą to odpowiednio definicje \ref{def:crossing}, \ref{def:alternating_link} i \ref{def:prime_knot}.
\index{węzeł!alternujący}%
\index{węzeł!pierwszy}%
\index{skrzyżowanie}%
Książka ma nieliniową budowę i należy przeczytać ją co najmniej dwa razy.

Pewne kryteria rozszczepialności konkretnych splotów znaleźć można u Kawauchiego \cite[s. 36-38]{kawauchi1996}.
% TODO: przepisać, a jeśli za trudne, to może chociaż szkic?


\subsubsection{Sploty Brunna}
\index{splot!Brunna|(}%
Hermann Brunn \cite{brunn1892} rozpatrywał w 1892 roku (a więc zanim jeszcze teoria węzłów przyszła na świat!) nierozszczepialne sploty, które po usunięciu dowolnego ogniwa stają się niesplotami.
\index[persons]{Brunn, Hermann}%
W~czasopiśmie Delta, nr 01/2011, przeczytaliśmy, że Rolfsen zaproponował nazywać je splotami Brunna i~tak też będziemy robić.
Najprostszym splotem Brunna są posiadające trzy ogniwa pierścienie Boromeuszy ($6_2^3$ w notacji Alexandera-Briggsa, \texttt{L6a4} w notacji Thistlethwaite'a).
\index{pierścienie Boromeuszy}%
Pokażemy na stronie \pageref{boromean_not_splittable}, że pierścienie Boromeuszy nie mają nietrywialnych trójkolorowań, więc nie mogą być niesplotem.

Pierścienie Boromeuszy są alternujące, hiperboliczne i drzewiaste.
\index{węzeł!alternujący}%
\index{węzeł!hiperboliczny}%
\index{węzeł!drzewiasty}%
\index{węzeł!algebraiczny|see {drzewiasty}}%
% DICTIONARY;arborescent;drzewiasty;węzeł
% TODO: jak jest arborescent po francusku?
Ich nazwa pochodzi od lombardzko-piemonckiego rodu kupieckiego, bankierskiego i arystokratycznego, z którego wywodziło się wielu kardynałów.
Herb tego rodu zawierał splecione ze sobą trzy okręgi.
Jest niemożliwe, by wykonać model przestrzenny tego splotu przy użyciu okrągłych pierścieni.
Zamiast tego można użyć na przykład elips.

Z dokładnością do homotopii sploty Brunna zostały sklasyfikowane przez Milnora \cite{milnor1954}, ale ponieważ ta książka nie tłumaczy, czym są $\mu$-niezmienniki Milnora, nie możemy dzisiaj wytłumaczyć, jak tego dokonał.
\index[persons]{Milnor, John}%
\index{splot!Brunna|)}%




\subsubsection{Sploty alternujące}

Zazwyczaj do zdefiniowania splotów alternujących potrzebne są najpierw diagramy.
% Zanim opowiemy, jak dotąd przebiegała klasyfikacja węzłów o małej liczbie skrzyżowań, zdefiniujemy klasę splotów ze specjalnymi diagramami.

% DICTIONARY;alternating;alternujący;węzeł
\begin{definition}[splot alternujący]
\label{def:alternating_link}%
\index{węzeł!alternujący}%
    Niech $D$ będzie diagramem splotu $L$.
    Jeżeli podczas poruszania się wzdłuż każdego ogniwa naprzemiennie mijamy podskrzyżowania oraz nadskrzyżowania, to diagram nazywamy alternujący.
    
    Splot $L$ jest alternujący, jeśli posiada alternujący diagram $D$s.
\end{definition}

Około 1961 roku Ralph Fox zapytał \emph{,,What is an alternating knot?''}.
\index[persons]{Fox, Ralph}%
Szukano takiej definicji węzła alternującego, która nie odnosi się bezpośrednio do diagramów, aż w~2015 roku Greene \cite{greene2017} podał geometryczną charakteryzację: nierozszczepialny splot w $S^3$ jest alternujący wtedy i~tylko wtedy, gdy ogranicza dodatnią oraz ujemną określoną powierzchnię rozpinającą.
\index[persons]{Greene, Joshua}%

\begin{remark}[Ralph Hartzler Fox]
    Matematyk amerykański urodzon w Morrisville, Pensylwanii w~1913 roku; zmarł w Filadelfii, tamże w 1973 roku.
    Był promotorem Johna Milnora, Lee Neuwirtha (o~których jeszcze wspomnimy!) i 23 innych osób, o których nie wspomnimy.
    Oprócz tego nadzorował pracę licencjacką Kennetha Perko.
    Zawdzięczamy mu spopularyzowanie $n$-kolorowania na koledżu Haverford w 1956 roku, podanie nowego sposobu na znalezienie wielomianu Alexandera przy użyciu rachunku różniczkowego Foxa oraz niektóre terminy teorii węzłów uzywane po dziś dzień: węzeł plastrowy, węzeł taśmowy, okrąg i powierzchnia Seiferta.
\end{remark}

Nie ma zwartego wzoru na liczbę splotów alternujących, ale wiemy, że rośnie co najmniej wykładniczo względem liczby skrzyżowań:

\begin{proposition}
\index{supeł}%
    Niech $a_n$ oznacza liczbę supłów o~$n$ skrzyżowaniach, które są alternujące oraz pierwsze.
    Wtedy
    \begin{equation}
        a_n \sim \frac{3c_1 \lambda^{n-3/2}}{4\sqrt{\pi n^{5}}},
    \end{equation}
    gdzie zarówno $c_1$, pierwszy współczynnik rozwinięcia Taylora funkcji $\Phi(\eta)$ zdefiniowanej w \cite{sundberg1998}, jak i $\lambda$ są jawnie znanymi stałymi:
    \begin{align}
        c_1 & = \sqrt{\frac{5^7 \cdot (21001 + 371 \sqrt{21001})^3}{2 \cdot 3^{10} \cdot (17 + 3\sqrt{21001})^5}} \\
        \lambda & = \frac {1}{40} (101 + \sqrt{21001})
    \end{align}
    Niech $A_n$ oznacza liczbę pierwszych, alternujących splotów o $n$ skrzyżowaniach.
    Wtedy $A_n \approx \lambda^n$, dokładniej: jeśli $n \ge 3$, to
    \begin{equation}
        \frac{a_{n-1}}{16n - 24} \le A \le \frac{a_n - 1}{2}.
    \end{equation}
\end{proposition}

Węzły pierwsze i~supły pojawiają się odpowiednio w definicjach \ref{def:prime_knot}, \ref{def:tangle}.

\begin{proof}[Niedowód]
\index[persons]{Sundberg, Carl}%
\index[persons]{Thistlethwaite, Morwen}%
    Zamiast przedstawić dowód albo chociaż jego szkic, wymienimy trzy narzędzia użyte przez Sundberga, Thistlethwaite'a \cite{sundberg1998}:
    algebraiczną metodę Conwaya znajdowania splotów,
    wynik Tuttego dotyczącego liczby ukorzenionych $c$-sieci
    oraz (wtedy już udowodnioną) hipotezę Taita.
\index[persons]{Conway, John}%
\index[persons]{Tutte, William}%
\index{hipoteza!Taita}%
\end{proof}

\begin{proposition}
    Niech $a_n$ oznacza liczbę supłów o~$n$ skrzyżowaniach, które są alternujące oraz pierwsze.
    Wtedy funkcja tworząca $f(z) = \sum_n a_n z^n$ spełnia równanie
    \begin{equation}
    f(1+z) - f(z)^2 - (1+f(z))q(f(z)) -z - \frac{2z^2}{1-z} = 0,
    \end{equation}
    gdzie $q(z)$ jest pomocniczą funkcją
    \begin{equation}
        q(z) = \frac{2z^2 - 10z - 1 + \sqrt{(1-4z)^3}} {2(z+2)^3} - \frac{2}{1+z} -z + 2.
    \end{equation}
\end{proposition}

Powyższa ciekawostka także pochodzi z cytowanej wcześniej pracy \cite{sundberg1998}.





\subsection{Dopełnienia węzłów i splotów}
Jeśli dwa węzły są równoważne, to ich dopełnienia są oczywiście homeomorficzne.
Pytanie o~prawdziwość implikacji odwrotnej jako pierwszy zadał najprawdopodobniej Tietze \cite{tietze1908} w~1908 roku.
\index[persons]{Tietze, Heinrich}%
O~jego trudności niech świadczy fakt, że dopiero w~roku 1987 pokazano, że istnieją co najwyżej dwa węzły o~zadanym dopełnieniu (Culler, Gordon, Luecke, Shalen \cite{culler1987}).
\index[persons]{Culler, Marc}%
\index[persons]{Shalen, Peter}%
\index[persons]{Gordon, Cameron}%
\index[persons]{Luecke, John}%
Dwa lata później poznaliśmy pozytywną odpowiedź na pytanie Tietzego: każdy węzeł jest wyznaczony jednoznacznie przez swoje dopełnienie.
Natomiast analogiczne stwierdzenie o~splotach jest fałszywe i wiedziano o tym od bardzo dawna: w~1937 roku Whitehead \cite{whitehead1937} podał nieskończenie wiele splotów, których dopełnienia wyglądają jak dopełnienia splotu Whiteheada.

\begin{theorem}[Gordon, Luecke, 1989]
\index[persons]{Gordon, Cameron}%
\index[persons]{Luecke, John}%
\index{twierdzenie!Gordona-Lueckego}%
    Niech $f \colon (\mathbb R^3 \setminus K_1) \to (\mathbb R^3 \setminus K_2)$ będzie zachowującym orientację homeomorfizmem dopełnień poskromionych węzłów $K_1, K_2$.
    Wtedy węzły $K_1 \cong K_2$ są izotopijne.
\end{theorem}

\begin{proof}[Niedowód]
    Wynika to z teorii Cerfa, kombinatorycznych technik w stylu Litherlanda, cienkich pozycji, cykli Scharlemanna i~ogólniejszego stwierdzenia: nietrywialna chirurgia Dehna na węźle w~3-sferze nigdy nie daje 3-sfery.
\index{chirurgia Dehna}%
\index{cykle Scharlemanna}%
\index{teoria Cerfa}%
    Pełny dowód zawiera praca \cite{gordon1989}.
\end{proof}

Twierdzenie to zamienia problem lokalny (czy dwa węzły w kuli $S^3$ są równoważne?) na problem globalny (czy dwie przestrzenie topologiczne są homeomorficzne?).
\index[persons]{Whitehead, John}%

% koniec sekcji Węzły i sploty
