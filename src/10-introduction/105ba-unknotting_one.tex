
\subsubsection{Sploty 1-gordyjskie}
Sploty o liczbie gordyjskiej 1 zasługują na szczególną uwagę.

\begin{proposition}
    Niech $L$ będzie wymiernym splotem 1-gordyjskim.
\index{węzeł!wymierny}%
    Wtedy na minimalnym diagramie $L$ jedno ze skrzyżowań jest rozwiązujące.
\end{proposition}

\begin{proof}
    Kanenobu, Murakami dla węzłów \cite{kanenobumurakami86}, wkrótce po tym Kohn dla splotów \cite{kohn91}.
\end{proof}

Coward, Lackenby dowiedli w~\cite{coward11}, że jeśli $K$ jest 1-gordyjski i~o genusie 1, to z~dokładnością do pewnej relacji równoważności, tylko jedna zmiana skrzyżowania rozwiązuje go; chyba że $K$ jest ósemką -- wtedy takie zmiany są dwie.

\begin{tobedone}
    Kawauchi, s. 151: $u = 1, g = 1$ to duble.
\end{tobedone}

\begin{proposition}
    \label{prp:unknotting_one_prime}
    Węzły $1$-gordyjskie są pierwsze.
\end{proposition}

Podejrzewał to Hilmar Wendt w~1937 roku, kiedy policzył liczbę gordyjską węzła babskiego używając homologii rozgałęzionego nakrycia cyklicznego \cite{wendt37}.

\begin{proof}[Niedowód]
    W pracy \cite{scharlemann85} z~1985 roku Scharlemann podał dość zawiłe uzasadnienie, w~które zamieszane były grafy planarne.
    Obecnie znamy prostsze dowody, patrz \cite{lackenby97} albo \cite{zhang91}.
\end{proof}

Scharlemann pokazał w \cite[wniosek 1.6]{scharlemann98}, że liczba gordyjska jest podaddytywna, to znaczy zachodzi $u(K_1 \shrap K_2) \le u(K_1) + u(K_2)$.
Stąd oraz z faktu \ref{prp:unknotting_one_prime} wynika, że suma dwóch $1$-gordyjskich węzłów jest $2$-gordyjska, ale od początku teorii węzłów podejrzewano dużo więcej:

\begin{conjecture}
    \index{hipoteza!o liczbie gordyjskiej}
    Niech $K_1, K_2$ będą węzłami.
    Wtedy $u(K_1 \shrap K_2) = u(K_1) + u(K_2)$, czyli liczba gordyjska jest addytywna.
\end{conjecture}

