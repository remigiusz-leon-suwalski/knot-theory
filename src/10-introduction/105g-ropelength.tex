
\subsection{Długość sznurowa}
\index{długość sznurowa|(}%
% DICTIONARY;ropelength;długość sznurowa;-
Matematyczne węzły nie mają grubości i można je dowolnie rozciągać.
Długość sznurowa, najsłabiej poznany niezmiennik numeryczny, pochodzi z~fizycznej teorii węzłów, która bierze pod uwagę obiekty wykonane z~nieelastycznych materiałów.

\begin{definition}
    Niech $L$ będzie splotem o długości $l$ oraz grubości $\tau$: posiada rurowe otoczenie bez samoprzecięć z~przekrojem poprzecznym o~promieniu $\tau$.
    Iloraz
    \begin{equation}
        \ropelength L = \frac l \tau
    \end{equation}
    nazywamy długością sznurową splotu.
\end{definition}

Przez wiele lat zastanawiano się: czy można zawiązać węzeł ze sznura o~długości jednej stopy i~promieniu jednego cala?
Lub równoważnie, czy $\ropelength K \le 12$ dla pewnego węzła $K$?
Na początku XXI wieku wiedzieliśmy z \cite{cantarella02}, że najkrótszy węzeł ma długość co najmniej $(2 + \sqrt 2)\pi \approx 10.726$, potem Diao udzielił negatywnej odpowiedzi na to pytanie w~\cite[s. 14]{diao03}.

Rozumowanie \cite{denne06} oparte o~czterosieczne pokazuje, że długość sznurowa nietrywialnego węzła wynosi co najmniej $15.66$.
Ponieważ eksperymenty komputerowe pokazują, że długość trójlistnika nie przekracza $16.372$, oszacowanie to jest dość ostre.

Prowadzono obszerne poszukiwania na temat zależności między długością sznurową i~innymi niezmiennikami.
Mamy na przykład:

\begin{proposition}
    Istnieją stałe $c_1, c_2$ takie, że $c_1 \crossing^{3/4} K \le \ropelength K \le c_2 \crossing^{3/2} K$.
\end{proposition}

Udowodniono, że dolnym ograniczeniem na czynnik $c_1$ jest $(4\pi/11)^{3/4} \approx 1.105$ (gdyż tak pisze Cantarella i~inni w~\cite[tw. 23]{cantarella02}).
Wiemy też, że stała $c_1$ nie może przekraczać $12.64$ ze względu na węzeł torusowy $T(3, 5)$, przeczytaliśmy o~tym w~\cite{klotz21}.

Klotz, Maldonado \cite{klotz21} piszą, że Diao dostał lepsze dolne ograniczenie (lepsze niż ,,3/4'') dla węzłów do 1850 skrzyżowań (!!!):
\begin{equation}
    \frac 12 \left(17.334 + \sqrt{17.334^2 + 64 \pi \crossing K}\right) \le \ropelength K.
\end{equation}

Dowód górnego ograniczenia opiera się na cyklach Hamiltona w~grafach zanurzonych w~kratach liczbowych \cite{yu04} i zostało później poprawione:

% Ograniczenie to realizowane jest przez pewne węzły torusowe oraz sploty Hopfa. - wikipedia o c_1

\begin{proposition}
    $\ropelength K = O(\crossing K \cdot \log^5(\crossing K)).$
\end{proposition}

\begin{proof}
    Świeży wynik z \cite{diao19}, którego dowód wykorzystuje kraty liczbowe.
\end{proof}

Jakościowy wynik znaleźliśmy znowu w~\cite{klotz21}: zachodzi $\ropelength L \le a_u \crossing K \log^5 \crossing K$ (nie tylko dla węzłów, ale też splotów), przy czym stała $a_u$ musi być większa od $8\pi/\log^5 2 \approx 78.5$, by dobrze ograniczała splot Hopfa.

Długość sznurowa nie pojawia się w~dalszych rozdziałach.

\index{długość sznurowa|)}%

% Koniec podsekcji Długość sznurowa

