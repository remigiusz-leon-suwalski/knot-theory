
% DICTIONARY;linking number;indeks zaczepienia;-
\subsection{Indeks zaczepienia}
\index{indeks zaczepienia|(}%
Gauß wprowadził indeks zaczepienia dwóch węzłów jako pewna całka, ale żyjemy w~XXI wieku i wystarczy nam definicja odwołująca się do diagramów.
\index[persons]{Gauß, Carl}%

Patrz też: \cite[s. 11]{kawauchi96}.

% DICTIONARY;sign;znak;skrzyżowanie
\begin{definition}[znak]
\index{znak skrzyżowania}%
    Liczbę $\pm 1$ przypisaną do skrzyżowania zgodnie z diagramem:
\begin{comment}
    \[
        \sign \left( \MediumPlusCrossingArrows \right) = +1 \quad
        \sign \left( \MediumMinusCrossingArrows \right) = -1
    \]
\end{comment}
    nazywamy znakiem skrzyżowania.
\end{definition}

Skrzyżowania dodatnie to takie, w których obrócenie dolnego łuku w prawo daje górny łuk, dlatego czasem nazywa się je także praworęcznymi.
Oczywiście skrzyżowania ujemne nazywamy wtedy leworęcznymi.
\index{skrzyżowanie!dodatnie i ujemne}%
\index{skrzyżowanie!lewo- i prawoskrętne}%

% DICTIONARY;smoothing;wygładzenie;-
\begin{definition}[wygładzenie]
\index{skrzyżowanie!... wygładzenie}%
    Niech dany będzie diagram z wyróżnionym skrzyżowaniem.
    Wtedy diagramy
\begin{comment}
    \begin{figure}[H]
        \begin{minipage}[b]{.48\linewidth}
            \[
                \LargeAlphaSmoothing
            \]
            \subcaption{wygładzenie dodatnie}
        \end{minipage}
        \begin{minipage}[b]{.48\linewidth}
            \[
                \LargeBetaSmoothing
            \]
            \subcaption{wygładzenie ujemne}
        \end{minipage}
    \end{figure}
\end{comment}
    powstałe przez zmianę małego otoczenia tego skrzyżowania nazywamy wygładzeniami.
    Jeżeli nie zaznaczono inaczej, wygładzamy zgodnie ze znakiem skrzyżowania.
\end{definition}

\begin{definition}[indeks zaczepienia]
    Niech $L = K_1 \sqcup K_2$ będzie splotem o dwóch ogniwach.
    Wielkość
    \begin{equation}
        \linking(K_1, K_2) = \frac 12 \sum_i \sign c_i,
    \end{equation}
    gdzie sumowanie rozciąga się na wszystkie skrzyżowania, na których spotykają się łuki z różnych ogniw, nazywamy indeksem zaczepienia węzłów $K_1, K_2$.
    Ogólniej, jeśli dany jest splot $L = K_1 \sqcup \ldots \sqcup K_n$ posiadający $n$ ogniw, to jego indeks zaczepienia wyznacza wzór
    \begin{equation}
        \linking(L) = \sum_{i < j} \linking(K_i, K_j).
    \end{equation}
\end{definition}

Zauważmy, że indeks zaczepienia splotu Hopfa wynosi $1$, natomiast splotu Whiteheada $0$.
\index{splot!Hopfa}%
\index{splot!Whiteheada}%
Są zatem istotnie różne.
W obydwu przypadkach indeks zaczepienia jest liczbą całkowitą.
Istotnie, na mocy twierdzenia Jordana $\linking$ jest funkcją o całkowitych wartościach.

\begin{proposition}
    Indeks zaczepienia jest dobrze określonym niezmiennikiem zorientowanych splotów.
\end{proposition}

\begin{proof}
    Sprawdźmy wpływ ruchów Reidemeistera na wartość $\linking L$:
\begin{comment}
    \begin{figure}[H]
    \centering
    %
    \begin{minipage}[b]{.3\linewidth}
        \[
            \MedLarReidemeisterOneLeft \cong \MedLarReidemeisterOneStraight
        \]
        \subcaption{ruch $R_1$}
    \end{minipage}
    %
    \begin{minipage}[b]{.3\linewidth}
        \[
            \MedLarReidemeisterTwoLinkingA \cong \MedLarReidemeisterTwoB
        \]
        \subcaption{ruch $R_2$}
    \end{minipage}
    %
    \begin{minipage}[b]{.35\linewidth}
        \[
            \MedLarReidemeisterThreeLinkingA \cong \MedLarReidemeisterThreeLinkingB
        \]
        \subcaption{ruch $R_3$}
    \end{minipage}
\end{figure}
\end{comment}
    Na mocy twierdzenia Reidemeistera dowód został zakończony.
\end{proof}

\index{indeks zaczepienia|)}%

% koniec podsekcji Indeks zaczepienia

