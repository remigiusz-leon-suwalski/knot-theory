% DICTIONARY;linking number;indeks zaczepienia;-
\subsection{Indeks zaczepienia}
\index{indeks zaczepienia|(}
Gauß wprowadził indeks zaczepienia dwóch węzłów jako pewna całka, ale żyjemy w~XXI wieku i wystarczy nam definicja odwołująca się do diagramów.

% DICTIONARY;sign;znak;skrzyżowanie
\begin{definition}[znak]
\index{znak skrzyżowania}%
    Liczbę $\pm 1$ przypisaną do skrzyżowania zgodnie z diagramem:
\begin{comment}
    \[
        \sign \left( \MediumSkeinPlus  \right) = +1 \quad
        \sign \left( \MediumSkeinMinus \right) = -1
    \]
\end{comment}
    nazywamy znakiem skrzyżowania.
\end{definition}

Skrzyżowania dodatnie to takie, w których obrócenie dolnego łuku w prawo daje górny łuk, dlatego czasem nazywa się je także praworęcznymi.
Oczywiście skrzyżowania ujemne nazywamy wtedy leworęcznymi.
\index{skrzyżowanie!dodatnie i ujemne}%
\index{skrzyżowanie!lewo- i prawoskrętne}%

% DICTIONARY;smoothing;wygładzenie;-
\begin{definition}[wygładzenie]
\index{wygładzenie (skrzyżowania)}%
    Niech dany będzie diagram z wyróżnionym skrzyżowaniem.
    Wtedy diagramy
\begin{comment}
    \begin{figure}[H]
        \begin{minipage}[b]{.48\linewidth}
            \[
                \RightCrossSmoothing
            \]
            \subcaption{wygładzenie dodatnie}
        \end{minipage}
        \begin{minipage}[b]{.48\linewidth}
            \[
                \LeftCrossSmoothing
            \]
            \subcaption{wygładzenie ujemne}
        \end{minipage}
    \end{figure}
\end{comment}
    powstałe przez zmianę małego otoczenia tego skrzyżowania nazywamy wygładzeniami.
    Jeżeli nie zaznaczono inaczej, wygładzamy zgodnie ze znakiem skrzyżowania.
\end{definition}

\begin{definition}[indeks zaczepienia]
\index{indeks zaczepienia}%
    Niech $L = K_1 \sqcup K_2$ będzie splotem o dwóch ogniwach.
    Wielkość
    \begin{equation}
        \linking(K_1, K_2) = \frac 12 \sum_i \sign c_i,
    \end{equation}
    gdzie sumowanie rozciąga się na wszystkie skrzyżowania, na których spotykają się łuki z różnych ogniw, nazywamy indeksem zaczepienia węzłów $K_1, K_2$.
    Ogólniej, jeśli dany jest splot $L = K_1 \sqcup \ldots \sqcup K_n$ posiadający $n$ ogniw, to jego indeks zaczepienia wyznacza wzór
    \begin{equation}
        \linking(L) = \sum_{i < j} \linking(K_i, K_j).
    \end{equation}
\end{definition}

Zauważmy, że indeks zaczepienia splotu Hopfa wynosi $1$, natomiast splotu Whiteheada $0$.
Są zatem istotnie różne.
W obydwu przypadkach indeks zaczepienia jest liczbą całkowitą.
Istotnie, na mocy twierdzenia Jordana $\linking$ jest funkcją o całkowitych wartościach.

\begin{proposition}
    Indeks zaczepienia jest dobrze określonym niezmiennikiem zorientowanych splotów.
\end{proposition}

\begin{proof}
    Sprawdźmy wpływ ruchów Reidemeistera na wartość $\linking L$:
\begin{comment}
    \[
        \fbox{
        \begin{tikzpicture}[baseline=-0.65ex,scale=0.07]
        \begin{knot}[clip width=5]
        \strand[semithick] (-10,10) .. controls (-10,2) and (-10,2) .. (-6,-2);
        \strand[semithick] (-10,-10) .. controls (-10,-2) and (-10,-1) .. (-9,0);
        \strand[semithick] (-7,1) -- (-6,2);
        \strand[semithick] (-6,2) .. controls (2,9) and (2,-9) .. (-6,-2);
        \end{knot}
        \end{tikzpicture}
        $\stackrel{R_1}{\cong} \,\,$
        \begin{tikzpicture}[baseline=-0.65ex,scale=0.07]
        \begin{knot}[clip width=5]
        \strand[semithick] (0,10) -- (0,-10);
        \end{knot}
        \end{tikzpicture}}
        %%%
        \quad \fbox{
        \begin{tikzpicture}[baseline=-0.65ex,scale=0.07]
        \begin{knot}[clip width=5]
        \strand[semithick] (4,-10) .. controls (4,-4) and (-4,-4) .. (-4,0);
        \strand[semithick] (4,10) .. controls (4, 4) and (-4, 4) .. (-4,0);
        \strand[semithick] (-4,-10) .. controls (-4,-4) and (4,-4) .. (4,0);
        \strand[semithick] (-4,10) .. controls (-4, 4) and (4,4) .. (4,0);
        \node[blue] at (-4,4)[left] {$a$};
        \node[blue] at (-4,-4)[left] {$-a$};
        \end{knot}
        \end{tikzpicture}
        $\stackrel{R_2}{\cong} \,\,$
        \begin{tikzpicture}[baseline=-0.65ex,scale=0.07]
        \begin{knot}[clip width=5]
        \strand[semithick] (4,-10) .. controls (4,-4) and (1,-4) .. (1,0);
        \strand[semithick] (4,10) .. controls (4, 4) and (1, 4) .. (1,0);
        \strand[semithick] (-4,-10) .. controls (-4,-4) and (-1,-4) .. (-1,0);
        \strand[semithick] (-4,10) .. controls (-4, 4) and (-1,4) .. (-1,0);
        \end{knot}
        \end{tikzpicture}}
        %%%
        \quad \fbox{
        \begin{tikzpicture}[baseline=-0.65ex,scale=0.07]
        \begin{knot}[clip width=5, flip crossing/.list={1,2,3}]
        \strand[semithick] (-10,-10) -- (10,10);
        \strand[semithick] (-10,10) -- (10,-10);
        \strand[semithick] (-10,-2) .. controls (-4, -2) and (-4,8) .. (0,8);
        \strand[semithick] (10,-2) .. controls (4, -2) and (4,8) .. (0,8);
        \node[blue] at (-6,4)[left] {$a$};
        \node[blue] at (6,4)[right] {$b$};
        \node[blue] at (0,-2)[below] {$c$};
        \end{knot}
        \end{tikzpicture}
        $\stackrel{R_3}{\cong} \,\,$
        \begin{tikzpicture}[baseline=-0.65ex,scale=0.07]
        \begin{knot}[clip width=5, flip crossing/.list={1,2,3}]
        \strand[semithick] (-10,-10) -- (10,10);
        \strand[semithick] (-10,10) -- (10,-10);
        \strand[semithick] (-10,2) .. controls (-4, 2) and (-4,-8) .. (0,-8);
        \strand[semithick] (10,2) .. controls (4, 2) and (4,-8) .. (0,-8);
        \node[blue] at (-6,-4)[left] {$a$};
        \node[blue] at (6,-4)[right] {$b$};
        \node[blue] at (0,2)[above] {$c$};
        \end{knot}
        \end{tikzpicture}}
    \]
\end{comment}
    Na mocy twierdzenia Reidemeistera dowód został zakończony.
\end{proof}

\index{indeks zaczepienia|)}

% koniec podsekcji Indeks zaczepienia
