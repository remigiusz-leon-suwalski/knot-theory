
\section{Diagramy. Ruchy Reidemeistera}

Chociaż w~świetle definicji \ref{def:knot} węzły są pewnymi regularnymi podzbiorami przestrzeni $\R^3$, z~kombinatorycznego punktu widzenia wygodniej jest rysować je na płaszczyźnie.

% DICTIONARY;oriented;zorientowany;węzeł
\begin{definition}[orientacja]
\index{węzeł!zorientowany}%
\index{orientacja|see {węzeł zorientowany}}%
    Węzeł, w~którym wybrano kierunek, w~którym należy się po nim poruszać, nazywamy zorientowanym.
    Splot nazywamy zorientowanym, jeśli wszystkie jego ogniwa traktowane jako węzły są zorientowane.
\end{definition}

Orientację na diagramie zaznaczamy małą strzałką wskazującą kierunek poruszania się.

% DICTIONARY;shadow;cień;-
\begin{definition}
\index{cień}%
    Rzut węzła $K \subseteq \R^3$ na płaszczyznę nazywamy cieniem.
\end{definition}

% DICTIONARY;crossing;skrzyżowanie;-
\begin{definition}[skrzyżowanie]
\label{def:crossing}%
\index{skrzyżowanie}%
    Podwójny punkt w cieniu nazywamy skrzyżowaniem.
\end{definition}

% DICTIONARY;diagram;diagram;-
\begin{definition}[diagram]
\index{diagram}%
    Cień razem z~informacją o~tym, jak przebiegają skrzyżowania i pozbawiony katastrof: potrójnych przecięć, stycznych czy dziobów nazywamy diagramem.
    % TODO: Narysować katastrofy
\end{definition}

Dla każdego ustalonego $n \ge 2$ i każdego węzła $K$ istnieje cień $D$, na którym wszystkie wielokrotne punkty są $n$-krotne (wiemy to od Hostego, College'a \cite[s. 11]{adams2021}, którzy nie napisali, skąd to wiedzą);
\index[persons]{Hoste, Jim}%
\index[persons]{College, Pitzer}%
dla co najmniej jednej wartości $n$ można dodatkowo wymagać, by diagram zawierał dokładnie jedno skrzyżowanie (Adams, Crawford, DeMeo, Landry, Lin, Montee, Park, Venkatesh, Yhee \cite{venkatesh2015} albo Brunn\footnote{Karl Hermann Brunn opisał w 1892 roku sploty Brunna, a więc zanim jeszcze teoria węzłów przyszła na świat.} ponad sto lat temu \cite[s. 28]{adams2021}!).
\index[persons]{Adams, Colin}%
\index[persons]{Crawford, Thomas}%
\index[persons]{DeMeo, Benjamin}%
\index[persons]{Landry, Michael}%
\index[persons]{Lin, Alex}%
\index[persons]{Montee, Murphy}%
\index[persons]{Park, Seojung}
\index[persons]{Venkatesh, Saraswathi}%
\index[persons]{Yhee, Farrah}%

\begin{definition}[włókno]
\index{włókno}%
    Fragment diagramu, który biegnie między dwoma kolejnymi tunelami, czyli podskrzyżowaniami, nazywamy włóknem.
\end{definition}

\begin{definition}[nić]
\index{nić}%
    Fragment diagramu, który biegnie między dwoma kolejnymi skrzyżowaniami, nazywamy nicią.
\end{definition}

Nici powstają z włókien przez rozcięcie ich przy każdym nadskrzyżowaniu.

\begin{proposition}
\label{prp:links_have_diagrams}%
    Niech $L$ będzie splotem.
    Jego diagramy tworzą otwarty i~gęsty podzbiór wszystkich rzutów.
\end{proposition}

Kawauchi \cite[s. 7]{kawauchi1996} wspomina w tym miejscu podręcznik Crowella, Foxa \cite[s. 7]{crowell1963}.
To samo jest na przykład u Burdego, Zieschanga, Heusenera \cite[s. 10]{burde2014}, ale oni odsyłają jeszcze do Reidemeistera \cite{reidemeister1927} i samego Burdego \cite{burde1978}.

\begin{proof}[Niedowód]
    Rzut splotu na równoległe płaszczyzny jest taki sam, a te można sparametryzować prostymi przechodzącymi przez początek układu współrzędnych, które tworzą przestrzeń rzutową $\R \mathbb P^2$.
    Niech $S$ będzie zbiorem prostych, które dają złe rzuty.
    Wystarczy pokazać jego nigdziegęstość.
    Okazuje się, że $S$ jest też jednowymiarowy.
\end{proof}

\begin{corollary}
    Każdy splot posiada diagram.
\end{corollary}


\subsection{Ruchy Reidemeistera}

Wiemy już, że węzły mają wiele diagramów.
Mając dane dwa różne diagramy chcielibyśmy wiedzieć, czy przedstawiają ten sam węzeł.
Stosowne narzędzie dostarczył Kurt Reidemeister w~latach dwudziestych ubiegłego wieku.
\index[persons]{Reidemeister, Kurt}%
Zdefiniujmy trzy lokalne operacje na diagramach, a~potem wysłowimy kryterium  Reidemeistera rozstrzygające problem równości węzłów.

% DICTIONARY;Reidemeister/Turaev/... move;ruch Reidemeistera/Turajewa/...;-
\begin{definition}[ruchy Reidemeistera]
\index{ruch!Reidemeistera}%
    Trzy gatunki lokalnych deformacji diagramu splotu:
    \begin{figure}[H]
    \centering
    \begin{minipage}[b]{.22\linewidth}%
        \centering%
        \MedLarReidemeisterOneLeft $\stackrel{R_1}{\cong}$ \MedLarReidemeisterOneStraight%
        \subcaption{ruch $R_1$}%
    \end{minipage}
    \quad\quad\quad
    \begin{minipage}[b]{.2\linewidth}
        \centering
        \MedLarReidemeisterTwoA $\stackrel{R_2}{\cong}$ \MedLarReidemeisterTwoB
        \subcaption{ruch $R_2$}
    \end{minipage}
    \quad\quad\quad
    \begin{minipage}[b]{.32\linewidth}
        \centering
        \MedLarReidemeisterThreeA $\stackrel{R_3}{\cong}$ \MedLarReidemeisterThreeB
        \subcaption{ruch $R_3$}
    \end{minipage}
    \end{figure}
    nazywamy ruchami Reidemeistera.
    Czasami używa się konkretnych nazw:
    \begin{itemize}
        \item skręcenie/rozkręcenie (dla $R_1$),
        \item wsunięcie/rozsunięcie (dla $R_2$) oraz
        \item przesunięcie łuku przez skrzyżowanie (dla $R_3$).
    \end{itemize}
\end{definition}

Reidemeister w~swojej pierwszej pracy przyjął inną kolejność, jego drugi ruch jest naszym pierwszym.
Dzięki temu ruch $R_k$ operuje na $k$ łukach diagramu.
Colberg \cite[s. 6]{colberg2013} pisze, że Maxwell znał ruchy Reidemeistera przed Reidemeisterem, ale mimo próśb Taita nigdy nie zgłosił swojego odkrycia w Royal Society of Edinburgh.
\index[persons]{Tait, Peter}%
\index[persons]{Maxwell, James}%

\begin{theorem}[Reidemeister, 1927]
\label{thm:reidemeister}%
\index{twierdzenie!Reidemeistera}%
\index[persons]{Reidemeister, Kurt}%
    Niech $D_1, D_2$ będą diagramami dwóch splotów $L_1, L_2$.
    Sploty $L_1, L_2$ są takie same wtedy i tylko wtedy, gdy diagram $D_2$ można otrzymać z $D_1$ wykonując skończony ciąg ruchów Reidemeistera oraz gładkich deformacji łuków, bez zmiany biegu skrzyżowań.
\end{theorem}
% https://math.stackexchange.com/questions/4399634/two-knots-k-and-k-prime-are-equivalent-if-and-only-if-their-projections-p
% Reidemeister, and pretty much every other author, has worked with the piecewise-linear case. In a way it doesn't matter which you choose, since there's a theorem that the categories of smooth and PL manifolds are equivalent in some sense. However, it's not so clear how you pass from one setting to the other (or at least I've never gone through the details myself!)

Twierdzenie Reidemeistera jest prawdziwe także dla splotów zorientowanych, ale wtedy trzeba uwzględnić różne orientacje łuków i~nie jest oczywiste, ile spośród $2^1 + 2^2 + 2^3 = 14$ wersji jest potrzebne.
Polyak \cite{polyak2010} pokazał, że minimalny zbiór zorientowanych ruchów składa się na przykład z~dwóch wersji ruchu $R_1$, jednej wersji ruchu $R_2$ i~jednej wersji ruchu $R_3$.
\index[persons]{Polyak, Michael}%

\begin{proof}[Niedowód]
Dowód podali niezależnie Reidemeister \cite{reidemeister1927} oraz Alexander, Briggs \cite{alexander1927}.
\index[persons]{Reidemeister, Kurt}%
\index[persons]{Briggs, Garland}%
\index[persons]{Alexander, James}%
    Szkielet dowodu można znaleźć w~książce Burdego i~Zieschanga \cite[s. 9-11]{burde2014}, ale kluczowe pomysły podają też Prasołow z~Sosińskim \cite[s. 11-12]{prasolov1997}.
\index[persons]{Burde, Gerhard}%
\index[persons]{Zieschang, Heiner}%
\index[persons]{Prasołow, Wiktor (Прасолов, Виктор Васильевич)}%
\index[persons]{Sosiński, Aleksiej (Сосинский, Алексей Брониславович)}%
    Innym przystępnym źródłem jest podręcznik Murasugiego \cite[s. 50-56]{murasugi1996}.
\index[persons]{Murasugi, Kunio}%
\end{proof}

Trace \cite{trace1983} zauważył, że dwa diagramy jednego węzła są związane tylko II i III ruchem (ale nie I) wtedy i tylko wtedy, gdy mają ten sam spin oraz indeksy nawinięcia.
\index[persons]{Trace, Bruce}%
Z prac Östlunda \cite{ostlund2001}, Manturowa \cite[s. ???]{manturov2004} oraz Haggego \cite{hagge2006} wynika, że dla każdego węzła istnieje para diagramów, do przejścia między którymi trzeba wykorzystać wszystkie trzy ruchy.
% TODO: ustalić, które strony w Manturowie
\index[persons]{Östlund, Olof}%
\index[persons]{Manturow, Wasilij}%
\index[persons]{Hagge, Tobias}%
% praca Haggego nazywa się "Every Reidemeister move is needed for each knot type" ale nawet w MathSciNecie wspomnieni są Ostlund i Manturow, więc zostawiam. Tekst skopiowany z Wiki
Coward \cite{coward2006} zademonstrował, że nawet jeśli wszystkie trzy ruchy są potrzebne, można je wykonywać w specjalnej kolejności: najpierw tylko I ruchy, potem tylko II ruchy, następnie tylko III ruchy i~znowu II ruchy.
\index[persons]{Coward, Alexander}%

Do pokazania, że dwa węzły $K_1, K_2$ nie są równoważne, powinniśmy na mocy twierdzenia \ref{thm:reidemeister} udowodnić, że żaden ciąg ruchów Reidemeistera nie przekształca jednego w drugi.
Oczywiście nikt o zdrowych zmysłach tak nie postępuje.
Zamiast tego wprowadza się stosowny niezmiennik, czyli funkcję $f$ określoną na zbiorze wszystkich węzłów (albo splotów, supłów, warkoczy itd.) tak, że jeśli węzły $K_1 \cong K_2$ są równoważne, to $f(K_1) = f(K_2)$.
% DICTIONARY;invariant;niezmiennik;-
Łatwo widać, że jeśli $f(K_1) \neq f(K_2)$, to węzły $K_1, K_2$ nie mogą być równoważne.
Natomiast gdy wartości są te same, nie dostajemy żadnej informacji.

Poznaliśmy jak na razie dwa niezmienniki: liczbę ogniw splotu oraz dopełnienie splotu do przestrzeni, w której jest zanurzony ($\mathbb R^3$ lub $S^3$).
Wiele, chociaż nie wszystkich, innych niezmienników definiuje się nie bezpośrednio na zbiorze węzłów, ale na zbiorze diagramów.
Należy wtedy sprawdzić, że każdy z trzech ruchów Reidemeistera nie ma wpływu na wartość niezmiennika.

Niezmienniki będą nam stale towarzyszyć w~wędrówce po krainie węzłów.

\begin{remark}[Kurt Werner Friedrich Reidemeister]
    ?
\end{remark}

\begin{remark}[James Waddell Alexander]
    ?
\end{remark}

\begin{remark}[Garland Baird Briggs]
    Matematyk amerykański, urodzon w Sebrell, Wirginii w 1894 roku; zmarł w Kolumbii w 1959 roku.
    Niestety nie wiemy za dużo o tym człowieku.
\end{remark}

\color{white}

\subsubsection{Dygresja -- wyniki ilościowe wokół twierdzenia Reidemeistera}
Załóżmy, że na dwóch diagramach tego samego węzła widać odpowiednio $n_1, n_2$ skrzyżowań.
Jak piszą Coward, Lackenby \cite{coward2011}, istnieje funkcja $f \colon \N \times \N \to \N$ taka, że między dwoma diagramami można przejść wykonując co najwyżej $f(n_1, n_2)$ ruchów.
\index[persons]{Coward, Alexander}%
\index[persons]{Lackenby, Marc}%
Wynika to z faktu, że istnieje skończenie wiele spójnych diagramów o danej liczbie skrzyżowań oraz twierdzenia Reidemeistera.
Okazuje się jednak, że od funkcji $f$ można żądać, by była obliczalna
(a to jest chyba równoważne istnienia algorytmu rozpoznającego, czy dwa diagramy przedstawiają jeden węzeł)
% http://people.dm.unipi.it/martelli/Cortona/Lackenby.pdf 7 of 90
i faktycznie, główny wynik \cite{coward2011} orzeka, że
\begin{equation}
    f(n_1, n_2) = 2^{2^{\ldots^{2^{n_1 + n_2}}}}
\end{equation}
jest taką funkcją.
Piętrowa potęga liczy sobie aż $10^{1000000 (n_1 + n_2)}$ warstw.

Natomiast jeżeli $n_2 = 0$, czyli drugi diagram przedstawia niewęzeł, ,,wystarcza'' $(236n_1)^{11}$ ruchów, przy czym liczba skrzyżowań podczas transformacji nigdy nie przekracza $49c^2$: to świeższy wynik samego Lackenby'ego \cite{lackenby2015}, gdzie poprawił wcześniejsze oszacowania Hassa, Lagariasa.
Przykład diagramu niewęzła, do rozwiązania którego nie można tylko usuwać istniejących skrzyżowań, przedstawiają Burde, Zieschang, Heusener \cite[s. 12]{burde2014}.

Hayashi \cite{hayashi2005} dowiódł, że liczbę ruchów Reidemeistera potrzebnych, by rozszczepić splot można ograniczyć z góry na podstawie indeksu skrzyżowaniowego.
\index[persons]{Hayashi, Chuichiro}%

% koniec sekcji Ruchy Reidemeistera



\input{10-introduction/102b-hardunknots}

\input{10-introduction/102c-alternating}


\subsection{Historia tablic węzłów}
% DICTIONARY;knot table;tablica węzłów;-
Pierwszą osobą, która podjęła się szukania węzłów, był Peter Guthrie Tait, szkocki fizyk.
\index[persons]{Tait, Peter}%
Razem z~Thomsonem (lordem Kelvinem) wierzyli, że węzły są kluczem do zrozumienia widma spektroskopowego różnych pierwiastków: na przykład atom sodu mógł być splotem Hopfa ze względu na dwie linie emisyjne.
\index[persons]{Thomson, William (lord Kelvin)}%
Eksperyment Michelsona-Morleya z~1887 roku zabił ich ,,wirową teorię atomu'', ale nie miało to znaczenia dla teorii węzłów jako działu matematyki.

Używana po dziś dzień strategia, którą przyjął Tait, jest stosunkowa prosta: narysować wszystkie możliwe diagramy o~zadanym indeksie skrzyżowaniowym, po czym połączyć ze sobą te, które przedstawiają jeden węzeł.
Na potrzeby pierwszego etapu Tait wymyślił schemat kodowania diagramów.
Opiszemy później jego ulepszenie, kod Dowkera-Thistlethwaite'a.


\subsubsection{Siedem i mniej skrzyżowań}
Tait wykorzystując swoją notację podał w~1876 pierwszą tablicę piętnastu węzłów o~mniej niż ośmiu skrzyżowaniach.
\index[persons]{Tait, Peter}%
Nie należy traktować tego jako skromny wynik: nie miał on do dyspozycji żadnych twierdzeń topologicznych do odróżniania węzłów.
Onieśmielony przez liczbę możliwych kodów dla kolejnych indeksów skrzyżowaniowych, powstrzymał się przed rozszerzaniem swojej tablicy.
To właśnie grupowanie diagramów przedstawiających ten sam węzeł, a~nie samo szukanie wszystkich możliwych diagramów, sprawia trudność.

Aby sobie pomóc, Tait znalazł lokalną modyfikację diagramu, która nie zmienia indeksu skrzyżowaniowego, znaną obecnie jako \textsc{flype}.
\index{flype}%
\index[persons]{Tait, Peter}%
Dla Taita ,,flype'' było innym ruchem, prostą transformacją związaną ze zmianą wyboru nieskończonego obszaru, ale pamięta o~tym teraz mało kto.
Dowiedzieliśmy się o tym z pracy \cite{menasco1993}; Menasco i~Thistlethwaite dowiedzieli się o~tym od Claude'a Webera.
\index[persons]{Menasco, William}%
\index[persons]{Thistlethwaite, Morwen}%
\index[persons]{Weber, Claude}%
Flype to stary szkocki czasownik oznaczający ,,wykręcać na drugą stronę''.

\begin{figure}[H]
    \centering
    \begin{tikzpicture}[baseline=-0.65ex, scale=0.15]
        \begin{knot}[clip width=6, end tolerance=1pt, flip crossing/.list={1}]
            \strand[ultra thick] (-10, -3) [in=180, out=0] to (-3, 3);
            \strand[ultra thick] (-10, 3) [in=180, out=0] to (-3, -3);
            \draw (-3, -5) rectangle (3, 5);
            \node at (0, 0) {\Huge {$T$}};
            \draw[ultra thick] (3, -3) to (10, -3);
            \draw[ultra thick] (3, 3) to (10, 3);
        \end{knot}
    \end{tikzpicture}
    \quad $\stackrel{\mathrm{flype}}{\cong}$ \quad
    \begin{tikzpicture}[baseline=-0.65ex, scale=0.15]
        \begin{knot}[clip width=6, end tolerance=1pt]
            \strand[ultra thick] (10, -3) [in=0, out=180] to (3, 3);
            \strand[ultra thick] (10, 3) [in=0, out=180] to (3, -3);
            \draw (-3, -5) rectangle (3, 5);
            \node at (0, 0) {\rotatebox[origin=c]{-180}{\Huge $T$}};
            \draw[ultra thick] (-3, -3) to (-10, -3);
            \draw[ultra thick] (-3, 3) to (-10, 3);
        \end{knot}
    \end{tikzpicture}
    \caption{Ruch flype}%
\end{figure}

Całkiem inną taktykę szukania węzłów przyjał wielebny Thomas Kirkman: zaczynał od małego zbioru "nieredukowalnych" rzutów, do których systematycznie dokładał skrzyżowania.
\index[persons]{Kirkman, Thomas}%
Od początku był zainteresowany głównie alternującymi węzłami; w~1885 roku wydrukował diagramy 634 węzłów o~dziesięciu skrzyżowaniach.
% wielebny => Adams, s. 31

\begin{definition}[węzła, Kirkmana, w stu słowach]
    \emph{By a Knot of $n$ crossings, I understand a~reticulation of any number of meshes of two or more edges, whose summits, all tessaraces, are each a~single crossing, as when you cross your forefingers straight or slightly curved, so as not to link them, and such meshes that every thread is either seen, when the projection of the Knot with its $n$ crossings and no more is drawn in double lines, or conceived by the reader of its course when drawn in single line, to pass alternately under and over the threads to which it comes at successive crossings.}
\end{definition}

Wiemy, że Tait czytał czasopismo zawierające diagramy Kirkmana i~wykorzystał je do opracowania prawie kompletnej listy węzłów alternujących o~mniej niż 11 skrzyżowaniach.
Tuż przed oddaniem jej do druku odkrył inny spis węzłów stworzony przez amerykańskiego naukowca Charlesa Little'a.
\index[persons]{Little, Charles}%
Znalazł wtedy jeden duplikat u~siebie, natomiast u Little'a jeden duplikat i~jedno pominięcie.


\input{10-introduction/102b-history-10}


\subsubsection{Jedenaście skrzyżowań}
Na dalszy postęp musieliśmy czekać do lat sześćdziesiątych.
Wtedy to John Conway \cite{conway1970} znalazł prawie wszystkie pierwsze węzły o~mniej niż 12 skrzyżowaniach oraz sploty o~mniej niż 11 skrzyżowaniach.
\index[persons]{Conway, John}%
Odkrył jeden duplikat oraz jedenaście pominięć w~starych tablicach Little'a (do 11 skrzyżowań), ale sam zgubił cztery węzły.
Naprawienie błędu zajęło chwilę: dwa brakujące węzły zawierała praca magisterska Lombardero z~1968 roku, dwa odkrył Caudron około 1980 roku \cite{caudron1982}.
\index[persons]{Caudron, Alain}%
\index[persons]{Lombardero, David}%
Conway też popełnił dwa duplikaty: współcześnie bardziej znany to para Perko, węzeł reprezentowany przez dwa diagramy aż do tablic Rolfsena.
\index{para Perko}%
Nazywamy ją tak, ponieważ została dostrzeżona przez Perko \cite{perko1974}, który bezskutecznie próbował odróżnić składniki pary diedralnym indeksem zaczepienia: dla obydwu diagramów wynosi on $32/11$.
\index[persons]{Perko, Kenneth}%
% https://www.researchgate.net/profile/Ken-Perko/publication/299560799_The_History_of_the_Perko_Pair/links/56ff0ba508aea6b77468d550/The-History-of-the-Perko-Pair.pdf both knots yielded a 5-fold dihedral linking number of 32/11
Jako przyczynę tak długiego niezauważenia pary Perko podaje się hipotezę Little'a, że spin minimalnego diagramu jest niezmiennikiem, gdyż błędnie założył, że 2-przejścia oraz flype wystarczają do zmiany dowolnego minimalnego diagramu w~inny.
% DICTIONARY;2-pass move;2-przejście;-
\index{spin}%
\index{2-przejście}%
\index{flype}%
\index[persons]{Little, Charles}% 
(Diagramy pary Perko mają różny spin).
Zachęcamy w~tym miejscu do nieprzeskakiwania strony \pageref{rolfsens_mistake}, gdzie para Perko pojawia się raz jeszcze.
Mniej znany duplikat wystąpił w~tablicy splotów do 10 skrzyżowań, gdzie \texttt{2.-2.-20.20} jest lustrem \texttt{8*-20:-20}.

Pomimo opisanych wyżej drobnych niepowodzeń, metodę Conwaya (mającą fundamenty w~pomysłach Kirkmana) uznaje się za bardzo dobrą.
Conway potrzebował zaledwie kilku godzin na przeprowadzenie swojej klasyfikacji; a my używamy jej po dziś dzień, na przykład Tuzun, Sikora zweryfikowali dzięki niej hipotezę \ref{con:jones} do 24 skrzyżowań.
\index[persons]{Tuzun, Robert}%
\index[persons]{Sikora, Adam}%

Rękopis \cite{siebenmann1979} (albo \cite{bonahon1989}?) Bonahona, Siebenmanna klasyfikuje węzły algebraiczne.
\index[persons]{Bonahon, Francis}%
\index[persons]{Siebenmann, Laurent}%
Kres ery ręcznych obliczeń nastąpił, gdy z~nielicznymi niealgebraicznymi węzłami do 11 skrzyżowań poradził sobie Perko \cite{perko1980}, \cite{perko1982}.
\index[persons]{Perko, Kenneth}%

\begin{remark}[John Horton Conway]
    Matematyk brytyjski urodzon w 1937 roku w Liverpoolu, zmarł w 2020 roku w New Brunswick, New Jersey.
    Był wszechstronnie uzdolniony: opracował grę w~życie (jeden z pierwszych i najbardziej znanych przykładów automatów komórkowych) oraz kropki (razem z Patersonem, polega na łączeniu kropek liniami tak, by się nie przecinały -- po angielsku \emph{sprouts}); ustalił, że poza sześcioma regularnymi 4-wielotopami oraz dwoma nieskończonymi seriami zaczynającymi się od dwugraniastosłupa i antydwugraniastosłupa istnieją 64 wypukłe jednorodne 4-wielotopy; znalazł grupę automorfizmów kraty Leecha, pokazał, że każą liczbę naturalną można zapisać jako 37 piątych potęg oraz wskazał przykład funkcji, która jest nigdzie ciągła i ma wszędzie własność Darboux.
    Dla nas najciekawsza jest jego praca \cite{conway1970} o supłach.
\end{remark}

% MAKOTO SAKUMA - A SURVEY OF THE IMPACT OF THURSTON’S WORK ON KNOT THEORY
% through hand calculation of homological invariants (in particular linking invariants) of finite branched coverings for those knots that are not covered by Bonahon and Siebenmann’s result described in Subsection 4.1. See [268] for an interesting historical note.



\input{10-introduction/102b-history-13}


\subsubsection{Szesnaście skrzyżowań}
Około roku 1998 Hoste z~Weeksem oraz niezależnie Thistlethwaite \cite{thistlethwaite1998} znaleźli 1 701 936 pierwszych węzłów do 16 skrzyżowań.
\index[persons]{Hoste, Jim}%
\index[persons]{Thistlethwaite, Morwen}%
\index[persons]{Weeks, Jeff}%
Używali przy tym różnych podejść: Hoste z Weeksem wykorzystywali niezmienniki hiperboliczne oraz program SnapPea; Thistlethwaite natomiast wzbogacił zestaw ruchów Reidemeistera o flype, przejście, 2-przejście, ruch Perko oraz kilka innych ezoterycznych przekształceńtak, żeby poradzić sobie z upartymi parami diagramów.
Wśród pierwszych węzłów do 16 skrzyżowań na 32 wyjątkowe (niehiperboliczne) węzły składa się 12 węzłów torusowych oraz 20 satelitów trójlistnika.
To samo jeszcze raz napiszemy na stronie \pageref{page:nonhyperbolic_below_16}.




\subsubsection{Dziewiętnaście skrzyżowań}
Artykuł \cite{thistlethwaite98} zawiera informację, że jego autorzy szukają węzłów o~17 skrzyżowaniach, ale my nie doszukaliśmy się żadnej późniejszej publikacji na ten temat.
\index[persons]{Hoste, Jim}%
\index[persons]{Thistlethwaite, Morwen}%
\index[persons]{Weeks, Jeff}%
W 2004 Flint, Rankin oraz Schermann \cite{rankin04} znaleźli alternujące węzły do 22 skrzyżowań (obliczenia na stacji roboczej z procesorem Xeon oraz 3 gigabajtami pamięci zajęły około 45 godzin), po czym długo nie działo się nic.
\index[persons]{Flint, Ortho}%
\index[persons]{Rankin, Stuart}%
\index[persons]{Schermann, John}%

Dopiero w 2020 Burton \cite{burton20} stablicował węzły pierwsze do 19 skrzyżowań: \emph{,,Here we extend the tables from 16 to 19 crossings, with a total of 352 152 252 distinct non-trivial prime knots.''}
\index[persons]{Burton, Benjamin}%

Thistlethwaite opublikował na swojej stronie internetowej, że znalazł węzły pierwsze do 20 skrzyżowań; czekamy na publikację w renomowanym czasopiśmie.
Jeżeli nie popełnił żadnego błędu, to mamy 199 631 989 alternujących i 1 647 687 439 niealternujących węzłów pierwszych o~dwudziestu skrzyżowaniach.
\index[persons]{Thistlethwaite, Morwen}%
% https://web.math.utk.edu/~morwen/k20v4.pdf



\subsubsection{Sploty}
Cerf \cite{cerf98} pisze, że Conway \cite{conway70} znalazł wcześniej sploty do 10 skrzyżowań, zaś Caudron \cite{caudron82} poprawił wynik do 11 skrzyżowań, ale wszystkie te sploty są niezorientowane, a~naukowcy potrzebują zorientowanych.
\index[persons]{Cerf, Corinne}%
\index[persons]{Conway, John}%
\index[persons]{Caudron, Alain}%
Problem został zaadresowany najpierw przez Dolla i Hoste'a \cite{doll91}, którzy wydali na mikrofilmie tablicę splotów zorientowanych do 9 skrzyżowań, ale ich diagramy nie zawsze pasowały do tych narysowanych w~książce Rolfsena.
\index[persons]{Doll, Helmut}%
\index[persons]{Hoste, Jim}%

Cerf obiecuje pogodzić punkty widzenia Rolfsena oraz Dolla/Hoste'a i tworzy własną tablicę zorientowanych splotów do 11 skrzyżowań.
Sprawdziła jednocześnie poprawność starszych tablic Conwaya -- i nie znalazła tam żadnych błędów.

% https://sci-hub.se/https://doi.org/10.1016/B978-044451452-3/50006-X
Wiemy od Hostego, że Jablan \cite{jablan99} znalazł alternujące sploty o 12 skrzyżowaniach w duchu metod Kirkmana i Conwaya.
\index[persons]{Jablan, Slavik}%
Praca Jablana została wydrukowana w 1999 roku.
W 2001 roku Thistlethwaite posiadał tablicę pierwszych splotów alternujących do 19 skrzyżowań, jednak wygląda na to, że nigdy jej nie opublikował.
% 2001 wiem to z M. Thistlethwaite, Prime unoriented alternating links to 19 crossings, unpublished table (2001)., które wiem z Mathematical Constants II - Strona 631
\index[persons]{Thistlethwaite, Morwen}%
W 2007 roku Fontaine, Rankin, Flint znaleźli liczbę pierwszych splotów alternujących o 20, 21, ..., 24 skrzyżowaniach.
% 2007 wiem to z https://oeis.org/A049344
\index[persons]{Fontaine, Bruce}%
\index[persons]{Rankin, Stuart}%
\index[persons]{Flint, Ortho}%

Klasyfikację splotów o 12 i 13 skrzyżowaniach znaleźliśmy w pracy magisterskiej Dylana Faullina (jego zdaniem jest ich, odpowiednio, 6447 i 28239).
\index[persons]{Faullin, Dylan}%
Z niezrozumiałych dla nas przyczyn pomimo upływu prawie dwóch dekad, ten wynik pozostaje raczej nieznany.
Nie ma go na przykład w ciągu \href{https://oeis.org/A086771}{A086771} w bazie danych OEIS.




\section{Hipotezy Taita}
\index{hipoteza!Taita|(}%

Tait na podstawie węzłów o małej liczbie skrzyżowań  wysunął około 1898 roku trzy lub cztery hipotezy.
Nie jest jasne, czy chodziło mu o wszystkie węzły, czy tylko te alternujące.
Uchylamy tutaj rąbka tajemnicy i~podajemy treść hipotez już teraz; dowód ze szczegółami odkładając na później, aż do sekcji \ref{sub:tait_conjectures}.
Tam też wspomnimy krótko o technikach użytych w dowodach pozostałych trzech.

\begin{conjecture}[I hipoteza Taita]
\index{indeks skrzyżowaniowy}%
\label{con:tait_1}%
    Zredukowany alternujący diagram splotu ma minimalny indeks skrzyżowaniowy.
\end{conjecture}

Najpierw znaleziono dowód korzystający z wielomianu Jonesa: dokonali tego w 1987 roku równocześnie Kauffman \cite{kauffman1987}, Murasugi \cite{murasugi1987} oraz Thistlethwaite \cite{thistlethwaite1987}.
\index[persons]{Kauffman, Louis}%
\index[persons]{Murasugi, Kunio}%
\index[persons]{Thistlethwaite, Morwen}%
Trzydzieści lat później Greene \cite{greene2017} zaprezentował geometryczne podejście do problemu.
\index[persons]{Greene, Joshua}%

\begin{conjecture}[II hipoteza Taita]
\index{spin}%
    Dwa zredukowane diagramy alternujące jednego węzła mają ten sam spin.
\end{conjecture}

Pierwsze dowody pochodzą znowu od Kauffmana \cite{kauffman1987} oraz Thistlethwaite'a \cite{thistlethwaite1987}.
\index[persons]{Kauffman, Louis}%
\index[persons]{Thistlethwaite, Morwen}%
Dla niektórych II hipoteza brzmi inaczej (,,achiralny splot alternujący ma zerowy spin''), dla innych jest prostym wnioskiem z naszego sformułowania.

\begin{conjecture}[III hipoteza Taita]
\index{flype}%
    Niech $D_1, D_2$ będą zredukowanymi alternującymi diagramami zorientowanego pierwszego splotu.
    Wtedy diagram $D_2$ można otrzymać z~$D_1$ korzystając jedynie z~ruchu \emph{flype}.
\end{conjecture}

Trzecią hipotezę udowodnił Menasco wspólnie z~Thistlethwaitem \cite{menasco1993}.
\index[persons]{Menasco, William}%
\index[persons]{Thistlethwaite, Morwen}%
Wynika z~niej, że dwa zredukowane diagramy alternujące tego samego węzła mają ten sam spin.
Nie jest prawdziwa dla niealternujących splotów, przez co w~tablicach węzłów tak długo mieliśmy duplikat -- parę Perko.
\index{para Perko}%

Czasami mówi się jeszcze o IV hipotezie: że zwierciadlane węzły mają parzysty indeks skrzyżowań.
\index{węzeł!zwierciadlany}
Ta okazała się fałszywa.

\index{hipoteza!Taita|)}%

% koniec podsekcji Hipotezy Taita



\input{10-introduction/102f-codes}

