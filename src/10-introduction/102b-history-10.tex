
\subsubsection{Dziesięć skrzyżowań}
Zachęcon przez Taita, Little zabrał się za alternujące węzły o~11 skrzyżowaniach i~za trudniejsze zadanie, stablicowanie węzłów niealternujących, czyli takich, które nie posiadają alternującego diagramu.
\index[persons]{Tait, Peter}%
\index[persons]{Little, Charles}%
Jak wynika z~pierwszej pracy Taita, początkowo nie wierzono, że takie w~ogóle istnieją.
Dowód znaleziono wiele lat później, niealternujące są $8_{19}$, $8_{20}$, $8_{21}$, ale nie pierwsze węzły o mniejszej liczbie skrzyżowań.
Patrz twierdzenie \ref{prp:bankwitz}.
Little pracował przez sześć lat (1893 -- 1899) i~znalazł 43 niealternujące węzły o~10 skrzyżowaniach.
Żadnego nie pominął, ale trafił mu się jeden duplikat.
\index[persons]{Little, Charles}%

W kolejnych dziesięcioleciach nie nastąpił znaczący postęp, zarówno w~rozszerzaniu tablic, jak i~sprawdzaniu tych już istniejących.
Haseman \cite{haseman1918} w~1918 roku znalazła achiralne węzły o~12 (takich jest 54, praca Haseman podaje 61, ponieważ zawiera 7 duplikatów) i~14 skrzyżowaniach.
\index[persons]{Haseman, Mary}%
W 1927 roku Alexander z~Briggsem \cite{alexander1927} przy użyciu pierwszej grupy homologii rozgałęzionego nakrycia cyklicznego (!) potrafili odróżnić od siebie dowolne dwa węzły (z~pominięciem 3 par) o~co najwyżej 9 skrzyżowaniach.
\index[persons]{Briggs, Garland}%
\index[persons]{Alexander, James}%
Reidemeister \cite{reidemeister1932} poradził sobie z~tymi wyjątkami w~1932 roku, korzystając z~indeksu zaczepienia i~homomorfizmów z~grupy węzła na grupy diedralne.
\index[persons]{Reidemeister, Kurt}%
