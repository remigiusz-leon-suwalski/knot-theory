\subsection{Metody kodowania}
\subsubsection{Notacja Alexandera-Briggsa}
\index{notacja!Alexandera-Briggsa}
Najbardziej tradycyjny, wprowadzony w~1927 roku sposób opisu węzłów do 9 skrzyżowań.
Węzły kodowane są przez indeks skrzyżowaniowy z dolnym indeksem informującym o miejscu w tablicy węzłów.
Porzadek jest umowny i nie ma żadnego głębszego znaczenia, jego wybór należy do osoby, która jako pierwsza znajdzie wszystkie węzły o danej liczbie skrzyżowań.
Jedyną regularnością jest to, że węzeł skręcony występuje zawsze po węźle torusowym.

Rolfsen w 1976 stworzył z kilkoma błędami tablicę diagramów pierwszych węzłów do 10 skrzyżowań.
Para Perko $10_{161}, 10_{162}$ przedstawia ten sam węzeł, zaś górne skrzyżowanie w~$10_{144}$ powinno być zmienione.
Ostatnie cztery węzły dostały nowe numery, by uniknąć duplikatu.
Kolejną usterką tablicy jest to, że notacja Conwaya oraz wielomian Alexandera dla węzłów $10_{83}$ oraz $10_{86}$ są zamienione miejscami.
Tu czyha pułapka: Stojmenow, nowe wydanie książki Rolfsena, atlas węzłów Bar-Natana oraz tablica niezmienników węzłowych Livingstona naprawiają to przez wymianę podpisów.
Podręcznik Kawauchiego wymienia diagramy.
% http://stoimenov.net/stoimeno/homepage/ptab/

\subsubsection{Notacja Dowkera-Thistlethwaite'a}
\index{notacja!Taita}
\index{notacja!Dowkera-Thistletwaite'a}
Poprawia notację Taita.
Należy ustalić minimalny diagram węzła, dowolny punkt początkowy oraz kierunek i zacząć przemierzać węzeł.
Za każdym razem, kiedy mijamy skrzyżowanie, przypisujemy mu kolejną liczbę naturalną, zaczynając od jedynki.
Jeżeli znajdujemy się nad skrzyżowaniem, parzyste etykiety zapisujemy z przeciwnym znakiem.
Kiedy skończymy, każde skrzyżowanie będzie mieć dwie etykiety.

\begin{definition}
    Ciąg parzystych liczb występujących na diagramie kolejno przy $1, 3, \ldots$ nazywamy kodem Dowkera-Thistlethwaite'a.
\end{definition}

Opisany powyżej kod nie jest idealny, ponieważ odtworzony z niego węzeł może być lustrzanym odbiciem wyjściowego.
Ogólniej, odbicie dowolnego składnika sumy spójnej nie zmienia kodu całego węzła.
Nie stanowi to jednak dużego problemu, ponieważ notacja została stworzona na potrzeby tablicowania węzłów pierwszych, a~te są niezorientowane.

Zaczynając od zredukowanego diagramu o $n$ skrzyżowaniach nie można doprowadzić do sytuacji, gdzie do pewnego skrzyżowania przypisane są dwie kolejne liczby całkowite.
Dzięki temu problem można przetłumaczyć na język teorii grafów.
Rozpatrzmy graf $G$, którego wierzchołkami są liczby $1, 2, \ldots, 2n$.
Połączmy niesąsiadujące modulo $2n$ wierzchołki o różnej parzystości krawędziami.
Graf ten powstaje przez usunięcie cyklu Hamiltona (łączącego kolejne liczby) z pełnego grafu dwudzielnego.
Zbiór par etykiet przy skrzyżowaniach węzła to skojarzenie doskonałe w grafie $G$.
Liczba skojarzeń prawie pokrywa się z rozwiązaniem zadania znanego w literaturze jako ,,problème des ménages'': na ile sposobów $n$ małżeństw można posadzić przy okrągłym stole tak, by żadne małżeństwo nie siedziało obok siebie i~każdy mężczyzna znalazł się obok dwóch kobiet?
Ustawienia, które powstają przez cykliczne permutowanie należy uznać za tożsame.
Gilbert znalazł w \cite{gilbert56} wzór na $a_n$, liczbę różnych kodów:
\begin{align}
u(m, t) & = 2m \sum_{k=0}^m {2m-k \choose k} \cdot (m-k)! \cdot \frac{(t-1)^k}{2m - k}  \\
a(n) & = \frac{1}{n} \sum_{d\mid n} \left(\frac{n}{d}\right)^d \cdot u \left(d, 1 - \frac{d}{n}\right) \cdot \varphi \left(\frac{n}{d}\right)
\end{align}

Kilka początkowych wartości to $a_3 = 1, 2, 5, 20, 87, 616, 4843, 44128, 444621, \ldots$ (ciąg A002484 w OEIS).

\subsubsection{Notacja Conwaya}
\index{notacja!Conwaya}
Wprowadzona przez Conwaya w~pracy \cite{conway70}.
Opiera się na pojęciu supła, dlatego więcej szczegółów przedstawiamy dopiero w definicji \ref{conway_notation}.

\subsubsection{Nazwy zwyczajowe}
Niektóre węzły i sploty, w szczególności te o niskim indeksie skrzyżowaniowym, występują tak często w teorii węzłów, że doczekały się nazw zwyczajowych.
Oto ich lista:
\begin{compactitem}
    \item węzeł $3_1$ to trójlistnik,
    \item węzeł $4_1$ to ósemka albo węzeł Listinga,
    \item węzeł $5_1$ to pięciolistnik albo węzeł Solomona (!),
    \item węzeł $6_1$ to węzeł dokerski,
    \item węzeł 11n34 to węzeł Conwaya,
    \item węzeł 11n42 to węzeł Kinoshity-Terasakiego,
    \item węzeł 12n242, czyli $(-2, 3, 7)$-precel, to węzeł Fintushela-Sterna,
    \item suma spójna takich samych trójlistników to węzeł babski,
    \item suma spójna lustrzanych trójlistników to węzeł prosty albo płaski (dość niefortunna nazwa),
    \item splot $2_1^2$ (L2a1) to splot Hopfa,
    \item splot $4_1^2$ (L4a1) to węzeł Solomona (!),
    \item splot $5_1^2$ (L5a1) to splot Whiteheada,
    \item splot $6_2^3$ (L6a4) to pierścienie Boromeuszy.
\end{compactitem}