
\section{Diagramy. Ruchy Reidemeistera}

Chociaż w~świetle definicji \ref{def:knot} węzły są pewnymi regularnymi podzbiorami przestrzeni $\R^3$, z~kombinatorycznego punktu widzenia wygodniej jest rysować je na płaszczyźnie.

% DICTIONARY;shadow;cień;-
% DICTIONARY;crossing;skrzyżowanie;-
\begin{definition}[cień, skrzyżowanie]
\index{cień}%
\index{skrzyżowanie}%  
\label{def:crossing}%
    Niech $\pi \colon \R^3 \to \R^2$ będzie rzutem na pewną płaszczyznę, zaś $L \subseteq \R^3$ ustalonym splotem.
    Obraz $\pi[L]$ nazywamy cieniem, punkty podwójne $p$ cienia (punkt $p \in \R^2$, którego przeciwobraz $\pi^{-1}[\{p\}]$ jest dwupunktowy) nazywamy skrzyżowaniami.
\end{definition}

Zazwyczaj nie wprowadza się osobnego terminu na cień, tylko oszczędza czas i zaczyna od razu od diagramów, tak jak Burde, Zieschang, Heusener \cite[s. 9]{burde2014}:

\begin{definition}[diagram]
% DICTIONARY;diagram;diagram;-
\index{diagram}%
    Niech $D$ będzie cieniem splotu $L$ takim, że cień $D$ zawiera skończenie wiele punktów wielokrotnych (i wszystkie są punktami podwójnymi) oraz skrzyżowania cienia $D$ nie zawierają wierzchołków splotu $L$.
    Wtedy cień $D$ wzbogacony o~informację o tym, jak przebiegają skrzyżowania (który odcinek łamanej biegnie dołem, a~który górą) nazywamy diagramem.
% TODO: dopisać, że będziemy to nazywać nad i pod skrzyżowania
\index{nadskrzyżowanie}%
\index{podskrzyżowanie}%
\end{definition}

Mieliśmy problemy ze znalezieniem ładnego rysunku katastrof, jakich nie dopuszczamy pracując z diagramami zamiast zwykłymi cieniami; nam nie do końca chciało się je rysować je tutaj.
Wybawieniem okazała się monografia Burdego, Zieschanga (i Heusenera) \cite[s. 9]{burde2014}.

Dla każdego ustalonego $n \ge 2$ i każdego węzła $K$ istnieje cień $D$, na którym wszystkie wielokrotne punkty są $n$-krotne (wiemy to od Hostego, College'a \cite[s. 11]{adams2021}, którzy nie napisali, skąd to wiedzą);
\index[persons]{Hoste, Jim}%
\index[persons]{College, Pitzer}%
dla co najmniej jednej wartości $n$ można dodatkowo wymagać, by diagram zawierał dokładnie jedno skrzyżowanie (Adams, Crawford, DeMeo, Landry, Lin, Montee, Park, Venkatesh, Yhee \cite{venkatesh2015} albo Brunn ponad sto lat temu \cite[s. 28]{adams2021}!).
\index[persons]{Adams, Colin}%
\index[persons]{Crawford, Thomas}%
\index[persons]{DeMeo, Benjamin}%
\index[persons]{Landry, Michael}%
\index[persons]{Lin, Alex}%
\index[persons]{Montee, Murphy}%
\index[persons]{Park, Seojung}
\index[persons]{Venkatesh, Saraswathi}%
\index[persons]{Yhee, Farrah}%

\begin{definition}[włókno, nić]
\index{włókno}%
\index{nić}%
    Fragment diagramu biegnący między dwoma kolejnymi skrzyżowaniami (odpowiednio: tunelami, czyli podskrzyżowaniami), nazywamy nicią (odpowiednio: włóknem)
\end{definition}

Skrzyżowania i diagramy są standardowymi terminami, zrozumiałymi przez każdego. 
Cienie, nici i włókna stanowią twórczość własną autorów!
Nici powstają z włókien przez rozcięcie ich przy każdym nadskrzyżowaniu.

\begin{proposition}
\label{prp:links_have_diagrams}%
    Niech $L$ będzie splotem.
    Jego diagramy tworzą otwarty i~gęsty podzbiór wszystkich rzutów.
\end{proposition}

Kawauchi \cite[s. 7]{kawauchi1996} wspomina w tym miejscu podręcznik Crowella, Foxa \cite[s. 7]{crowell1963}.
To samo jest na przykład u Burdego, Zieschanga, Heusenera \cite[s. 10]{burde2014}, ale oni odsyłają jeszcze do Reidemeistera \cite{reidemeister1927} i samego Burdego \cite{burde1978}.

\begin{proof}[Niedowód]
    Rzut splotu na równoległe płaszczyzny jest taki sam, a te można sparametryzować prostymi przechodzącymi przez początek układu współrzędnych, które tworzą przestrzeń rzutową $\R \mathbb P^2$.
    Niech $S$ będzie zbiorem prostych, które dają złe rzuty.
    Wystarczy pokazać jego nigdziegęstość.
    Okazuje się, że $S$ jest też jednowymiarowy.
\end{proof}

\begin{corollary}
    Każdy splot posiada diagram.
\end{corollary}

% DICTIONARY;oriented;zorientowany;węzeł
\begin{definition}[orientacja]
\index{węzeł!zorientowany}%
\index{orientacja|see {węzeł zorientowany}}%
    Węzeł, w~którym wybrano kierunek, w~którym należy się po nim poruszać, nazywamy zorientowanym.
    Splot nazywamy zorientowanym, jeśli wszystkie jego ogniwa traktowane jako węzły są zorientowane.
\end{definition}

Orientację na diagramie zaznaczamy małą strzałką wskazującą kierunek poruszania się.


\subsection{Ruchy Reidemeistera}

Wiemy już, że węzły mają wiele diagramów.
Mając dane dwa różne diagramy chcielibyśmy wiedzieć, czy przedstawiają ten sam węzeł.
Stosowne narzędzie dostarczył Kurt Reidemeister w~latach dwudziestych ubiegłego wieku.
\index[persons]{Reidemeister, Kurt}%
Zdefiniujmy trzy lokalne operacje na diagramach, a~potem wysłowimy kryterium  Reidemeistera rozstrzygające problem równości węzłów.

% DICTIONARY;Reidemeister/Turaev/... move;ruch Reidemeistera/Turajewa/...;-
\begin{definition}[ruchy Reidemeistera]
\index{ruch!Reidemeistera}%
    Trzy gatunki lokalnych deformacji diagramu splotu:
    \begin{figure}[H]
    \centering
    \begin{minipage}[b]{.22\linewidth}%
        \centering%
        \MedLarReidemeisterOneLeft $\stackrel{R_1}{\cong}$ \MedLarReidemeisterOneStraight%
        \subcaption{ruch $R_1$}%
    \end{minipage}
    \quad\quad\quad
    \begin{minipage}[b]{.2\linewidth}
        \centering
        \MedLarReidemeisterTwoA $\stackrel{R_2}{\cong}$ \MedLarReidemeisterTwoB
        \subcaption{ruch $R_2$}
    \end{minipage}
    \quad\quad\quad
    \begin{minipage}[b]{.32\linewidth}
        \centering
        \MedLarReidemeisterThreeA $\stackrel{R_3}{\cong}$ \MedLarReidemeisterThreeB
        \subcaption{ruch $R_3$}
    \end{minipage}
    \end{figure}
    nazywamy ruchami Reidemeistera.
    Czasami używa się konkretnych nazw:
    \begin{itemize}
        \item skręcenie/rozkręcenie (dla $R_1$),
        \item wsunięcie/rozsunięcie (dla $R_2$) oraz
        \item przesunięcie łuku przez skrzyżowanie (dla $R_3$).
    \end{itemize}
\end{definition}

Reidemeister w~swojej pierwszej pracy przyjął inną kolejność, jego drugi ruch jest naszym pierwszym.
Dzięki temu ruch $R_k$ operuje na $k$ łukach diagramu.
Colberg \cite[s. 6]{colberg2013} pisze, że Maxwell znał ruchy Reidemeistera przed Reidemeisterem, ale mimo próśb Taita nigdy nie zgłosił swojego odkrycia w Royal Society of Edinburgh.
\index[persons]{Tait, Peter}%
\index[persons]{Maxwell, James}%

\begin{theorem}[Reidemeister, 1927]
\label{thm:reidemeister}%
\index{twierdzenie!Reidemeistera}%
\index[persons]{Reidemeister, Kurt}%
    Niech $D_1, D_2$ będą diagramami dwóch splotów $L_1, L_2$.
    Sploty $L_1, L_2$ są takie same wtedy i tylko wtedy, gdy diagram $D_2$ można otrzymać z $D_1$ wykonując skończony ciąg ruchów Reidemeistera oraz gładkich deformacji łuków, bez zmiany biegu skrzyżowań.
\end{theorem}
% https://math.stackexchange.com/questions/4399634/two-knots-k-and-k-prime-are-equivalent-if-and-only-if-their-projections-p
% Reidemeister, and pretty much every other author, has worked with the piecewise-linear case. In a way it doesn't matter which you choose, since there's a theorem that the categories of smooth and PL manifolds are equivalent in some sense. However, it's not so clear how you pass from one setting to the other (or at least I've never gone through the details myself!)

Twierdzenie Reidemeistera jest prawdziwe także dla splotów zorientowanych, ale wtedy trzeba uwzględnić różne orientacje łuków i~nie jest oczywiste, ile spośród $2^1 + 2^2 + 2^3 = 14$ wersji jest potrzebne.
Polyak \cite{polyak2010} pokazał, że minimalny zbiór zorientowanych ruchów składa się na przykład z~dwóch wersji ruchu $R_1$, jednej wersji ruchu $R_2$ i~jednej wersji ruchu $R_3$.
\index[persons]{Polyak, Michael}%

\begin{proof}[Niedowód]
Dowód podali niezależnie Reidemeister \cite{reidemeister1927} oraz Alexander, Briggs \cite{alexander1927}.
\index[persons]{Reidemeister, Kurt}%
\index[persons]{Briggs, Garland}%
\index[persons]{Alexander, James}%
    Szkielet dowodu można znaleźć w~książce Burdego i~Zieschanga \cite[s. 9-11]{burde2014}, ale kluczowe pomysły podają też Prasołow z~Sosińskim \cite[s. 11-12]{prasolov1997}.
\index[persons]{Burde, Gerhard}%
\index[persons]{Zieschang, Heiner}%
\index[persons]{Prasołow, Wiktor (Прасолов, Виктор Васильевич)}%
\index[persons]{Sosiński, Aleksiej (Сосинский, Алексей Брониславович)}%
    Innym przystępnym źródłem jest podręcznik Murasugiego \cite[s. 50-56]{murasugi1996}.
\index[persons]{Murasugi, Kunio}%
\end{proof}

Trace \cite{trace1983} zauważył, że dwa diagramy jednego węzła są związane tylko II i III ruchem (ale nie I) wtedy i tylko wtedy, gdy mają ten sam spin oraz indeksy nawinięcia.
\index[persons]{Trace, Bruce}%
Z prac Östlunda \cite{ostlund2001}, Manturowa \cite[s. ???]{manturov2004} oraz Haggego \cite{hagge2006} wynika, że dla każdego węzła istnieje para diagramów, do przejścia między którymi trzeba wykorzystać wszystkie trzy ruchy.
% TODO: ustalić, które strony w Manturowie
\index[persons]{Östlund, Olof}%
\index[persons]{Manturow, Wasilij}%
\index[persons]{Hagge, Tobias}%
% praca Haggego nazywa się "Every Reidemeister move is needed for each knot type" ale nawet w MathSciNecie wspomnieni są Ostlund i Manturow, więc zostawiam. Tekst skopiowany z Wiki
Coward \cite{coward2006} zademonstrował, że nawet jeśli wszystkie trzy ruchy są potrzebne, można je wykonywać w specjalnej kolejności: najpierw tylko I ruchy, potem tylko II ruchy, następnie tylko III ruchy i~znowu II ruchy.
\index[persons]{Coward, Alexander}%

Do pokazania, że dwa węzły $K_1, K_2$ nie są równoważne, powinniśmy na mocy twierdzenia \ref{thm:reidemeister} udowodnić, że żaden ciąg ruchów Reidemeistera nie przekształca jednego w drugi.
Oczywiście nikt o zdrowych zmysłach tak nie postępuje.
Zamiast tego wprowadza się stosowny niezmiennik, czyli funkcję $f$ określoną na zbiorze wszystkich węzłów (albo splotów, supłów, warkoczy itd.) tak, że jeśli węzły $K_1 \cong K_2$ są równoważne, to $f(K_1) = f(K_2)$.
% DICTIONARY;invariant;niezmiennik;-
Łatwo widać, że jeśli $f(K_1) \neq f(K_2)$, to węzły $K_1, K_2$ nie mogą być równoważne.
Natomiast gdy wartości są te same, nie dostajemy żadnej informacji.

Poznaliśmy jak na razie dwa niezmienniki: liczbę ogniw splotu oraz dopełnienie splotu do przestrzeni, w której jest zanurzony ($\mathbb R^3$ lub $S^3$).
Wiele, chociaż nie wszystkich, innych niezmienników definiuje się nie bezpośrednio na zbiorze węzłów, ale na zbiorze diagramów.
Należy wtedy sprawdzić, że każdy z trzech ruchów Reidemeistera nie ma wpływu na wartość niezmiennika.

Niezmienniki będą nam stale towarzyszyć w~wędrówce po krainie węzłów.

\begin{remark}[Kurt Werner Friedrich Reidemeister]
    ?
\end{remark}

\begin{remark}[James Waddell Alexander]
    ?
\end{remark}

\begin{remark}[Garland Baird Briggs]
    Matematyk amerykański, urodzon w Sebrell, Wirginii w 1894 roku; zmarł w Kolumbii w 1959 roku.
    Niestety nie wiemy za dużo o tym człowieku.
\end{remark}

\color{white}

\subsubsection{Dygresja -- wyniki ilościowe wokół twierdzenia Reidemeistera}
Załóżmy, że na dwóch diagramach tego samego węzła widać odpowiednio $n_1, n_2$ skrzyżowań.
Jak piszą Coward, Lackenby \cite{coward2011}, istnieje funkcja $f \colon \N \times \N \to \N$ taka, że między dwoma diagramami można przejść wykonując co najwyżej $f(n_1, n_2)$ ruchów.
\index[persons]{Coward, Alexander}%
\index[persons]{Lackenby, Marc}%
Wynika to z faktu, że istnieje skończenie wiele spójnych diagramów o danej liczbie skrzyżowań oraz twierdzenia Reidemeistera.
Okazuje się jednak, że od funkcji $f$ można żądać, by była obliczalna
(a to jest chyba równoważne istnienia algorytmu rozpoznającego, czy dwa diagramy przedstawiają jeden węzeł)
% http://people.dm.unipi.it/martelli/Cortona/Lackenby.pdf 7 of 90
i faktycznie, główny wynik \cite{coward2011} orzeka, że
\begin{equation}
    f(n_1, n_2) = 2^{2^{\ldots^{2^{n_1 + n_2}}}}
\end{equation}
jest taką funkcją.
Piętrowa potęga liczy sobie aż $10^{1000000 (n_1 + n_2)}$ warstw.

Natomiast jeżeli $n_2 = 0$, czyli drugi diagram przedstawia niewęzeł, ,,wystarcza'' $(236n_1)^{11}$ ruchów, przy czym liczba skrzyżowań podczas transformacji nigdy nie przekracza $49c^2$: to świeższy wynik samego Lackenby'ego \cite{lackenby2015}, gdzie poprawił wcześniejsze oszacowania Hassa, Lagariasa.
Przykład diagramu niewęzła, do rozwiązania którego nie można tylko usuwać istniejących skrzyżowań, przedstawiają Burde, Zieschang, Heusener \cite[s. 12]{burde2014}.

Hayashi \cite{hayashi2005} dowiódł, że liczbę ruchów Reidemeistera potrzebnych, by rozszczepić splot można ograniczyć z góry na podstawie indeksu skrzyżowaniowego.
\index[persons]{Hayashi, Chuichiro}%

% koniec sekcji Ruchy Reidemeistera



\input{10-introduction/102b-history}

\input{10-introduction/102c-codes}


\section{Wykrywanie niewęzła}
Jednym z pierwszych dużych problemów teorii węzłów było poszukiwanie odpowiedzi na pytanie, kiedy diagram przedstawia niewęzeł.
\index{niewęzeł}%
Stosowny algorytm wykrywający niewęzły podał Haken \cite{haken1961}, ale długo nikt nie podjął się jego implementacji.
\index[persons]{Haken, Wolfgang}%
Epple pisze \emph{,,this algorithm was extremely impractical''}, w recenzji z MathSciNet proponuje, żeby przed przeczytaniem pełnej niepotrzebnych dygresji pracy Hakena poznać artykuł \cite{schubert1961} Schuberta.
\index[persons]{Epple, Moritz}%
\index[persons]{Schubert, Horst}%
W życie pomysły Hakena udało się wdrożyć Burtonowi, Budneyowi oraz Petterssonowi w~komputerowym programie Regina\footnote{\url{https://regina-normal.github.io/}.} na przełomie tysiącleci.

\index[persons]{Burton, Benjamin}%
\index[persons]{Budney, Ryan}%
\index[persons]{Pettersson, William}%
%=% https://mathscinet.ams.org/mathscinet-getitem?mr=141107
% DICTIONARY;incompressible;nieściśliwy;-

Burton, Rubinstein i~Tillman \cite{burton2012} pokazali, jak sprawdzać, w~czasie wykładniczym czy powierzchnia normalna na striangulowanej 3-rozmaitości jest (nie)ściśliwa.
\index[persons]{Rubinstein, Joachim}%
\index[persons]{Tillman, Stephan}%
To wystarczyło do udzielenia negatywnej odpowiedzi na pytanie Thurstona: \emph{,,czy przestrzeń Seiferta-Webera jest rozmaitością Hakena?''}, a zatem wykraczającego poza poziom naszego skromnego dzieła.
\index[persons]{Thurston, William}%
\index{przestrzeń!Seiferta-Webera}%
\index{rozmaitość!Hakena}%

SnapPea\footnote{\url{http://geometrygames.org/SnapPea/index.html}.} to inny popularny wśród niskowymiarowych topologów program pozwalający badać hiperboliczne 3-rozmaitości, patrz sekcja \ref{sec:hyperbolic}.

Wiadomo, że genus oraz zredukowana kohomologia Chowanowa wykrywa niewęzły (fakty \ref{prp:genus_detects_unknot}, \ref{khovanov_detects_unknot}) i nie wiadomo, czy wielomian Jonesa to robi (hipoteza \ref{con:jones}).
\index{genus}%
\index{homologia!Chowanowa}%
\index{wielomian!Jonesa}%
Od dawna wiadomo, że wielomian Alexandera nie wykrywa niewęzła (fakt \ref{alexander_no_detects_unknot}).
\index{wielomian!Alexandera}%
W lutym 2021 Lackenby ogłosił nowy algorytm rozpoznający niewęzły w~quasiwielomianowym czasie.
% i nie zrobił tego w THE EFFICIENT CERTIFICATION OF KNOTTEDNESS AND THURSTON NORM, bo to wyszło na arxiv w 2016
\index[persons]{Lackenby, Marc}%

Wyśmienitym punktem wyjścia do poszukiwań trudnych niewęzłów jest praca \cite{schleimer2021}, dzieło Burtona, Changa, Löfflera, de Mesmaya, Marii, Schleimera, Sedgwicka oraz Spreera.
\index[persons]{Burton, Benjamin}%
\index[persons]{Chang, Hsien-Chih}%
\index[persons]{Mesmay, Arnaud@de Mesmay, Arnaud}%
\index[persons]{Löffler, Maarten}%
\index[persons]{Maria, Clément}%
\index[persons]{Schleimer, Saul}%
\index[persons]{Sedgwick, Eric}%
\index[persons]{Spreer, Jonatha}%
Cytuje ona artykuł Lackenby'a \cite{lackenby2015}, gdzie poznaliśmy stary (z 1934 roku!) przykład Goeritza \cite{goeritz1934} diagramu niewęzła o~11 skrzyżowaniach, który można zmienić w~zwykły diagram niewęzła tylko zwiększając po drodze liczbę skrzyżowań.
% 9781470454999 s. 5
\index[persons]{Goeritz, Lebrecht}%
\index{niewęzeł!Goeritza}%
Na kolejne teksty przyszło poczekać ponad pół wieku.
Autorzy przywołują jeszcze klasyczny przykład Freedmana, He, Wanga \cite{freedman1994}; ale też podstępne niewęzły Hakena, Ochiai, Thistlethwaite'a oraz mocno doświadczalną pracę Petronio, Zanellatiego \cite{zanellati2016}.
\index[persons]{Freedman, Michael}%
\index[persons]{He, Zheng-Xu}%
\index[persons]{Wang, Zhenghan}%
\index{niewęzeł!Freedmana}%
\index[persons]{Petronio, Carlo}%
\index[persons]{Zanellati, Adolfo}%

% TODO: sprawdzić, czy w \cite{heinrich2014} nie ma więcej trudnych niewęzłów

My nie potrafimy albo nie chcemy potrafić ładnie rysować, więc pozwolimy sobie pokazać tylko, jak wyglądał wspomniany wcześniej przykład Goeritza.
Na swoim blogu\footnote{\url{https://mickburton.co.uk/2015/06/05/how-do-you-construct-hakens-gordian-knot/}} Burton (inny Burton niż w poprzednim akapicie!) zamieścił wpis pełen rysunków, które tłumaczą, jak powstał niewęzeł Hakena.

\begin{comment}
\begin{figure}[H]
    \centering
    \begin{tikzpicture}[baseline=-0.65ex, scale=0.08]
        \begin{knot}[clip width=5, end tolerance=1pt, flip crossing/.list={1,2,3,4,8,9}]
            % horizontal lines
            \strand[ultra thick] (-40, -10) to (-10, -10);
            \strand[ultra thick] (-40, 10) to (-10, 10);
            \strand[ultra thick] (10, 10) to (30, 10);
            \strand[ultra thick] (10, -10) to (30, -10);
            % 
            \strand[ultra thick] (5-45, -10) [in=left,out=left] to (5-45, 3.33);
            \strand[ultra thick] (5-45, 10) [in=left,out=left] to (5-45, -3.33);
            \strand[ultra thick] (-5-35, -3.33) [in=left,out=right] to (5-35, 3.33);
            \strand[ultra thick] (-5-35, 3.33) [in=left,out=right] to (5-35, -3.33);
            \strand[ultra thick] (-5-25, -3.33) [in=left,out=right] to (5-25, 3.33);
            \strand[ultra thick] (-5-25, 3.33) [in=left,out=right] to (5-25, -3.33);
            \strand[ultra thick] (-5-15, -3.33) [in=left,out=right] to (5-15, 3.33);
            \strand[ultra thick] (-5-15, 3.33) [in=left,out=right] to (5-15, -3.33);
            %
            \strand[ultra thick] (-5+15, -3.33) [in=left,out=right] to (5+15, 3.33);
            \strand[ultra thick] (-5+15, 3.33) [in=left,out=right] to (5+15, -3.33);
            \strand[ultra thick] (-5+25, -3.33) [in=left,out=right] to (5+25, 3.33);
            \strand[ultra thick] (-5+25, 3.33) [in=left,out=right] to (5+25, -3.33);
            \strand[ultra thick] (-5+35, -3.33) [in=right,out=right] to (-5+35, 10);
            \strand[ultra thick] (-5+35, 3.33) [in=right,out=right] to (-5+35, -10);
            %
            \strand[ultra thick] (-5+5, 3.33) [in=left,out=right] to (5+5, 10);
            \strand[ultra thick] (-5+5, 10) [in=left,out=right] to (5+5, 3.33);
            \strand[ultra thick] (-5-5, 3.33) [in=left,out=right] to (5-5, 10);
            \strand[ultra thick] (-5-5, 10) [in=left,out=right] to (5-5, 3.33);
            %
            \strand[ultra thick] (-5+5, -10) [in=left,out=right] to (5+5, -3.33);
            \strand[ultra thick] (-5+5, -3.33) [in=left,out=right] to (5+5, -10);
            \strand[ultra thick] (-5-5, -10) [in=left,out=right] to (5-5, -3.33);
            \strand[ultra thick] (-5-5, -3.33) [in=left,out=right] to (5-5, -10);
        \end{knot}
    \end{tikzpicture}
    \caption{niewęzeł Goeritza}
\end{figure}
\end{comment}




\section{Hipotezy Taita}
\index{hipoteza!Taita|(}%

Tait na podstawie węzłów o małej liczbie skrzyżowań  wysunął około 1898 roku trzy lub cztery hipotezy.
Nie jest jasne, czy chodziło mu o wszystkie węzły, czy tylko te alternujące.
Uchylamy tutaj rąbka tajemnicy i~podajemy treść hipotez już teraz; dowód ze szczegółami odkładając na później, aż do sekcji \ref{sub:tait_conjectures}.
Tam też wspomnimy krótko o technikach użytych w dowodach pozostałych trzech.

\begin{conjecture}[I hipoteza Taita]
\index{indeks skrzyżowaniowy}%
\label{con:tait_1}%
    Zredukowany alternujący diagram splotu ma minimalny indeks skrzyżowaniowy.
\end{conjecture}

Najpierw znaleziono dowód korzystający z wielomianu Jonesa: dokonali tego w 1987 roku równocześnie Kauffman \cite{kauffman1987}, Murasugi \cite{murasugi1987} oraz Thistlethwaite \cite{thistlethwaite1987}.
\index[persons]{Kauffman, Louis}%
\index[persons]{Murasugi, Kunio}%
\index[persons]{Thistlethwaite, Morwen}%
Trzydzieści lat później Greene \cite{greene2017} zaprezentował geometryczne podejście do problemu.
\index[persons]{Greene, Joshua}%

\begin{conjecture}[II hipoteza Taita]
\index{spin}%
    Dwa zredukowane diagramy alternujące jednego węzła mają ten sam spin.
\end{conjecture}

Pierwsze dowody pochodzą znowu od Kauffmana \cite{kauffman1987} oraz Thistlethwaite'a \cite{thistlethwaite1987}.
\index[persons]{Kauffman, Louis}%
\index[persons]{Thistlethwaite, Morwen}%
Dla niektórych II hipoteza brzmi inaczej (,,achiralny splot alternujący ma zerowy spin''), dla innych jest prostym wnioskiem z naszego sformułowania.

\begin{conjecture}[III hipoteza Taita]
\index{flype}%
    Niech $D_1, D_2$ będą zredukowanymi alternującymi diagramami zorientowanego pierwszego splotu.
    Wtedy diagram $D_2$ można otrzymać z~$D_1$ korzystając jedynie z~ruchu \emph{flype}.
\end{conjecture}

Trzecią hipotezę udowodnił Menasco wspólnie z~Thistlethwaitem \cite{menasco1993}.
\index[persons]{Menasco, William}%
\index[persons]{Thistlethwaite, Morwen}%
Wynika z~niej, że dwa zredukowane diagramy alternujące tego samego węzła mają ten sam spin.
Nie jest prawdziwa dla niealternujących splotów, przez co w~tablicach węzłów tak długo mieliśmy duplikat -- parę Perko.
\index{para Perko}%

Czasami mówi się jeszcze o IV hipotezie: że zwierciadlane węzły mają parzysty indeks skrzyżowań.
\index{węzeł!zwierciadlany}
Ta okazała się fałszywa.

\index{hipoteza!Taita|)}%

% koniec podsekcji Hipotezy Taita



