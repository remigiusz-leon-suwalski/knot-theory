\section{Węzły pierwsze}
\label{sec:prime_knots}
Istnieje węzłowy odpowiednik liczb pierwszych.
Jest on ściśle związany z~podaną wyżej operacją sumy spójnej.
Do jego dostatecznie dobrego zrozumienia wymagana jest znajomość powierzchni Seiferta (opisanych w~sekcji \ref{sec:genus}).

\begin{definition}
    \label{primeknot}
    \index{węzeł!pierwszy}
    Nietrywialny węzeł nazywamy \textbf{pierwszym},
    kiedy nie można przedstawić go jako sumy spójnej $K_1 \shrap K_2$
    dwóch nietrywialnych węzłów $K_1, K_2$ (nie jest złożony).
\end{definition}

Okazuje się, że jeśli alternujący splot nie jest pierwszy,
to każdy jego alternujący diagram jest złożony.
Jako pierwszy fakt ten został wykazany przez Menasco w~\cite{menasco84}.
Dowód opiera się na multiplikatywności wielomianu BLM/Ho (opisuje go definicja \ref{def:blm_ho}).

Czy węzłów pierwszych jest nieskończenie wiele?
Tak (patrz fakt \ref{infty_primes}), potrafimy nawet oszacować liczbę $K_n$ węzłów pierwszych oraz $L_n$ splotów pierwszych.
W roku 1987 C. Ernst, D. Sumners w~oparciu o~wyniki Thistlethwaite'a, Kauffmana, oraz Murasugiego dotyczące węzłów alternujących pokazali w~\cite{ernst87}, że $K_n \ge \frac 1 3 (2^{n- 2} - 1)$, przy czym węzły lustrzane traktowane są jako różne.
Dokładniej:

\begin{proposition}
    Niech $f(n)$ oznacza liczbę węzłów dwumostowych o indeksie skrzyżowaniowym $n$.
    Wtedy
    \begin{equation}
        f(n) = \begin{cases}
        \frac 13 (2^{n-2} - 1) & \text{dla } n = 2k \ge 4 \\
        \frac 13 (2^{n-2} + 2^{(n-1)/2}) & \text{dla } n = 4k + 1 \ge 5 \\
        \frac 13 (2^{n-2} + 2^{(n-1)/2} + 2) & \text{dla } n = 4k + 3 \ge 7
        \end{cases}
    \end{equation}
\end{proposition}


Welsh rozpatruje w \cite{welsh92} węzły bez orientacji i znajduje poniższe ograniczenia.
Nie wiadomo, czy zwykłe granice istnieją.
\begin{equation}
    2.68 \le \liminf_{n \to \infty}  \sqrt[n]{K_n} \le \limsup_{n \to \infty} \sqrt[n]{L_n} \le \frac {27}{2}.
\end{equation}

% "On the number of knots and links" (MR1218230)

Czy niewęzeł nie daje się zapisać jako suma dwóch innych węzłów?
Byłoby to skrajnie niepożądane, gdyż każdy węzeł jest naturalnie spójną sumą siebie oraz niewęzła.
Na szczęście przy pomocy powierzchni Seiferta można pokazać, że tak się nie dzieje (jest to wniosek \ref{no_inverses}).
Prawdziwe jest dużo mocniejsze stwierdzenie,
którego nie udowodnimy ze względu na niedostatecznie rozwinięty aparat matematyczny.
Należy o~nim myśleć jak o~analogonie zasadniczego twierdzenia arytmetyki.

\begin{theorem}[Schubert, 1949]
    \label{thm:schubert}
    Każdy różny od niewęzła węzeł rozkłada się jednoznacznie na węzły pierwsze
    (jeśli tylko pominąć kolejność składników).
\end{theorem}

Schubert podał geometryczny dowód oparty o powierzchnie Seiferta; wyraził go w języku PL-rozmaitości (\cite{schubert49}), ale niedużym wysiłkiem można dokonać adaptacji do gładkiego świata.
Praca Schuberta korzysta z twierdzenia Alexandera, że 2-sfera w przestrzeni $\R^3$ ogranicza dysk, i jego odpowiednika dla torusów w $S^3$

My pokażemy tylko, że rozkład istnieje i jest skończony, to fakt \ref{genus_sum}.
% koniec sekcji Węzły pierwsze
