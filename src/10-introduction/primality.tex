\section{Węzły pierwsze}
\label{sec:prime_knots}
Suma spójna jest dla węzłów tym, czym mnożenie dla liczb naturalnych.
Analogia ta nabiera sensu, gdy zdefiniujemy węzły pierwsze, odpowiedniki liczb pierwszych.
Do ich dobrego zrozumienia warto znać powierzchnie Seiferta z~sekcji \ref{sec:genus}.

\begin{definition}[węzeł pierwszy]
    \label{primeknot}
    \index{węzeł!pierwszy}
    Niech $K$ będzie węzłem różnym od niewęzła.
    Jeśli nie przedstawia się jako suma spójna $K_1 \shrap K_2$
    dwóch nietrywialnych węzłów $K_1, K_2$, nazywamy go węzłem pierwszym.

    W~przeciwnym razie mówimy, że jest złożony.
\end{definition}

Okazuje się, że jeśli alternujący splot nie jest pierwszy,
to każdy jego alternujący diagram jest złożony.
Jako pierwszy fakt ten został wykazany przez Menasco w~\cite{menasco84}.
Dowód opiera się na multiplikatywności wielomianu BLM/Ho z~definicji \ref{def:blm_ho}.

Czy węzłów pierwszych jest nieskończenie wiele?
Tak (patrz fakt \ref{infty_primes}), potrafimy nawet oszacować liczbę $K_n$ węzłów pierwszych oraz $L_n$ splotów pierwszych.
W roku 1987 C. Ernst, D. Sumners w~oparciu o~wyniki Thistlethwaite'a, Kauffmana, oraz Murasugiego dotyczące węzłów alternujących pokazali w~\cite{ernst87}, że $K_n \ge \frac 1 3 (2^{n- 2} - 1)$, przy czym węzły lustrzane traktowane są jako różne.
Dokładniej:

\begin{proposition}
    Niech $f(n)$ oznacza liczbę węzłów dwumostowych o indeksie skrzyżowaniowym $n$.
    Wtedy
    \begin{equation}
        f(n) = \begin{cases}
        \frac 13 (2^{n-2} - 1) & \text{dla } n = 2k \ge 4 \\
        \frac 13 (2^{n-2} + 2^{(n-1)/2}) & \text{dla } n = 4k + 1 \ge 5 \\
        \frac 13 (2^{n-2} + 2^{(n-1)/2} + 2) & \text{dla } n = 4k + 3 \ge 7
        \end{cases}
    \end{equation}
\end{proposition}


Welsh rozpatruje w \cite{welsh92} węzły bez orientacji i znajduje poniższe ograniczenia.
Nie wiadomo, czy zwykłe granice istnieją.
\begin{equation}
    2.68 \le \liminf_{n \to \infty}  \sqrt[n]{K_n} \le \limsup_{n \to \infty} \sqrt[n]{L_n} \le \frac {27}{2}.
\end{equation}

% "On the number of knots and links" (MR1218230)

Każdy węzeł jest sumą spójną siebie oraz niewęzła, dlatego byłoby miło, gdyby niewęzeł nie dał się zapisać jako suma dwóch innych węzłów.
Wniosek \ref{no_inverses} mówi więcej, że suma spójna nie posiada elementów odwrotnych.
Ponadto odpowiednik zasadniczego twierdzenia arytmetyki jest prawdziwy dla węzłów.
Później, czyli jako fakt \ref{genus_sum}, pokażemy, że rozkład na węzły pierwsze istnieje i~wspomnimy, dlaczego jest jedyny.

% koniec sekcji Węzły pierwsze
