
\subsection{Sploty alternujące}

Zanim opowiemy, jak dotąd przebiegała klasyfikacja węzłów o małej liczbie skrzyżowań, zdefiniujemy klasę splotów ze specjalnymi diagramami.

% DICTIONARY;alternating;alternujący;węzeł
\begin{definition}[alternacja]
\index{węzeł!alternujący}%
    Diagram splotu, gdzie podczas poruszania się wzdłuż każdego ogniwa nad- oraz podskrzyżowania mijane są naprzemiennie, nazywamy alternującym.
    Splot jest alternujący, jeśli posiada alternujący diagram.
\end{definition}

Około 1961 roku Fox zapytał ,,What is an alternating knot?''.
\index[persons]{Fox, Ralph}%
Szukano takiej definicji węzła alternującego, która nie odnosi się bezpośrednio do diagramów, aż w~2015 roku Joshua Greene podał geometryczną charakteryzację: nierozszczepialny splot w $S^3$ jest alternujący wtedy i tylko wtedy, gdy ogranicza dodatnią oraz ujemną określoną powierzchnię rozpinającą \cite{greene17}.
\index[persons]{Greene, Joshua}%
% definite spanning surface

Sundberg, Thistlethwaite \cite{sundberg98} pokazali w 1998 roku, że liczba splotów alternujących rośnie co najmniej wykładniczo:
\index[persons]{Sundberg, Carl}%
\index[persons]{Thistlethwaite, Morwen}%

\begin{proposition}
\index{supeł}%
    Niech $a_n$ oznacza liczbę pierwszych, alternujących supłów o~$n$ skrzyżowaniach.
    Wtedy
    \begin{equation}
        a_n \sim (3c_1/4\sqrt{\pi})n^{-5/2}\lambda^{n-3/2},
    \end{equation}
    gdzie zarówno $c_1$, pierwszy współczynnik rozwinięcia Taylora funkcji $\Phi(\eta)$ zdefiniowanej w \cite{sundberg98}, jak i $\lambda$ są jawnie znanymi stałymi:
    \begin{align}
        c_1 & = \sqrt{\frac{5^7 \cdot (21001 + 371 \sqrt{21001})^3}{2 \cdot 3^{10} \cdot (17 + 3\sqrt{21001})^5}} \\
        \lambda & = \frac {1}{40} (101 + \sqrt{21001})
    \end{align}
    Niech $A_n$ oznacza liczbę pierwszych, alternujących splotów o $n$ skrzyżowaniach.
    Wtedy $A_n \approx \lambda^n$, dokładniej: jeśli $n \ge 3$, to
    \begin{equation}
        \frac{a_{n-1}}{16n - 24} \le A \le \frac{a_n - 1}{2}.
    \end{equation}
\end{proposition}

Czasami będziemy używać słów przed ich zdefiniowaniem, tak jak uczyniliśmy tutaj: węzły pierwsze i~supły pojawiają się odpowiednio w definicjach \ref{def:prime_knot}, \ref{def:tangle}.
Książkę trzeba więc przeczytać co~najmniej dwa razy.

\begin{proof}
\index[persons]{Conway, John}%
\index[persons]{Tutte, William}%
\index[persons]{Tait, Peter}%
    Dowód korzysta z metody Conwaya znajdowania splotów, rozwiązania hipotezy Taita oraz wyniku Tuttego dotyczącego liczby ukorzenionych $c$-sieci.
\end{proof}

\begin{proposition}
    Niech $a_n$ oznacza liczbę pierwszych, alternujących supłów o~$n$ skrzyżowaniach.
    Wtedy funkcja tworząca $f(z) = \sum_n a_n z^n$ spełnia równanie
    \begin{equation}
    f(1+z) - f(z)^2 - (1+f(z))q(f(z)) -z - \frac{2z^2}{1-z} = 0,
    \end{equation}
    gdzie $q(z)$ jest pomocniczą funkcją
    \begin{equation}
        q(z) = \frac{2z^2 - 10z - 1 + \sqrt{(1-4z)^3}} {2(z+2)^3} - \frac{2}{1+z} -z + 2.
    \end{equation}
\end{proposition}

Powyższa ciekawostka także pochodzi z cytowanej wcześniej pracy \cite{sundberg98}.

