\subsection{Spin} % (fold)

% DICTIONARY;writhe;spin
Z angielskiego \emph{writhe}.
\begin{definition}[spin]
    \index{spin}
    Niech $D$ będzie diagramem zorientowanego splotu.
    Wielkość
    \begin{equation}
        \writhe D = \sum_c \operatorname{sign} c,
    \end{equation}
    gdzie sumowanie przebiega po wszystkich skrzyżowaniach diagramu $D$ nazywamy spinem.
\end{definition}

Co ważne, spin nie jest niezmiennikiem splotów ani węzłów.
Para Perko przedstawia ten sam węzeł z~minimalną liczbą skrzyżowań i~spinem równym siedem lub dziewięć.
Dzięki temu przez wiele lat nie została dostrzeżona.
Spin jest za to niezmiennikiem węzłów alternujących, mówi o~tym druga hipoteza Taita.

\begin{lemma}
    \label{writhe_not_invariant}
    Spin nie zależy od orientacji.
    Tylko I ruch Reidemeistera zmienia spin:
\begin{comment}
    \begin{equation}
        \writhe (\MalyreidemeisterIa) = \writhe (\MalyreidemeisterIb) - 1.
    \end{equation}
\end{comment}
    Pozostałe ruchy nie mają na niego wpływu.
\end{lemma}

% Koniec sekcji Spin
