
\subsubsection{Dziewiętnaście skrzyżowań}
Artykuł \cite{thistlethwaite98} zawiera informację, że jego autorzy szukają węzłów o~17 skrzyżowaniach, ale my nie doszukaliśmy się żadnej późniejszej publikacji na ten temat.
\index[persons]{Hoste, Jim}%
\index[persons]{Thistlethwaite, Morwen}%
\index[persons]{Weeks, Jeff}%
W 2004 Flint, Rankin oraz Schermann \cite{rankin04} znaleźli alternujące węzły do 22 skrzyżowań (obliczenia na stacji roboczej z procesorem Xeon oraz 3 gigabajtami pamięci zajęły około 45 godzin), po czym długo nie działo się nic.
\index[persons]{Flint, Ortho}%
\index[persons]{Rankin, Stuart}%
\index[persons]{Schermann, John}%

Dopiero w 2020 Burton \cite{burton20} stablicował węzły pierwsze do 19 skrzyżowań: \emph{,,Here we extend the tables from 16 to 19 crossings, with a total of 352 152 252 distinct non-trivial prime knots.''}
\index[persons]{Burton, Benjamin}%

Thistlethwaite opublikował na swojej stronie internetowej, że znalazł węzły pierwsze do 20 skrzyżowań; czekamy na publikację w renomowanym czasopiśmie.
Jeżeli nie popełnił żadnego błędu, to mamy 199 631 989 alternujących i 1 647 687 439 niealternujących węzłów pierwszych o~dwudziestu skrzyżowaniach.
\index[persons]{Thistlethwaite, Morwen}%
% https://web.math.utk.edu/~morwen/k20v4.pdf
Wśród nich mamy 915 satelitów trójlistnika, 5 satelitów ósemki oraz węzeł (3,10)-torusowy; pozostałe węzły są hiperboliczne.

