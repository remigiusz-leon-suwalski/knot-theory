
\subsubsection{Jedenaście skrzyżowań}
Dopiero John Conway\footnote{TODO: biografia Conwaya...} w~latach sześćdziesiątych minionego wieku znalazł pierwsze węzły o~mniej niż 12 skrzyżowaniach oraz wszystkie sploty o~mniej niż 11 skrzyżowaniach w~oparciu o~pomysły Kirkmana.
\index[persons]{Conway, John}%
% An enumeration of knots and links, 1970.
Zajęło mu to jedynie kilka godzin!
Metoda Conwaya jest tak dobra, że używamy jej po dziś dzień, na przykład Tuzun, Sikora zweryfikowali dzięki niej hipotezę \ref{con:jones} do 24 skrzyżowań.
\index[persons]{Tuzun, Robert}%
\index[persons]{Sikora, Adam}%

Conway znalazł 1 duplikat oraz 11 pominięć w~starych tablicach Little'a, ale sam popełnił 4 pominięcia.
Przeoczył między innymi słynny duplikat w~niealternującej tablicy, parę Perko.
% 1974?
\index{para Perko}%
\index{spin}%
Przyczyną było prawdopodobnie to, że dwa diagramy miały różny spin:
% DICTIONARY;2-pass move;2-przejście;-
Little błędnie twierdził, że spin minimalnego diagramu jest niezmiennikiem, gdyż błędnie założył, że 2-przejścia oraz flype wystarczają do zmiany dowolnego minimalnego diagramu w~inny.
\index[persons]{Little, Charles}%

Naprawienie błędu tego błędu zajęło chwilę: pominęcia w~tablicy Conwaya znalazł Caudron około 1980 roku \cite{caudron1982}.
\index[persons]{Caudron, Alain}%
Rękopis \cite{siebenmann1979} (albo \cite{bonahon1989}?) Bonahona, Siebenmanna klasyfikuje węzły algebraiczne.
\index[persons]{Bonahon, Francis}%
\index[persons]{Siebenmann, Laurent}%
Z~nielicznymi niealgebraicznymi węzłami do 11 skrzyżowań poradził sobie Perko w \cite{perko1980} oraz \cite{perko1982}, co było kresem ery ręcznych obliczeń.
\index[persons]{Perko, Kenneth}%

% MAKOTO SAKUMA - A SURVEY OF THE IMPACT OF THURSTON’S WORK ON KNOT THEORY
% through hand calculation of homological invariants (in particular linking invariants) of finite branched coverings for those knots that are not covered by Bonahon and Siebenmann’s result described in Subsection 4.1. See [268] for an interesting historical note.

