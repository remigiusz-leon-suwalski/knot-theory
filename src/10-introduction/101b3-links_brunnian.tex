
\subsubsection{Sploty Brunna}
\index{splot!Brunna|(}%
Hermann Brunn \cite{brunn1892} rozpatrywał w 1892 roku (a więc zanim jeszcze teoria węzłów przyszła na świat!) nierozszczepialne sploty, które po usunięciu dowolnego ogniwa stają się niesplotami.
\index[persons]{Brunn, Hermann}%
W~czasopiśmie Delta, nr 01/2011, przeczytaliśmy, że Rolfsen zaproponował nazywać je splotami Brunna i~tak też będziemy robić.
Najprostszym splotem Brunna są posiadające trzy ogniwa pierścienie Boromeuszy ($6_2^3$ w notacji Alexandera-Briggsa, \texttt{L6a4} w notacji Thistlethwaite'a).
\index{pierścienie Boromeuszy}%
Pokażemy na stronie \pageref{boromean_not_splittable}, że pierścienie Boromeuszy nie mają nietrywialnych trójkolorowań, więc nie mogą być niesplotem.

Pierścienie Boromeuszy są alternujące, hiperboliczne i drzewiaste.
\index{węzeł!alternujący}%
\index{węzeł!hiperboliczny}%
\index{węzeł!drzewiasty}%
% DICTIONARY;arborescent;drzewiasty;węzeł
% TODO: jak jest arborescent po francusku?
Ich nazwa pochodzi od lombardzko-piemonckiego rodu kupieckiego, bankierskiego i arystokratycznego, z którego wywodziło się wielu kardynałów.
Herb tego rodu zawierał splecione ze sobą trzy okręgi.
Jest niemożliwe, by wykonać model przestrzenny tego splotu przy użyciu okrągłych pierścieni.
Zamiast tego można użyć na przykład elips.

Z dokładnością do homotopii sploty Brunna zostały sklasyfikowane przez Milnora \cite{milnor1954}, ale ponieważ ta książka nie tłumaczy, czym są $\mu$-niezmienniki Milnora, nie możemy dzisiaj wytłumaczyć, jak tego dokonał.
\index[persons]{Milnor, John}%
\index{splot!Brunna|)}%

