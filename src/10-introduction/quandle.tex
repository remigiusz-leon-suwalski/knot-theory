\section{Kwandle i wraki} % (fold)
% TODO: Problems  on invariants of knots and 3-manifolds, rozdział 3
\label{sec:quandle}
Sekcja ta powstała częściowo w~oparciu o~notatki autorstwa Andrew Bergera, Chrisa Geriga\footnote{dostępne pod adresem \url{https://math.berkeley.edu/~cgerig/notes}} oraz Andrew Bergera, Brandona Flannery'ego i~Chrisa Sumnichta\footnote{dostępne pod adresem \url{https://github.com/thyrgle/191_Final_Project/blob/master/paper.pdf}}.
Kwandle, z~angielskiego \emph{quandle}, są strukturami algebraicznymi przypominającymi grupy.
Aksjomaty grupy stanowią uogólnienie symetrii -- symetrie są odwracalne, można je składać, identyczność jest symetrią.
Aksjomaty kwandli będą odzwierciedlać ruchy Reidemeistera.

Niech $X$ będzie skończonym zbiorów, zaś $K$ węzłem.
Elementy $x \in X$ będą dla nas kolorami, którymi oznaczymy długie łuki na diagramie węzła $K$.
Gdy trzy kolory spotykają się przy jednym skrzyżowaniu, definiujemy funkcję $\triangleright \colon X \times X \to X$, jak na rysunku.
To znaczy: kiedy łuk o kolorze $x$ przechodzi pod łukiem koloru $y$, staje się łukiem w kolorze $x \triangleright y$.

\[
    \begin{tikzpicture}[scale=0.18, baseline=0]
        \path[TIKZ_ARCH,Latex-] (-4,0) -- (4,8);
        \path[TIKZ_ARCH] (4,0) -- (1,3);
        \path[TIKZ_ARCH] (-1,5) -- (-4,8);
        \node[darkblue] at (-4, 0)[above left] {$y$};
        \node[darkblue] at (4, 0)[above right] {$x \triangleright y$};
        \node[darkblue] at (-4, 8)[below left] {$x$};
    \end{tikzpicture}
\]

Teraz możemy przetłumaczyć ruchy Reidemeistera w aksjomaty kwandli.

\begin{definition}[kwandl]
    \index{quandle}
    Zbiór $X$ wyposażony w dwuargumentowe działanie $\triangleright$ taki, że dla wszystkich elementów $x, y, z \in X$ zachodzi:
    \begin{enumerate}
        \item $x \triangleright x = x$,
        \item odwzorowanie $\beta_y \colon X \to X$ dane wzorem $\beta_y(x) = x \triangleright y$ jest odwracalne,
        \item $(x \triangleright y) \triangleright z = (x \triangleright z) \triangleright (y \triangleright z)$
    \end{enumerate}
    nazywamy kwandlem.
\end{definition}

Ta definicja pochodzi z~nieopublikowanej korespondencji między Johnem Conwayem i~Gavenem Wraithem, którzy w 1959 byli studentami I stopnia na uniwersytecie w Cambridge.
Ponownie odkryto ją w latach 80. XX wieku: Joyce w 1982 po raz pierwszy nazwał te obiekty kwandlami, Matwiejow w tym samym roku jako grupoidy rozdzielne, Brieskorn w 1986 jako zbiory automorficzne.

Drugi aksjomat nazywa się czasem odwracalnością z prawej strony: znając $x \triangleright y$ oraz $y$ możemy odtworzyć element $x$, jednak znając $x$ być może nie jesteśmy w stanie odtworzyć elementu $y$.
Jedyny element $x$ taki, że $x \triangleright y = z$ nazwijmy $y \triangleleft z$.
To pozwala podać trochę inną definicję kwandli, my nie będziemy jej używać.

\begin{definition}
    Zbiór $X$ z dwuargumentowymi działaniami $\triangleright, \triangleleft$ taki, że dla wszystkich $x, y, z \in X$ zachodzi:
    \begin{align*}
    x \triangleright x = x \triangleleft x & = x \\
    (x \triangleleft y) \triangleright x & = y \\
    x \triangleleft (y \triangleright x) & = y \\
     (x \triangleright z) \triangleright (y \triangleright z) & = (x \triangleright y) \triangleright z \\
    (x \triangleleft y) \triangleleft (x \triangleleft z) & = x \triangleleft (y \triangleright z)
    \end{align*}
    nazywamy kwandlem.
\end{definition}

Homomorfizmy definiujemy standardowo, przez analogię do grup:

\begin{definition}
    Niech $Q_1, Q_2$ będą kwandlami.
    Odwzorowanie $f \colon Q_1 \to Q_2$, które dla waszystkich $x,y \in Q_1$ spełnia warunek $f(x \triangleright y) = f(x) \triangleright f(y)$, nazywamy homomorfizmem.
\end{definition}

Wiele znanych struktur algebraicznych okazuje się być źródłem kwandli.

\begin{example}[kwandl cykliczny/diedralny]
    Grupa abelowa z działaniem $x \triangleright y = 2y - x$.
\end{example}

\begin{example}[kwandle sprzężone]
    Grupa z działaniem $x \triangleright y = y^{-n} x y^n$.
\end{example}

\begin{example}[kwandl Alexandera]
    Moduł nad pierścieniem $\Z[t, 1/t]$ wielomianów Laurenta z~działaniem $x \triangleright y =tx + (1-t) y$.
\end{example}

\begin{example}[kwandl symplektyczny]
    Przestrzeń liniowa i antysymetryczna forma dwuliniowa $\langle \cdot | \cdot \rangle$ z działaniem $x \triangleright y = x + \langle x | y \rangle y$.
\end{example}

D. Joyce w swojej rozprawie doktorskiej przypisał każdemu węzłowi $K$ pewien szczególny kwandl $Q(K)$, kwandl podstawowy.
Definicja tego obiektu jest dość zawiła: łuki diagramu są generatorami, zaś skrzyżowania odpowiadają za relacje.
Joyce pokazał, że kwandl $Q(K)$ wyznacza węzeł $K$ jednoznacznie z dokładnością do orientacji.
Nie czyni to jednak nowego niezmiennika użytecznym, gdyż wyznaczenie go nawet w najprostszych przypadkach stanowi trudność.
Niebrzydowski, Przytycki pokazali w 2008 roku, że kwandl podstawowy trójlistnika jest izomorficzny z~rzutowym pierwotnym podkwandlem pewnych odwzorowań liniowych przestrzeni symplektycznej $\Z \oplus \Z$, cokolwiek to znaczy.

Niech $X$ będzie ustalonym kwandlem.
Liczba homomorfizmów $Q(K) \to X$ stanowi niezmiennik węzłów, zwany czasem niezmiennikiem zliczającym.

Aksjomaty grupy można wzmacniać (grupy abelowe) lub osłabiać (monoidy).
Podobnie czyni się z aksjomatami kwandli.
Kwandle inwolutywne odpowiadają węzłom bez orientacji, wraki dobrze opisują węzły obramowane (\emph{framed}), i tak dalej.

\begin{definition}[kwandl inwolutywny]
    Kwandl $Q$, w którym dla wszystkich $x, y \in Q$ zachodzi $x \triangleleft (x \triangleleft y) = y$, nazywamy inwolutywnym (albo kei).
\end{definition}

Kwandle inwolutywne badał jako pierwszy Mituhisa Takasaki (1943).
Szukał niełącznej struktury, która dobrze opisywałaby odbicia w skończonej geometrii.

\begin{definition}[półka]
    \index{shelf}
    Zbiór $X$ z dwuargumentowym działaniem $\triangleright$ taki, że dla wszystkich elementów $x, y, z \in X$ zachodzi $(x \triangleright y) \triangleright z = (x \triangleright z) \triangleright (y \triangleright z)$ nazywamy półką (z angielskiego \emph{shelf}).
\end{definition}

Półkę można uogólnić na dwa sposoby:

\begin{definition}[wrak]
    \index{wrack}
    Zbiór $X$ z dwuargumentowym działaniem $\triangleright$ taki, że dla wszystkich elementów $x, y, z \in X$ zachodzi:
    \begin{enumerate}
        \item odwzorowanie $\beta_y \colon X \to X$ dane wzorem $\beta_y(x) = x \triangleright y$ jest odwracalne,
        \item $(x \triangleright y) \triangleright z = (x \triangleright z) \triangleright (y \triangleright z)$
    \end{enumerate}
    nazywamy wrakiem (z angielskiego \emph{wrack}).
\end{definition}

\begin{definition}[wrzeciono]
    \index{spindle}
    Zbiór $X$ z dwuargumentowym działaniem $\triangleright$ taki, że dla wszystkich elementów $x, y, z \in X$ zachodzi:
    \begin{enumerate}
        \item $x \triangleright x = x$,
        \item $(x \triangleright y) \triangleright z = (x \triangleright z) \triangleright (y \triangleright z)$
    \end{enumerate}
    nazywamy wrzecionem (z angielskiego \emph{spindle}).
\end{definition}

Zatem kwandle to wraki, które są też wrzecionami.
Muszę w~tym miejscu wtrącić uwagę językową.
Conway nazwał wraki wrakami (\emph{wracks}), by częściowo zażartować z~nazwiska jego kolegi Gavina Wraitha, a częściowo by zaznaczyć, że są one tym, co zostaje z~grupy, w~której zapomniano o~mnożeniu, ale nie sprzęganiu (w~języku angielskim co najmniej od XVI wieku funkcjonuje zwrot ,,wrack and ruin'' oznaczający zniszczenie).
Obecnie dominuje określenie \emph{racks}.

%Pokażemy, że kwandle uogólniają kolorowania.
%Niech $X$ będzie zbiorem kolorów z~operacją $\triangleright$, które spełnia aksjomaty z~definicji kwandli.
%Wtedy przy każdym skrzyżowaniu występują trzy kolory: $x$, $y$ oraz $x \triangleright y$.

%Przypomnijmy, że 3-kolorowanie diagramu polegało na przypisaniu każdemu włóknu pewnego koloru (z trzech) tak, by każdy został użyty, a~żadne skrzyżowanie nie stało się dwubarwne.
%Ogólniej, jeśli kolorami były liczby $0, \ldots, n - 1$, żądaliśmy od skrzyżowań, by kolor $y$ po przejściu pod kolorem $x$ stawał się $z$, gdzie $z \equiv 2x - y$ modulo $n$.
%Można to uogólnić jeszcze bardziej, właśnie do quandli: $\Z/n$-kolorowanie węzła to quandle związany z~pierścieniem $\Z/n$ operacją $x \triangleright y =  2y - x$}

% Koniec sekcji Wraki i~kwandle