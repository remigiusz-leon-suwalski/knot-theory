\documentclass{LegrandOrangeBook}

\makeindex % Tells LaTeX to create the files required for indexing
\makeindex[title=Skorowidz]
\makeindex[name=persons,title=Indeks osób]
\makeindex[name=knotslinks,title=Wykaz małych węzłów i splotów]

\begin{document}

\input{include/head_1}
\input{include/head_2}
\input{include/head_3}

% strona czwarta

\thispagestyle{empty}
\begin{figure}[H]
\begin{minipage}[b]{.48\linewidth}
{\noindent Prof. Casimir Allard\\
Université Bordeaux I\\
351 Cours de la Libération\\
33400 Talence, Francja}
\end{minipage}
\begin{minipage}[b]{.48\linewidth}
{\noindent Adélaïde Gauthier\\
École polytechnique\\
Route de Saclay\\
91128 Palaiseau, Francja}
\end{minipage}
\end{figure}

{\noindent \textbf{Kategorie MSC 2020}\\57K10 (niskowymiarowa teoria węzłów),\\57K30 (topologia 3-rozmaitości)} \vspace{5mm}

{\noindent \textbf{Tytuł oryginału}\\La théorie combinatoire des næuds}
\vspace{5mm}

{\noindent \textbf{Z francuskiego przetłumaczyła}\\Charlotte Ocelot} 
\vspace{5mm}

{\noindent \textbf{Okładkę zaprojektował}\\Raphaël Embusqueur}
\vspace{5mm}

{\noindent \textbf{Zredagował}\\Gniewomir Grzywiasty}
\vspace{5mm}

{\noindent \textbf{Zredagowała technicznie}\\Jagoda Kłos}
\vspace{5mm}

{\noindent \textbf{Złożyli i połamali}\\Porte de Versailles, Paryż}
\vspace{5mm}

{\noindent \textbf{Korekty dokonali}\\Ignacy Filemon, Gabriel Bonifacy}

\vfill

{\noindent Copyleft © 2024 by Antykwariat Czarnoksięski.
Książka, a żeby było śmieszniej także każda jej część, mogą być przedrukowywane oraz w jakikolwiek inny sposób reprodukowane czy powielane mechanicznie, fotooptycznie, zapisywane elektronicznie lub magnetycznie, oraz odczytywane w środkach publicznego przekazu bez pisemnej zgody wydawcy.
}

\vspace{5mm}
{
    \noindent
    Tekst udostępniany na licencji Creative Commons: uznanie autorstwa, użycie niekomercyjne. Przeczytaj więcej na \url{https://creativecommons.org/licenses/by-nc/4.0/deed.pl}.
}

\vspace{5mm}

{\noindent Przygotowano w systemie \TeX, wydrukowano na siarczystym papierze.}

% koniec strony czwartej



% strona piąta

\chapter*{Przedmowa}
Książka wprowadza we współczesną teorię węzłów, najważniejsze metody i obszary tej wciąż niedocenianej, choć żywo rozwijającej się dyscypliny matematycznej.
Tor wykładu wzorowany jest na seminarium z teorii węzłów jakie odbyło się na Wydziale Matematyki Uniwersytetu Wrocławskiego w semestrze zimowym 2013/2014 i dlatego podręcznik nadaje się szczególnie do pierwszego czytania dla studentów wyższych lat studiów matematycznych, natomiast dla młodych pracowników naukowych może okazać się pożytecznym źródłem odsyłaczy do prac, które zainspirują do dalszych badań.
Zwięźle i na tyle, na ile było to możliwe opisuję rozmaite narzędzia stosowane do badania węzłów, splotów, supłów i innych: klasyczne niezmienniki numeryczne i ruchy Reidemeistera, niezmienniki kolorujące i spokrewnione z nimi kwandle, potem diagramatyczny wielomian Jonesa, homologiczny wielomian Alexandera oraz dość świeże niezmienniki typu skończonego.
Ze względu na niepełne zrozumienie maszynerii topologi algebraicznej, rozdział czwarty został napisany trochę niechlujnie, jednakże żywię nadzieję poprawić się w następnym wydaniu trzymanej przez Ciebie książki.
W ostatnim, piątym rozdziale przedstawiam krótko charakterystyczne rodziny: warkocze, dwumostowe sploty, mutanty, precle, sploty torusowe, satelitarne i hiperboliczne, wreszcie węzły plastrowe i taśmowe, obiekt zainteresowań czterowymiarowej topologii.
Dołączam również tablice węzłów pierwszych o małej liczbie skrzyżowań.
Z takim zakresem i ujęciem materiału pozycja jest unikalna nie tylko we francuskiej, ale i w światowej literaturze matematycznej.

Serdecznie dziękuję Ś. G. oraz J. Ś. bez których ten tekst nigdy by nie powstał.

\begin{flushright}
Casimir Allard,\\Marsylia, 4 lutego 2019
\end{flushright}

\newpage
\section*{Przedmowa do wydania drugiego}
Od poprzedniego wydania minął ponad rok i~trochę się w~tym czasie zmieniło.

Dodałem informację o~różnych notacjach dla węzłów, poprawiłem informację na temat płci niektórych osób, napisałem nową sekcję o~macierzy Seiferta, wymieniłem niezmienniki nieodróżniające mutantów.
Pojawiła się klasyfikacja kwandli, nierówność Mortona, homologia Floera i kilka mniejszych matematycznych obiektów.
Niektóre istniejące sekcje przeniosłem w~lepsze mam nadzieję miejsca, niektóre trudne dowody otrzymały swój zarys.
Tam, gdzie to możliwe, podałem rys historyczny, na przykład w~sekcji o~kolorowaniach wspomniałem o~Foxie, który chciał uczynić teorię węzłów prostszą dla swoich studentów.
Na końcu książki umieściłem krótki słowniczek angielsko-polski.
Wreszcie usunąłem znalezione literówki i~uzupełniłem luki, zarówno w~bibliografii jak i~rysunkach.
Poprawiłem drobnie typografię, dla przykładu od teraz wszystkie równania powinny być numerowane.

Żywię nadzieję, że korzystanie z~książki będzie równie (jeśli nie bardziej) przyjemne, co w~przypadku pierwszego jej wydania.

\begin{flushright}
Casimir Allard,\\Marsylia, 19 sierpnia 2020
\end{flushright}

\section*{Przedmowa do wydania trzeciego}
Lata mijały, wiele wody upłynęło w Loarze, a~wydania trzeciego wciąż nie było widać na horyzoncie.
Głównym winowajcą jest tutaj Amazon, który z~niejasnych dla mnie powodów przestał wspierać język francuski.
Pomimo przeciwności losu nie zapomniałem jednak o miłośnikach teorii węzłów i~nie przestawałem pracować w domowym zaciszu nad książką.
Nieznacznie rozrosła się ze 114 stron w 2020 roku do 1XX stron teraz, kiedy oddaję ją do druku.
Cytowań już wcześniej było bez liku (dokładniej: 240), a~teraz jest ich jeszcze więcej (aż 437).
Autor był sam jak paluch, a teraz jest nas dwoje!
% TODO: to jest licząc ostatnią stronę 5. rozdziału
Staraliśmy się nie zmieniać znacząco układu, w~myśl dewizy, że lepsze jest wrogiem dobrego.
Zupełnych nowości jest tyle, co kot napłakał.
W rozdziale drugim na osobną sekcję zasłużyły sobie etykietowania.
Natomiast w całej książce długie bloki tekstu zostały podzielone na sekcji i podsekcje tak, że odnajdywanie potrzebnych informacji w pośpiechu powinno sprawiać mniej trudu.
Z czystej przyzwoitości wymieniamy też, czego nie ma (i raczej szybko nie będzie):
\begin{enumerate}
    \item \textbf{chirurgii Dehna}.
    \item \textbf{niezmienników Reszetichina-Turajewa}.
    (Wspominamy o nich raz, za definicją \ref{def:jones_polynomial}.)
    \item \textbf{rachunku Kirby'ego}.
    Każdą zamkniętą, orientowalną 3-rozmaitość można otrzymać wykonując chirurgię Dehna na sferze wzdłuż pewnego splotu. Robion Kirby udowodnił około 1978 roku, że dwie takie rozmaitości są homeomorficzne wtedy i tylko wtedy, gdy między odpowiadającymi im splotami można przejść ruchami Kirby'ego.
    (Porównaj to ze sformułowaniem twierdzenia Reidemeistera).
    % Kirby, Robion (1978). "A calculus for framed links in S3". Inventiones Mathematicae. 45 (1): 35–56. Bibcode:1978InMat..45...35K. doi:10.1007/BF01406222. MR 0467753. S2CID 120770295. ?
    % TODO: Yanga-Baxtera?
\end{enumerate}
I to by było na tyle.
\begin{flushright}
    Casimir Allard i Adélaïde Gauthier,\\Bordeaux, \today
\end{flushright}

\section*{Przedmowa do wydania czwartego?}

Wydanie czwarte prawdopodobnie nigdy nie powstanie.
Dlaczego nam nie wyszło? 
Myślimy, że powodów da się wymienić wiele.
Opuściliśmy środowisko akademickie, zapał już teraz jest trochę słomiany i wypala się z każdym nowym pracodawcą szybciej i szybciej.
My nie do końca sprawdziliśmy się jako matematycy, a was nie można nazwać wzorowymi czytelnikami -- ale było warto!
Stworzyliśmy kilka cudownych akapitów godynch zapamiętania.
I nie wiem jak wy, ale my się naprawdę dobrze bawiliśmy.
Dziękujemy wszystkim, którzy zgłosili choć jedną uwagę i ożywili jakoś treść, ale w szczególności wdzięczni jesteśmy A. W. W., która nigdy tego nie przeczyta i której nikt tu nie zna, ale bez której nie powstałaby tak dziwna publikacja.

\begin{flushright}
Casimir Allard i Adélaïde Gauthier,\\gdzieś, kiedyś
\end{flushright}

% koniec strony piątej


\tableofcontents

\pagestyle{fancy} % Enable default headers and footers again
\cleardoublepage % Start the following content on a new page

\chapterimage{orange2.jpg} % Chapter heading image
\chapterspaceabove{5.75cm} % Whitespace from the top of the page to the chapter title on chapter pages
\chapterspacebelow{7.25cm} % Amount of vertical whitespace from the top margin to the start of the text on chapter pages

\part{Preludium}
\chapter{Preludium}
\input{10-introduction/100-intro}

\section{Węzły i~sploty}
Wprowadzamy pojęcia węzła i splotu, fundamentalnych obiektów teorii, o której napisana została ta książka.
Oprócz tego podajemy definicję, kiedy dwa węzły lub sploty uznajemy za tożsame oraz uzasadniamy, dlaczego akurat ta definicja jest właściwa.

Istnieją odnogi teorii węzłów, które badają inne, pokrewne obiekty.
Mamy na przykład węzły obramowane:

% DICTIONARY;framed;obramowany;węzeł
% GLOSSAIRE;encadré;obramowany;węzeł
% DICTIONARY;framing;obramowanie;-
% GLOSSAIRE;encadrement;obramowanie;-
\begin{definition}[obramowanie]
\index{obramowanie|see {węzeł obramowany}}%
\index{węzeł!obramowany}%
    Każde nieznikające normalne pole wektorowe $V$ na splocie $L$ nazywamy jego obramowaniem.
    Szczególnie interesujący jest przypadek, gdy wszystkie wektory tego pola są równoległe do płaszczyzny, na której leży diagram tego splotu.
\end{definition}

% DICTIONARY;virtual;wirtualny;węzeł
% GLOSSAIRE;virtuel;wirtualny;węzeł
% DICTIONARY;welded;zespawany;węzeł
% GLOSSAIRE;soudé;zespawany;węzeł
% DICTIONARY;long;długi;węzeł
% GLOSSAIRE;long;długi;węzeł
Są jeszcze węzły wirtualne, zespawane (iloraz węzłów wirtualnych przez ruch znany jako ,,nadskrzyżowania komutują''), długie (gdzie końce nie są ze sobą zszyte, ale umieszczone tak daleko, że są nieosiągalne) i inne.
% http://katlas.math.toronto.edu/ester/weldedknots/explanations.html => because the "overcrossings commute" move is not symmetric
\index{węzeł!wirtualny}%
\index{węzeł!zespawany}%
\index{węzeł!długi}%
Ta książka nie zawiera zbyt wiele informacji o wspomnianych bytach.


\subsection{Węzły}
Matematyczne węzły można traktować jako model elastycznej oraz pozbawionej grubości liny, której luźne końce zostały ze sobą połączone.
Sugeruje to przyjęcie naiwnej definicji:

\begin{definition}[węzeł (prawie)]
\index{węzeł}%
    Ciągłe oraz różnowartościowe odwzorowanie $S^1 \to \R^3$ nazywamy węzłem.
\end{definition}

Takie rozwiązanie nie jest doskonałe, ponieważ oprócz pożądanych (cokolwiek to znaczy) węzłów, obejmuje wiele innych, patologicznych obiektów takich jak ten z~rysunku \ref{fig_wild_knot}.
Milnor \cite{milnor1964} udowodnił, że ,,większość'' węzłów jest dzika.

\begin{figure}[H]
    \centering
\begin{comment}
    \includegraphics[height=0.14\linewidth]{wild_knot.png}
\end{comment}
    \caption[caption-wild-knot]{Węzeł dziki, źródło: Wikimedia{\footnotemark}}
\index{węzeł!dziki}%
\label{fig_wild_knot}%
\end{figure}
\footnotetext{\url{https://upload.wikimedia.org/wikipedia/commons/2/2f/Wildknot.svg}}

Zamiast wyjaśnić, jakie są jego niepożądane właściwości, podamy od razu dobrą definicję.

\begin{definition}[węzeł]
    Różnowartościowe włożenie $S^1 \to \R^3$, którego pochodna istnieje wszędzie i~nie znika nigdzie, nazywamy węzłem.
\end{definition}

\begin{example}[niewęzeł]
    Węzeł zadany w przestrzeni $\R^3$ parametrycznie $(\sin \theta, \cos \theta, 0)$ dla $\theta \in [0, 2\pi]$ nazywamy niewęzłem i oznaczamy $\SmallUnknot$.
\end{example}

Potrzeba jeszcze matematycznego opisu manipulacji, jakim możemy poddawać sznur trzymany w~ręce.
Izotopia jest niewłaściwym narzędziem do tego celu: powiedzielibyśmy, że dwa węzły $K_1, K_2$ są izotopijne, jeśli istnieje ciągła funkcja
\begin{equation}
    F \colon S^1 \times [0, 1] \to \R^3
\end{equation}
taka, że $K_1 = F(-, 0)$ jest pierwszym, zaś $K_2 = F(-,1)$ drugim węzłem (funkcję $F$ nazywa się izotopią).
Tym razem źródło problemów można wskazać jawnie.
Dowolny zaplątany fragment z węzła można usunąć wykonując sztuczkę Alexandera (ponieważ jak powiedziałyby mądre głowy, ,,przestrzeń homeomorfizmów dysku w siebie $D^{n+1} \to D^{n+1}$, które zgadzają się z odwzorowaniem tożsamościowym na brzegu dysku -- sferze $S^n$, jest spójna''):
\index[persons]{Alexander, James}%
\index{sztuczka Alexandera}%

\begin{comment}
\begin{figure}[H]
    \centering
    \fbox{\begin{minipage}[b]{.12\linewidth}
        \centering
        \includegraphics[width=\linewidth]{../data/alexander-trick/0.png}
        \subcaption{$t = 0$}
    \end{minipage}}\,\,
    \fbox{\begin{minipage}[b]{.12\linewidth}
        \centering
        \includegraphics[width=\linewidth]{../data/alexander-trick/1.png}
        \subcaption{$t = 1/4$}
    \end{minipage}}\,\,
    \fbox{\begin{minipage}[b]{.12\linewidth}
        \centering
        \includegraphics[width=\linewidth]{../data/alexander-trick/2.png}
        \subcaption{$t=1/2$}
    \end{minipage}}\,\,
    \fbox{\begin{minipage}[b]{.12\linewidth}
        \centering
        \includegraphics[width=\linewidth]{../data/alexander-trick/3.png}
        \subcaption{$t=3/4$}
    \end{minipage}}\,\,
    \fbox{\begin{minipage}[b]{.12\linewidth}
        \centering
        \includegraphics[width=\linewidth]{../data/alexander-trick/4.png}
        \subcaption{$t = 1$}
    \end{minipage}}
    \caption[caption-alexander-trick]{Sztuczka Alexandera \cite[s. 2]{burde2014}}
\end{figure}
\end{comment}

W podobny sposób moglibyśmy przekształcić dowolny węzeł w~niewęzeł.
Teoria, w~której wszystkie obiekty są takie same, nie jest zbyt ciekawa.
Od izotopii należy wymagać gładkości albo lokalnej płaskości,
% https://math.stackexchange.com/questions/1311865/equivalence-of-knots-ambient-isotopy-vs-homeomorphism
co zdaje się prowadzić do pojęcia izotopii otaczającej, która uwzględnia nie tylko sam węzły, ale też to, jak leżą w otaczającej je przestrzeni.

% DICTIONARY;isotopy;izotopia;-
% GLOSSAIRE;isotopie;izotopia;-
% DICTIONARY;ambient;otaczająca;izotopia
% GLOSSAIRE;ambiante;otaczająca;izotopia
\begin{definition}[izotopia otaczająca]
    \index{izotopia otaczająca}%
    Niech $K_1, K_2 \colon N \hookrightarrow M$ będą włożeniami dwóch rozmaitości $N, M$.
    Ciągłe odwzorowanie $F \colon M \times [0,1] \to M$ spełniające następujące warunki:
    \begin{enumerate}
        \item funkcja $F(-, 0)$ jest odwzorowaniem tożsamościowym,
        \item każda z~funkcji $F(-, t)$ jest homeomorfizmem,
        \item złożenie $F(-, 1)$ z~pierwszym włożeniem $K_1$ daje drugie włożenie $K_2$
    \end{enumerate}
    nazywamy izotopią otaczającą przenoszącą włożenie $K_1$ na $K_2$.
\end{definition}

W~topologii rozważa się włożenia dowolnych rozmaitości, nam wystarczy jeden szczególny przypadek $N = S^1$ oraz $M = \R^3$.
Intuicyjnie, funkcja $F$ zniekształca przestrzeń $\R^3$ tak, że w~chwili początkowej $t = 0$ widzimy pierwszy, zaś w~chwili końcowej $t = 1$ drugi węzeł.
Izotopia otaczająca nie pozwala na ściąganie zaplątanych fragmentów do punktu.

\begin{definition}
    Dwa węzły są równoważne wtedy i tylko wtedy, gdy istnieje pomiędzy nimi izotopia otaczająca.
\end{definition}

Znając już izotopię otaczającą, można podać alternatywny opis węzłów:

% DICTIONARY;knot;węzeł;-
% GLOSSAIRE;nœud;węzeł;-
% DICTIONARY;tame;poskromiony;węzeł
% GLOSSAIRE;lisse/régulier;poskromiony;węzeł
% DICTIONARY;wild;dziki;węzeł
% GLOSSAIRE;sauvage;dziki;węzeł
\begin{definition}[węzeł]
\index{węzeł!poskromiony}%
\label{def:knot}%
    Gładkie włożenie $S^1 \hookrightarrow \R^3$ otaczająco izotopijne z~zamkniętą łamaną bez samoprzecięć nazywamy węzłem poskromionym.
\end{definition}

Czasami wygodniej jest rozpatrywać węzeł jako włożenie $S^1 \hookrightarrow S^3$ albo dopuścić do myśli węzły nieposkromione.
Ale jeśli nie zaznaczono inaczej, nie robimy tego: pisząc węzeł mamy na myśli poskromione włożenie w przestrzeń $\R^3$, nie $S^3$.

Formalnie węzły to pewne odwzorowania, więc prawidłowym sposobem na zapisanie, że są izotopijne (czyli dla nas: równe), jest $K_1 \cong K_2$.
Ponieważ nie prowadzi to do problemów, będziemy jednak stosować zapis $K_1 = K_2$.
Jednocześnie często węzeł jako odwzorowanie nie będzie odróżniany od obrazu tego odwzorowania.

Istnieje jeszcze jedna, konkurencyjna definicja węzłów równoważnych:

\begin{proposition}
\label{def:equivalent_knots_2}%
    Dwa węzły są równoważne wtedy i~tylko wtedy, gdy jeden z~nich jest obrazem drugiego przez zachowujący orientację homeomorfizm $\R^3 \to \R^3$.
\end{proposition}

\begin{proof}
    Podany niżej dowód pochodzi z~książki ,,Topology from the differentiable viewpoint'' Johna Milnora.
\index[persons]{Milnor, John}%
    Musimy pokazać, że dyfeomorfizm $f \colon \R^m \to \R^m$ jest gładko izotopijny z~identycznością.
    Translacje są izotopiami, więc bez straty ogólności zakładamy, że $f(0) = 0$.
    Pochodna $f$ w~zerze jest dana wzorem $\mathrm{d}f_0(x) = \lim_{t \to 0} f(tx) /t$, więc
    \begin{equation}
        F(x, t) = \begin{cases}
            \mathrm{d}f_0(x) & t = 0 \\
            f(tx) / t & 0 < t \le 1
        \end{cases} .
    \end{equation}
    stanowi naturalną definicję izotopii $F \colon \R^m \times [0, 1] \to \R^m$.
    Funkcja $f$ zapisuje się na mocy lematu Hadamarda jako suma $x_1 g_1(x) + \ldots + x_mg_m(x)$, gdzie funkcje $g_i$ są gładkie, więc funkcja $F$ też jest gładka, co jakoś kończy dowód.
\index{lemat Hadamarda}%    
\end{proof}

Milnor zauważa, że istnieje dyfeomorfizm $S^6 \to S^6$ stopnia $+1$, który nie jest gładko izotopijny z~identycznością!
\index[persons]{Milnor, John}%

\begin{remark}[John Willard Milnor]
    Matematyk amerykański urodzon w 1931 roku w Orange, New Jersey.
    Odkrył egzotyczną 7-wymiarową sferę (czyli zwykłą sferę z niezwykłą strukturą różniczkową) w 1956 roku, za co został później odznaczon medalem Fieldsa.
    Obalił Hauptvermutung pięć lat później: hipotezę Steinitza i Tietzego z 1908 roku, że każde dwie triangulacje przestrzeni mają kombinatorycznie równoważne podpodziały.
    Do jego zainteresowań należą topologia różniczkowa, algebraiczna K-teorią, ale też algebry Hopfa, grupy Liego i holomorficzne układy dynamiczne.
    W~świecie węzłów jest znany przez wprowadzenie niezmienników $\mu$ Milnora, które uogólniają grupę podstawową dopełnienia oraz pewne wyniki dotyczące hipotezy plastrowo-taśmowej.
\end{remark}
\index{niezmiennik!$\mu$ Milnora}%
\index{hipoteza!plastrowo-taśmowa}%
\index[persons]{Milnor, John}%




\subsection{Sploty}
% DICTIONARY;link;splot;-
% GLOSSAIRE;un entrelacs;splot;-
% DICTIONARY;component;ogniwo splotu;-
% GLOSSAIRE;composante/boucle;ogniwo splotu;-
\begin{definition}[splot, ogniwo]
\index{splot}%
    Sumę parami rozłącznych węzłów
    \begin{equation}
        L = K_1 \sqcup K_2 \sqcup \ldots \sqcup K_n
    \end{equation}
    nazywamy splotem, a~składniki $K_i$ -- ogniwami splotu.
\end{definition}

Przez analogię do węzłów mówimy, że dwa sploty są takie same, jeśli jeden jest obrazem drugiego przez zachowujący orientację homeomorfizm $\R^3 \to \R^3$.
To oczywiste, że liczba ogniw jest niezmiennikiem splotów.
Później podamy mniej oczywiste niezmienniki.

\begin{example}[niesplot]
\index{niesplot}%
    Splot zadany w przestrzeni $\R^3$ parametrycznie
    \begin{equation}
        \bigcup_{k=1}^n \{(\sin \theta, \cos \theta, k) : \theta \in [0, 2\pi]\}
    \end{equation}
    nazywamy niesplotem i oznaczamy $U_n$.
\end{example}
    
\begin{example}
\index{splot!Hopfa}%
\index[persons]{Hopf, Heinz}%
    Splot Hopfa (rys. \ref{small_links_diagram}a), najprostszy nietrywialny splot. Ma dwa ogniwa.
\end{example}

\begin{remark}[Heinz Hopf]
    Matematyk niemiecki urodzon w 1894 roku w Gräbschen (obecnie część Wrocławia); zmarł w 1971 roku w Zurychu, Szwajcarii.
    Głównym wynikiem jego pracy doktorskiej z~1925 roku było, że każda jednospójna zupełna 3-rozmaitość Riemanna o~stałej krzywiźnie sekcyjnej jest globalnie izometryczna do przestrzeni euklidesowej, sferycznej lub hiperbolicznej.
    Był pionierem topologii algebraicznej.
    W~1931 roku prowadził badania nad tzw. rozwłóknieniem: odwzorowaniem $S^3 \to S^2$ takim, że przeciwobrazy punktów są okręgami wielkimi na 3-sferze.
    Podczas tych badań zajmował się splotem nazywanym teraz splotem Hopfa.
\end{remark}

\begin{example}
\index{splot!Whiteheada}%
\index[persons]{Whitehead, John}%
    Splot Whiteheada (rys. \ref{small_links_diagram}b).
\end{example}

\begin{remark}[John Henry Constantine Whitehead]
    Matematyk brytyjski urodzon w 1903 roku w~Madrasie, Indiach; zmarł w 1960 roku w Princeton, New Jersey.
    Był jednym z założycieli teorii homotopii, podał definicję CW-kompleksów.
\index{CW kompleks}%
    Próbował udowodnić hipotezę Poincarégo, ale popełnił błąd twierdząc, że nie istnieje ściągalna otwarta 3-rozmaitość, która nie jest homeomorficzna z $\R^3$.
\index{hipoteza!Poincarégo}%
\index{rozmaitość!ściągalna}%
    W 1935 roku sam wskazał taką rozmaitość, do jej konstrukcji wykorzystując splot Whiteheada.
\end{remark}

\begin{comment}
    {\setlength{\intextsep}{4pt plus 2pt minus 2pt}
    \begin{figure}[H]
        \centering
        \begin{minipage}[b]{.3\linewidth}
            \centering
            \includegraphics[height=0.6\linewidth]{../data/links/2_2_1.png}
            \subcaption{splot Hopfa}
        \end{minipage}\,\,
        \begin{minipage}[b]{.3\linewidth}
            \centering
            \includegraphics[height=0.6\linewidth]{../data/links/5_2_1.png}
            \subcaption{splot Whiteheada}
        \end{minipage}\,\,
        \begin{minipage}[b]{.3\linewidth}
            \centering
            \includegraphics[height=0.6\linewidth]{../data/links/6_3_2.png}
            \subcaption{pierścienie Boromeuszy}
            \index{pierścienie Boromeuszy}%
        \end{minipage}
        \caption[small-links]{Sploty o małej liczbie skrzyżowań}
        \label{small_links_diagram}
    \end{figure}
    }
\end{comment}

\subsubsection{Sploty rozszczepialne}
Aby wytłumaczyć, czemu trzeci splot z rysunku \ref{small_links_diagram} jest interesujący, potrzebujemy zdefiniować sploty rozszczepialne.

% DICTIONARY;splittable;rozszczepialny;splot
\begin{definition}[rozszczepialność]
\index{splot!rozszczepialny}%
    Jeżeli splot $L$ można zanurzyć w przestrzeni $\R^3$ tak, że niektóre jego ogniwa będą leżeć nad pewną rozłączną ze splotem płaszczyzną, zaś pozostałe pod nią, to powiemy, że splot $L$ jest rozszczepialny.
\end{definition}

Liczbę nierozszczepialnych splotów pierwszych kopiujemy z bazy danych LinkInfo \cite{linkinfo24}:
\renewcommand*{\arraystretch}{1.4}
\footnotesize
\begin{longtable}{lcccccccccccc}
    \hline
    \textbf{skrzyżowania} & 0 & 1 & 2 & 3 & 4 & 5 &  6 &  7 &  8 & 9 & 10 & 11 \\ \hline \endhead
    sploty pierwsze, nierozszczepialne & 0 & 0 & 1 & 0 & 1 & 1 & 6 & 9 & 29 & 83 & 287 & 1007 \\
    (w tym) alternujące & 0 & 0 & 1 & 0 & 1 & 1 & 5 & 7 & 21 & 55 & 174 & 548 \\
    (w tym) niealternujące & 0 & 0 & 0 & 0 & 0 & 0 & 1 & 2 & 8 & 28 & 113 & 459 \\
    \hline
\end{longtable}
\normalsize

W bazie liczb OEIS trafiliśmy tylko na ciąg \href{https://oeis.org/A086826}{A086826} opisujący liczbę nierozszczepialnych pierwszych i złożonych węzłów i splotów, na przykład $a_5 = 4$, bo mamy dwa węzły pierwsze, splot Whiteheada oraz trójlistnik spleciony z~niewęzłem.
Słowa ,,skrzyżowanie'' , ,,alternujący'' oraz ,,pierwszy'' definiujemy w~przyszłości, będą to odpowiednio definicje \ref{def:crossing}, \ref{def:alternating_link} i \ref{def:prime_knot}.
\index{węzeł!alternujący}%
\index{węzeł!pierwszy}%
\index{skrzyżowanie}%
Książka ma nieliniową budowę i należy przeczytać ją co najmniej dwa razy.

Pewne kryteria rozszczepialności konkretnych splotów znaleźć można u Kawauchiego \cite[s. 36-38]{kawauchi1996}.
% TODO: przepisać, a jeśli za trudne, to może chociaż szkic?


\subsubsection{Sploty Brunna}
\index{splot!Brunna|(}%
Hermann Brunn \cite{brunn1892} rozpatrywał w 1892 roku (a więc zanim jeszcze teoria węzłów przyszła na świat!) nierozszczepialne sploty, które po usunięciu dowolnego ogniwa stają się niesplotami.
\index[persons]{Brunn, Hermann}%
W~czasopiśmie Delta, nr 01/2011, przeczytaliśmy, że Rolfsen zaproponował nazywać je splotami Brunna i~tak też będziemy robić.
Najprostszym splotem Brunna są posiadające trzy ogniwa pierścienie Boromeuszy ($6_2^3$ w notacji Alexandera-Briggsa, \texttt{L6a4} w notacji Thistlethwaite'a).
\index{pierścienie Boromeuszy}%
Pokażemy na stronie \pageref{boromean_not_splittable}, że pierścienie Boromeuszy nie mają nietrywialnych trójkolorowań, więc nie mogą być niesplotem.

Pierścienie Boromeuszy są alternujące, hiperboliczne i drzewiaste.
\index{węzeł!alternujący}%
\index{węzeł!hiperboliczny}%
\index{węzeł!drzewiasty}%
\index{węzeł!algebraiczny|see {drzewiasty}}%
% DICTIONARY;arborescent;drzewiasty;węzeł
% TODO: jak jest arborescent po francusku?
Ich nazwa pochodzi od lombardzko-piemonckiego rodu kupieckiego, bankierskiego i arystokratycznego, z którego wywodziło się wielu kardynałów.
Herb tego rodu zawierał splecione ze sobą trzy okręgi.
Jest niemożliwe, by wykonać model przestrzenny tego splotu przy użyciu okrągłych pierścieni.
Zamiast tego można użyć na przykład elips.

Z dokładnością do homotopii sploty Brunna zostały sklasyfikowane przez Milnora \cite{milnor1954}, ale ponieważ ta książka nie tłumaczy, czym są $\mu$-niezmienniki Milnora, nie możemy dzisiaj wytłumaczyć, jak tego dokonał.
\index[persons]{Milnor, John}%
\index{splot!Brunna|)}%




\subsubsection{Sploty alternujące}

Zazwyczaj do zdefiniowania splotów alternujących potrzebne są najpierw diagramy.
% Zanim opowiemy, jak dotąd przebiegała klasyfikacja węzłów o małej liczbie skrzyżowań, zdefiniujemy klasę splotów ze specjalnymi diagramami.

% DICTIONARY;alternating;alternujący;węzeł
\begin{definition}[splot alternujący]
\label{def:alternating_link}%
\index{węzeł!alternujący}%
    Niech $D$ będzie diagramem splotu $L$.
    Jeżeli podczas poruszania się wzdłuż każdego ogniwa naprzemiennie mijamy podskrzyżowania oraz nadskrzyżowania, to diagram nazywamy alternujący.
    
    Splot $L$ jest alternujący, jeśli posiada alternujący diagram $D$s.
\end{definition}

Około 1961 roku Ralph Fox zapytał \emph{,,What is an alternating knot?''}.
\index[persons]{Fox, Ralph}%
Szukano takiej definicji węzła alternującego, która nie odnosi się bezpośrednio do diagramów, aż w~2015 roku Greene \cite{greene2017} podał geometryczną charakteryzację: nierozszczepialny splot w $S^3$ jest alternujący wtedy i~tylko wtedy, gdy ogranicza dodatnią oraz ujemną określoną powierzchnię rozpinającą.
\index[persons]{Greene, Joshua}%

\begin{remark}[Ralph Hartzler Fox]
    Matematyk amerykański urodzon w Morrisville, Pensylwanii w~1913 roku; zmarł w Filadelfii, tamże w 1973 roku.
    Był promotorem Johna Milnora, Lee Neuwirtha (o~których jeszcze wspomnimy!) i 23 innych osób, o których nie wspomnimy.
    Oprócz tego nadzorował pracę licencjacką Kennetha Perko.
    Zawdzięczamy mu spopularyzowanie $n$-kolorowania na koledżu Haverford w 1956 roku, podanie nowego sposobu na znalezienie wielomianu Alexandera przy użyciu rachunku różniczkowego Foxa oraz niektóre terminy teorii węzłów uzywane po dziś dzień: węzeł plastrowy, węzeł taśmowy, okrąg i powierzchnia Seiferta.
\end{remark}

Nie ma zwartego wzoru na liczbę splotów alternujących, ale wiemy, że rośnie co najmniej wykładniczo względem liczby skrzyżowań:

\begin{proposition}
\index{supeł}%
    Niech $a_n$ oznacza liczbę supłów o~$n$ skrzyżowaniach, które są alternujące oraz pierwsze.
    Wtedy
    \begin{equation}
        a_n \sim \frac{3c_1 \lambda^{n-3/2}}{4\sqrt{\pi n^{5}}},
    \end{equation}
    gdzie zarówno $c_1$, pierwszy współczynnik rozwinięcia Taylora funkcji $\Phi(\eta)$ zdefiniowanej w \cite{sundberg1998}, jak i $\lambda$ są jawnie znanymi stałymi:
    \begin{align}
        c_1 & = \sqrt{\frac{5^7 \cdot (21001 + 371 \sqrt{21001})^3}{2 \cdot 3^{10} \cdot (17 + 3\sqrt{21001})^5}} \\
        \lambda & = \frac {1}{40} (101 + \sqrt{21001})
    \end{align}
    Niech $A_n$ oznacza liczbę pierwszych, alternujących splotów o $n$ skrzyżowaniach.
    Wtedy $A_n \approx \lambda^n$, dokładniej: jeśli $n \ge 3$, to
    \begin{equation}
        \frac{a_{n-1}}{16n - 24} \le A \le \frac{a_n - 1}{2}.
    \end{equation}
\end{proposition}

Węzły pierwsze i~supły pojawiają się odpowiednio w definicjach \ref{def:prime_knot}, \ref{def:tangle}.

\begin{proof}[Niedowód]
\index[persons]{Sundberg, Carl}%
\index[persons]{Thistlethwaite, Morwen}%
    Zamiast przedstawić dowód albo chociaż jego szkic, wymienimy trzy narzędzia użyte przez Sundberga, Thistlethwaite'a \cite{sundberg1998}:
    algebraiczną metodę Conwaya znajdowania splotów,
    wynik Tuttego dotyczącego liczby ukorzenionych $c$-sieci
    oraz (wtedy już udowodnioną) hipotezę Taita.
\index[persons]{Conway, John}%
\index[persons]{Tutte, William}%
\index{hipoteza!Taita}%
\end{proof}

\begin{proposition}
    Niech $a_n$ oznacza liczbę supłów o~$n$ skrzyżowaniach, które są alternujące oraz pierwsze.
    Wtedy funkcja tworząca $f(z) = \sum_n a_n z^n$ spełnia równanie
    \begin{equation}
    f(1+z) - f(z)^2 - (1+f(z))q(f(z)) -z - \frac{2z^2}{1-z} = 0,
    \end{equation}
    gdzie $q(z)$ jest pomocniczą funkcją
    \begin{equation}
        q(z) = \frac{2z^2 - 10z - 1 + \sqrt{(1-4z)^3}} {2(z+2)^3} - \frac{2}{1+z} -z + 2.
    \end{equation}
\end{proposition}

Powyższa ciekawostka także pochodzi z cytowanej wcześniej pracy \cite{sundberg1998}.





\subsection{Dopełnienia węzłów i splotów}
Jeśli dwa węzły są równoważne, to ich dopełnienia są oczywiście homeomorficzne.
Pytanie o~prawdziwość implikacji odwrotnej jako pierwszy zadał najprawdopodobniej Tietze \cite{tietze1908} w~1908 roku.
\index[persons]{Tietze, Heinrich}%
O~jego trudności niech świadczy fakt, że dopiero w~roku 1987 pokazano, że istnieją co najwyżej dwa węzły o~zadanym dopełnieniu (Culler, Gordon, Luecke, Shalen \cite{culler1987}).
\index[persons]{Culler, Marc}%
\index[persons]{Shalen, Peter}%
\index[persons]{Gordon, Cameron}%
\index[persons]{Luecke, John}%
Dwa lata później poznaliśmy pozytywną odpowiedź na pytanie Tietzego: każdy węzeł jest wyznaczony jednoznacznie przez swoje dopełnienie.
Natomiast analogiczne stwierdzenie o~splotach jest fałszywe i wiedziano o tym od bardzo dawna: w~1937 roku Whitehead \cite{whitehead1937} podał nieskończenie wiele splotów, których dopełnienia wyglądają jak dopełnienia splotu Whiteheada.

\begin{theorem}[Gordon, Luecke, 1989]
\index[persons]{Gordon, Cameron}%
\index[persons]{Luecke, John}%
\index{twierdzenie!Gordona-Lueckego}%
    Niech $f \colon (\mathbb R^3 \setminus K_1) \to (\mathbb R^3 \setminus K_2)$ będzie zachowującym orientację homeomorfizmem dopełnień poskromionych węzłów $K_1, K_2$.
    Wtedy węzły $K_1 \cong K_2$ są izotopijne.
\end{theorem}

\begin{proof}[Niedowód]
    Wynika to z teorii Cerfa, kombinatorycznych technik w stylu Litherlanda, cienkich pozycji, cykli Scharlemanna i~ogólniejszego stwierdzenia: nietrywialna chirurgia Dehna na węźle w~3-sferze nigdy nie daje 3-sfery.
\index{chirurgia Dehna}%
\index{cykle Scharlemanna}%
\index{teoria Cerfa}%
    Pełny dowód zawiera praca \cite{gordon1989}.
\end{proof}

Twierdzenie to zamienia problem lokalny (czy dwa węzły w kuli $S^3$ są równoważne?) na problem globalny (czy dwie przestrzenie topologiczne są homeomorficzne?).
\index[persons]{Whitehead, John}%

% koniec sekcji Węzły i sploty


\section{Diagramy. Ruchy Reidemeistera}

Chociaż w~świetle definicji \ref{def:knot} węzły są pewnymi regularnymi podzbiorami przestrzeni $\R^3$, z~kombinatorycznego punktu widzenia wygodniej jest rysować je na płaszczyźnie.

% DICTIONARY;shadow;cień;-
% DICTIONARY;crossing;skrzyżowanie;-
\begin{definition}[cień, skrzyżowanie]
\index{cień}%
\index{skrzyżowanie}%  
\label{def:crossing}%
    Niech $\pi \colon \R^3 \to \R^2$ będzie rzutem na pewną płaszczyznę, zaś $L \subseteq \R^3$ ustalonym splotem.
    Obraz $\pi[L]$ nazywamy cieniem, punkty podwójne $p$ cienia (punkt $p \in \R^2$, którego przeciwobraz $\pi^{-1}[\{p\}]$ jest dwupunktowy) nazywamy skrzyżowaniami.
\end{definition}

Zazwyczaj nie wprowadza się osobnego terminu na cień, tylko oszczędza czas i zaczyna od razu od diagramów, tak jak Burde, Zieschang, Heusener \cite[s. 9]{burde2014}:

\begin{definition}[diagram]
% DICTIONARY;diagram;diagram;-
\index{diagram}%
    Niech $D$ będzie cieniem splotu $L$ takim, że cień $D$ zawiera skończenie wiele punktów wielokrotnych (i wszystkie są punktami podwójnymi) oraz skrzyżowania cienia $D$ nie zawierają wierzchołków splotu $L$.
    Wtedy cień $D$ wzbogacony o~informację o tym, jak przebiegają skrzyżowania (który odcinek łamanej biegnie dołem, a~który górą) nazywamy diagramem.
% TODO: dopisać, że będziemy to nazywać nad i pod skrzyżowania
\index{nadskrzyżowanie}%
\index{podskrzyżowanie}%
\end{definition}

Mieliśmy problemy ze znalezieniem ładnego rysunku katastrof, jakich nie dopuszczamy pracując z diagramami zamiast zwykłymi cieniami; nam nie do końca chciało się je rysować je tutaj.
Wybawieniem okazała się monografia Burdego, Zieschanga (i Heusenera) \cite[s. 9]{burde2014}.

Dla każdego ustalonego $n \ge 2$ i każdego węzła $K$ istnieje cień $D$, na którym wszystkie wielokrotne punkty są $n$-krotne (wiemy to od Hostego, College'a \cite[s. 11]{adams2021}, którzy nie napisali, skąd to wiedzą);
\index[persons]{Hoste, Jim}%
\index[persons]{College, Pitzer}%
dla co najmniej jednej wartości $n$ można dodatkowo wymagać, by diagram zawierał dokładnie jedno skrzyżowanie (Adams, Crawford, DeMeo, Landry, Lin, Montee, Park, Venkatesh, Yhee \cite{venkatesh2015} albo Brunn ponad sto lat temu \cite[s. 28]{adams2021}!).
\index[persons]{Adams, Colin}%
\index[persons]{Crawford, Thomas}%
\index[persons]{DeMeo, Benjamin}%
\index[persons]{Landry, Michael}%
\index[persons]{Lin, Alex}%
\index[persons]{Montee, Murphy}%
\index[persons]{Park, Seojung}
\index[persons]{Venkatesh, Saraswathi}%
\index[persons]{Yhee, Farrah}%

\begin{definition}[włókno, nić]
\index{włókno}%
\index{nić}%
    Fragment diagramu biegnący między dwoma kolejnymi skrzyżowaniami (odpowiednio: tunelami, czyli podskrzyżowaniami), nazywamy nicią (odpowiednio: włóknem)
\end{definition}

Skrzyżowania i diagramy są standardowymi terminami, zrozumiałymi przez każdego. 
Cienie, nici i włókna stanowią twórczość własną autorów!
Nici powstają z włókien przez rozcięcie ich przy każdym nadskrzyżowaniu.

\begin{proposition}
\label{prp:links_have_diagrams}%
    Niech $L$ będzie splotem.
    Jego diagramy tworzą otwarty i~gęsty podzbiór wszystkich rzutów.
\end{proposition}

Kawauchi \cite[s. 7]{kawauchi1996} wspomina w tym miejscu podręcznik Crowella, Foxa \cite[s. 7]{crowell1963}.
To samo jest na przykład u Burdego, Zieschanga, Heusenera \cite[s. 10]{burde2014}, ale oni odsyłają jeszcze do Reidemeistera \cite{reidemeister1927} i samego Burdego \cite{burde1978}.

\begin{proof}[Niedowód]
    Rzut splotu na równoległe płaszczyzny jest taki sam, a te można sparametryzować prostymi przechodzącymi przez początek układu współrzędnych, które tworzą przestrzeń rzutową $\R \mathbb P^2$.
    Niech $S$ będzie zbiorem prostych, które dają złe rzuty.
    Wystarczy pokazać jego nigdziegęstość.
    Okazuje się, że $S$ jest też jednowymiarowy.
\end{proof}

\begin{corollary}
    Każdy splot posiada diagram.
\end{corollary}

% DICTIONARY;oriented;zorientowany;węzeł
\begin{definition}[orientacja]
\index{węzeł!zorientowany}%
\index{orientacja|see {węzeł zorientowany}}%
    Węzeł, w~którym wybrano kierunek, w~którym należy się po nim poruszać, nazywamy zorientowanym.
    Splot nazywamy zorientowanym, jeśli wszystkie jego ogniwa traktowane jako węzły są zorientowane.
\end{definition}

Orientację na diagramie zaznaczamy małą strzałką wskazującą kierunek poruszania się.


\subsection{Ruchy Reidemeistera}

Wiemy już, że węzły mają wiele diagramów.
Mając dane dwa różne diagramy chcielibyśmy wiedzieć, czy przedstawiają ten sam węzeł.
Stosowne narzędzie dostarczył Kurt Reidemeister w~latach dwudziestych ubiegłego wieku.
\index[persons]{Reidemeister, Kurt}%
Zdefiniujmy trzy lokalne operacje na diagramach, a~potem wysłowimy kryterium  Reidemeistera rozstrzygające problem równości węzłów.

% DICTIONARY;Reidemeister/Turaev/... move;ruch Reidemeistera/Turajewa/...;-
\begin{definition}[ruchy Reidemeistera]
\index{ruch!Reidemeistera}%
    Trzy gatunki lokalnych deformacji diagramu splotu:
    \begin{figure}[H]
    \centering
    \begin{minipage}[b]{.22\linewidth}%
        \centering%
        \MedLarReidemeisterOneLeft $\stackrel{R_1}{\cong}$ \MedLarReidemeisterOneStraight%
        \subcaption{ruch $R_1$}%
    \end{minipage}
    \quad\quad\quad
    \begin{minipage}[b]{.2\linewidth}
        \centering
        \MedLarReidemeisterTwoA $\stackrel{R_2}{\cong}$ \MedLarReidemeisterTwoB
        \subcaption{ruch $R_2$}
    \end{minipage}
    \quad\quad\quad
    \begin{minipage}[b]{.32\linewidth}
        \centering
        \MedLarReidemeisterThreeA $\stackrel{R_3}{\cong}$ \MedLarReidemeisterThreeB
        \subcaption{ruch $R_3$}
    \end{minipage}
    \end{figure}
    nazywamy ruchami Reidemeistera.
    Czasami używa się konkretnych nazw:
    \begin{itemize}
        \item skręcenie/rozkręcenie (dla $R_1$),
        \item wsunięcie/rozsunięcie (dla $R_2$) oraz
        \item przesunięcie łuku przez skrzyżowanie (dla $R_3$).
    \end{itemize}
\end{definition}

Reidemeister w~swojej pierwszej pracy przyjął inną kolejność, jego drugi ruch jest naszym pierwszym.
Dzięki temu ruch $R_k$ operuje na $k$ łukach diagramu.
Colberg \cite[s. 6]{colberg2013} pisze, że Maxwell znał ruchy Reidemeistera przed Reidemeisterem, ale mimo próśb Taita nigdy nie zgłosił swojego odkrycia w Royal Society of Edinburgh.
\index[persons]{Tait, Peter}%
\index[persons]{Maxwell, James}%

\begin{theorem}[Reidemeister, 1927]
\label{thm:reidemeister}%
\index{twierdzenie!Reidemeistera}%
\index[persons]{Reidemeister, Kurt}%
    Niech $D_1, D_2$ będą diagramami dwóch splotów $L_1, L_2$.
    Sploty $L_1, L_2$ są takie same wtedy i tylko wtedy, gdy diagram $D_2$ można otrzymać z $D_1$ wykonując skończony ciąg ruchów Reidemeistera oraz gładkich deformacji łuków, bez zmiany biegu skrzyżowań.
\end{theorem}
% https://math.stackexchange.com/questions/4399634/two-knots-k-and-k-prime-are-equivalent-if-and-only-if-their-projections-p
% Reidemeister, and pretty much every other author, has worked with the piecewise-linear case. In a way it doesn't matter which you choose, since there's a theorem that the categories of smooth and PL manifolds are equivalent in some sense. However, it's not so clear how you pass from one setting to the other (or at least I've never gone through the details myself!)

Twierdzenie Reidemeistera jest prawdziwe także dla splotów zorientowanych, ale wtedy trzeba uwzględnić różne orientacje łuków i~nie jest oczywiste, ile spośród $2^1 + 2^2 + 2^3 = 14$ wersji jest potrzebne.
Polyak \cite{polyak2010} pokazał, że minimalny zbiór zorientowanych ruchów składa się na przykład z~dwóch wersji ruchu $R_1$, jednej wersji ruchu $R_2$ i~jednej wersji ruchu $R_3$.
\index[persons]{Polyak, Michael}%

\begin{proof}[Niedowód]
Dowód podali niezależnie Reidemeister \cite{reidemeister1927} oraz Alexander, Briggs \cite{alexander1927}.
\index[persons]{Reidemeister, Kurt}%
\index[persons]{Briggs, Garland}%
\index[persons]{Alexander, James}%
    Szkielet dowodu można znaleźć w~książce Burdego i~Zieschanga \cite[s. 9-11]{burde2014}, ale kluczowe pomysły podają też Prasołow z~Sosińskim \cite[s. 11-12]{prasolov1997}.
\index[persons]{Burde, Gerhard}%
\index[persons]{Zieschang, Heiner}%
\index[persons]{Prasołow, Wiktor (Прасолов, Виктор Васильевич)}%
\index[persons]{Sosiński, Aleksiej (Сосинский, Алексей Брониславович)}%
    Innym przystępnym źródłem jest podręcznik Murasugiego \cite[s. 50-56]{murasugi1996}.
\index[persons]{Murasugi, Kunio}%
\end{proof}

Trace \cite{trace1983} zauważył, że dwa diagramy jednego węzła są związane tylko II i III ruchem (ale nie I) wtedy i tylko wtedy, gdy mają ten sam spin oraz indeksy nawinięcia.
\index[persons]{Trace, Bruce}%
Z prac Östlunda \cite{ostlund2001}, Manturowa \cite[s. ???]{manturov2004} oraz Haggego \cite{hagge2006} wynika, że dla każdego węzła istnieje para diagramów, do przejścia między którymi trzeba wykorzystać wszystkie trzy ruchy.
% TODO: ustalić, które strony w Manturowie
\index[persons]{Östlund, Olof}%
\index[persons]{Manturow, Wasilij}%
\index[persons]{Hagge, Tobias}%
% praca Haggego nazywa się "Every Reidemeister move is needed for each knot type" ale nawet w MathSciNecie wspomnieni są Ostlund i Manturow, więc zostawiam. Tekst skopiowany z Wiki
Coward \cite{coward2006} zademonstrował, że nawet jeśli wszystkie trzy ruchy są potrzebne, można je wykonywać w specjalnej kolejności: najpierw tylko I ruchy, potem tylko II ruchy, następnie tylko III ruchy i~znowu II ruchy.
\index[persons]{Coward, Alexander}%

Do pokazania, że dwa węzły $K_1, K_2$ nie są równoważne, powinniśmy na mocy twierdzenia \ref{thm:reidemeister} udowodnić, że żaden ciąg ruchów Reidemeistera nie przekształca jednego w drugi.
Oczywiście nikt o zdrowych zmysłach tak nie postępuje.
Zamiast tego wprowadza się stosowny niezmiennik, czyli funkcję $f$ określoną na zbiorze wszystkich węzłów (albo splotów, supłów, warkoczy itd.) tak, że jeśli węzły $K_1 \cong K_2$ są równoważne, to $f(K_1) = f(K_2)$.
% DICTIONARY;invariant;niezmiennik;-
Łatwo widać, że jeśli $f(K_1) \neq f(K_2)$, to węzły $K_1, K_2$ nie mogą być równoważne.
Natomiast gdy wartości są te same, nie dostajemy żadnej informacji.

Poznaliśmy jak na razie dwa niezmienniki: liczbę ogniw splotu oraz dopełnienie splotu do przestrzeni, w której jest zanurzony ($\mathbb R^3$ lub $S^3$).
Wiele, chociaż nie wszystkich, innych niezmienników definiuje się nie bezpośrednio na zbiorze węzłów, ale na zbiorze diagramów.
Należy wtedy sprawdzić, że każdy z trzech ruchów Reidemeistera nie ma wpływu na wartość niezmiennika.

Niezmienniki będą nam stale towarzyszyć w~wędrówce po krainie węzłów.

\begin{remark}[Kurt Werner Friedrich Reidemeister]
    ?
\end{remark}

\begin{remark}[James Waddell Alexander]
    ?
\end{remark}

\begin{remark}[Garland Baird Briggs]
    Matematyk amerykański, urodzon w Sebrell, Wirginii w 1894 roku; zmarł w Kolumbii w 1959 roku.
    Niestety nie wiemy za dużo o tym człowieku.
\end{remark}

\color{white}

\subsubsection{Dygresja -- wyniki ilościowe wokół twierdzenia Reidemeistera}
Załóżmy, że na dwóch diagramach tego samego węzła widać odpowiednio $n_1, n_2$ skrzyżowań.
Jak piszą Coward, Lackenby \cite{coward2011}, istnieje funkcja $f \colon \N \times \N \to \N$ taka, że między dwoma diagramami można przejść wykonując co najwyżej $f(n_1, n_2)$ ruchów.
\index[persons]{Coward, Alexander}%
\index[persons]{Lackenby, Marc}%
Wynika to z faktu, że istnieje skończenie wiele spójnych diagramów o danej liczbie skrzyżowań oraz twierdzenia Reidemeistera.
Okazuje się jednak, że od funkcji $f$ można żądać, by była obliczalna
(a to jest chyba równoważne istnienia algorytmu rozpoznającego, czy dwa diagramy przedstawiają jeden węzeł)
% http://people.dm.unipi.it/martelli/Cortona/Lackenby.pdf 7 of 90
i faktycznie, główny wynik \cite{coward2011} orzeka, że
\begin{equation}
    f(n_1, n_2) = 2^{2^{\ldots^{2^{n_1 + n_2}}}}
\end{equation}
jest taką funkcją.
Piętrowa potęga liczy sobie aż $10^{1000000 (n_1 + n_2)}$ warstw.

Natomiast jeżeli $n_2 = 0$, czyli drugi diagram przedstawia niewęzeł, ,,wystarcza'' $(236n_1)^{11}$ ruchów, przy czym liczba skrzyżowań podczas transformacji nigdy nie przekracza $49c^2$: to świeższy wynik samego Lackenby'ego \cite{lackenby2015}, gdzie poprawił wcześniejsze oszacowania Hassa, Lagariasa.
Przykład diagramu niewęzła, do rozwiązania którego nie można tylko usuwać istniejących skrzyżowań, przedstawiają Burde, Zieschang, Heusener \cite[s. 12]{burde2014}.

Hayashi \cite{hayashi2005} dowiódł, że liczbę ruchów Reidemeistera potrzebnych, by rozszczepić splot można ograniczyć z góry na podstawie indeksu skrzyżowaniowego.
\index[persons]{Hayashi, Chuichiro}%

% koniec sekcji Ruchy Reidemeistera



\input{10-introduction/102b-history}

\input{10-introduction/102c-codes}


\section{Wykrywanie niewęzła}
Jednym z pierwszych dużych problemów teorii węzłów było poszukiwanie odpowiedzi na pytanie, kiedy diagram przedstawia niewęzeł.
\index{niewęzeł}%
Stosowny algorytm wykrywający niewęzły podał Haken \cite{haken1961}, ale długo nikt nie podjął się jego implementacji.
\index[persons]{Haken, Wolfgang}%
Epple pisze \emph{,,this algorithm was extremely impractical''}, w recenzji z MathSciNet proponuje, żeby przed przeczytaniem pełnej niepotrzebnych dygresji pracy Hakena poznać artykuł \cite{schubert1961} Schuberta.
\index[persons]{Epple, Moritz}%
\index[persons]{Schubert, Horst}%
W życie pomysły Hakena udało się wdrożyć Burtonowi, Budneyowi oraz Petterssonowi w~komputerowym programie Regina\footnote{\url{https://regina-normal.github.io/}.} na przełomie tysiącleci.

\index[persons]{Burton, Benjamin}%
\index[persons]{Budney, Ryan}%
\index[persons]{Pettersson, William}%
%=% https://mathscinet.ams.org/mathscinet-getitem?mr=141107
% DICTIONARY;incompressible;nieściśliwy;-

Burton, Rubinstein i~Tillman \cite{burton2012} pokazali, jak sprawdzać, w~czasie wykładniczym czy powierzchnia normalna na striangulowanej 3-rozmaitości jest (nie)ściśliwa.
\index[persons]{Rubinstein, Joachim}%
\index[persons]{Tillman, Stephan}%
To wystarczyło do udzielenia negatywnej odpowiedzi na pytanie Thurstona: \emph{,,czy przestrzeń Seiferta-Webera jest rozmaitością Hakena?''}, a zatem wykraczającego poza poziom naszego skromnego dzieła.
\index[persons]{Thurston, William}%
\index{przestrzeń!Seiferta-Webera}%
\index{rozmaitość!Hakena}%

SnapPea\footnote{\url{http://geometrygames.org/SnapPea/index.html}.} to inny popularny wśród niskowymiarowych topologów program pozwalający badać hiperboliczne 3-rozmaitości, patrz sekcja \ref{sec:hyperbolic}.

Wiadomo, że genus oraz zredukowana kohomologia Chowanowa wykrywa niewęzły (fakty \ref{prp:genus_detects_unknot}, \ref{khovanov_detects_unknot}) i nie wiadomo, czy wielomian Jonesa to robi (hipoteza \ref{con:jones}).
\index{genus}%
\index{homologia!Chowanowa}%
\index{wielomian!Jonesa}%
Od dawna wiadomo, że wielomian Alexandera nie wykrywa niewęzła (fakt \ref{alexander_no_detects_unknot}).
\index{wielomian!Alexandera}%
W lutym 2021 Lackenby ogłosił nowy algorytm rozpoznający niewęzły w~quasiwielomianowym czasie.
% i nie zrobił tego w THE EFFICIENT CERTIFICATION OF KNOTTEDNESS AND THURSTON NORM, bo to wyszło na arxiv w 2016
\index[persons]{Lackenby, Marc}%

Wyśmienitym punktem wyjścia do poszukiwań trudnych niewęzłów jest praca \cite{schleimer2021}, dzieło Burtona, Changa, Löfflera, de Mesmaya, Marii, Schleimera, Sedgwicka oraz Spreera.
\index[persons]{Burton, Benjamin}%
\index[persons]{Chang, Hsien-Chih}%
\index[persons]{Mesmay, Arnaud@de Mesmay, Arnaud}%
\index[persons]{Löffler, Maarten}%
\index[persons]{Maria, Clément}%
\index[persons]{Schleimer, Saul}%
\index[persons]{Sedgwick, Eric}%
\index[persons]{Spreer, Jonatha}%
Cytuje ona artykuł Lackenby'a \cite{lackenby2015}, gdzie poznaliśmy stary (z 1934 roku!) przykład Goeritza \cite{goeritz1934} diagramu niewęzła o~11 skrzyżowaniach, który można zmienić w~zwykły diagram niewęzła tylko zwiększając po drodze liczbę skrzyżowań.
% 9781470454999 s. 5
\index[persons]{Goeritz, Lebrecht}%
\index{niewęzeł!Goeritza}%
Na kolejne teksty przyszło poczekać ponad pół wieku.
Autorzy przywołują jeszcze klasyczny przykład Freedmana, He, Wanga \cite{freedman1994}; ale też podstępne niewęzły Hakena, Ochiai, Thistlethwaite'a oraz mocno doświadczalną pracę Petronio, Zanellatiego \cite{zanellati2016}.
\index[persons]{Freedman, Michael}%
\index[persons]{He, Zheng-Xu}%
\index[persons]{Wang, Zhenghan}%
\index{niewęzeł!Freedmana}%
\index[persons]{Petronio, Carlo}%
\index[persons]{Zanellati, Adolfo}%

% TODO: sprawdzić, czy w \cite{heinrich2014} nie ma więcej trudnych niewęzłów

My nie potrafimy albo nie chcemy potrafić ładnie rysować, więc pozwolimy sobie pokazać tylko, jak wyglądał wspomniany wcześniej przykład Goeritza.
Na swoim blogu\footnote{\url{https://mickburton.co.uk/2015/06/05/how-do-you-construct-hakens-gordian-knot/}} Burton (inny Burton niż w poprzednim akapicie!) zamieścił wpis pełen rysunków, które tłumaczą, jak powstał niewęzeł Hakena.

\begin{comment}
\begin{figure}[H]
    \centering
    \begin{tikzpicture}[baseline=-0.65ex, scale=0.08]
        \begin{knot}[clip width=5, end tolerance=1pt, flip crossing/.list={1,2,3,4,8,9}]
            % horizontal lines
            \strand[ultra thick] (-40, -10) to (-10, -10);
            \strand[ultra thick] (-40, 10) to (-10, 10);
            \strand[ultra thick] (10, 10) to (30, 10);
            \strand[ultra thick] (10, -10) to (30, -10);
            % 
            \strand[ultra thick] (5-45, -10) [in=left,out=left] to (5-45, 3.33);
            \strand[ultra thick] (5-45, 10) [in=left,out=left] to (5-45, -3.33);
            \strand[ultra thick] (-5-35, -3.33) [in=left,out=right] to (5-35, 3.33);
            \strand[ultra thick] (-5-35, 3.33) [in=left,out=right] to (5-35, -3.33);
            \strand[ultra thick] (-5-25, -3.33) [in=left,out=right] to (5-25, 3.33);
            \strand[ultra thick] (-5-25, 3.33) [in=left,out=right] to (5-25, -3.33);
            \strand[ultra thick] (-5-15, -3.33) [in=left,out=right] to (5-15, 3.33);
            \strand[ultra thick] (-5-15, 3.33) [in=left,out=right] to (5-15, -3.33);
            %
            \strand[ultra thick] (-5+15, -3.33) [in=left,out=right] to (5+15, 3.33);
            \strand[ultra thick] (-5+15, 3.33) [in=left,out=right] to (5+15, -3.33);
            \strand[ultra thick] (-5+25, -3.33) [in=left,out=right] to (5+25, 3.33);
            \strand[ultra thick] (-5+25, 3.33) [in=left,out=right] to (5+25, -3.33);
            \strand[ultra thick] (-5+35, -3.33) [in=right,out=right] to (-5+35, 10);
            \strand[ultra thick] (-5+35, 3.33) [in=right,out=right] to (-5+35, -10);
            %
            \strand[ultra thick] (-5+5, 3.33) [in=left,out=right] to (5+5, 10);
            \strand[ultra thick] (-5+5, 10) [in=left,out=right] to (5+5, 3.33);
            \strand[ultra thick] (-5-5, 3.33) [in=left,out=right] to (5-5, 10);
            \strand[ultra thick] (-5-5, 10) [in=left,out=right] to (5-5, 3.33);
            %
            \strand[ultra thick] (-5+5, -10) [in=left,out=right] to (5+5, -3.33);
            \strand[ultra thick] (-5+5, -3.33) [in=left,out=right] to (5+5, -10);
            \strand[ultra thick] (-5-5, -10) [in=left,out=right] to (5-5, -3.33);
            \strand[ultra thick] (-5-5, -3.33) [in=left,out=right] to (5-5, -10);
        \end{knot}
    \end{tikzpicture}
    \caption{niewęzeł Goeritza}
\end{figure}
\end{comment}




\section{Hipotezy Taita}
\index{hipoteza!Taita|(}%

Tait na podstawie węzłów o małej liczbie skrzyżowań  wysunął około 1898 roku trzy lub cztery hipotezy.
Nie jest jasne, czy chodziło mu o wszystkie węzły, czy tylko te alternujące.
Uchylamy tutaj rąbka tajemnicy i~podajemy treść hipotez już teraz; dowód ze szczegółami odkładając na później, aż do sekcji \ref{sub:tait_conjectures}.
Tam też wspomnimy krótko o technikach użytych w dowodach pozostałych trzech.

\begin{conjecture}[I hipoteza Taita]
\index{indeks skrzyżowaniowy}%
\label{con:tait_1}%
    Zredukowany alternujący diagram splotu ma minimalny indeks skrzyżowaniowy.
\end{conjecture}

Najpierw znaleziono dowód korzystający z wielomianu Jonesa: dokonali tego w 1987 roku równocześnie Kauffman \cite{kauffman1987}, Murasugi \cite{murasugi1987} oraz Thistlethwaite \cite{thistlethwaite1987}.
\index[persons]{Kauffman, Louis}%
\index[persons]{Murasugi, Kunio}%
\index[persons]{Thistlethwaite, Morwen}%
Trzydzieści lat później Greene \cite{greene2017} zaprezentował geometryczne podejście do problemu.
\index[persons]{Greene, Joshua}%

\begin{conjecture}[II hipoteza Taita]
\index{spin}%
    Dwa zredukowane diagramy alternujące jednego węzła mają ten sam spin.
\end{conjecture}

Pierwsze dowody pochodzą znowu od Kauffmana \cite{kauffman1987} oraz Thistlethwaite'a \cite{thistlethwaite1987}.
\index[persons]{Kauffman, Louis}%
\index[persons]{Thistlethwaite, Morwen}%
Dla niektórych II hipoteza brzmi inaczej (,,achiralny splot alternujący ma zerowy spin''), dla innych jest prostym wnioskiem z naszego sformułowania.

\begin{conjecture}[III hipoteza Taita]
\index{flype}%
    Niech $D_1, D_2$ będą zredukowanymi alternującymi diagramami zorientowanego pierwszego splotu.
    Wtedy diagram $D_2$ można otrzymać z~$D_1$ korzystając jedynie z~ruchu \emph{flype}.
\end{conjecture}

Trzecią hipotezę udowodnił Menasco wspólnie z~Thistlethwaitem \cite{menasco1993}.
\index[persons]{Menasco, William}%
\index[persons]{Thistlethwaite, Morwen}%
Wynika z~niej, że dwa zredukowane diagramy alternujące tego samego węzła mają ten sam spin.
Nie jest prawdziwa dla niealternujących splotów, przez co w~tablicach węzłów tak długo mieliśmy duplikat -- parę Perko.
\index{para Perko}%

Czasami mówi się jeszcze o IV hipotezie: że zwierciadlane węzły mają parzysty indeks skrzyżowań.
\index{węzeł!zwierciadlany}
Ta okazała się fałszywa.

\index{hipoteza!Taita|)}%

% koniec podsekcji Hipotezy Taita





\chapter{Operacje węzłowe}

\section{Operacje na węzłach: lustra, rewersy i sumy}
Mając dany diagram splotu zorientowanego, można odwrócić jego wszystkie ogniwa albo wszystkie skrzyżowania.
Działania te nazywamy odpowiednio rewersem i~lustrem, opisujemy je w~pierwszej podsekcji.
Dalej pojawi się suma niespójna oraz spójna, odpowiednik mnożenia liczb naturalnych zbadany dokładniej przez Schuberta około 1954 roku.
\index[persons]{Schubert, Horst}%
Znacznie później (bo dopiero w~sekcji \ref{sec:tangle}) wprowadzimy jeszcze sumę i~iloczyn supłów.


\subsection{Lustro i~rewers. Węzły skrętne i zwierciadlane}
\begin{definition}[lustro]
% DICTIONARY;mirror;lustro/lustrzany;węzeł
\index{lustro}%
\index{węzeł!lustrzany}% TODO: to się może mylić ze zwieciadlanym
    Niech $L$ będzie zorientowanym splotem.
    Splot $\operatorname{m} L$ powstały przez odbicie splotu $L$ względem dowolnej płaszczyzny nazywamy lustrem.
\end{definition}

\begin{definition}[rewers]
% DICTIONARY;reverse;rewers/odwrotny;węzeł
\index{rewers}%
\index{węzeł!odwrotny}%
    Niech $L$ będzie zorientowanym splotem.
    Splot $\operatorname{r} L$ powstały przez odwrócenie orientacji wszystkich ogniw splotu $L$ nazywamy rewersem.
\end{definition}

\begin{comment}
\begin{figure}[H]
    \begin{minipage}[b]{.32\linewidth}
        \centering
        \includegraphics[height=2.2cm]{../data/links/9_3_2_mirror.png}
        \subcaption{lustro $mL$}
    \end{minipage}
    \begin{minipage}[b]{.32\linewidth}
        \centering
        \includegraphics[height=2.2cm]{../data/links/9_3_2_base.png}
        \subcaption{przykładowy splot $L$}
    \end{minipage}
    \begin{minipage}[b]{.32\linewidth}
        \centering
        \includegraphics[height=2.2cm]{../data/links/9_3_2_reverse.png}
        \subcaption{rewers $rL$}
    \end{minipage}
\end{figure}
\end{comment}

Na lewym obrazku odbiliśmy diagram względem pionowej płaszczyzny, ale ten sam splot dostalibyśmy odwracając wszystkie nad- i podskrzyżowania (czyli odbijając go względem płaszczyzny papieru, ale programy graficzne nie pozwalają na wykonanie tej operacji zbyt łatwo, a~my jesteśmy trochę leniwi).

Niektórzy mówią o odbiciach lustrzanych i odwrotnościach.
Zaletą naszych oznaczeń jest to, że trudniej jest pomylić lustro z odwrotnością; ale w literaturze dominuje $L^*$ jako symbol lustra oraz $-L$ jako symbol odwrotności splotu $L$; używają ich na przykład Burde, Zieschang, Heusener \cite{burde2014} czy Murasugi \cite[s. 14, 24]{murasugi1996}.
Lickorish \cite[s. 4]{lickorish1997} używa $r L$ dla odwrotności oraz $\overline{L}$ dla lustra; zapewne ktoś gdzieś używa jeszcze innych znaczków.

Zauważmy, że wzięcie lustra i/lub rewersu węzła nie musi prowadzić do nowych obiektów.
Na przykład trójlistnik jest własnym rewersem: $3_1 = \operatorname{r} 3_1$, ale nie lustrem.

Dlatego wyróżniamy pięć typów symetrii splotów:

\begin{definition}[sploty zwierciadlane, odwracalne, chiralne]
% DICTIONARY;chiral;skrętny/chiralny;węzeł
% DICTIONARY;reversible;odwracalny;węzeł
% DICTIONARY;achiral/amphicheiral;zwierciadlany;węzeł
    Niech $L$ będzie splotem.
    Wtedy $L$ jest równoważny ze swoim rewersem, lustrem, rewersem lustra, wszystkimi albo żadnym z trzech wymienionych splotów;
    splot $L$ nazywamy odpowiednio:
    \begin{itemize}
        \item odwracalnym ($L = rL$),
        \item dodatnio zwierciadlanym ($L = mL$),
        \item ujemnie zwierciadlanym ($L = mrL$),
        \item całkowicie zwierciadlanym ($mL = L = rL$),
        \item całkowicie chiralnym ($mL \neq L \neq rL$).
    \end{itemize}
\index{węzeł!zwierciadlany}%
\index{węzeł!odwracalny}%
\index{węzeł!chiralny}%
\index{węzeł!skrętny|see {węzeł chiralny}}%
\end{definition}

Węzły całkowicie chiralne nazywa się czasem skrętnymi.

%    Węzły $K$, $rK$, $mK$ są parami nierównoważne. % chiral 9_32
%    Węzły $K \cong rK$ są równoważne. % reversible 3_1
%    Węzły $K \cong mrK$ są równoważne. % negative amphicheiral 8_17
%    Węzły $K \cong mK$ są równoważne. % positive amphicheiral 12a_427
%    Węzły $K, rK, mK$ są parami równoważne. % fully amphicheiral 4_1

\begin{example}
    Węzeł $9_{32}$ jest całkowicie skrętny.
\end{example}

% Całkowicie skrętne są też między innymi wszystkie węzły torusowe.
% TODO: wiki pisze Each nontrivial torus knot is prime[4] and chiral.[2]

\begin{example}
    \label{exm:trefoil_is_chiral}
    Trójlistnik jest odwracalny, ale nie zwierciadlany.
\end{example}

Przypuszczał to już Listing \cite{listing1847} w 1847 roku, ale pierwszy dowód podał dużo później, bo w 1914 roku Dehn \cite{dehn1914}. 
Oto, jak tego dokonał.
% równik = equator
% równoleżnik = parallel (of latitude), najdłuższy równoleżnik to równik
% południk = meridian (of longitude)
% TODO: naprostować bałagan z meridian/longitude w całej książce
% https://math.stackexchange.com/questions/2511364/how-did-dehn-prove-that-the-trefoil-is-chiral myli te pojęcia: pisze o meridian i longitude, kiedy oryginalna praca Dehna operowała na longitude i latitude
Iloraz grafu Cayleya dla grupy podstawowej trójlistnika, $G = \pi_1(S^3 - K)$, zanurza się w~produkt $\mathbb H^2 \times \R$, co pozwala wyznaczyć grupę zewnętrznych automorfizmów grupy $G$, $\Z/2\Z$.
\index{grupa!podstawowa}
% DICTIONARY;latitude;szerokość geograficzna;geografia
% DICTIONARY;longitude;długość geograficzna;geografia
% DICTIONARY;meridian (of longitude);południk;geografia
% DICTIONARY;parallel (of latitude);równoleżnik;geografia
% DICTIONARY;---;geografia;-
Korzystając z~południków i~równoleżników pokazał następnie, że nietrywialny automorfizm zewnętrzny odwraca orientację przestrzeni otaczającej.

My przekonamy się o~tym po wyznaczeniu wielomianu Jonesa trójlistnika, patrz wniosek \ref{cor:jones_of_amphicheiral}.

\begin{example}
    Węzeł $8_{17}$ jest zwierciadlany ujemnie, ale nie odwracalny.
\end{example}

Sześćdziesiąt lat temu matematycy nie byli pewni, czy węzły nieodwracalne w~ogóle istnieją \cite[problem 10]{fox1962};
obecnie wiadomo, że nieodwracalne są prawie wszystkie węzły (Murasugi \cite[s.~46]{murasugi1996}).
\index[persons]{Murasugi, Kunio}%
W~roku 1962 Ralph Fox wskazał kilku kandydatów do tego tytułu.
\index[persons]{Fox, Ralph}%
Hale Trotter odkrył rok później nieskończoną rodzinę nieodwracalnych precli, patrz \ref{prp:pretzel_not_invertible}.
\index[persons]{Trotter, Hale}%

% MAKOTO SAKUMA - A SURVEY OF THE IMPACT OF THURSTON’S WORK ON KNOT THEORY
% Hartley [129] realized that one can apply this method to the problem of identifying noninvertible knots, as follows. Suppose no automorphism of Γ maps γ to γ−1. Then the set R(G(K), Γ, γ) is possibly different from the set R(G(K), Γ, γ−1), and there is a chance to show noninvertibility of K by comparing the homology invariants associated with φ ∈ R(G(K), Γ, γ) with those associated with φ′ ∈ R(G(K), Γ, γ−1). Hartley showed that this method is quite effective: he completely determined the 36 non-invertible knots up to 10 crossings claimed by Conway to be noninvertible.

\begin{example}
    Węzeł $12a_{427}$ jest zwierciadlany dodatnio, ale nie odwracalny.
\end{example}

Żaden inny węzeł pierwszy o mniej niż 13 skrzyżowaniach nie ma tej cechy.

\begin{example}
\label{property_of_eight_knot}%
    Ósemka $4_1$ jest całkowicie zwierciadlana.
\end{example}

To najprostszy typ symetrii, wystarczy jawnie wskazać przekształcenie między diagramem węzła, jego lustra oraz odwrotności.

\label{con:tait_fourth}%
Tait odnosił wrażenie, że zwierciadlane węzły mają parzysty indeks skrzyżowań, ale Hoste, Thistlethwaite znaleźli w~1998 kontrprzykład o~piętnastu skrzyżowaniach, $15_{700}$. % wg https://mathworld.wolfram.com/AmphichiralKnot.html
(Czwarta) hipoteza Taita jest prawdziwa dla węzłów pierwszych, alternujących.
\index{hipoteza Taita}%

Poniższa tabela oparta jest (kolejno) o~ciągi
\href{https://oeis.org/A051766}{51766},
\href{https://oeis.org/A051769}{51769},
\href{https://oeis.org/A051768}{51768},
\href{https://oeis.org/A051767}{51767},
\href{https://oeis.org/A052400}{52400},
z bazy danych ``The On-Line Encyclopedia of Integer Sequences'' (OEIS).

\begin{table}[h]
    \centering
    \begin{tabular}{@{}*{20}l@{}} \toprule
        skrzyżowania & 3 & 4 & 5 & 6 & 7 & 8 & 9 & 10 & 11 & 12 & 13 & 14 \\ \midrule
        całkowicie skrętne & 0 & 0 & 0 & 0 & 0 & 0 & 2 & 27 & 187 & 1103 & 6919 & 37885 \\
        odwracalne & 1 & 0 & 2 & 2 & 7 & 16 & 47 & 125 & 365 & 1015 & 3069 & 8813 \\
        $-$ zwierciadlane & 0 & 0 & 0 & 0 & 0 & 1 & 0 & 6 & 0 & 40 & 0 & 227 \\
        $+$ zwierciadlane & 0 & 0 & 0 & 0 & 0 & 0 & 0 & 0 & 0 & 1 & 0 & 6 \\
        zwierciadlane & 0 & 1 & 0 & 1 & 0 & 4 & 0 & 7 & 0 & 17 & 0 & 41 \\
        \bottomrule
        \hline
    \end{tabular}
    \caption{Liczba węzłów pierwszych o~poszczególnych typach symetrii}
\end{table}

\begin{definition}
    Niech $K \subseteq S^3$ będzie węzłem.
    Jeśli istnieje inwolucja pary $(S^3, K)$, która zachowuje orientację sfery, ale odwraca orientację węzła, to węzeł $K$ nazywamy silnie odwracalnym.
\end{definition}

To jest definicja 10.3.2 z monografii Kawauchiego \cite{kawauchi1996}.

\begin{proposition}
    Jeśli węzeł jest silnie odwracalny, to jest też odwracalny.
\end{proposition}

Hipotezę, że każdy odwracalny węzeł jest też silnie odwracalny, postawił Montesinos \cite[problem 1.6]{kirby1978}, on też zdefiniował klasę silnie odwracalnych węzłów \cite{montesinos1975}.
\index[persons]{Montesinos, José}%
Jednakże...

\begin{proposition}
    Istnieją odwracalne węzły, które nie są silnie odwracalne.
\end{proposition}

\begin{proof}
\index[persons]{Hartley, Richard}%
\index[persons]{Whitten, Wilbur}%
    Stosowne przykłady podali niezależnie od siebie Hartley \cite[s. 183]{hartley1980} oraz Whitten \cite{whitten1981} (węzeł $K$ jest silnie odwracalny wtedy i tylko wtedy, gdy każdy jego dubel jest silnie nieodwracalny, wynika stąd, że pewien dubel $8_{17}$ nie jest silnie odwracalny; jednocześnie Schubert \cite[s. 235]{schubert1953} pokazał, że duble są odwracalne).
\end{proof}

Ale hiperboliczny węzeł odwracalny jest silnie odwracalny, wspomina o tym bez dowodu Kawauchi.

\begin{proposition}
    Każdy wielomian Alexandera jest realizowany przez pewien silnie odwracalny węzeł.
\end{proposition}

\begin{proof}
\index[persons]{Sakai, Tsuyoshi}%
    Sakai konstruuje w \cite{sakai1983} silnie odwracalny węzeł o dowolnie wybranym cyklicznym module Alexandera.
\end{proof}

% Koniec podsekcji Lustro i rewers




\subsection{Węzły okresowe}
\index{węzeł!okresowy|(}%
Można wyróżnić jeszcze jeden rodzaj symetrii.

% DICTIONARY;period;okres;-
% DICTIONARY;periodic;okresowy;węzeł
\begin{definition}
\label{def:period}%
    Węzeł $K$ nazywamy $n$-okresowym, jeśli istnieje obrót $f \colon \R^3 \to \R^3$ o~kąt $2\pi/n$ wokół pewnej prostej $l$, rozłącznej z~węzłem $K$, taki że $f(K) = K$.
\end{definition}

Zamiast obrotów można rozpatrywać dowolne odwzorowania okresowe $f \colon S^3 \to S^3$, których zbiór punktów stałych jest rozłączny z węzłem $K$, homeomorficzny z $S^1$ oraz które trzymają węzeł $K$ w miejscu, ale dostaje się wtedy dokładnie taką samą klasę węzłów.
Czemu?
Wynika to z hipotezy Smitha, otrzymanej z połączenia głębokich teorii dotyczących geometrii i topologii 3-rozmaitości.
\index{hipoteza!Smitha}%
Kawauchi \cite[s. 125]{kawauchi1996} odsyła tu do książki Morgana, Bassa \cite{morgan1984}, gdzie znajdziemy problem, jego historię i rozwiązanie.
\index[persons]{Morgan, John}%
\index[persons]{Bass, Hyman}%

\begin{proposition}
    Zbiór wszystkich okresów jest niezmiennikiem węzłów.
\end{proposition}

Nieodwracalny węzeł $8_{17}$ nie posiada żadnych okresów.
% ćwiczenie 10.1.5 w Kawauchi
Węzeł $5_1$ jest 5-okresowy, co widać na standardowym diagramie, oraz 2-okresowy, tę drugą symetrię można dostrzec na diagramie realizującym liczbę mostową.
Trójlistnik ma dokładnie dwa okresy, $2$ i~$3$.
Ogólniej, jak głosi Kawauchi \cite[ćwiczenie 10.1.9]{kawauchi1996}:

\begin{proposition}
    Jedynymi okresami węzła $(p, q)$-torusowego są dzielniki liczb $p$ oraz $q$.
\end{proposition}

Murasugi podał dwa warunki, które musi spełniać węzeł o~okresie $n = p^r$, gdzie $r$ jest liczbą pierwszą.
Do ich zrozumienia potrzebujemy dwóch definicji.

\begin{definition}
    Niech $f$ będzie obrotem z definicji \ref{def:period}, zaś $p \colon \R^3 \to \R^3/f \simeq \R^3$ rzutem na przestrzeń ilorazową.
% DICTIONARY;quotient;ilorazowy;węzeł
\index{węzeł!ilorazowy}%
    Wtedy $p(K)$ nazywamy \emph{węzłem ilorazowym}, zaś $K$ to jego $n$-krotne nakrycie.    
\end{definition}

Ładny rysunek węzła ilorazowego znaleźliśmy u Kawauchiego \cite[s. 122]{kawauchi1996}.

\begin{definition}
    Niech $K$ będzie zorientowanym węzłem, zaś $l$ zorientowaną półprostą, która nie jest styczna do węzła $K$.
    Wtedy różnicę między liczbą skrzyżowań dodatnich oraz ujemnych wzdłuż półprostej (bez znaku) nazywamy indeksem zaczepienia $\lambda$ węzła $p(K)$.
\end{definition}

\begin{proposition}[warunek Murasugiego]
\index{warunek!Murasugiego}%
\label{prp:murasugi_periodic}%
    Niech $K$ będzie węzłem o~okresie $n = p^r$, gdzie $p$ jest liczbą pierwszą.
    Niech $J$ będzie jego węzłem ilorazowym, z~indeksem zaczepienia $\lambda$.
    Wtedy wielomian $\alexander_J$ jest dzielnikiem wielomianu $\alexander_K$ oraz istnieje pewna całkowita liczba $k$, taka że
    \begin{equation}
        \alexander_K(t) \equiv \pm t^k \alexander_J(n)^n \left(1 + t + t^2 + \ldots + t^{\lambda - 1}\right)^{n-1} \mod p.
    \end{equation}
\end{proposition}

\begin{proof}[Niedowód]
    Mozolne operacje na macierzach, których wyznacznikiem jest wielomian Alexandera, patrz \cite{murasugi1971}.
    Kawauchi przedstawia inny dowód: najpierw dowodzi tego dla węzła torusowego $T_{n, d}$, którego węzłem ilorazowym jest niewęzeł.
    W ogólnym przypadku, korzysta z relacji kłębiastej dla wielomianu Conwaya.
    Szczegóły oraz odsyłacze do dalszych prac znaleźć można w jego przeglądowej publikacji \cite[s. 122-124]{kawauchi1996}.
\end{proof}

Z prac Borodzika (m.in. \cite{grabowski20} napisanej z Grabowskim, Królem i Marchwicką) dowiedzieliśmy się, że podany wyżej warunek Murasugiego jest tylko jednym z wielu ograniczeń, jakie musi spełniać węzeł okresowy.
Autorzy wymieniają:
\begin{itemize}
    \item warunek Murasugiego udoskonalony przez Davisa, Livingstona \cite{davis1991},
\index[persons]{Davis, James}%
\index[persons]{Livingston, Charles}%
    \item kryterium Naika z homologiami rozgałęzionego nakrycia \cite{naik1997},
\index[persons]{Naik, Swatee}%
\index{homologie}%
\index{rozgałęzione nakrycie}%
    \item kryterium Traczyka z wielomianem Jonesa \cite{traczyk1991},
\index{wielomian Jonesa}%
\index[persons]{Traczyk, Paweł}%
    \item kryterium Przytyckiego z wielomianem HOMFLY-PT \cite{przytyckij1989},
\index{wielomian HOMFLY-PT}%
\index[persons]{Przytycki, Józef}%
    \item kryterium Naika z niezmiennikiem Cassona-Gordona,
\index{niezmiennik Cassona-Gordona}%
\index[persons]{Naik, Swatee}%
    \item kryterium Hillmana, Livingstona, Naika ze skręconym wielomianem Alexandera \cite{hillman2006},
\index{wielomian Alexandera!skręcony}%
\index[persons]{Hillman, Jonathan}%
\index[persons]{Livingston, Charles}%
\index[persons]{Naik, Swatee}%
    \item kryterium Jabuki, Naika z homologią Floera \cite{jabuka2016},
\index{homologia Floera}%
\index[persons]{Jabuka, Stanisław}%
\index[persons]{Naik, Swatee}%
    \item kryteria Borodzika, Politarczyka z homologią Chowanowa, \cite{politarczyk2017}, \cite{politarczyk2021},
\index{homologia Chowanowa}%
\index[persons]{Borodzik, Maciej}%
\index[persons]{Politarczyk, Wojciech}%
    \item kryterium Chena \cite{chen2018} z grupą podstawową.
\index{grupa podstawowa}%
\index[persons]{Chen, Haimiao}%
\end{itemize}
% Sakumy nie ma, bo nie został wymieniony ze słowem criterion; być może napisał coś ogólnego? 
\index{kryterium Naika (okresowości)}%

\index{węzeł!okresowy|)}%

% koniec podsekcji Węzły okresowe




\section{Suma niespójna i~suma spójna}

\begin{definition}[suma niespójna]
% DICTIONARY;distant union;suma niespójna;-
\index{suma niespójna}%
    Niech $L_1$ oraz $L_2$ będą splotami, które leżą po różnych stronach ustalonej płaszczyzny w przestrzeni $\R^3$.
    Teoriomnogościową sumę $L_1 \sqcup L_2$ nazywamy sumą niespójną splotów $L_1$ i $L_2$.
\end{definition}

\begin{definition}[suma spójna]
% DICTIONARY;connected sum;suma spójna;-
\index{suma spójna}%
\label{def:connected_sum}%
    Niech $K_1, K_2$ będą zorientowanymi węzłami.
    Natnijmy każdy z nich w dwóch punktach tego samego krótkiego łuku, a następnie zszyjmy dwoma łukami, które nie przecinają już istniejących, jak na obrazku.
    Otrzymany węzeł nazywamy sumą spójną węzłów $K_1$ oraz $K_2$ i oznaczamy przez $K_1 \shrap K_2$.
\end{definition}

Sumę spójną wprowadził Schubert \cite{schubert1949}, jak wspominają Burde, Zieschang, Heusener \cite[s. 21]{burde2014}.
\index[persons]{Schubert, Horst}%

\begin{example}
    Suma spójna $10_{17} \shrap 9_{44}$:
    \[
        \raisebox{-0.42\height}{\includegraphics[height=2.2cm]{../data/connected-sum/knot_sum_a.png}}
        \scalebox{2.5}{\ensuremath{\shrap}}
        \raisebox{-0.42\height}{\includegraphics[height=2.2cm]{../data/connected-sum/knot_sum_b.png}}
        \scalebox{2.5}{\ensuremath{=}}
        \raisebox{-0.42\height}{\includegraphics[height=2.2cm]{../data/connected-sum/knot_sum_ab.png}}
    \]
\end{example}

W topologii rozważa się podobną operację: z~każdej $n$-rozmaitości wycina się kulę, po czym skleja wzdłuż brzegowej sfery w~jedną rozmaitość.
Ale kiedy zajmujemy się węzłami, nie interesuje nas struktura rozmaitości (gdyż każdy węzeł jest homeomorficzny z~okręgiem), tylko zanurzenie w otaczającą przestrzeń.
Pojęcie sumy spójnej węzłów (oraz opisane później satelity) wprowadził do matematyki Schubert \cite{schubert1949}.
\index[persons]{Schubert, Horst}%

Żadnego elementu definicji \ref{def:connected_sum} nie można pominąć:
\begin{itemize}
    \item \emph{składniki $K_1, K_2$ muszą być zorientowane}. 
    Węzeł prosty, czyli suma dwóch przeciwnie skręconych trójlistników, ma zerową sygnaturę i jest plastrowy.
    \index{węzeł!plastrowy}%
    \index{sygnatura}%
    Węzeł babski, czyli suma tak samo skręconych trójlistników ma niezerową sygnaturę.
    (To jedno z niewielu miejsc, gdzie nomenklatura pochodzi od żeglarzy.).
    \label{two_sums_of_two_trefoils}%
    Uzasadnienie, że te węzły są różne, nie jest łatwym zadaniem.
    Fox twierdzi, że Seifert \cite{seifert1933} wiedział o tym.
    Pokazał też w~króciutkim artykule \cite{fox1952}, że ich dopełnienia nie są homeomorficzne.
    \item \emph{składniki $K_1, K_2$ muszą być węzłami, nie splotami}. Nie istnieje kanoniczny wybór, które ogniwa łączyć ze sobą.
    \item \emph{zszywajace łuki nie mogą przecinać diagramów}.
    Cromwell \cite[s. 90]{cromwell2004} pokazuje przykład dwóch niewęzłów, z~których otrzymano niepoprawnie dwie różne sumy, $6_1$ oraz $8_{20}$.
\end{itemize}

\begin{proposition}
    Suma spójna węzłów jest dobrze określonym działaniem.
    Jest przemienna oraz łączna; niewęzeł stanowi jej element neutralny.
\end{proposition}

\begin{proof}
    Niech dane będą węzły $K_1$ oraz $K_2$ oraz dwa różne łuki $\gamma_1$, $\gamma_2$, których można użyć do konstrukcji sumy spójnej.
    Skurczmy $K_1$ tak, by był bardzo mały, przeciągnijmy najpierw przez łuk $\gamma_1$, a~następnie wzdłuż węzła $K_2$ do miejsca, gdzie zaczyna się łuk $\gamma_2$.
    Na koniec odwróćmy proces, z łukiem~$\gamma_2$ w~miejscu łuku $\gamma_1$.

    Prosty dowód drugiego zdania pozostawiamy Czytelnikowi.
\end{proof}

Algebra powiedziałaby, że węzły z~sumą spójną tworzą półgrupę, podobnie jak liczby naturalne z~działaniem dodawania.
Do bycia grupą brakuje istnienia elementów przeciwnych. 
Znane są nam co najmniej trzy różne sposoby na uzasadnienie tego faktu.

Najpierw wywnioskowaliśmy to z~własności powierzchni Seiferta i~ich genusu: faktów \ref{prp:genus_detects_unknot} oraz \ref{prp:genus_of_sum}.
O~tym samym dowodzie wspomina Kawauchi \cite[s. 33]{kawauchi1996}, a~fakt nazywa twierdzeniem o~nieanulowaniu.
Potem poznaliśmy szwindel Mazura, technikę dowodzenia bardziej przystępną dla kogoś, kto nie zna topologii algebraicznej, ale niestety wykorzystującą węzły dziki.
Długo myśleliśmy, że inaczej się nie da, ale elementarny dowód istnieje!
% odkryte w https://aperiodical.com/2018/07/the-big-internet-math-off-round-1-jim-propp-v-zoe-griffiths/
Trzeba zajrzeć do \cite[s. 18-20]{kauffman1995} dla dwóch rysunków tamże.

\begin{proposition}
\label{first_time_sum_is_trivial}%
    Niech $K_1, K_2$ będą takimi węzłami, że $K_1 \shrap K_2 = \SmallUnknot$. Wtedy $K_1 = K_2 = \SmallUnknot$.
\end{proposition}

\begin{proof}[Niedowód]
% DICTIONARY;Mazur swindle;szwindel Mazura;-
    Technika ta zwana jest szwindlem Mazura.
\index{szwindel Mazura}%
    Załóżmy, że $K \shrap L = \SmallUnknot$ i~dopuśćmy wyjątkowo węzły dzikie.
    Skonstruujmy sumę $K \shrap L \shrap K \shrap \ldots$,
    przy czym kolejne składniki powinny zmniejszać się,
    aby ich suma nadal była węzłem.
    Wtedy
    \begin{align*}
        K & \simeq K \shrap [(L \shrap K) \shrap (L \shrap K) \ldots] \\
         & \simeq (K \shrap L) \shrap (K \shrap L) \shrap \ldots
         \simeq \SmallUnknot \shrap \SmallUnknot \shrap \ldots
         \simeq \SmallUnknot.
    \end{align*}
    Analogicznie pokazujemy, że $L \simeq \SmallUnknot$.
    To jedyne miejsce w~całej książce, gdzie użyte zostają (zostały?) węzły dzikie.
\end{proof}

\begin{proof}
    (Jak pan Kauffman \cite[s. 18-20]{kauffman1995} napisał).
    Wyobraźmy sobie, że węzeł oraz torus połykająco-podążający $T$ został zawieszony między dwiema ścianami pokoju i~załóżmy nie wprost, że suma $K = K_1 \shrap K_2$ jest trywialna.
    Wtedy pewien homeomorfizm pokoju, który nie rusza ścian, prostuje sumę: zamienia pozornie zaplątany węzeł $K$ w odcinek $L$. 

    Niech $\pi$ będzie dowolną płaszczyzną zawierającą wyprostowaną sumę $K$.
    Tnie część wspólną torusa $T$ oraz ścian w czterech punktach, oznaczmy je $A, B$ (na lewej ścianie) oraz $C, D$ (na prawej).
    Zauważmy, że $\pi$ tnie $T$ w łukach, które wychodzą z $A, B, C, D$ oraz pewnych zamkniętych krzywych.
    Łuk wychodzący z~punktu $A$ nie może łączyć go z punktami $B$ lub $D$, ponieważ te leżą po drugiej stronie odcinka $L$ na płaszczyźnie $\pi$.
    
    Łuk $AC$ przedstawia niewęzeł.
    Jednocześnie jest on obrazem pewnego łuku, który łączył końce torusa $T$, zatem musi być równoważny z~węzłem-towarzyszem.
\end{proof}

Półgrupę węzłów z~operacją sumy spójnej można uczynić grupą na dwa sposoby: albo zmieniając działanie, albo osłabiając równoważność węzłów.
Drugi pomysł jest lepszy niż pierwszy.
Na początku lat pięćdziesiątych Milnor wprowadził pojęcie zgodności.
\index[persons]{Milnor, John}%
\index{węzeł!plastrowy}%
\index{węzeł!zgodny}%
Element neutralny nowej grupy to węzły plastrowe, ich opis leży w~sekcji \ref{sec:slice}.
Zgodność i plastrowe węzły to zagadnienia zakorzenione w~czterowymiarowej topologii.

Kawauchi \cite[s. 50-53]{kawauchi1996} opisuje $2n$-sumę Murasugiego tak, że $2$-suma to nasza suma spójna, zaś $4$-suma to \textsc{,,plumbing''} (coś, czego nie znamy) i dodaje komentarz, że jest bardzo przydatna do badania powierzchni Seiferta czy genusu.
\index{plumbing}%
\index{suma Murasugiego}%
Została wprowadzona dawno temu w~\cite{murasugi1958}, by szacować stopień wielomianu Alexandera alternujących węzłów.
\index[persons]{Murasugi, Kunio}%
\index{wielomian!Alexandera}%
\index{węzeł!alternujący}%

% DICTIONARY;suma paskowa;band sum;-
Innym uogólnieniem jest suma paskowa, patrz \cite[s. 31-32, 43]{kawauchi1996}, specjalny przypadek hiperbolicznej transformacji splotu oraz fuzji splotu.
\index{suma paskowa}%
% TODO: sprawdzić, czy fuzja splotu ma trafić do indeksu

% Koniec podsekcji Suma niespójna i suma spójna





% DICTIONARY;prime;pierwszy;węzeł
% DICTIONARY;composite;złożony;węzeł
\section{Węzły pierwsze}
\index{węzeł!pierwszy|(}%
Suma spójna jest dla węzłów tym, czym mnożenie dla liczb naturalnych.
Analogia ta nabiera sensu, gdy zdefiniujemy węzły pierwsze, odpowiedniki liczb pierwszych.
Do ich dobrego zrozumienia warto znać powierzchnie Seiferta (ale przy pierwszym czytaniu nie trzeba).

\begin{definition}[węzeł pierwszy]
\label{def:prime_knot}%
    Niech $K$ będzie węzłem różnym od niewęzła.
    Jeśli nie przedstawia się jako suma spójna $K_1 \shrap K_2$ dwóch nietrywialnych węzłów $K_1, K_2$, nazywamy go węzłem pierwszym.
    W~przeciwnym razie mówimy, że jest złożony.
\index{węzeł!złożony}%
\end{definition}
% TODO: jak przedłużyć mimo niejednoznaczności na sploty    ?

Pokażemy później, że rozkład na węzły pierwsze istnieje i~wspomnimy, dlaczego jest jedyny.
Fakt \ref{prp:genus_of_sum} stanowi odpowiednik zasadniczego twierdzenia arytmetyki (i jest od niego trochę trudniejszy w~dowodzie).
Każdy węzeł jest sumą spójną siebie oraz niewęzła, dlatego byłoby miło, gdyby niewęzeł nie dał się zapisać jako suma dwóch innych węzłów.
Jest to dokładnie wniosek \ref{cor:connected_sum_no_inverses}: suma spójna nie posiada elementów odwrotnych.

Pewne kryteria pierwszości konkretnych splotów oparte o prace Nakanishiego na temat supłów znaleźć można u Kawauchiego \cite[s. 38-41]{kawauchi1996}.

Przesmyk to wąskie skrzyżowanie między dwiema rozłącznymi częśćmi diagramu.
\index{przesmyk}%

\begin{proposition}
\index{splot!rozszczepialny}%
    Niech $L$ będzie alternującym splotem bez przesmyków.
    Jeśli diagram splotu $L$ jest spójny, to splot jest nierozszczepialny.
    Jeśli splot nie jest rozszczepialny, to jest też pierwszy jeśli dla każdego okręgu przecinającego diagram w dwóch niepodwójnych punktach, przekrój wnętrza okręgu z~diagramem jest łukiem.
\end{proposition}

Innymi słowy, jeśli alternujący splot jest złożony, widać to bezpośrednio na każdym jego alternującym diagramie.
Jako pierwszy pokazał to Menasco \cite{menasco1984}.
\index[persons]{Menasco, William}%
Jego dowód opiera się na multiplikatywności wielomianu BLM/Ho.
\index{wielomian!BLM/Ho}%

Czy węzłów pierwszych jest nieskończenie wiele?
% achtung
Tak, patrz fakt \ref{prp:infinitely_many_prime_knots}, potrafimy nawet oszacować liczbę $K_n$ węzłów pierwszych oraz $L_n$ splotów pierwszych.
W roku 1987 Ernst, Sumners \cite{ernst1987} w~oparciu o~dowód hipotez Taita pokazali, że $K_n \ge \frac 1 3 (2^{n- 2} - 1)$, przy czym węzły lustrzane traktowane są jako różne.
\index[persons]{Ernst, Claus}%
\index[persons]{Sumners, De Witt}%
Dokładniej:

\begin{proposition}
\index{węzeł!dwumostowy}%
    Niech $f(n)$ oznacza liczbę węzłów dwumostowych o indeksie skrzyżowaniowym $n$.
    Wtedy
    \begin{equation}
        f(n) = \begin{cases}
        \frac 13 (2^{n-2} - 1) & \text{dla } n = 2k \ge 4 \\
        \frac 13 (2^{n-2} + 2^{(n-1)/2}) & \text{dla } n = 4k + 1 \ge 5 \\
        \frac 13 (2^{n-2} + 2^{(n-1)/2} + 2) & \text{dla } n = 4k + 3 \ge 7
        \end{cases}
    \end{equation}
\end{proposition}

Welsh \cite{welsh1992} rozpatruje węzły bez orientacji i znajduje poniższe ograniczenia:
\index[persons]{Welsh, Dominic}%

\begin{proposition}
    Niech $K_n$ (odpowiednio: $L_n$) oznacza liczbę węzłów (splotów) pierwszych o $n$ skrzyżowaniach.
    % TODO: n? co najwyżej n? coś innego?
    % TODO: linijka pod % achtung używa K_n przed zdefiniowaniem K_n
    Wtedy
    \begin{equation}
        2.68 \le \liminf_{n \to \infty} \sqrt[n]{K_n} \le \limsup_{n \to \infty} \sqrt[n]{L_n} \le \frac {27}{2}.
    \end{equation}
\end{proposition}

\begin{proof}[Niedowód]
    Dolne ograniczenie bierze się ze zliczenia alternujących splotów Montesinosa.
    Dowód górnego korzysta z teoriografowych wyników Tuttego oraz wzoru Stirlinga.
\end{proof}

Stojmenow \cite{stoimenowb2004} pokazał jakiś czas później, jak poprawić górne ograniczenie do
\begin{equation}
    \limsup_{n \to \infty} \sqrt[n]{L_n} \le \frac{\sqrt{13681} + 91}{20} \approx 10.398
\end{equation}
i wytłumaczył, czemu jego metody z funkcjami tworzącymi nie można użyć do dalszych poprawek.
\index[persons]{Stojmenow, Alexander}%
Dla splotów dolną granicą jest co najmniej $4$ (w miejsce $2.68$).

Pytanie, czy zwykłe granice istnieją, pozostaje otwarte (stan na 2004?).

\index{węzeł!pierwszy|)}

% koniec sekcji Węzły pierwsze



\part{Niezmienniki}
\chapter{Niezmienniki numeryczne}

Wspomnieliśmy na s. \pageref{page_first_invariant}, że pytanie, czy dwa dane diagramy splotów przedstawiają ten sam czy różne sploty, bywa trudne.
Napomknęliśmy też, że świadkiem równości dwóch splotów jest ciąg ruchów Reidemeistera, natomiast niezmienników używa się do wykazania różności.
Czym jest niezmiennik?
To pewna wielkość, która nie ulega zmianie (w naszym przypadku: podczas izotopii otaczającej przestrzeni, w której jest zanurzony splot).
Jeśli dany niezmiennik przyjmuje różne wartości dla dwóch splotów, to nie mogą być równoważne.

Jak pisze Przytycki w~,,Dwustu latach teorii węzłów'', kiedy badamy nowy niezmiennik, powinniśmy zadać sobie trzy pytania.
\index[persons]{Przytycki, Józef}%
Dobrze byłoby o nich nie zapominać.
Oto pytania Przytyckiego:
\begin{enumerate}
    \item czy łatwo wyznaczyć wartość niezmiennika?
    \item czy w zbiorze wartości niezmiennika łatwo odróżnia się elementy?
    \item czy niezmiennik odróżnia wiele splotów?
\end{enumerate}

Poznaliśmy dotychczas dwa niezmienniki: liczbę ogniw (którą łatwo wyznaczyć i której wartości -- liczby naturalne -- łatwo odróżniać; ale nie odróżnia wielu splotów) oraz topologię dopełnienia splotu (tym razem odróżnia wszystkie węzły pierwsze, ale trudno wyznaczyć jej ,,wartość'').
Tutaj przedstawiamy kilka więcej; przede wszystkim te, które nie wymagają mocnej znajomości reszty książki.
Niektóre z~nich są miarą złożoności splotów zgodnie z~następującym przepisem: niech $f$ będzie pewną funkcją określoną dla dowolnego diagramu splotu.
Wtedy odwzorowanie
\begin{equation}
    f(L) = \min \{f(D) : D \text{ jest diagramem splotu } L\}
\end{equation}
stanowi niezmiennik splotów.
Dowód jest trywialny i~pozostawiamy go jako ćwiczenie dla Czytelnika.
Im większa wartość funkcji $f$, tym bardziej skomplikowany splot.

Później poznamy inne niezmienniki, oprócz opisanych poniżej miarą złożoności jest też liczba warkoczowa (definicja \ref{def:braid_number}), ale nie wyznacznik (definicja \ref{def:determinant}) czy sygnatura (definicja \ref{def:signature}).
Przekonamy się też, że istnieją użyteczne niezmienniki, które są wielomianami albo innymi obiektami algebraicznymi.

\input{30-invariants-numeric/301a-crossing}


% DICTIONARY;unknotting number;liczba gordyjska;-
\section{Liczba gordyjska}
\index{liczba!gordyjska|(}%

\begin{definition}
    Niech $L$ będzie splotem.
    Minimalną liczbę skrzyżowań, które trzeba odwrócić na pewnym jego diagramie, by dostać niewęzeł, nazywamy liczbą gordyjską i~oznaczamy $\unknotting L$.
\end{definition}

Zgodnie z ,,klasyczną'' definicją, między odwracaniem kolejnych skrzyżowań mamy prawo wykonać izotopie otaczające; natomiast zgodnie ze ,,standardową'' definicją, takie izotopie są zabronione.
Obie definicje są równoważne: tłumaczy to książka Adamsa \cite[s. 58]{adams1994}.
(Książka nie tłumaczy, czemu te dwie definicje zostały określone akurat takimi przymiotnikami.)

\begin{lemma}
\label{lem:unknotting_well_defined}%
    W dowolnym rzucie splotu można odwrócić pewne skrzyżowania tak, by uzyskać diagram niesplotu.
\end{lemma}

(To jest \cite[ćwiczenie E 1.6, s. 15]{burde2014}.)

\begin{proof}
    Bez straty ogólności załóźmy, że diagram przedstawia węzeł.
    Ustalmy zatem diagram węzła i~wybierzmy jakiś początkowy punkt na nim, różny od skrzyżowania wraz z~kierunkiem, wzdłuż którego będziemy przemierzać węzeł.
    Za każdym razem, kiedy odwiedzamy nowe skrzyżowanie, zmieniamy je w~razie potrzeby na takie, przez które przemieszczamy się wzdłuż górnego łuku.
    Skrzyżowań już odwiedzonych nie zmieniamy wcale.

    Teraz wyobraźmy sobie nasz nowy węzeł w~trójwymiarowej przestrzeni $\mathbb R^3$, przy czym oś $z$ skierowana jest z~płaszczyzny, w~której leży diagram, w~naszą stronę.
    Umieśćmy początkowy punkt tak, by jego trzecią współrzędną była $z = 1$.

    Przemierzając węzeł, zmniejszamy stopniowo tę współrzędną, aż osiągniemy wartość $0$ tuż przed punktem, z~którego wyruszyliśmy.
    Połączmy obydwa punkty (początkowy oraz ten, w~którym osiągamy współrzędną $z = 0$) pionowym odcinkiem.
    Zauważmy, że kiedy patrzymy na węzeł w~kierunku osi $z$, nie widzimy żadnych skrzyżowań.

    Oznacza to, że nasza procedura przekształciła początkowy diagram w~diagram niewęzła, co należało okazać.
\end{proof}

Nakanishi \cite{nakanishi1983} znalazł 2-gordyjski diagram 1-gordyjskiego węzła $6_2$, a po trzynastu latach udowodnił, że każdy nietrywialny węzeł ma diagram, który nie jest 1-gordyjski \cite{nakanishi1996}.
\index[persons]{Nakanishi, Yasutaka}%
Jego wyniki uogólnia praca Taniyamy \cite{taniyama2009}: dla każdego nietrywialnego splotu istnieje diagram wymagający odwrócenia dowolnie wielu skrzyżowań.
\index[persons]{Taniyama, Kouki}%

\begin{proposition}
    Niech $L$ będzie nietrywialnym splotem.
    Dla każdej liczby naturalnej $n \in \N$ istnieje diagram $D$ splotu $L$ taki, że $\unknotting D \ge n$.
\end{proposition}

Pokazany jest tam jeszcze jeden godny uwagi fakt.
Jeśli odwrócenie pewnych skrzyżowań daje niewęzeł, to odwrócenie pozostałych także.
Zatem dla splotów $L$ o $n$ skrzyżowaniach mamy $2 \crossing L \le n$.
Nie jest to zbyt pomocne, daje rozstrzygnięcie pięć razy dla pierwszych węzłów do 12 skrzyżowań: $3_{1}$, $5_{1}$, $7_{1}$, $9_{1}$, $11a_{367}$.
Ale...

\begin{proposition}
    Jeśli liczba gordyjska diagramu $D$ węzła $K$ wynosi $\frac 12 (\crossing D - 1)$, to węzeł jest $(2,p)$-torusowy albo wygląda jak diagram niewęzła po pierwszym ruchu Reidemeistera.
\end{proposition}

W powyższym stwierdzeniu nie można zastąpić słowa ,,węzeł'' przez ,,splot''.

\input{30-invariants-numeric/301ba-unknotting_one}


\subsubsection{Znane wartości}
Cha, Livingston \cite{cha2018} podają, że znamy liczby gordyjskie wszystkich węzłów pierwszych do dziesięciu skrzyżowań poza dziewięcioma wyjątkami: $10_{11}$, $10_{47}$, $10_{51}$, $10_{54}$, $10_{61}$, $10_{76}$, $10_{77}$, $10_{79}$, $10_{100}$ (gdzie nie mamy pewności, czy $\unknotting = 2$, czy $\unknotting = 3$).
\index[persons]{Cha, Jae}%
\index[persons]{Livingston, Charles}%
Kto pierwszy znalazł liczbę gordyjską którego węzła ostaraliśmy się bezbłędnie przepisać z bazy danych KnotInfo\footnote{Patrz \url{https://knotinfo.math.indiana.edu/descriptions/unknotting_number.html}}.
Według KnotInfo oprócz węzłów torusowych, tych wymienionych poniżej oraz 1-gordyjskich, do 10 skrzyżowań mamy jeszcze 2 węzły o~siedmiu skrzyżowaniach, 3 o~ośmiu, 15 o~dziewięciu i~68 o~dziesięciu, których liczba gordyjska zdaje się należeć do folkloru matematycznego.

{
    \setlength{\intextsep}{4pt plus 2pt minus 2pt}
\begin{table}[H]
    \raggedright
    \footnotesize
    \centering
    \begin{tabular}{l|p{100mm}} \toprule
    rok & węzły i odkrywcy ich liczb gordyjskich \\ \midrule
    1982 & $7_{4}$ (Lickorish \cite{lickorish1985}) \\
    1986 & $8_{4}, 8_{6}, 8_{8}, 8_{12}, 9_{5}, 9_{8}, 9_{15}, 9_{17}, 9_{31}$ (Kanenobu, Murakami \cite{kanenobumurakami1986}) \\
    1989 & $9_{25}$ (Kobayashi \cite{kobayashi1989}) \\
    1994 & $10_{8}$ (Adams \cite[s. 62]{adams1994}?) \\
    1998 & $10_{65}, 10_{69}, 10_{89}, 10_{97}, 10_{108}, 10_{163}, 10_{165}$ (Miyazawa \cite{miyazawa1998}), $10_{154}, 10_{161}$ (Tanaka \cite{tanaka1998}) \\
    1999 & $10_{67}$ (Traczyk \cite{traczyk1999}) \\
    2000 & $8_{16}$ (Murakami, Yasuhara \cite{yasuhara2000}) \\
    2002 & $10_{139}, 10_{145}, 10_{152}, 10_{154}, 10_{161}$ (Gibson, Ishikawa \cite{ishikawa2002}) \\
    2004 & $8_{18}, 9_{37}, 9_{40}, 9_{46}, 9_{48}, 9_{49}, 10_{86}, 10_{103}, 10_{105}, 10_{106}, 10_{109}, 10_{121}, 10_{131}$ (Stojmenow \cite{stoimenow2004}; ostatni węzeł zdaje się być jedynym 1-gordyjskim na tej liście!) \\
    2005 & $9_{29}$, $10_{79}$, $10_{81}$, $10_{87}$, $10_{90}$, $10_{93}$, $10_{94}$, $10_{96}$, $10_{148}$, $10_{151}$, $10_{153}$ (Gordon, Luecke \cite{gordon2006}), $8_{10}$, $10_{48}$, $10_{52}$, $10_{54}$, $10_{57}$, $10_{58}$, $10_{64}$, $10_{68}$, $10_{70}$, $10_{77}$, $10_{110}$, $10_{112}$, $10_{116}$, $10_{117}$, $10_{125}$, $10_{126}$, $10_{130}$, $10_{135}$, $10_{138}$, $10_{158}$, $10_{162}$ (Ozsváth, Szabó \cite{szabo2005}), $10_{83}$ (Nakanishi \cite{nakanishi2005}) \\
    2008 & $9_{10}, 9_{13}, 9_{35}, 9_{38}, 10_{53}, 10_{101}, 10_{120}$ (Owens \cite{owens2008}) \\
    \bottomrule
    \hline
    \end{tabular}
% 50 za dużo -< aż do przykład NB
% 25 trochę za dużo  \vspace{-25pt} ^ to samo
% 5 za mało
\end{table}
}
\index[persons]{Lickorish, William}%
\index[persons]{Kanenobu, Taizo}%
\index[persons]{Murakami, Hitoshi}%
\index[persons]{Kobayashi, Tsuyoshi}%
\index[persons]{Adams, Colin}%
\index[persons]{Miyazawa, Yasuyuki}%
\index[persons]{Tanaka, Toshifumi}%
\index[persons]{Traczyk, Paweł}%
\index[persons]{Yasuhara, Akira}%
\index[persons]{Gibson, William}%
\index[persons]{Ishikawa, Masaharu}%
\index[persons]{Stojmenow, Aleksander}%
\index[persons]{Gordon, Cameron}%
\index[persons]{Luecke, John}%
\index[persons]{Nakanishi, Yasutaka}%
\index[persons]{Szabó, Zoltán}%
\index[persons]{Ozsváth, Peter}%
\index[persons]{Owens, Brendan}%
\normalsize




\subsubsection{Dolne ograniczenia liczby gordyjskiej}
Dokładna wartość liczby gordyjskiej jest znana tylko dla niektórych klas węzłów, na przykład torusowych (fakt \ref{prp:torus_unknotting_number}) albo skręconych.
\index{węzeł!torusowy}%
\index{węzeł!skręcony}%

Borodzik oraz Friedl podali niedawno całkiem mocne ograniczenia na liczbę gordyjską w~pracach \cite{borodzik2014} i~\cite{borodzik2015}.
\index[persons]{Borodzik, Maciej}%
\index[persons]{Friedl, Stefan}%
Ich narzędziem jest parowanie Blanchfielda.
\index{parowanie Blanchfielda}%
Poprawiają tam starsze estymaty wynikające z~sygnatury Levine'a-Tristrama, indeksu Nakanishiego oraz przeszkody Lickorisha.
\index{indeks Nakanishiego}%
\index{przeszkoda Lickorisha}%
\index{sygnatura!Levine'a-Tristrama}%
% DICTIONARY;Lickorish obstruction;przeszkoda Lickorisha;-
Wśród pierwszych węzłów o~co najwyżej 12 skrzyżowaniach dwadzieścia pięć ma liczbę gordyjską równą co najmniej trzy, trudno było uzasadnić to innymi metodami.



\input{30-invariants-numeric/301bd-bleiler}

\input{30-invariants-numeric/301be-metric}

\subsubsection{Inne operacje rozwiązujące węzły}

Shimizu w pracy \cite{shimizu2014} rozpatruje różne operacje, które rozwiązują węzły lub sploty.
\index[persons]{Shimizu, Ayaka}%
Nie będziemy się nimi zajmować, podamy tylko przykład: zamiana pod- i nadskrzyżowań wokół obszaru na diagramie rozwiązuje węzły, ale nie sploty; kontrprzykładem jest splot Hopfa.
\index{splot!Hopfa}%
(Ciąg dalszy tamtej pracy razem z Oshikirim, Tamurą \cite{oshikiri2024}).
\index[persons]{Tamura, Junya}%
\index[persons]{Oshikiri, Tokio}%
Patrz też, co pisze Kawauchi w \cite[s. 141-154]{kawauchi1996}.

Mieliśmy też:

\begin{conjecture}
    Dowolny splot można rozwiązać wykonując ciąg 3-ruchów (zastępując dwie równoległe nici przez trzy półskręty lub odwrotnie).
\end{conjecture}

Ze zbioru problemów Kirby'ego \cite{kirby1978} wiemy, że Nakanishi zastanawiał się nad tym w 1981 roku.
\index[persons]{Nakanishi, Yasutaka}%
Nie to samo, ale podobne pytanie zadał wcześniej Montesinos w związku z nakryciami i~dlatego Kirby nazwał problem hipotezą Nakanishiego-Montesinosa.
\index[persons]{Montesinos, José}%
Conway zauważył, że hipoteza jest prawdziwa dla węzłów algebraicznych.
\index[persons]{Conway, John}%
Coxeter rozprawił się z nią dla prawie wszystkich splotów o~indeksie warkoczowym mniejszym niż $6$ oraz indeksie mostowym mniejszym niż $4$.
\index[persons]{Coxeter, Harold}%
Nakanishi w 1994 pokazał splot zbudowany z pierścieni Boromeuszy wobec którego podejrzewał, że jest kontrprzykładem.
\index{pierścienie Boromeuszy}%
Żeby zdobyć więcej informacji o postępie prac nad hipotezą, musieliśmy sięgnąć po artykuł Przytyckiego, Dąbkowskiego \cite{dabkowski2002}.
\index[persons]{Przytycki, Józef}%
\index[persons]{Dąbkowski, Mieczysław}%
Chen w~1999 zasugerował inny kontrprzykład, domknięcie 5-warkocza $(\sigma_1\sigma_2\sigma_3\sigma_4)^{10}$.
\index[persons]{Chen, Qi}%
Artykuł \cite{dabkowski2002} dowodzi, że te dwa sploty istotnie obalają hipotezę.
Używa się w~nim nieprzemiennej wersji $n$-kolorowań Foxa, tak zwanej $n$-tej grupy Burnside'a splotu.
\index{grupa!Burnside'a}%
\index{kolorowanie}%

Nakanishi w 1979, a więc zanim ogłosił powyższą hipotezę, miał wrażenie, że $4$-ruchy rozwiązują wszystkie sploty.
Najpierw sprawdzono, że jest prawdziwa dla wszystkich dwu- i trzymostowych węzłów, a także węzłów do 12 skrzyżowań, ale potem Askitas ogłosił, że pewien węzeł o 16 skrzyżowaniach obala ją.
\index[persons]{Askitas, Nikos}%
Później pojawili się inni podejrzani, ale nie wiemy, czy naprawdę są kontrprzykładami.

% znalezione przypadkiem w MR3143585
% 1979 Nakanishi: hipoteza że 4-ruch jest rozwiązujący
% dowody: 2-mostowe i 3-mostowe węzły, wszystkie do 12 skrzyżowań

\index{liczba!gordyjska|)}%

% Koniec podsekcji Liczba gordyjska



\input{30-invariants-numeric/301c-bridge}


\subsection{Spin}
\index{spin|(}%

Niektórzy, na przykład Przytycki, używają określenia ,,liczba Taita'', ale nam się ono nie podoba, więc proponujemy nasze, lepsze.
\index{liczba!Taita}%

% DICTIONARY;writhe;spin;-
\begin{definition}[spin]
    Niech $D$ będzie diagramem zorientowanego splotu $L$.
    Wtedy sumę wszystkich znaków skrzyżowań diagramu:
    \begin{equation}
        \writhe D = \sum_c \operatorname{sign} c,
    \end{equation}
    nazywamy jego spinem (diagramu, nie splotu!).
\end{definition}

Co ważne, spin nie jest niezmiennikiem splotów ani węzłów.
Para Perko przedstawia ten sam węzeł z~minimalną liczbą skrzyżowań i~spinem równym siedem lub dziewięć.
\index{para Perko}%
Dzięki temu przez wiele lat nie została dostrzeżona.
Spin jest za to niezmiennikiem węzłów alternujących, mówi o~tym druga hipoteza Taita.
\index{hipoteza!Taita}%

\begin{lemma}
\label{lem:writhe_reidemeister}%
    Spin nie zależy od orientacji diagramu.
    Tylko I ruch Reidemeistera zmienia spin:
\begin{comment}
    \begin{equation}
        \writhe \left(\MedLarReidemeisterOneLeft\right) =
        \writhe \left(\MedLarReidemeisterOneStraight\right) - 1.
    \end{equation}
\end{comment}
    Pozostałe ruchy nie mają na niego wpływu.
\end{lemma}

\index{spin|)}%

% Koniec sekcji Spin




% DICTIONARY;linking number;indeks zaczepienia;-
\section{Indeks zaczepienia}
\index{indeks!zaczepienia|(}%
Około 1833 roku Gauß wyraził indeks zaczepienia dwóch węzłów jako pewną (nieciekawą dla nas) całkę, co czyni go najstarszym niezmiennikiem splotów.
\index[persons]{Gauß, Carl}%
Po przeczytaniu \cite{colberg2013} wydaje nam się, że osiągnął to korzystając z praw fizyki: prawa Ampère'a i Biota-Savarta.
My żyjemy w~XXI wieku, wystarczy nam diagramatyczna definicja.
(Ale patrz też: \cite[s. 11]{kawauchi1996}.)
% Erin Colberg - A brief history of knot theory
% TODO: czemu patrz też?

% DICTIONARY;sign;znak;skrzyżowanie
\begin{definition}[znak]
\index{znak skrzyżowania}%
    Liczbę $\pm 1$ przypisaną do skrzyżowania zgodnie z regułą:
\begin{comment}
    \setlength{\intextsep}{4pt plus 2pt minus 2pt}
    \begin{figure}[H]
        \begin{minipage}[b]{.48\linewidth}
            \[
                \sign \left( \MedLarPlusCrossingArrows \right) = +1
            \]
        \end{minipage}
        \begin{minipage}[b]{.48\linewidth}
            \[
                \sign \left( \MedLarMinusCrossingArrows \right) = -1
            \]
        \end{minipage}

    \end{figure}
\end{comment}
\noindent
    nazywamy znakiem skrzyżowania.
\end{definition}

Skrzyżowania dodatnie to takie, w których obrócenie dolnego łuku w prawo daje górny łuk, dlatego czasem nazywa się je także praworęcznymi.
Oczywiście skrzyżowania ujemne nazywamy wtedy leworęcznymi.
\index{skrzyżowanie!dodatnie i ujemne}%
\index{skrzyżowanie!lewo- i prawoskrętne}%

% DICTIONARY;smoothing;wygładzenie;-
\begin{definition}[wygładzenie]
\index{skrzyżowanie!... wygładzenie}%
    Diagramy powstałe przez zmianę biegu łuków danego skrzyżowania zgodnie z poniższymy rysunkami:
\begin{comment}
    {\setlength{\intextsep}{4pt plus 2pt minus 2pt}
    \begin{figure}[H]
        \setlength{\intextsep}{4pt plus 2pt minus 2pt}
        \begin{minipage}[b]{.48\linewidth}
            \[
                \MedLarAlphaSmoothing
            \]
            \subcaption{wygładzenie dodatnie}
        \end{minipage}
        \begin{minipage}[b]{.48\linewidth}
            \[
                \MedLarBetaSmoothing
            \]
            \subcaption{wygładzenie ujemne}
        \end{minipage}
    \end{figure}
    }
\end{comment}
\noindent
    nazywamy wygładzeniem.
    Jeżeli nie zaznaczono inaczej, wygładzamy zgodnie ze znakiem skrzyżowania.
\end{definition}

\begin{definition}[indeks zaczepienia]
    Niech $L = K_1 \sqcup K_2$ będzie splotem o dwóch ogniwach, zaś $D$ jego diagramem.
    Wielkość
    \begin{equation}
        \linking(K_1, K_2) = \frac 12 \sum_i \sign c_i,
    \end{equation}
    gdzie sumowanie rozciąga się na wszystkie skrzyżowania, na których spotykają się łuki różnych ogniw, nazywamy indeksem zaczepienia węzłów $K_1, K_2$.
    Ogólniej, jeśli dany jest splot $L = K_1 \sqcup \ldots \sqcup K_n$ posiadający $n$ ogniw, to jego indeks zaczepienia wyznacza wzór
    \begin{equation}
        \linking(L) = \sum_{i < j} \linking(K_i, K_j).
    \end{equation}
\end{definition}

Zauważmy, że indeks zaczepienia splotu Hopfa wynosi $1$, natomiast splotu Whiteheada $0$.
\index{splot!Hopfa}%
\index{splot!Whiteheada}%
Są zatem różne.
W obydwu przypadkach indeks zaczepienia jest liczbą całkowitą.
Istotnie, na mocy twierdzenia Jordana $\linking$ jest funkcją o całkowitych wartościach.

\begin{proposition}
    Indeks zaczepienia jest dobrze określonym niezmiennikiem zorientowanych splotów.
\end{proposition}

\begin{proof}
    Wielkość $\linking L$ jest sumą znaków pewnych skrzyżowań, zatem na mocy twierdzenia Reidemeistera wystarczy sprawdzić, jaki jest wpływ ruchów Reidemeistera na te składniki:
\begin{comment}
{\setlength{\intextsep}{4pt plus 2pt minus 2pt}
\begin{figure}[H]
\centering
    %
    \begin{minipage}[b]{.3\linewidth}
        \[
            \MedLarReidemeisterOneLeft \cong \MedLarReidemeisterOneStraight
        \]
        \subcaption{ruch $R_1$}
    \end{minipage}
    %
    \begin{minipage}[b]{.3\linewidth}
        \[
            \MedLarReidemeisterTwoLinkingA \cong \MedLarReidemeisterTwoB
        \]
        \subcaption{ruch $R_2$}
    \end{minipage}
    %
    \begin{minipage}[b]{.35\linewidth}
        \[
            \MedLarReidemeisterThreeLinkingA \cong \MedLarReidemeisterThreeLinkingB
        \]
        \subcaption{ruch $R_3$}
    \end{minipage}
\end{figure}
}
\end{comment}
\noindent
    Ὅπερ ἔδει δεῖξαι...
\end{proof}

\index{indeks!zaczepienia|)}%

% koniec podsekcji Indeks zaczepienia




\section{Liczba patykowa}
\index{liczba patykowa|(}%

% DICTIONARY;stick number;liczba patykowa;-

\begin{definition}
    Niech $L$ będzie splotem.
    Minimalną liczbę odcinków łamanej, która przedstawia splot $L$, nazywamy jego liczbą patykową i~oznaczamy $\stick(L)$.
\end{definition}

Liczbę patykową $\stick$ do matematyki wprowadził Randell \cite{randell1994}, gdzie znalazł od razu jej wartość dla niewęzła (3), trójlistnika (6), ósemki (7) i pozostałych węzłów pierwszych do sześciu skrzyżowań: $5_1, 5_2, 6_1, 6_2, 6_3$ (8).
\index[persons]{Randell, Richard}%
Według Cromwella \cite{cromwell2004}, z węzłami od siedmiu do dziewięciu częściowo rozprawiali się Randell \cite{randell1994}, Calvo \cite{calvo1998} i~Negami \cite{negami1991}.
\index[persons]{Randell, Richard}%
\index[persons]{Calvo, Jorge}%
\index[persons]{Negami, Seiya}%
Niedawno dołączył do nich Shonkwiler \cite{shonkwiler2022}.
\index[persons]{Shonkwiler, Clayton}%
Baza danych KnotInfo podpowiada, jaki jest obecny stan wiedzy.

Jest siedem węzłów z $\stick K = 8$: wymienione wcześniej, $8_{19}$ i $8_{20}$;
dwadzieścia pięć gdzie $\stick K = 9$; oprócz tego $\stick 10_{124} = 10$.
O pozostałych węzłach pierwszych do 10 skrzyżowań wciąż brakuje nam danych.
Zachodzi $9 \le \stick 10_{37}, \stick 10_{76} \le 12$, dla pozostałych węzłów mamy lepsze oszacowania.
88 węzłów spełnia $9 \le \stick K \le 11$, zaś 124 innych $9 \le \stick K \le 10$.

Encyklopedia Wolfram Mathworld plecie farmazony, że dokładnie jeden węzeł pierwszy do 10 skrzyżowań ma liczbę patykową równą 14: jest to $10_{84}$, dokładnie pięć ma liczbę patykową równą 13: $10_{39}$, $10_{64}$, $10_{73}$, $10_{76}$, $10_{80}$, a liczba patykowych pozostałych nie przekracza 12.
Niestety Wolfram nie podaje źródeł tych rewelacji.
Ugh!

Negami \cite{negami1991} pokazał przy użyciu teorii grafów, że dla nietrywialnych węzłów prawdziwe są nierówności
\index[persons]{Negami, Seiya}%
\begin{equation}
    \frac{5+\sqrt{9 + 8 \crossing K}}{2} \le \stick K \le 2 \crossing K.
\end{equation}
Trójlistnik to jedyny węzeł realizujący górne ograniczenie.
% Huh, Oh 2011 z trójlistnikiem?
Z~pracy Elrifaia \cite{elrifai2006} (a wiemy o~niej z~\cite{huh2011}) wynika, że dla węzłów o~co najwyżej 26 skrzyżowaniach, dolne ograniczenie jest ostre: można pisać $<$ w~miejsce $\le$.
\index[persons]{Elrifai, Elsayed}%

Jin oraz Kim w 1993 ograniczyli liczby patykowe dla węzłów torusowych korzystając z~liczby supermostowej.
\index[persons]{Jin, Gyo}%
\index[persons]{Kim, Hyoung-Seok}%
Wkrótce wynik został poprawiony przez samego Jina, w pracy \cite{jin1997} znalazł dokładne wartości dla niektórych węzłów.
I~tak, jeśli $2 \le p < q < 2p$, to $\stick T_{p,q} = 2q$ oraz $\stick T_{p, 2p} = 4p-1$.
Ten sam wynik, choć dla węższego zakresu parametrów, odkryli Adams, Brennan, Greilsheimer, Woo \cite{greilsheimer1997}.
\index[persons]{Adams, Colin}%
\index[persons]{Brennan, Bevin}%
\index[persons]{Greilsheimer, Deborah}%
\index[persons]{Woo, Alexander}%
\index{suma spójna}%
Autorzy niezależnie od siebie znaleźli proste oszacowanie z~góry dla liczby patykowej sumy spójnej:
\label{stick_bounded_factors}%
\begin{equation}
    \stick(K_1 \shrap K_2) \le \stick(K_1) + \stick(K_2) - 3.
\end{equation}

Koniec dekady przyniósł jeszcze jedną pracę McCabe z~nierównością $\stick(K) \le 3 + \crossing (K)$ dla węzłów dwumostowych (\cite{mccabe1998}) oraz odkrycie Calvo \cite{calvo2001}: jeśli ograniczymy się do łamanych o co najwyżej siedmiu odcinkach, ósemka przestaje być odwracalna.
\index[persons]{McCabe, Cynthia}%
\index[persons]{Calvo, Jorge}%

Na początku XXI wieku nierówności Negamiego poprawiono, z dołu dokonał tego Calvo \cite{calvo2001}, z góry natomiast Huh, Oh \cite{huh2011}.
\index[persons]{Calvo, Jorge}%
\index[persons]{Huh, Youngsik}%
\index[persons]{Oh, Seungsang}%
% huh11: simple and self-contained
Górne ograniczenie można zmniejszyć o $3/2$, jeżeli $K$ jest niealternującym węzłem pierwszym.
\begin{equation}
    \frac{7+\sqrt{1 + 8 \crossing K}}{2} \le \stick K \le \frac{3}{2} (1 + \crossing K).
\end{equation}

Liczba patykowa nie pojawia się już nigdzie w następnych rozdziałach.

% https://knotinfo.math.indiana.edu/descriptions/polygon_index.html wspomina jeszcze <[6] Shonkwilker, C., "All prime knots through 10 crossings have superbridge Index <= 5. Arxiv preprint.>

\index{liczba patykowa|)}%

% Koniec podsekcji Liczba patykowa



\input{30-invariants-numeric/301g-ropelength}

\subsection{Podsumowanie}
(Podana dalej lista może być niezrozumiała przy pierwszym czytaniu).
Między niektórymi niezmiennikami nie ma bezpośredniego związku:
\begin{enumerate}
    \item między liczbą gordyjską i~mostową (fakt \ref{no_relation_bridge_unknotting}),
    \item między liczbą mostową i~genusem (wzmianka po fakcie \ref{no_relation_bridge_unknotting}),
    \item między defektami modulo różne liczby pierwsze (fakt \ref{no_relation_defects}),
    \item między liczbą mostową i~wielomianem Alexandera (fakt \ref{no_relation_bridge_alexander}),
    \item między wielomianem Jonesa i~Alexandera (paragraf przed faktem \ref{homfly_stronger}),
    \item między liczbą mostową i~sygnaturą (wniosek \ref{no_relation_signature_bridge}),
    \item między liczbą gordyjską i~wielomianem Alexandera (dowód faktu \ref{balanced_iff_four_conditions}).
\end{enumerate}

Ale między niektórymi innymi takie związki są:
\begin{enumerate}
    \item między indeksem skrzyżowaniowym i genusem % TODO: \cite{livingston1993}, 142
    \item między liczbą gordyjską i sygnaturą % TODO: \cite{livingston1993}, 142
\end{enumerate}

% Koniec sekcji Niezmienniki liczbowe



\chapter{Niezmienniki kolorowe}

Powoli kończymy definiować niezmienniki węzłów, po tym rozdziale brakować będzie tylko liczby warkoczowej.
Zaczniemy opisu grupy podstawowej dopełnienia węzła oraz prezentacji tej grupy znalezionej przez Wirtingera.
Następnie pokażemy, że każdy węzeł jest brzegiem pewnej zorientowanej powierzchni (zwanej powierzchnią Seiferta) i odkryjemy tak źródło kolejnych niezmienników: genusu, wyznacznika, sygnatury.
Nie do końca wiadomo dlaczego, ale wspomnimy krótko o~niezmienniku Arfa.
Na koniec spróbujemy przekonać czytelnika, że najciekawszą teorią homologii dla węzłów są homologie Chowanowa.

% koniec wstępu do rozdziału 4: topologia



\section{Macierz kolorująca}
\index{macierz!kolorująca|(}%
Zajmiemy się dwoma niezmiennikami pochodzącymi od macierzy kolorującej: defektem oraz wyznacznikiem.
Podamy też opis macierzy Goeritza, która prowadzi do tych samych niezmienników, ale pozwala na uniknięcie części rachunków.

\begin{definition}[macierz kolorująca]
    Ustalmy diagram bez zamkniętych krzywych dla splotu $L$ z~łukami $x_0, \ldots, x_m$ oraz skrzyżowaniami $0, \ldots, m$.
    Definiujemy macierz $A_+$, której wyraz $a_{lj}$ jest współczynnikiem przy $x_j$ w~$l$-tym równaniu kolorującym:
\begin{center}
\begin{comment}
    \LargePlusCrossingMatrix
\end{comment}
\end{center}
    Macierz kolorująca $A$ powstaje z~macierzy $A_+$ przez skreślenie dowolnego wiersza oraz dowolnej kolumny.
\end{definition}

Zauważmy, że pierwszy ruch Reidemeistera usuwa ,,zamknięte krzywe'', czyli pojedyncze łuki bez skrzyżowań.
Diagram bez takich krzywych ma tyle samo skrzyżowań, co łuków.
Z~każdego skrzyżowania wychodzą (tunelem) dwa włókna mające dwa końce i otrzymana macierz jest kwadratowa.

Wykreślenie wiersza i~kolumny jest konieczne.
Gdybyśmy tego zaniechali, otrzymana macierz nie byłaby odwracalna, bowiem wiersze sumują się do zera (patrz fakt \ref{prp:colouring_sum_zero}).
Dla alternujących diagramów możemy żądać, by górą $i$-tego skrzyżowania biegło $i$-te włókno, wtedy na diagonali macierzy $A$ znajdą się same minus dwójki.

\input{40-colours/401a-determinant}


\subsection{Defekt}
\index{defekt|(}%
\begin{definition}[defekt]
    Niech $K$ będzie węzłem, zaś $A$ jego macierzą kolorującą modulo $p$.
    Wymiar jądra $\ker A$ nazywamy defektem węzła.
\end{definition}

Prawdopodobnie jest to ostatnia strona, gdzie wprowadzamy do teorii węzłów nowy termin z algebry liniowej.

\begin{proposition}
\label{no_relation_defects}%
    Defekty modulo różne liczby pierwsze są niezależne od siebie.
\end{proposition}

Wspomina o tym Livingston \cite[s. 145]{livingston1993}.

\begin{proof}[Niedowód]
    Na przykład suma spójna $k$ trójlistników i~$j$ węzłów $T_{2,5}$ posiada defekt $k$ modulo $3$ oraz $j$ modulo $5$.
    Podobne przykłady istnieją dla innych zbiorów liczb pierwszych.
\end{proof}

Defekt także jest niezmiennikiem, choć rzadziej używanym (nie pojawia się na żadnej późniejszej stronie w tej książce).
Węzeł o~defekcie $n$ modulo $p$ posiada $p(p^n-1)$ kolorowań $p$ kolorami, jest \cite[twierdzenie 2]{taalman2005}.
Węzły $8_{18}$ oraz $9_{24}$ mają ten sam wyznacznik, $45$.
Ich defekty modulo $3$ to $1$ i~$2$, zatem są różne.
\index{defekt|)}%



\input{40-colours/401c-goeritz}

\index{macierz!kolorująca|)}

% Koniec sekcji Macierz i~wyznacznik



\section{Grupa kolorująca}
\index{grupa kolorująca|(}%

\begin{definition}
\label{def:colouring_group}%
    Niech $L$ będzie splotem.
    Grupa kolorująca $L$, oznaczana $\ColoringGroup(L)$, to abelowa grupa generowana przez łuki diagramu przedstawiającego $L$ z~równaniami skrzyżowań tegoż diagramu jako relacjami, oprócz tego mamy jeszcze jedną relację, $a = 0$, gdzie $a$ jest ustalonym łukiem.
\end{definition}

\begin{proposition}
\label{prp:colourings_are_morphisms}%
    Splot $L$ koloruje się modulo $n$, wtedy i~tylko wtedy gdy istnieje nietrywialny homomorfizm $\ColoringGroup(L) \to \Z/n\Z$.
\end{proposition}

\begin{proof}
    Niech $a$ będzie ustalonym łukiem na diagramie splotu $L$ z definicji \ref{def:colouring_group}.
    Funkcja $\varphi \colon \ColoringGroup(L) \to \Z/n\Z$ jest nietrywialnym morfizmem, jeśli przyjmuje choć raz wartość różną od zera, spełnia warunek $\varphi(a) = 0$ i~dla każdego skrzyżowania prawdziwe jest równanie
    \begin{equation}
        \varphi(x_j) + \varphi(x_k) - 2\varphi(x_i) = 0
    \end{equation}
    przy oznaczeniach z definicji \ref{def:colouring_equation}.
    To pokazuje, że niezerowe morfizmy $\ColoringGroup(L) \to \Z/n\Z$ to dokładnie kolorowania modulo $n$, z kolorem $\varphi(l)$ na łuku $l$.
\end{proof}

\begin{corollary}
\index{wyznacznik}%
\label{cor:knot_determinant_odd}%
    Niech $K$ będzie węzłem.
    Wtedy jego wyznacznik, $\det K$, jest liczbą nieparzystą.
\end{corollary}

\begin{proof}
    Bezpośredni wniosek z faktów \ref{prp:no_colourings_mod_2} oraz \ref{prp:colourings_are_morphisms}.
\end{proof}

Homomorfizm $\ColoringGroup (L) \to \Z/n\Z$ nie musi być surjekcją, zatem...

\begin{proposition}
    Niech splot $L$ koloruje się modulo $n$.
    Wtedy koloruje się też modulo $mn$, dla każdego $m \in \N$.
\end{proposition}

\begin{proposition}
    Grupa kolorująca jest z dokładnością do izomorfizmu niezmiennikiem węzłów.
\end{proposition}

\begin{proof}[Szkic dowodu]
    Do wyznaczenia grupy kolorującej potrzebujemy diagramu $D$ z wybranym łukiem $a$.
    Musimy zatem sprawdzić, że ustalenie innego diagramu lub łuku prowadzi do grupy izomorficznej z wyjściową.
    Dla diagramów wystarczy sprawdzić, co dzieje się podczas ruchów Reidemeistera jak w dowodzie faktu \ref{prp:colouring_invariance}.
    W przypadku łuku rozumowanie przebiega analogicznie do dowodu lematu \ref{lem:colouring_arc}.
\end{proof}

Oto metoda pozwalająca na znalezienie grupy kolorującej.
Wybierzmy diagram splotu $L$ bez zamkniętych krzywych, etykietowanie $x_0, \ldots, x_m$ dla łuków oraz $0, \ldots, m$ dla skrzyżowań.
Utwórzmy macierz kolorującą $A$.
Grupy abelowe $\ColoringGroup(L)$ oraz $\Z^m / A^t \Z^m$ są izomorficzne, wynika to bezpośrednio z~definicji macierzy $A$.
Następnie znajdźmy macierz diagonalną $D = \operatorname{diag}(d_1, \ldots, d_m)$, taką że $D = RAC$, gdzie całkowite macierze $R, C$ mają wyznacznik $1$.
Z~algebry liniowej wiemy, że to zawsze się uda: macierz $D$ nazywamy postacią normalną Smitha.
Wtedy funkcja
\begin{equation}
    f(x) \colon \frac{\Z^m}{A^t \Z^m} \to \frac{\Z^m}{D\Z^m}, \quad f(x) = C^t x
\end{equation}
stanowi izomorfizm, a~skoro $D$ jest macierzą diagonalną, to
\begin{equation}
    \frac{\Z^m}{D\Z^m} \cong \bigoplus_{k=1}^m \frac{\Z}{|d_k| \Z}.
\end{equation}

Dokładnie to samo można uczynić z macierzą Goeritza $G$ zamiast macierzy kolorującej $A$.
Podsumujmy.

\begin{proposition}
\label{prp:colouring_group_summands}%
    Niech $L$ będzie splotem z~macierzą kolorującą $A$, zaś $D = \operatorname{diag}(d_1, \ldots, d_n)$ postacią normalną Smitha macierzy $A$.
    Wtedy grupy 
    \begin{equation}
        \ColoringGroup(L) \cong \bigoplus_{k=1}^n \frac{\Z}{|d_k| \Z}
    \end{equation}
    są izomorficzne.
\end{proposition}

\begin{corollary}
    Grupa kolorująca splotu $L$ jest skończona wtedy i tylko wtedy, gdy wyznacznik tego splotu, $\det L$, jest niezerowy.
\end{corollary}

\begin{proof}
    Przy oznaczeniach z faktu \ref{prp:colouring_group_summands}, $\det(L) = |d_1| \cdot \ldots \cdot |d_n|$.
\end{proof}

\begin{corollary}
\index{wyznacznik}%
\label{cor:determinant_invariant}%
    Wyznacznik jest niezmiennikiem splotów.
\end{corollary}

Wyznacznik jest słabszym niezmiennikiem od grupy kolorującej.
Na przykład węzły $6_1$ oraz $3_1 \shrap 3_1$ mają ten sam wyznacznik, $9$, ale różne grupy kolorujące: odpowiednio $\Z/9$ i~$(\Z/3)^2$.

Jednakowoż wyznacznik jest na tyle mocnym niezmiennikiem, że pozwala elementarnie udowodnić, że węzłów pierwszych jest nieskończenie wiele.
Będziemy potrzebować lematu:

\begin{lemma}
\label{lem:det_multiplicativ}%
    Niech $K_1, K_2$ będą węzłami.
    Wtedy $\det(K_1 \shrap K_2) = \det(K_1) \det(K_2)$.
\end{lemma}

\begin{proof}
    Dla oszczędności papieru zamiast podać pełen dowód napiszemy tylko, że wynika to z faktów \ref{prp:alexander_multiplicative} oraz \ref{prp:alexander_determinant}.
    (Podczas czytania dowodu faktu \ref{prp:alexander_multiplicative} należy zastąpić każde wystąpienie $t$ przez $-1$).
\end{proof}

\begin{proposition}
\label{prop:infinite_prime_knots_1}%
    Istnieje nieskończenie wiele węzłów pierwszych.
\end{proposition}

Po poznaniu genusu oraz węzłów skręconych poznamy jeszcze jeden dowód (patrz fakt \ref{prp:infinitely_many_prime_knots}).

\begin{proof}
    Tak jak Cygan dawno temu, rozpatrzmy rodzinę węzłów $K_k$ dla $k \ge 4$:
\index[persons]{Cygan, Szymon}%

\begin{figure}[H]
    \centering
    \begin{comment}
    \begin{tikzpicture}[baseline=-0.65ex, scale=0.1]
    %\useasboundingbox (-5, -5) rectangle (5,5);
    \begin{knot}[clip width=5, end tolerance=1pt, flip crossing/.list={1,2,3,6}]
        \strand[semithick] (-10, +3) .. controls (-4, +3) and (-4, -3) .. (0, -3);
        \strand[semithick] (-10, -3) .. controls (-4, -3) and (-4, +3) .. (0, +3);
        \node at (5, 0) {$\ldots$};
        \strand[semithick] (10+10, +3) .. controls (10+ 4, +3) and (10+ 4, -3) .. (10+0, -3);
        \strand[semithick] (10+ 10, -3) .. controls (10+ 4, -3) and (10+4, +3) .. (10+0, +3);
        \strand[semithick] (20+10, +3) .. controls (20+ 4, +3) and (20+ 4, -3) .. (20+0, -3);
    \strand[semithick] (20+ 10, -3) .. controls (20+ 4, -3) and (20+4, +3) .. (20+0, +3);
        \strand[semithick] (30, 3) [in=up, out=right] to (35, -3);
        \strand[semithick,  ] (30, -3) [in=down, out=right] to (35, 3);
        \strand[semithick] (35, 3) [in=right, out=up] to (0, 10);
        \strand[semithick] (35, -3) [in=right, out=down] to (0, -10);
        \strand[semithick] (-10, -3) [in=down, out=left] to (-20, 0) to [in=left, out=up] (-10, 3);
        \strand[semithick] (-15, 5) [in=left, out=up] to (0, 10);
        \strand[semithick] (-15, 5) to (-15, -5) [in=left, out=down] to (0, -10);
        \node at (-5, -5) {$c_3$};
        \node at (15, -5) {$c_{k-2}$};
        \node at (25, -5) {$c_{k-1}$};
    \end{knot}
    \end{tikzpicture}
\end{comment}
    \caption[caption-cygan]{Węzeł $K_k$}
\end{figure}

    Niech $W_n$ będzie macierzą $n \times n$, na przekątnej której znajdują się $2$, zaś bezpośrednio nad i~pod nią -- wyrazy $-1$.
    W macierzy kolorującej (ze skreślonym wierszem ,,$c_k$'' oraz kolumną ,,$a_k$'') zamieńmy miejscami dwie pierwsze kolumny, dodajmy do drugiej dwa razy pierwszą, zaś do drugiego wiersza -- dwa razy pierwszy wiersz.
    Otrzymamy macierz
    \begin{equation}
        \begin{pmatrix}
            -1 & 0 & 0 \\
            0 & 3 & -1 \\
            0 & -1 & W_{k-3}
        \end{pmatrix}
    \end{equation}

    Powtarzając operacje: zamiana miejscami skrajnie lewych kolumn, dodanie do drugiej $2m+1$ razy pierwszej, odjęcie pierwszego wiersza od trzeciego, czyli sprowadzając naszą macierz do postaci normalnej Smitha przekonamy się, że na jej przekątnej znajdują się wyrazy $-1, -1, \ldots, -1, 2k-3$.
    To oznacza, że $\det K_k = 2k-3$.

    I to już prawie koniec!
    Wskażemy injekcję ze zbioru liczb pierwszych w zbiór węzłów pierwszych.
    Niech $2k-3$ będzie liczbą pierwszą.
    Wtedy któryś składnik w rozkładzie węzła $K_k$ na węzły pierwsze (tak jak w fakcie \ref{prp:knots_decompose_into_primes}) musi mieć wyznacznik $2k-3$.
    Jest jasne (po wniosku \ref{cor:determinant_invariant}), dlaczego te składniki są parami różne.
\end{proof}

\begin{proposition}
    Niech $k \ge 9$ będzie takie, że wyznacznik węzła $K_k$ jest pewną potęgą $3$.
    Wtedy $K_k$ nie jest splotem trójlistników.
\end{proposition}

\begin{proof}
    Macierz diagonalna otrzymana po wniosku \ref{cor:determinant_invariant} także jest niezmiennikiem węzłów.

    Macierzą dla splotu trójlistników $(3_1)^{\# n}$ trójlistników jest $\operatorname{diag}(1, \ldots, 1, 3, \ldots, 3)$, zaś dla węzła $K_k$: $\operatorname{diag} (1, \ldots, 1, 3^n)$.
\end{proof}

\index{grupa kolorująca|)}%

% koniec sekcji grupa


\input{40-colours/403-quandle}

\chapter{Niezmienniki wielomianowe}

Wszystkie poznane dotąd niezmienniki (poza długością sznurową) przyjmowały całkowite wartości.
Teraz poszerzymy skrzynkę z~narzędziami o~klasyczne wielomiany Alexandera, Jonesa, HOMFLY; ale też późniejsze: BLM/Ho, Kauffmana oraz niezmienniki skończonego typu.
Co ciekawe, wielomiany te wywodzą się z~różnych działów matematyki: wielomian $\alexander$ Alexandera z~homologii pewnej przestrzeni nakrywającej, $\jones$ Jonesa: z~algebr von Neumanna.
HOMFLY (albo raczej HOMFLY-PT) to ich naturalne uogólnienie.

Atrakcyjnym wprowadzeniem była przygotowana przez matematyków niemieckich (przez to dostępna tylko w~ich języku) praca \cite{gellert2009}.
Potem okazało się, że praca została wycieknięta do internetu bez ich wiedzy, a następnie skasowana.
A to heca!
Pierwotnymi artykułami były \cite{alexander1928}, \cite{jones1985} oraz \cite{homfly1985}, wszystkie należą do przełomowych w~kombinatorycznej teorii węzłów.



\section{Wielomian Alexandera}
\index{wielomian!Alexandera|(}%
Wielomian Alexandera to najstarszy niezmiennik tego typu, odkryty w 1923 roku \cite{alexander1923}.
Jego pierwsza definicja była czysto topologiczna algebraicznie:

\begin{definition}
    Niech $K$ będzie węzłem w~3-sferze $S^3$, zaś $X$ nieskończonym nakryciem cyklicznym jego dopełnienia otrzymanym przez rozcięcie dopełnienia wzdłuż powierzchni Seiferta.
\index{powierzchnia!Seiferta}%
    Na przestrzeni $X$ oraz grupie homologii $H_1(X)$, działa automorfizm $t$, który czyni z~niej moduł nad pierścieniem $\Z[t, t^{-1}]$, i~to skończenie prezentowalny.
    Załóżmy, że wiemy mniej niż mało; w szczególności nie wiemy, czy nasz moduł posiada przedstawienie z~$r$ generatorami i~$s$ relacjami, gdzie $r \le s$ (jeśli tak jest, rozpatrzmy ideał generowany przez minory $r \times r$ macierzy prezentacji; jeśli nie, weźmy ideał zerowy).
    Alexander pokazał, że ideał, o którym mowa, jest zawsze niezerowy, główny i generowany przez coś, co teraz nazywamy wielomianem Alexandera.
\end{definition}

Nie jest to definicja, z którą praca stanowi przyjemność.
Dlatego podamy prostszy opis, oparty o równania kolorujące.
Później sprawdzimy, jaki wpływ na wielomian mają suma spójna, lustro i~rewers oraz jak Conway odkrył na nowo wielomian Alexandera, jednocześnie przyspieszając jego liczenie.
Na koniec pokażemy, co łączy go ze zdefiniowanymi wcześniej i~później numerycznymi niezmiennikami.

\input{50-polynomials/501a-colouring}

\input{50-polynomials/501b-operations}

\input{50-polynomials/501c-skein}

\input{50-polynomials/501d-numerical}

\input{50-polynomials/501e-fox}

\input{50-polynomials/501f-miscellaneous}

\index{wielomian!Alexandera|)}%



\section{Wielomian Jonesa}
\index{wielomian!Jonesa|(}%
Drugi wielomianowy niezmiennik, jaki poznamy, spełnia bardzo podobną relację kłębiastą (porównaj: \ref{eqn:alexander_skein} versus \ref{eqn:jones_skein}), co ten z poprzedniej sekcji.
Zaczniemy od bardzo ogólnikowego opisu algebry Temperleya-Lieba i śladu Markowa; dwóch składników w~konstrukcji Jonesa.
Szczegółowo zajmiemy się późniejszym odkryciem Kauffmana, bo wykorzystał lubiane przez nas diagramy i~ruchy Reidemeistera.
Opiszemy krótko, jak zmienia się wielomian podczas odbijania, odwracania i dodawania.
Na koniec podamy dowód I hipotezy Taita, wspomnimy też, jaką rolę wielomian Jonesa odegrał w dowodzie pozostałych dwóch hipotez.

\input{50-polynomials/502b-temperley}

\input{50-polynomials/502a-kauffman}

\input{50-polynomials/502d-distinguish}

\input{50-polynomials/502e-skein}

\input{50-polynomials/502f-mirror_reverse}

\input{50-polynomials/502g-roots_of_unity}

\index{wielomian!Jonesa|)}%

\input{50-polynomials/502c-span}


\input{50-polynomials/503-homfly}
\input{50-polynomials/504-blmho}
\input{50-polynomials/505-kauffman}

\section{Niezmienniki Wasiljewa}
\index{niezmiennik!Wasiljewa|(}%
\label{sec:vassiliev}%
\input{50-polynomials/506g-intro}

\input{50-polynomials/506c-small_order}

Czas na garść przykładów niezmienników skończonego typu oraz niezmienników, które nie są skończonego typu.

\input{50-polynomials/506a-finite_yes}

\input{50-polynomials/506b-finite_no}


\subsection{Diagramy cięciw, układy ciężarów, algebra chińskich znaków}

Okazuje się, że wartość niezmiennika Wasiljewa $v$ nie zależy wprost od tego, jak zaplątany jest węzeł osobliwy $K$, ale od tego, jak ułożone są wierzchołki wzdłuż węzła.
Standardową metodą kodowania tej informacji jest diagram cięciw.

\begin{definition}[diagram cięciw]
% DICTIONARY;chord;cięciw;diagram
\index{diagram cięciw}%
    Zorientowany okrąg razem z~$2n$ punktami leżącymi na nim (oraz~połączonymi w pary) nazywamy diagramem cięciw rzędu $n$, albo stopnia $n$.
    Dwa homeomorficzne z zachowaniem orientacji diagramy uznajemy za takie same.
\end{definition}

\vspace{-15pt}

\begin{figure}[H]
    \centering
\begin{comment}
\input{50-polynomials/506f-chords-diagrams}
\end{comment}
    \caption{Wszystkie dwadzieścia sześć diagramów cięciw stopnia 1, 2, 3, 4}
\end{figure}

Jak zamienić węzeł osobliwy w diagram cięciw?
Wybierzmy dowolny punkt na węźle, różny od wierzchołka i~przemierzmy węzeł.
Mijanym skrzyżowaniom przypiszmy liczby $1, 2, \ldots, 2n$.
Następnie na okręgu zaznaczmy kolejno te same punkty $1, 2, \ldots 2n$.
Wreszcie połączmy ze sobą liczby, które występują na tych samych skrzyżowaniach.

% TODO: wstawić obrazek.

\begin{proposition}
    Niech $K_1, K_2$ będą dwoma osobliwymi węzłami o~tym samym diagramie cięciw, zaś $v$~niezmiennikiem Wasiljewa.
    Wtedy $v(K_1) = v(K_2)$.
\end{proposition}

\begin{proof}
    Umieśćmy węzły osobliwe $K_1, K_2$ w przestrzeni tak, by ich wierzchołki oraz obie gałęzie wychodzące z wierzchołków leżały tak samo. Wtedy można tak zdeformować łuki $K_1$ tak, by jedynymi osobliwościami, jakie się pojawią lub znikną, były podwójne punkty.
    Teraz relacja akłębiasta Wasiljewa mówi, że wartość $v$ nie zmienia się podczas tego procesu, zatem $v(K_1) = v(K_2)$, co należało okazać.
    % chmutov12
    % TODO: (\cite{duzhin2012}, prop. 3.4.2)
\end{proof}

\begin{definition}[symbol niezmiennika]
\index{niezmiennik!Wasiljewa!symbol ...}%
    Niech $v$ będzie niezmiennikiem Wasiljewa.
    Obcięcie $v$ do zbioru węzłów osobliwych o~dokładnie $n$ wierzchołkach traktowane jako funkcja ze zbioru diagramów cięciw nazywamy symbolem tego niezmiennika.
\end{definition}

Jeśli $v_1, v_2$ są niezmiennikami Wasiljewa rzędu co najwyżej $n$ o~tych samych symbolach, to ich różnica jest niezmiennikiem rzędu co najwyżej $n - 1$.
Oznacza to, że przestrzeń $\mathcal V_n/\mathcal V_{n-1}$ pokrywa się z przestrzenią wszystkich symboli niezmienników Wasiljewa rzędu co najwyżej $n$.
Zbiór diagramów cięciw rzędu $n$ jest skończony, więc przestrzeń funkcji na tym zbiorze też jest skończona, a zatem przestrzenie $\mathcal V_n$ są skończonego wymiaru.

Stojmenow w \cite{stoimenowa2001} znalazł jakościowy wynik bez praktycznego znaczenia (ponieważ już dla $k = 4$ musielibyśmy znać wszystkie węzły o 36 skrzyżowaniach, a~tych jest tak dużo, że prędzej skonamy niż poznamy pełną ich listę).
\index[persons]{Stojmenow, Aleksander}%
Dokładniej:

\begin{proposition}
    Każdy niezmiennik Wasiljewa rzędu $k$ jest jednoznacznie określony przez wartości, jakie przyjmuje na alternujących węzłach o co najwyżej $2k^2 + k$ skrzyżowaniach.
\end{proposition}

Symbol nie jest byle jaką funkcją, spełnia dwie relacje:
\index{relacje 1T i 4T}%
\begin{figure}[H]
\begin{comment}
%
\begin{minipage}[b]{.065\linewidth}
    $\color{white}?$\hfill$\color{white}?$
\end{minipage}
%
\begin{minipage}[b]{.20\linewidth}
    \centering
    \MedLarOneTerm $= 0,$
\end{minipage}
%
\begin{minipage}[b]{.65\linewidth}
    \centering
    \MedLarFourTermA $-$ \MedLarFourTermB $=$ \MedLarFourTermC $-$ \MedLarFourTermD
\end{minipage}
%
\begin{minipage}[b]{.065\linewidth}
    $\color{white}?$\hfill$\color{white}?$
\end{minipage}
%
\end{comment}
    \caption{Relacje ,,one-term'' (1T) i ,,four-term'' (4T)}
\end{figure}

Diagramy mogą mieć więcej cięciw z końcami tam, gdzie linia jest kropkowana, natomiast wszystkie końce cięciw na czarnych, pogrubionych łukach zostały zaznaczone explicite.

Zastosowaliśmy tutaj mały skrót dla oszczędności miejsca: oczywiście nie umiemy jeszcze odejmować od siebie diagramów, dlatego powyższe relacje należy rozumieć tak, że na każdym diagramie liczymy symbol niezmiennika i porównujemy tak otrzymane liczby zespolone.

\begin{example}
Mamy:
\begin{equation}
    \begin{tikzpicture}[baseline=-0.65ex, scale=0.11001]
    \begin{knot}[clip width=15, end tolerance=1pt]
        \useasboundingbox (-7, -5) rectangle (7, 5); % REMOVE ME
        \draw[thick, first_colour] (0:5) [in=240, out=180] to (60:5);
        \draw[thick, first_colour] (120:5) [in=360, out=300] to (180:5);
        \draw[thick, first_colour] (240:5) [in=480, out=420] to (300:5);
        \draw[very thick] (-0, 0) circle (5);
    \end{knot}
    \end{tikzpicture}
    =
    \begin{tikzpicture}[baseline=-0.65ex, scale=0.11001]
    \begin{knot}[clip width=15, end tolerance=1pt]
        \useasboundingbox (-7, -5) rectangle (7, 5); % REMOVE ME
        \draw[thick, first_colour] (0:5) [in=240, out=180] to (60:5);
        \draw[thick, first_colour] (120:5) [in=480, out=300] to (300:5);
        \draw[thick, first_colour] (180:5) [in=420, out=360] to (240:5);
        \draw[very thick] (-0, 0) circle (5);
    \end{knot}
    \end{tikzpicture},
    \,\,\,
    \begin{tikzpicture}[baseline=-0.65ex, scale=0.11001]
        \begin{knot}[clip width=15, end tolerance=1pt]
            \useasboundingbox (-7, -5) rectangle (7, 5); % REMOVE ME
            \draw[thick, first_colour] (0:5) [in=360, out=180] to (180:5);
            \draw[thick, first_colour] (60:5) [in=420, out=240] to (240:5);
            \draw[thick, first_colour] (120:5) [in=480, out=300] to (300:5);
            \draw[very thick] (-0, 0) circle (5);
        \end{knot}
        \end{tikzpicture}
        +
        \begin{tikzpicture}[baseline=-0.65ex, scale=0.11001]
            \begin{knot}[clip width=15, end tolerance=1pt]
                \useasboundingbox (-7, -5) rectangle (7, 5); % REMOVE ME
                \draw[thick, first_colour] (0:5) [in=240, out=180] to (60:5);
                \draw[thick, first_colour] (120:5) [in=420, out=300] to (240:5);
                \draw[thick, first_colour] (180:5) [in=480, out=360] to (300:5);
                \draw[very thick] (-0, 0) circle (5);
            \end{knot}
            \end{tikzpicture}
        =
        2\begin{tikzpicture}[baseline=-0.65ex, scale=0.11001]
            \begin{knot}[clip width=15, end tolerance=1pt]
                \useasboundingbox (-7, -5) rectangle (7, 5); % REMOVE ME
                \draw[thick, first_colour] (0:5) [in=300, out=180] to (120:5);
                \draw[thick, first_colour] (60:5) [in=420, out=240] to (240:5);
                \draw[thick, first_colour] (180:5) [in=480, out=360] to (300:5);
                \draw[very thick] (-0, 0) circle (5);
            \end{knot}
            \end{tikzpicture}
\end{equation}
    (na mocy relacji 1T i 4T).
    % https://people.math.osu.edu/chmutov.1/preprints/cdbook20jul10.pdf
    % Introduction to Vassiliev Knot Invariants First draft. Comments welcome. July 20, 2010 S. Chmutov S. Duzhin J. Mostovoy
\end{example}

\begin{definition}[układ ciężarów]
% DICTIONARY;weight system;układ ciężarów;-
\index{układ ciężarów}%
    Funkcję określoną na zbiorze diagramów $n$ cięciw, która spełnia relacje 1T oraz 4T, nazywamy układem ciężarów.
\end{definition}

Okazuje się, że wszystkie zależności, jakie występują między niezmiennikami Wasiljewa, są konsekwencjami relacji 1T oraz 4T.
Mówi o~tym głębokie twierdzenie Koncewicza:

\begin{proposition}
    Każdy układ ciężarów jest symbolem pewnego niezmiennika Wasiljewa. % rzędu co najwyżej $n$ - nie mieści się, przenosi samo $n$ do nowej linii.
\end{proposition}

\begin{proof}
    Koncewicz w \cite{kontsevich1993}. % chmutow12/chmutov11 theorem 3.4
\end{proof}

% DICTIONARY;actuality table;tablica rzeczywistości;-
\index{tablica rzeczywistości}
Z tego, co napisaliśmy wyżej wynika, że wszystkie informacje o niezmienniku Wasiljewa można zakodować w~postaci tak zwanej ,,tablicy rzeczywistości''.
Dla każdego diagramu cięciw wybiera się reprezentanta, węzeł osobliwy, oraz podaje wartość niezmiennika na tym węźle.
Jest to pięknie zilustrowane w \cite[sekcja 3.7]{duzhin2012}.

\begin{definition}[chiński znak]
    Spójny graf złożony z pojedynczego zorientowanego okręgu oraz pewnej liczby niezorientowanych, kreskowanych linii, które mogą się spotykać w~jednym z dwóch typów wierzchołków:
    \begin{itemize}
        \item wewnętrznych wierzchołkach, gdzie spotykają się trzy kreskowane linie;
        \item zewnętrznych wierzchołkach, gdzie kreskowane linie kończą się na okręgu.
    \end{itemize}
    Wierzchołki wewnętrzne są zorientowane w lewo lub prawo.
\end{definition}

Diagramy cięciw modulo relacja 4T jest tym samym, co algebra chińskich znaków modulo relacja STU.
% DICTIONARY;algebra of Chinese characters;algebra chińskich znaków;-
\index{algebra!chińskich znaków}%
\index{chiński znak|see {algebra chińskich znaków}}%
W~tej drugiej spełnione są jeszcze relacje AS oraz IHX, nie mamy siły tego rysować, ale wszystko można znaleźć w pracy Bar-Natana \cite{barnatand1995}.
\index[persons]{Bar-Natan, Dror}%



\input{50-polynomials/506d-large_order}

\input{50-polynomials/506e-kontsevich}

\index{niezmiennik!Wasiljewa|)}%

% dummy commit
% koniec sekcji niezmienniki Wasiljewa



\chapter{Topologia algebraiczna}

Powoli kończymy definiować niezmienniki węzłów, po tym rozdziale brakować będzie tylko liczby warkoczowej.
Zaczniemy opisu grupy podstawowej dopełnienia węzła oraz prezentacji tej grupy znalezionej przez Wirtingera.
Następnie pokażemy, że każdy węzeł jest brzegiem pewnej zorientowanej powierzchni (zwanej powierzchnią Seiferta) i odkryjemy tak źródło kolejnych niezmienników: genusu, wyznacznika, sygnatury.
Nie do końca wiadomo dlaczego, ale wspomnimy krótko o~niezmienniku Arfa.
Na koniec spróbujemy przekonać czytelnika, że najciekawszą teorią homologii dla węzłów są homologie Chowanowa.

% koniec wstępu do rozdziału 4: topologia



\section{Grupa węzła}
Ponieważ dopełnienie dowolnego węzła, zarówno w przestrzeni $\R^3$ jak i $S^3$, jest łukowo spójne, jego grupa podstawowa nie zależy od wyboru punktu bazowego.
Dzięki temu poniższa definicja ma sens:

\begin{definition}[grupa węzła]
\index{grupa!węzła}%
    Niech $K$ będzie węzłem.
    Grupę podstawową jego dopełnienia,
    \begin{equation}
        \pi(L) := \pi_1 \left(\R^3 \setminus L\right),
    \end{equation}
    nazywamy grupą węzła i oznaczamy $\pi(K)$.
\end{definition}

Nie należy (i dosyć trudno jest) mylić grupy węzła z grupą kolorującą (patrz definicja \ref{def:colouring_group}).
Jak wkrótce się przekonamy, ta pierwsza jest zawsze nieskończona i poza grupą niewęzła, nieprzemienna.
Natomiast grupa kolorująca jest skończona dla splotów o~niezerowym wyznaczniku i zawsze przemienna.

Podamy najpierw kilka przykładów węzłów oraz ich grup.

\begin{example}
    Grupa niewęzła: $\Z$.
\end{example}

Z twierdzenia o pętli, czyli uogólnienia lematu Dehna wynika, że niewęzeł jest jedynym węzłem, którego grupą podstawową jest $\Z$.
% TODO: https://math.stackexchange.com/questions/3468034/knot-group-is-mathbbz-iff-k-is-the-unknot
(Pierwszy dowód lematu Dehna podał Max Dehn w 1910 roku, ale Hellmuth Kneser znalazł w 1929 lukę w dowodzie.
Z odsieczą przyszedł matematyk grecki Christos Papakyriakopoulos, który około 1957 roku nie tylko podał poprawny dowód używając jego ,,konstrukcji wieżowej'', ale uogólnił go do twierdzenia o~pętli, o~sferze).
% Wiem to z: https://en.wikipedia.org/wiki/Dehn%27s_lemma oraz Burde, Zieschang, Heusener, strona 52

\begin{example}
\label{exm:trefoil_group}%
    Grupa trójlistnika: $\langle x, y \mid x^2 = y^3\rangle$.
\end{example}

\begin{proof}
    Wynika to z (prezentacji Wirtingera i) równości
    % https://en.wikipedia.org/wiki/Tietze_transformations
    \begin{align}
        \pi_1(S^3 \setminus 3_1) & = \langle x, y, z \mid xz = yx, zy = xz, yx = zy \rangle \\
                                 & = \langle x, y \mid xyx = yxy \rangle \\
                                 & = \langle x, y, a, b \mid xyx = yxy, a = yx, b = xyx \rangle \\
                                 & = \langle x, a, b \mid xa = a^2x^{-1}, b = xa \rangle \\
                                 & = \langle a, b \mid b = a^2(ba^{-1})^{-1} \rangle \\
                                 & = \langle a, b \mid a^3 = b^2 \rangle,
    \end{align}
    prawdziwych na mocy transformacji Tietzego.
\end{proof}

Trójlistnik jest węzłem $(3, 2)$-torusowym, więc powyższy przykład stanowi szczególny przypadek grupy węzła torusowego.
Jej wyznaczenie to popularne ćwiczenie w~podręcznikach topologii algebraicznej:

\begin{example}
    Grupa węzła $(p,q)$-torusowego: $\langle x, y \mid x^p = y^q \rangle$.
\end{example}

\begin{proof}
    Wniosek z twierdzenia Seiferta-van Kampena, patrz \cite[s. 77]{kawauchi96} albo \cite[s. 47]{hatcher02}.
\end{proof}

\begin{example}
    Grupa ósemki: $\langle x, y \mid yxy^{{-1}}xy=xyx^{{-1}}yx \rangle$.
\end{example}

\begin{proposition}
    \label{prop:knot_group_invariant}
    Jeżeli węzły $L_1, L_2$ są równoważne, to grupy $\pi(L_1), \pi(L_2)$ są izomorficzne.
    Innymi słowy, grupa jest niezmiennikiem węzłów.
\end{proposition}

\begin{proof}
    Gdy dwa węzły są równoważne, istnieje izotopijny z~identycznością homeomorfizm $\R^3 \to \R^3$, który posyła pierwszy węzeł na drugi.
    Obcięty do dopełnień węzłów indukuje izomorfizm grup podstawowych.
\end{proof}

Na przykładzie grupy $\langle x,y,z \mid xyx=yxy,xzx=zxz\rangle$, która odpowiada zarówno sumie prostej różno-, jak i~jednoskrętnych trójlistników, widać że implikacja odwrotna nie zachodzi: mają one różne sygnatury (patrz uwaga za wnioskiem \ref{cor:acheiral_signature}).
Prawdziwe jest nawet ogólniejsze stwierdzenie:

\begin{proposition}
    Niech $K_1, K_2$ będą zorientowanymi węzłami.
    Wtedy węzłom $K_1 \shrap K_2$, $K_1 \shrap mr K_2$ odpowiadają izomorficzne grupy.
\end{proposition}

\begin{proof}
    Wniosek z twierdzeń o podgrupie południkowo-równoleżnikowej \cite[s. 75]{kawauchi96}.
\end{proof}

Twierdzenie odwrotne do faktu \ref{prop:knot_group_invariant} jest prawdziwe w klasie węzłów pierwszych:

\begin{proposition}
    Niech $K_1, K_2$ będą węzłami pierwszymi.
    Jeżeli ich grupy są izomorficzne, to same węzły są równoważne.
\end{proposition}

,,\emph{The group of a prime knot does not, however, necessarily determine the topological type of the exterior. Dehn hips on certain “essential” solid tori in the exteriors of torus knots and of cable knots produce Haken manifolds that are homotopically equivalent but not homeomorphic to the original exteriors and that, in fact, cannot be imbedded in $S^3$}'' (Whitten, \cite{whitten87}).
\index{rozmaitość!Hakena}%

\begin{proof}
\index[persons]{Gordon, Cameron}%
\index[persons]{Luecke, John}%
\index[persons]{Whitten, Wilbur}%
    % to jest kopia \cite[s. 76]{kawauchi96}
    Whitten pokazał w \cite{whitten87}, że węzły o~izomorficznych grupach mają homeomorficzne dopełnienia.
    Wkrótce po tym Gordon, Luecke udowodnili w~\cite{gordon89}, że nietrywialna chirurgia Dehna na nietrywialnym węźle nigdy nie daje sfery $S^3$, a~stąd wynika, że każdy homeomorfizm dopełnień węzłów można przedłużyć do homeomorfizmu $S^3$ w~siebie posyłającego jeden węzeł na drugi jako zbiory.
\index{chirurgia Dehna}%
\end{proof}

Waldhausen \cite{waldhausen68} pokazał dla pewnej dużej klasy 3-rozmaitości, że są scharakteryzowane topologicznie przez ich grupy podstawowe.
\index[persons]{Waldhausen, Friedhelm}%
Potem odkryto, że:

\begin{proposition}
    Grupa węzła wyznacza wartość jego genusu.
\end{proposition}
\begin{proof}[Niedowód]
    Jest to wniosek 3 z~pracy \cite{feustel78} Feustela.
\end{proof}
\begin{proposition}
    Grupa węzła złożonego wyznacza wartość jego liczby mostowej.
\end{proposition}
\begin{proof}[Niedowód]
    Jest to wniosek 3 z~pracy \cite{feustel78} Feustela.
\end{proof}

Wreszcie Thurston \cite{thurston82} jako efekt uboczny uzyskał kolejny wynik o grupie: hiperboliczne rozmaitości o skończonej objętości (domknięte rozmaitości lub takie, których brzeg jest zbudowany z torusów) są wyznaczone przez ich grupę podstawową.
\index[persons]{Thurston, William}%

\subsection{Grupa splotu}

Kawauchi \cite[s. 73]{kawauchi96} definiuje grupę splotu jako grupę podstawową jego dopełnienia.
Dla Milnora \cite{milnor54} grupą splotu był pewien iloraz\footnote{%
Niech $L$ będzie splotem w otwartej 3-rozmaitości $M$, $\pi(L)$ grupą podstawową dopełnienia $L$, zaś $L^i$ splotem powstałym przez usunięcie $i$-tego ogniwa.
Niech $A_i(L)$ oznacza jądro naturalnej inkluzji $\pi(L) \to \pi(L^i)$, zaś $[A_i]$ komutanta.
Wtedy $E(L) := [A_1][A_2] \cdots [A_n]$ jest podgrupą normalną $\pi(L)$.
Milnor nazywa iloraz $\pi(L) / E(L)$ grupą splotu.%
} tamtej grupy, dlatego należy zachować ostrożność i~sprawdzić, która konwencja obowiązuje.
% TODO: John Milnor (1954). Link groups. Ann. of Math. (2), 59, 177–195. https://mathscinet.ams.org/mathscinet/relay-station?mr=71020
My będziemy mieć do czynienia jedynie z~grupą podstawową dopełnienia.

\begin{example}
    Grupa splotu Hopfa: $\Z \oplus \Z$.
\end{example}

Kawauchi wspomina, że istnieją sploty, których grupa splotu nie odróżnia, podaje w~formie ćwiczenia \cite[s. 73]{kawauchi96}, że grupa sumy niespójnej splotów $L_1, L_2$ to $\pi(L_1) * \pi(L_2)$ i~przytacza twierdzenie:

\begin{proposition}
    Niech $L \subseteq S^3$ będzie splotem.
    Następujące warunki są równoważne:
    \begin{enumerate}
        \item splot $L$ nie jest rozszczepialny,
\index{splot!rozszczepialny}%
        \item splot $L$ jest niewęzłem lub jego dopełnienie jest rozmaitością Hakena o~nieściśliwym brzegu,
\index{rozmaitość!Hakena}%
        \item grupa podstawowa splotu $L$ jest nierozkładalna względem produktu wolnego.
    \end{enumerate}
\end{proposition}

\begin{proof}
    Z twierdzenia o pętli (niech $M$ będzie spójną 3-rozmaitością o niepustym brzegu, zaś $F$ powierzchnią na $\partial M$; jeżeli homomorfizm $\pi_1(F) \to \pi_1(M)$ indukowany przez inkluzję nie jest różnowartościowy, to istnieje dysk ściskający dla $F$ w $M$) i sferze (niech $M$ będzie spójną zorientowaną 3-rozmaitością, jeżeli $\pi_2(M)$ jest nietrywialna, to istnieje sfera właściwa w $M$) wynika, że $1 \implies 2$.
    Implikacja $2 \implies 3$ jest wnioskiem z hipotezy Knesera (niech $M$ będzie zwartą, spójną 3-rozmaitością, której brzeg jest pusty lub złożony z nieściśliwych powierzchni; jeśli $\pi_1(M) \cong G_1 * G_2$, to istnieje rozkład $M$ na sumę $M_1 \shrap M_2$ taką, że $\pi_1(M_i) \cong G_i$), zaś wynikanie $3 \implies 1$ jest oczywiste.
    To kończy dowód.
\end{proof}

\begin{corollary}
    Niech $L \subseteq S^3$ będzie splotem.
    Następujące warunki są równoważne:
    \begin{enumerate}
        \item grupa podstawowa splotu $L$ jest wolna, rangi $n$,
        \item splot $L$ jest trywialny, złożony z $n$ ogniw.
    \end{enumerate}
\end{corollary}

\begin{proposition}
    Niech $L \subseteq S^3$ będzie splotem.
    Następujące warunki są równoważne:
    \begin{enumerate}
        \item splot $L$ jest prosty i niepierścieniowaty\footnote{simple, anannular},
        \item grupa $\pi(L)$ jest nieprzemienną, nierozkładalną względem produktu wolnego grupą, izomorficzną z dyskretną podgrupą $PSL_2(\C)$.
    \end{enumerate}
\end{proposition}

\begin{proof}
    Wniosek z twierdzenia Thurstona o hiperbolizacji, patrz \cite[s. 76]{kawauchi96}.
\end{proof}

Jak wspomina Kawauchi \cite[s. 83]{kawauchi96}, jeśli $G$ jest nietrywialną przemienną podgrupą $\pi(L)$, to $G \cong \Z$ lub $\Z \oplus \Z$.
Wynika to z~klasyfikacji abelowych grup podstawowych 3-rozmaitości.
W~szczególności, jeśli $\pi(L) \cong \Z \oplus \Z$, to $L$ jest splotem Hopfa.

\begin{proposition}
    Niech $L$ będzie splotem.
    Centrum grupy $\pi(L)$ jest nietrywialne wtedy i tylko wtedy, gdy dopełnienie splotu $L$ jest rozmaitością Seiferta.
\end{proposition}

\begin{corollary}
    Niech $K$ będzie węzłem.
    Centrum grupy $\pi(K)$ jest nietrywialne wtedy i tylko wtedy, gdy $K$ jest węzłem torusowym.
\end{corollary}

O tym samym, ale nie tak samo, pisaliśmy już w \ref{prp:torus_nontrivial_center}.
Kawauchi \cite[s. 85]{kawauchi96} żegna się z grupami splotów:

\begin{proposition}
    Grupa splotu jest rezydualnie skończona (dla każdego nietrywialnego elementu $x \in \pi$, istnieje homomorfizm $f: \pi \to H$ taki, że grupa $H$ jest skończona i $f(x) \neq 1$) i lokalnie indeksowalna (dla każdej  nietrywialnej skończenie generowanej podgrupy $H \le \pi$, istnieje epimorfizm $H \to \Z$).
\end{proposition}


\subsection{Prezentacja Wirtingera}
\index{prezentacja!Wirtingera|(}%
Perko \cite{perko2016} pisze, że na wykładzie w 1905 roku (którego nigdy nie spisano) ,,Wirtinger otworzył drzwi do barwnego świata przestrzeni nakryciowych''\footnote{\emph{,,I think covering spaces can be understood by fifth graders''} -- Kenneth Perko tamże}, rozkładając dopełnienie węzła na komórki, co doprowadziło do prezentacji grupy podstawowej.
\index[persons]{Wirtinger, Wilhelm}%
\index[persons]{Perko, Kenneth}%
Jest to skończona prezentacja, w~której wszystkie relacje są postaci $w g_i w^{-1} = g_j$, gdzie $w$ to pewne słowo na generatorach $g_1, \ldots, g_k$.
% In a 1905 lecture that he never wrote down, Wirtinger opened the door to a colorful world of covering spaces. His cellular decomposition of a knot's complement in 3-space furnished an algorithm for presenting its fundamental group. - Perko, "Historical highlights of non-cyclic knot theory"
Pokażemy zarys konstruktywnego algorytmu, który w wygodny sposób zamienia skrzyżowania diagramu w relacje.

\begin{proposition}
    Grupa każdego zorientowanego węzła posiada prezentację Wirtingera.
\end{proposition}

\begin{proof}
    Niech $K$ będzie zorientowanym węzłem.
    Wybierzmy dowolny diagram, początkowy punkt na tym diagramie i przemierzajmy węzeł, nazywając kolejne włókna $x_1, x_2, \ldots x_{n}$.
    To będą generatory grupy.
    Do każdego skrzyżowania przypisujemy relację zgodnie z poniższymi regułami, w zależności od znaku skrzyżowania:
\begin{comment}
    \begin{figure}[H]
        \begin{minipage}[b]{.48\linewidth}
            \[
                \HugeWirtingerPlus
            \]
            \subcaption{skrzyżowanie dodatnie: $x_j = x_k x_{j+1} x_k^{-1}$}
        \end{minipage}
        \begin{minipage}[b]{.48\linewidth}
            \[
                \HugeWirtingerMinus
            \]
            \subcaption{skrzyżowanie ujemne: $x_j = x_k^{-1} x_{j+1} x_k$}
        \end{minipage}
    \end{figure}
\end{comment}
\noindent
    Oto mnemotechnika ułatwiająca zapamiętywnaie relacji.
    Wyobraźmy sobie zorientowaną przeciwnie do ruchu wskazówek zegara ścieżkę wokół skrzyżowania.
    Poruszamy się po niej i kiedy mijamy włókno biegnące do skrzyżowania, zapisujemy jego symbol, a kiedy włókno biegnie od skrzyżowania na zewnątrz, zapisujemy odwrotność jego symbolu.
    Tak uzyskane czteroliterowe słowo jest równe $1$ w grupie węzła.
\end{proof}

Porządny dowód znaleźliśmy w podręczniku Stillwella \cite[s. 144-147]{stillwell1993}: używa twierdzenia Seiferta-van Kampena.
Nie będziemy go przepisywać, ograniczymy się tylko do pożyczenia dwóch obrazków:

\begin{figure}[H]
    \centering
\begin{comment}
    \includegraphics[height=0.28\linewidth]{../data/stillwell-157.png}
    \includegraphics[height=0.28\linewidth]{../data/stillwell-158.png}
\end{comment}
    \caption[caption-stillwell]{Kontrakcja krzywych do punktu}
\end{figure}

\begin{corollary}
    Niech $G$ będzie grupą węzła.
    Wtedy jej abelianizacja jest nieskończoną grupą cykliczną: $G^{\operatorname{ab}} \cong \Z$.
\end{corollary}

Jedna z relacji w prezentacji Wirtingera jest zbędna -- wynika z pozostałych.
Fakt ten zazwyczaj podaje się bez dowodu, dlatego warto wspomnieć, że nie zapomniał o nim Rolfsen \cite[s. 56-60]{rolfsen1976}, za co serdecznie mu dziękujemy.

\begin{proof}
    Relacja $a_ia_ja_i^{-1}a_k^{-1}=1$ po przejściu do abelianizacji przyjmuje postać $a_j = a_k$.
    Oznacza to, że etykieta łuku nie zmienia się podczas przejścia pod każdym skrzyżowaniem, zatem wszystkie etykiety są takie same.

    Można też zauważyć, że abelianizacją grupy podstawowej węzła jest pierwsza grupa homologii okręgu, czyli $\Z$.
\end{proof}

Michel Kervaire \cite{kervaire1965} pokazał, jakie własności musi posiadać grupa węzła (i~wiemy o~tym, bo przeczytaliśmy książkę Kawauchiego \cite[tw. 14.1.1]{kawauchi1996}):
\index[persons]{Kervaire, Michel}%

\begin{proposition}
\index{południk}%
\index{homologia!druga grupa homologii}%
    Niech $G$ będzie grupą węzła $S^n \subseteq S^{n+2}$.
    Wtedy:
    \begin{enumerate}
        \item grupa $G$ jest skończenie prezentowana,
        \item abelianizacja $G/G'$ jest nieskończoną grupą cykliczną,
        \item druga grupa homologii $H_2(G) = 0$ jest trywialna,
        \item istnieje element $x \in G$ zwany południkiem taki, że $G$ jest najmniejszą podgrupą normalną $G$, która zawiera $x$.
        \index{południk}%
    \end{enumerate}
\end{proposition}

Wyżej wymienione warunki konieczne są także wystarczające, jeżeli $n \ge 3$, jednakże problem pełnej charakteryzacji w~czwartym wymiarze jest otwarty.
Warunki 2. i 3. wynikają z~dualności\footnote{Niech $X$ będzie zwartą, lokalnie ściągalną podprzestrzenią sfery $S^n$. Wtedy $\tilde {H}_{q}(S^n \setminus X) \cong \tilde H^{n-q-1}(X)$. Założenie o~lokalnej ściągalności można pominąć, jeśli pracuje się z kohomologią Čecha.\index{kohomologia!Čecha}} Alexandera, zaś 1. i 4. stanowią przeformułowanie prezentacji Wirtingera.

\index{prezentacja!Wirtingera|)}%
% koniec podsekcji Prezentacja Wirtingera



% Koniec sekcji Grupa splotu


\input{40-topology/402-seifert}
\input{40-topology/403-arf}

\section{Homologie Chowanowa}

Reszta książki nie zależy od tej sekcji, więc Czytelnik nie dozna większego uszczerbku na wiedzy, jeśli pominie tę i kilka następnych stron (na przykład dlatego, że nie zna kompleksów łańcuchowych, różniczek, grup homologii, i tak dalej).
Homologię Chowanowa nazywa się kategoryfikacją wielomianu Jonesa.
Zanim zagłębimy się w szczegóły, rozpatrzmy prostszy przykład tego procesu.
Niech $X$ będzie przestrzenią topologiczną, wtedy charakterystykę Eulera oraz grupy homologii łączy zależność
\begin{equation}
    \chi(X) = \sum_{n = 0}^{\dim X} (-1)^n \operatorname{rk} H_n(X),
\end{equation}
a przy tym grupy homologii dostarczają więcej informacji, co więcej można o nich myśleć jako (jak o? \texttt{:|}) funktorach.
Homologie są kategoryfikacją charakterystyki Eulera.

\index{homologia!Chowanowa|(}
Niech $L$ będzie splotem, zaś $D$ jego diagramem.
Chowanow \cite{khovanov2000} odkrył, że całkowite współczynniki we wzorze o sumowaniu stanów można zamienić na kompleksy grup abelowych.
Skonstruował rodzinę grup $\mathcal H^{i, j}(D)$ takich, że
\begin{equation}
    K(L, q) = \sum_{i, j} q^j (-1)^i \dim_\Q (\mathcal H^{i, j}(D) \otimes \Q),
\end{equation}
gdzie $K$ jest wersją wielomianu Jonesa.
Niestety opis konstrukcji grup $\mathcal H^{i, j}$ przeładował algebraicznymi szczegółami, dlatego później Bar-Natan \cite{barnatan2002}, Viro \cite{viro2002} przygotowali swoje teksty z~myślą o~topologach.
\index[persons]{Bar-Natan, Dror}%
\index[persons]{Viro, Oleg}%
My opierać się będziemy o~późniejszy artykuł Viro \cite{viro2004}, który został napisany, gdyż \emph{,,Nonetheless, in most of the papers on Khovanov homology, the differences between \cite{barnatan2002} and \cite{viro2002} are taken too seriously. In this paper I discuss the constructions again. I begin with the approach of \cite{viro2002}. (...) Then I identify this construction with the construction of \cite{barnatan2002} and \cite{khovanov2000}''}.
Viro \cite{viro2004} wymienia kilka innych dobrych miejsc, gdzie można zacząć wycieczkę do zrozumienia homologii Chowanowa: Turnera \cite{turner2017} albo Bar-Natana \cite{barnatan2002}, a potem Szumakowicza \cite{shumakovitch2012}, Chowanowa \cite{khovanov2000}.
\index[persons]{Chowanow, Michaił (Хованов, Михаил Гелиевич)}%
\index[persons]{Bar-Natan, Dror}%
\index[persons]{Szumakowicz, Aleksander}%
\index[persons]{Turner, Paul}%

Viro zauważa, że powszechna definicja wielomianu Jonesa sprawia problem dla pustego splotu (którego nigdy wcześniej nie rozpatrywaliśmy).
Mamy:
\begin{equation}
    \jones_\varnothing = \frac{1}{-t^{1/2} - t^{-1/2}},
\end{equation}
a to nie jest wielomian Laurenta jednej zmiennej.
\index{wielomian!Jonesa!powiększony}%
Dlatego definiuje powiększony wielomian Jonesa:
% DICTIONARY;Jones polynomial;wielomian Jonesa;-
% DICTIONARY;augmented;powiększony;wielomian Jonesa
\begin{equation}
    \widetilde{\jones_L}(t) = (-t^{1/2} - t^{-1/2}) \cdot \jones_L(t),
\end{equation}
i mówi, że będzie kategoryfikować powiększony wielomian Jonesa, a właściwie powiększoną klamrę Kauffmana.
Niech $q := -t^{1/2}$.
Dostaje się tak nowy wielomian, nazwijmy go $K$.
Spełnia trzy aksjomaty:
\begin{itemize}
\item (normalizacja) $K(\SmallUnknot) = q + 1/q$;
\item (stabilizacja) $K(L \sqcup \SmallUnknot) = (q + 1/q) K(L)$;
\item (relacja kłębiasta) \begin{equation}
    q^{-2}     K\left( \MediumPlusCrossingArrows \right) -
    q^{2}      K\left( \MediumMinusCrossingArrows \right) =
    (q^{-1}-q) K\left( \MediumJustSmoothing \right).
\end{equation}
\end{itemize}

Stąd widać już, jakie grupy dobrać dla niewęzła:
\begin{equation}
    H^{i,j} = \begin{cases}
        \Z & \textrm{ jeśli } i = 0, j = \pm 1 \\
        0  & \textrm{ w przeciwnym razie}.
    \end{cases}
\end{equation}
Wtedy spełniona jest równość
\begin{equation}
    K(L, q) = \sum_{i, j} (-1)^i q^j \operatorname{rk} H^{i, j} (L).
\end{equation}
Pozostało powtórzyć to dla dowolnego splotu.
Wzór o sumowaniu stanów przybiera postać:
\begin{equation}
    K(L, q) = \sum_s (-1)^{(\writhe D - |s|)/2} q^{(3\writhe D - |s|)/2} (q+1/q)^{|sD|}.
\end{equation}

Reprezentacja ta ma jedną wadę: każdy składnik z prawej strony przyczynia się do różnych jednomianów, zatem ma wpływ na różne grupy (których dopiero szukamy).
,,Surowe'' stany nie są prawdziwym odpowiednikiem sympleksów, jakie spotyka się podczas kategoryfikacji charakterystyki Eulera.
Najprostszym pomysłem, jak to naprawić, jest rozbicie ostatniej potęgi $q + 1/q$.
Zauważmy, że ma tyle czynników, ile wygładzenie diagramu ma składowych.
To motywuje definicję:

\begin{definition}[stan wzbogacony]
\index{stan diagramu!wzbogacony}%
    Stan diagramu $D$ razem z przypisaniem znaku $+$ lub $-$ do każdego okręgu $sD$ nazywamy stanem wzbogaconym.
\end{definition}

Dla ustalonego wzbogaconego stanu $S$ diagramu $D$ oznaczmy przez $\tau(S)$ sumę znaków przypisanych do okręgów\footnote{Oznaczenie wzięte z pracy Viro, żywimy nadzieję, że nikt nie weźmie $\tau$ za liczbę kolorowań z rodziału drugiego. Poza tym, Viro pisze $\sigma(s)$ zamiast naszego $|s|$ oraz $|s|$ zamiast naszego $|sD|$. Ostrożność wskazana.}.
Wtedy
\begin{equation}
    q^{(3 \writhe D - |s|)/2} (q + 1/q)^{|sD|} = \sum_{S/s} q^{(3 \writhe D - |s| + 2 \tau(S))/2},
\end{equation}
gdzie sumowanie odbywa się po wszystkich stanach $S$ wzbogacających stan $s$.
Niech
\begin{equation}
    j(S) := \frac 12 (3 \writhe D - |s| + 2 \tau(S)).
\end{equation}
Dobrnęliśmy do
\begin{equation}
    K(L, q) = \sum_S (-1)^{(\writhe D - |s|)/2} q^{j(S)},
\end{equation}
tym razem sumujemy po wszystkich wzbogaconych stanach diagramu $D$.

Potrzebujemy jeszcze trochę nowych obiektów.
Niech $C(D)$ oznacza wolną abelową grupę generowaną przez wzbogacone stany diagramu $D$, a $C^j(D)$ będzie jej podgrupą generowaną przez wzbogacone stany $S$ takie, że $j(S) = j$.

Czyni to $C(D)$ wolną grupą abelową z $\Z$-gradacją:
\begin{equation}
    C(D) = \bigoplus_{j \in \Z} C^j (D).
\end{equation}

Dla ustalonego stanu wzbogaconego $S$, niech $i(S) = (\writhe D - |s|)/2$.
Określmy ostatnią podgrupę, $C^{i,j}(D) \le C^j(S)$ generowaną przez wzbudzone stany $S$, dla których $i(S) = i$.
Dostajemy wreszcie
\begin{equation}
    K(L, q) = \sum_{j = -\infty}^\infty q^j \sum_{i = -\infty}^\infty (-1)^i \operatorname{rk} C^{i, j}(D).
\end{equation}

Teraz ,,wystarczy'' zdefiniować funkcję $d \colon C^{i, j} \to C^{i+1, j}$ oraz sprawdzić, że $d^2 = 0$, czyli że $d$ jest różniczką.
I to byłby już koniec, ale nam brakuje sił, by przybliżyć konstrukcję.
Różniczka pozwala przejść z grup $C^{i,j}$ do grup homologii.
Pewne wyjaśnienia znaleźć można w~\cite[s. 42]{przytycki2015}, gdzie podano przepis wymagający tylko ponumerowania skrzyżowań.
\index[persons]{Przytycki, Józef}%

Topolog izraelski Bar-Natan \cite{barnatan2007} podał algorytm\footnote{Źródło: komentarze pod postem \url{https://mathoverflow.net/a/232267}} o~złożoności obliczeniowej $O(\exp(c \sqrt n))$ liczący homologie Chowanowa, gdzie $c$ jest pewną stałą, zaś $n$ liczbą skrzyżowań na diagramie.
\index[persons]{Bar-Natan, Dror}%
Nie możemy liczyć na istotne przyspieszenie:
znalezienie przybliżenia wielomianu Jonesa jest w ogólności problemem \#P-trudnym (Kuperberg \cite{kuperberg2015}, Vertigan \cite{vertigan2005}),
\index[persons]{Kuperberg, Greg}%
\index[persons]{Vertigan, Dirk}%
ale trywialnym przy znanych homologiach.
(Patrz też fakt \ref{prp:jones_at_roots_of_unity}).
% TODO: Gukov, Halverson, Ruehle, Sulkowski: Learning to unknot
% [31] = Hass J, Lagarias J C and Pippenger N 1999 The computational complexity of knot and link problems J. ACM 46 185 proved that the unknotting problem, i.e. the decision problem whether a given knot K is actually an unknot, is in complexity class NP
% [32] = Kuperberg G 2014 Knottedness is in np, modulo GRH Adv. Math. 256 493: assuming GRH, unknot recognition problem is in coNP, this assumption was later relaxed in [33] = Lackenby M 2017 The efficient certification of knottedness and thurston norm (arXiv:1604.00290)

Kronheimer, Mrówka \cite{kronheimer2011} pokazali:
\index[persons]{Kronheimer, Peter}%
\index[persons]{Mrówka, Tomasz}%

\begin{proposition}
\label{khovanov_detects_unknot}%
    Zredukowana kohomologia Chowanowa wykrywa niewęzeł.
\end{proposition}

\begin{proof}[Niedowód]
% DICTIONARY;sutured;szwowa;rozmaitość
\index{rozmaitość!szwowa}%
    Dowód składa się z~dwóch kroków.
    W~pierwszym panowie pokazują, że istnieje ciąg spektralny zaczynający się od zredukowanej kohomologii Chowanowa, po którym następuje koniec: homologia zdefiniowana osobliwymi instantonami.
    Potem dowodzą, że ta homologia jest izomorficzna z~instantonową homologią Floera szwowego dopełnienia węzła, o~której wiadomo, że wykrywa niewęzeł.
\index{homologia!Floera}%
\end{proof}

\index{homologia!Chowanowa|)}

% Koniec sekcji Homologie



\part{Rodziny węzłów}
\chapter{Wybrane rodziny węzłów}
\input{50-families/intro}
\input{50-families/braid}

\section{Supły}
\index{supeł|(}%
\label{sec:tangle}%
Na przełomie lat sześćdziesiątych i~siedemdziesiątych Conway szukał sposobu na zbudowanie kompletnej tablicy węzłów.
Niezmienniki znane w~tym czasie nie były dostatecznie mocne, by sprostać temu wyzwaniu.
Conway wprowadził pojęcie supła i~chociaż wszystkich węzłów nie można z~nich uzyskać, teoria została pchnięta do przodu.
Supły stanowią budulec splotów takich jak na przykład precle z~definicji~\ref{def:pretzel}.

Sekcja oparta jest na podręczniku Murasugiego \cite{murasugi1996} i~pracach Conwaya \cite{conway1970}, Kauffmana, Goldmana \cite{kauffman1997}, Kauffmana, Lambropoulou \cite{kauffman2004}, a~także Schuberta \cite{schubert1956}, ale nie przeglądowej książce Kawauchiego \cite[s. 34-36]{kawauchi1996}.
\index[persons]{Conway, John}%
\index[persons]{Goldman, Jay}%
\index[persons]{Kauffman, Louis}%
\index[persons]{Lambropoulou, Sofia}%
\index[persons]{Schubert, Horst}%

Supły występują także w polskojęzycznym artykule Janiak-Osajcy, Pogody \cite{janiak2004}, ale ten zawiera nieprzyjemną pułapkę: wprowadza notację sprzeczną z~powszechnie akceptowaną.
\index[persons]{Janiak-Osajca, Agnieszka}%
\index[persons]{Pogoda, Zdzisław}%

% DICTIONARY;tangle;supeł;-
\begin{definition}[supeł]
    \label{def:tangle}
    Zawarty w~kole fragment diagramu splotu o~dwóch łukach wyjściowych oraz dwóch wejściowych, nazywamy supłem.
\end{definition}

% z: AMPHICHEIRALS ACCORDING TO TAIT AND HASEMAN
Słowo ,,supeł'' zaproponowała Haseman, już ona rysowała supły wewnątrz pomocniczego okręgu, który tnie diagram w~czterech punktach.
\index[persons]{Haseman, Mary}%

Istnieją dwa rodzaje supłów:
\begin{comment}
\begin{figure}[H]
    \centering
    \begin{minipage}[b]{.48\linewidth}
        \[\LargeTangleAlternatingYes\]
        \subcaption{supeł naprzemienny}
    \end{minipage}
    \begin{minipage}[b]{.48\linewidth}
        \centering
        \[\LargeTangleAlternatingNo\]
        \subcaption{supeł sąsiądujący}
    \end{minipage}
\end{figure}
\end{comment}

Podobnie jak dla węzłów, pojawia się naturalne pytanie o~równoważność dwóch supłów.
Jest tak wtedy, gdy istnieje homeomorfizm kuli na siebie, który przekształca jeden supeł na drugi, ale nie rusza sfery otaczającej.
Dla diagramów odpowiada to ruchom Reidemeistera, nie mamy jednak prawa opuszczać kuli zawierającej supeł.

Dużo dokładniej mówi o tym Turajew \cite{turaev1990}:
\index[persons]{Turajew, Władymir (Тураев, Владимир Георгиевич)}%

\begin{proposition}
    Oznaczmy przez OTa kategorię zorientowanych supłów.
    Jej obiektami są skończone ciągi złożone z~$\pm 1$, razem z~ciągiem pustym.
    Morfizm ciągu $\varepsilon = (\varepsilon_1, \ldots, \varepsilon_k)$ w ciąg $\nu = (\nu_1, \ldots, \nu_l)$ jest klasą izotopii zorientowanego $(k, l)$-supła $L$ tak, że źródłem $L$ jest $\varepsilon$, zaś celem $\nu$.
    Na przykład supły $\curvearrowright$, $\curvearrowleft$ oraz $X_+$ opisane są przez morfizmy $\varnothing \to (-1, 1)$, $\varnothing \to (1, -1)$ oraz $(1, 1) \to (1, 1)$.
    Składanie morfizmów odpowiada mnożeniu supłów.

    Wprowadźmy iloczyn tensorowy $\otimes$.
    Iloczynem obiektów $\varepsilon, \nu$ (które znaczą to, co wcześniej) jest obiekt $(\varepsilon_1, \ldots, \varepsilon_k, \nu_1, \ldots, \nu_l)$, zaś iloczyn tensorowy morfizmów będzie iloczynem tensorowym splotów i łatwo widać, że $(OTa, \otimes, \varnothing)$ jest ściśle monoidalną kategorią (cokolwiek to znaczy).

    Zdefiniujmy cztery słowa:
    \begin{align}
        A & = (\downarrow \, \downarrow \, \curvearrowright) \circ (\downarrow \, \downarrow \, \uparrow \, \curvearrowright \, \downarrow) \circ (\downarrow \, \downarrow X_\pm \downarrow \, \downarrow) \circ (\downarrow \inversedcurvearrowright \, \uparrow \, \downarrow \, \downarrow) \circ (\inversedcurvearrowright \downarrow \, \downarrow) \\
        B & = (\curvearrowleft \, \downarrow \, \downarrow) \circ (\downarrow \, \curvearrowleft \, \uparrow \, \downarrow \, \downarrow) \circ (\downarrow \, \downarrow X_\pm \downarrow \, \downarrow) \circ (\downarrow \, \downarrow \, \uparrow  \inversedcurvearrowleft \downarrow) \circ (\downarrow \, \downarrow \inversedcurvearrowleft) \\
        T & = (\curvearrowleft \, \uparrow \, \downarrow) \circ (\downarrow X_- \downarrow) \circ (\downarrow \, \uparrow \inversedcurvearrowleft) \\
        Y & = (\uparrow \, \downarrow \, \curvearrowright) \circ (\downarrow X_+ \downarrow) \circ (\inversedcurvearrowright \uparrow \, \downarrow)
    \end{align}
    Kategorię OTa można przedstawić przez morfizmy $\inversedcurvearrowright, \inversedcurvearrowleft, \curvearrowright, \curvearrowleft, X_+, X_-$ (generatory) oraz relacje:
    \begin{align}
        (\curvearrowright \, \uparrow) \circ (\uparrow \inversedcurvearrowright) = & \uparrow \, = (\uparrow \, \curvearrowleft) \circ (\inversedcurvearrowleft \uparrow) \\
        (\curvearrowleft \, \downarrow) \circ (\downarrow \inversedcurvearrowleft) = & \downarrow \, = (\downarrow \, \curvearrowright) \circ (\inversedcurvearrowright \downarrow) \\
        A & = B \\
        X_+ \circ X_- & = X_- \circ X_+ = \, \uparrow \, \uparrow \\
        (X_+ \uparrow) \circ (\uparrow X_-) \circ (X_+ \uparrow) & = (\uparrow X_+) \circ (X_+ \uparrow) \circ (\uparrow X_+) \\
        (\uparrow \, \curvearrowright) \circ (X_\pm \downarrow) \circ (\uparrow \inversedcurvearrowleft) & = \, \uparrow \\
        Y \circ T = \, \downarrow \, \uparrow, & \quad T \circ Y = \, \uparrow \, \downarrow
    \end{align}
\end{proposition}

\begin{proof}
    Dowód twierdzenia oraz graficzne przedstawienie relacji z kategorii OTa zawiera praca Turajewa \cite{turaev1990}.
    Wszystkie relacje odpowiadają ruchom Reidemeistera.
\index{ruch!Reidemeistera}%
    Trzecie od końca równanie to geometryczny wariant równania Yanga-Baxtera.
\index{równanie Yanga-Baxtera}%
% TODO: przerysować... do kodu

Patrz też \cite[s. 29-30]{duzhin2012} (Czmutow, Dużin, Mostovoy przygotowali tam śliczne rysunki) albo \cite[s. 31]{schieber2018} (gdzie Schieber przedstawił ruchy Reidemeistera i~cięte diagramy).
\index[persons]{Czmutow, Siergiej (Чмутов, Сергей Владимирович)}%
\index[persons]{Dużin, Siergiej (Дужин, Сергей Васильевич)}%
\index[persons]{Mostovoy, Jacob}%
\index[persons]{Schieber, Nathaniel}%
% sliced diagrams
% DICTIONARY;sliced;cięty;diagram
\end{proof}

Wszystkich supłów jest bardzo dużo, więc ograniczymy się do końca rozdziału do pewnej ich regularnej rodziny.
Oto cztery podstawowe supły:
\begin{comment}
\begin{figure}[H]
    \centering
    \begin{minipage}[b]{.23\linewidth}
        \[
            \LargeTangleBasicZero
        \]
        \subcaption{$(0)$}
    \end{minipage}
    \begin{minipage}[b]{.23\linewidth}
        \[
            \LargeTangleBasicInfinity
        \]
        \subcaption{$(\infty) = (0, 0)$}
    \end{minipage}
    \begin{minipage}[b]{.23\linewidth}
        \[
            \LargeTangleBasicMinus
        \]
        \subcaption{$(-1)$}
    \end{minipage}
    \begin{minipage}[b]{.23\linewidth}
        \[
            \LargeTangleBasicPlus
        \]
        \subcaption{$(+1)$}
    \end{minipage}
\end{figure}
\end{comment}

\begin{definition}
    Supły powstające z~$(0)$ lub $(\infty)$ przez homeomorfizm kuli na siebie permutujący wejścia i~wyjścia nazywamy wymiernymi.
\end{definition}

Pokażemy teraz, jak zamienić dowolny skończony ciąg liczb całkowitych w~supeł, jako że jest to prostsze od procesu odwrotnego.
Nazwijmy jednak najpierw dwa rodzaje skrętów:
\begin{comment}
\begin{figure}[H]
    \centering
    \begin{minipage}[b]{.48\linewidth}
        \[\LargeTwistsRight\]
        \subcaption{skręty prawe}
    \end{minipage}
    \begin{minipage}[b]{.48\linewidth}
        \centering
        \[\LargeTwistsLeft\]
        \subcaption{skręty lewe}
    \end{minipage}
\end{figure}
\end{comment}

Mając ciąg $(a_1, a_2, \ldots, a_n)$ wykonujemy naprzemiennie obroty półsferą dolną (SW--SE, takie nazywamy pionowymi) oraz prawą (SW--NW, a takie poziomymi) tak, by ostatni był obrót poziomy.
Oto reguła zgodnie z którą wybieramy kierunek obrotów.
Podczas pionowych obrotów, prawy skręt jest dodatni, zaś lewy ujemny.
Podczas poziomych, zamieniamy znaki: prawy odpowiada ujemnym wyrazom ciągu, lewy dodatnim.
Wreszcie, jeżeli $n$ jest nieparzyste, zaczynamy od supła $T(0)$, w przeciwnym razie od supła $T(0, 0)$.

Różnym ciągom mogą odpowiadać te same supły, na przykład $T(-2, 3, 3) = T(3, -2)$, więc notacja nie jest jednoznaczna, ale to nic złego.
Każdemu supłowi przypiszmy pewną liczbę wymierną, według przepisu:
\begin{equation}
    T(a_1, a_2, \ldots, a_n) \mapsto a_n + \frac{1}{\ldots + 1/a_1} = \frac \alpha \beta.
\end{equation}

\begin{proposition}
    Istnieje bijekcja między supłami wymiernymi oraz ułamkami łańcuchowymi.
\end{proposition}

\begin{proof}[Niedowód]
    Praca Conwaya \cite[s. 331-332]{conway1970}.
\end{proof}

\begin{proposition}[ćwiczenie 9.2.6 w \cite{murasugi1996}]
    \label{prp:continued_fractions}
    Niech $T(a_1, a_2, \ldots, a_n)$ będzie supłem różnym od $0$ oraz $\infty$.
    Wtedy bez straty ogólności można założyć, że wszystkie liczby $a_i$ są tego samego znaku.
\end{proposition}

Z każdym supłem $T$ związane jest jego odbicie $\overline T$, obraz wyjściowego przez symetrię względem prostej $y = -x$.
Mając dwa supły obok siebie, można dokonać ich sklejenia wzdłuż połówek kul, w~których leżą:
\begin{comment}
\begin{figure}[H]
    \centering
    \begin{minipage}[b]{.23\linewidth}
        \[
            \LargeTangleSummandA
        \]
        \subcaption{jakiś supeł}
    \end{minipage}
    \begin{minipage}[b]{.23\linewidth}
        \centering
        \[
            \LargeTangleSummandB
        \]
        \subcaption{jakiś inny supeł}
    \end{minipage}
    \begin{minipage}[b]{.48\linewidth}
        \centering
        \[
            \LargeTangleSumAB
        \]
        \subcaption{suma tych supłów}
    \end{minipage}
\end{figure}
\end{comment}

Oznaczmy tak otrzymany splot przez $T_1 + T_2$.
Niektórzy definiują dalsze działania, jak produkt: $T_1 \cdot T_2 = \overline T_1 + T_2$ czy rozgałęzienie, $\overline T_1 + \overline T_2$.
Rodzina supłów wymiernych jest zamknięta na branie produktów, ale nie sum.
Wprowadzamy więc następującą, ogólniejszą definicję.
Supeł będący skończoną sumą supłów wymiernych, ich luster, odbić lub odbić luster nazywamy algebraicznym.

Conway korzystając ze skończonej listy ,,wielościanów podstawowych'' (pewnych grafów planarnych) był w stanie zakodować wszystkie węzły o~małej liczbie skrzyżowań.
Ale ponieważ notacja ta nie jest uniwersalna -- im więcej skrzyżowań, tym więcej wielościanów potrzeba, by opisać wszystkie węzły -- nie opiszemy, jak działa.

Przez zszycie par łuków wejściowych (lub wyjściowych) zamieniamy supły w~węzły:
\begin{figure}[H]
    \centering
    \begin{minipage}[b]{.3\linewidth}
        \centering
        \LargeTangleFraction
        \subcaption{supeł $T$}
    \end{minipage}
    \begin{minipage}[b]{.3\linewidth}
        \centering
        \LargeTangleFractionNumerator
        \subcaption{licznik, $N(T)$}
    \end{minipage}
    \begin{minipage}[b]{.3\linewidth}
        \centering
        \LargeTangleFractionDenominator
        \subcaption{mianownik, $D(T)$}
    \end{minipage}
\end{figure}

% DICTIONARY;... numerator;licznik ...;supeł
% DICTIONARY;... denominator;mianownik ...;supeł
Oznaczenia $N(T)$ oraz $D(T)$ pochodzą od angielskich słów \emph{numerator}, \emph{denominator}.
Być może nie jest jasne, dlaczego terminy stosowane zazwyczaj do opisu ułamków stosujemy wobec diagramów splotów.
Nazewnictwo nie jest przypadkowe.

\begin{proposition}
\index{ułamek supła}%
    Ułamek supła zadany wzorem
    \begin{equation}
        F(A) = \frac{\conway_{N(A)}(z)}{\conway_{D(A)}(z)}
    \end{equation}
    spełnia zależność $F(A+B) = F(A) + F(B)$.
\end{proposition}

\begin{proof}
    Praca \cite{conway1970} Conwaya.
\end{proof}

Praca \cite{conway1970} zawiera jeszcze jeden ciekawy rezultat, uogólniony przez Lickorisha i~Milletta w~\cite[fakt 12]{lickorish1987}.
Używamy tu wersji wielomianu HOMFLY o zmiennych $l, m$.

\begin{proposition}
    Niech $A, B$ będą supłami, zaś $T_n$ (odpowiednio: $T_d$)  wielomianem HOMFLY licznika (mianownika) supła $T$.
    Wtedy
    \begin{equation}
        (\mu^2 - 1)(A+B)_n = \mu(A_nB_n + A_dB_d) - (A_nB_d + A_dB_n),
    \end{equation}
    gdzie $\mu = -(l + 1/l)/m$. % oraz
    %\begin{equation}
        %(A+B)_d = A_dB_d.
    %\end{equation}
    % TODO: tego nie potrafię znaleźć w pracy Lickorisha
\end{proposition}

\input{50-families/tangle-bridge}

\input{50-families/tangle-mutants}

\index{supeł|)}%


\input{50-families/pretzel}

\section{Węzły Lissajous}
% szkielet: praca Lamma
\index{węzeł!Lissajous|(}%
Węzły Lissajous zdefiniowali Bogle, Hearst, Jones i Stoiłow \cite{bogle1994} jako węzły, których pewien diagram jest krzywą Lissajous.
\index[persons]{Bogle, Miles}%
\index[persons]{Hearst, John}%
\index[persons]{Jones, Vaughan}%
\index[persons]{Stoiłow, Luben}%

\begin{definition}[węzeł Lissajous]
    Niech $n_x, n_y, n_z$ będą liczbami całkowitymi (zwanymi dalej ,,częstotliwościami''), zaś  $\varphi_x, \varphi_y, \varphi_z$ liczbami rzeczywistymi (,,fazami'').
    Wtedy węzeł zadany w $\R^3$ parametrycznie:
    \begin{equation}
        x = \cos(n_xt + \varphi_x), \quad
        y = \cos(n_yt + \varphi_y), \quad
        z = \cos(n_zt + \varphi_z),
    \end{equation}
    nazywamy węzłem Lissajous.
\end{definition}

Węzeł nie może posiadać samoprzecięć, dlatego żadna z~wielkości $n_i\varphi_j-n_j\varphi_i$, dla różnych indeksów $i, j$ nie może być krotnością $\pi$.
Bez straty ogólności możemy założyć, że $\varphi_z = 0$.
Dodatkowo stałe $n_x, n_y, n_z$ muszą być parami względnie pierwsze.

Wiele węzłów jest węzłami Lissajous:

\begin{example}
    Dla $n_x = 3$, $n_y = 2$, $n_z = 7$, $\varphi_x = 7/10$, $\varphi_y = 2/10$ mamy węzeł $5_2$.
\end{example}

\begin{example}
    Węzły pierwsze $6_1$, $7_4$, $8_{15}$, $10_1$, $10_{35}$, $10_{58}$ są węzłami Lissajous.
\end{example}

\begin{example}
    Suma prawego i~lewego trójlistnika, suma dwóch kopii $5_2$ są węzłami Lissajous.
\end{example}

Jest wiele węzłów Lissajous:

\begin{proposition}
    Istnieje nieskończenie wiele węzłów Lissajous.
\end{proposition}

\begin{proof}
    Niech $a, b > 1$ będą względnie pierwsze.
    Lamm \cite{lamm1997} pokazał dużo ogólniejszy wynik.
    Mianowicie: węzeł Lissajous o~częstotliwościach $n_x = a$, $n_y = b$, $n_z = 2ab-a-b$ oraz fazach:
    \begin{equation}
        \varphi_x = \frac{2n_x-1}{n_z} \pi, \quad
        \varphi_y = \frac{\pi}{n_z}, \quad
        \varphi_z = 0
    \end{equation}
    posiada sygnaturę $\sigma = a+b-ab-1$ i genus $g = -\sigma/2$.
\index{sygnatura}%
\index{genus}%
\end{proof}

Węzły Lissajous są bardzo symetryczne.

\begin{proposition}
\index{węzeł!silnie dodatnio achiralny}%
\label{prp:lissajus_odd}%
    Jeśli wszystkie częstotliwości $n_x, n_y, n_z$ węzła Lissajous są nieparzyste, to węzeł ten jest silnie dodatnio achiralny.
\end{proposition}

\begin{proposition}
\index{węzeł!okresowy}%
\label{prp:lissajous_two_periodic}%
    Jeśli jedna z częstotliwości $n_x$ węzła Lissajous jest parzysta, to węzeł ten jest 2-okresowy.
\end{proposition}

\begin{proof}
    Półobrót węzła $K$ wokół osi $x$ odwzorowuje go w~siebie.
\end{proof}

To nakłada ograniczenia na wielomian Alexandera.

\begin{proposition}
\index{wielomian!Alexandera}
\label{prp:lissajous_alexander}%
    Niech $K$ będzie węzłem Lissajous.
    Wtedy $\alexander(t)$, jego wielomian Alexandera, jest kwadratem w~pierścieniu $(\Z/2\Z)[t]$.
\end{proposition}

\begin{proof}
    Rozpatrzmy dwa przypadki.

    \paragraph{Przypadek I}
    Jeżeli wszystkie trzy częstotliwości $n_x, n_y, n_z$ są nieparzyste, to węzeł $K$ jest silnie dodatnio achiralny (fakt \ref{prp:lissajus_odd}).
    Hartley, Kawauchi \cite{hartley1979} pokazali, że wymierna grupa homologii pewnego konkretnego nakrycia $S^3 \setminus K$ jest sumą prostą $M \oplus M$, gdzie $M$ jest $\Q\langle t \rangle$-modułem, więc wielomian Alexandera $\alexander_K(t)$ jest kwadratem w~pierścieniu $\Z[t]$.
    Szczegóły ich pracy są dla nas nieistotne; nam wystarczy zauważyć, że stąd już wynika kwadratowość wielomianu $\alexander_K(t)$ w~pierścieniu $(\Z/2\Z)[t]$.

    \paragraph{Przypadek II}
    Jeżeli jedna z częstotliwości jest nieparzysta, to z faktów~\ref{prp:lissajous_two_periodic} oraz~\ref{prp:murasugi_periodic} (dla $n=2^1$ oraz $\lambda = 1$) dostajemy równość
    \begin{equation}
\label{eqn:lissajous_squared}%
        \alexander_K(t) \equiv \alexander^2_{J}(t) \mod 2,
    \end{equation}
    która kończy dowód.
\end{proof}

Warto zwrócić uwagę, że bycie silnie dodatnio achiralnym jest bardzo restrykcyjnym warunkiem.
Spośród pierwszych węzłów o~co najwyżej 12 skrzyżowaniach, spełnia go mniej niż dziesięć.
Lamm \cite{lamm2021} wymienia: 10a103 ($10_{99}$), 10a121 ($10_{123}$), 12a427, 12a1019, 12a1105, 12a1202, 12n706 oraz być może 12a435.
\index{węzeł!10-99}%
\index{węzeł!10a103}%
\index{węzeł!10-123}%
\index{węzeł!10a121}%
\index{węzeł!12a427}%
\index{węzeł!12a435}%
\index{węzeł!12n706}%
\index{węzeł!12a1019}%
\index{węzeł!12a1105}%
\index{węzeł!12a1202}%
W 2018 roku wiedzieliśmy tylko, że 10a103 jest węzłem Lissajous i nie wiedzieliśmy, co z pozostałymi węzłami.

\begin{example}
    Trójlistnik oraz ósemka nie są węzłami Lissajous.
\end{example}

Wygodnie jest przeformułować warunek z faktu~\ref{prp:lissajous_alexander} do następującej postaci.
Wielomian $\alexander(t) = A_0 + A_1(t+1/t) + \cdots + A_n(t^n + 1/t^n)$ jest kwadratem modulo $2$ wtedy i tylko wtedy, gdy współczynniki $A_{2k+1}$ są parzyste dla $k \ge 0$.

\begin{corollary}
\index{niezmiennik!Arfa}%
    Niech $K$ będzie węzłem Lissajous, wtedy niezmiennik $\operatorname{Arf} K = 0$ znika.
\end{corollary}

\begin{proof}
    Niech $\alexander(t) = A_0 + A_1(t+1/t) + \cdots + A_n(t^n + 1/t^n)$ będzie  wielomianem Alexandera.
    Wtedy $\alexander_K(1) = \pm 1$, zatem $0 = \pm 1 + A_0 + 2A_1 + \cdots + 2A_n$.
    Możemy teraz odjąć to od $\alexander_K(-1) = A_0 - 2A_1 \pm \cdots$, by uzyskać $\alexander_K(-1) = \pm 1 - 4A_1 - 4A_3 - \cdots$.
    Razem z wygodnym przeformułowaniem daje to $\alexander_K(-1) \equiv \pm 1 \mod 8$, co w połączeniu z~\ref{prp:arf_murasugi} kończy dowód.
\end{proof}

\begin{corollary}
\index{węzeł!rozwłókniony}%
\label{cor:lissajous_fibered}%
    Włókniste węzły o nieparzystym genusie nie są węzłami Lissajous.
\end{corollary}

%=% Lamm
\begin{proof}
    Wynika to z~naszego wygodnego przeformułowania oraz faktu \ref{prp:fibered_alexander_monic}.
\end{proof}

\begin{corollary}
\index{węzeł!dwumostowy}%
\label{cor:lissajous_twobridge}%
    Niech $K$ będzie dwumostowym węzłem Lissajous.
    Wtedy $\alexander_K(t) \equiv 1 \mod 2$.
\end{corollary}

\begin{proof}
    Nietrywialny węzeł o dwóch mostach nie może być silnie dodatnio achiralny (patrz \cite{hartley1979}).
    Jeśli jest węzłem Lissajous, jedna z jego częstotliwości okazuje się być parzysta.
    Wtedy jego faktor jest trywialny i~kongruencja~\ref{eqn:lissajous_squared} daje $\alexander_K(t) \equiv 1 \mod 2$.
\index{faktor}%
    % TODO: co to jest faktor? To jedyne miejsce w całej książce, gdzie używa się tego słowa. factor?
\end{proof}

%=% Lamm
\begin{corollary}
\index{węzeł!rozwłókniony}%
    Dwumostowe węzły rozwłóknione nie są węzłami Lissajous.
\end{corollary}

Można skorzystać z tego samego argumentu, co w dowodzie wniosku~\ref{cor:lissajous_fibered} oraz~\ref{cor:lissajous_twobridge}.

%=% Lamm
\begin{proposition}
    Niech $p, q$ będą względnie pierwszymi liczbami różnymi od $0, \pm 1$, zaś $K$ węzłem.
    Wtedy $(p, q)$-kabel węzła $K$ nie jest węzłem Lissajous.
\end{proposition}

Lamm pisze, że obiekt ten -- $(p, q)$-kabel -- zdefiniowano w książce Eisenbuda/Neumanna ,,Three-dimensional link theory and invariants of plane curve singularities''.
\index[persons]{Neumann, Walter}%
\index[persons]{Eisenbud, David}%
Celowo pomijamy ją w bibliografii, prawdopodobnie chodzi o zwykłe węzły satelitarne.

\begin{proof}
\index[persons]{Seifert, Herbert}%
    Niech $L$ będzie $(p, q)$-kablem węzła $K$.
    Seifert \cite{seifert1950} pokazał, że
    \begin{equation}
        \alexander_L(t) = \frac{(t^{pq}-1)(t-1)}{(t^p-1)(t^q-1)} \alexander_K(t^p),
    \end{equation}
    przy czym wyjątkowo wielomian $\alexander$ nie jest symetryczny w $t$ oraz $1/t$, tylko unormowany.
    Dwa najwyższe współczynniki w wielomianie ,,ułamkowym'' to $\pm 1$, a skoro $|p| > 1$, dwa najwyższe współczynniki $\alexander_L$ to także $\pm 1$.
    ,,Wygodne sformułowanie'' kończy dowód.
\end{proof}

\begin{corollary}
\index{węzeł!torusowy}%
    Nietrywialne węzły torusowe nie są węzłami Lissajous.
\end{corollary}

\begin{proof}
    Węzły torusowe to kable niewęzła.
\end{proof}

\begin{corollary}
    Nietrywialne węzły algebraiczne nie są węzłami Lissajous.
\end{corollary}

Co to są węzły algebraiczne?
Nie wiemy.

\begin{proof}
    Wynika to z ogólniejszego stwierdzenia: jeśli spełniony jest warunek $|p_n|, |q_n| > 1$, to iterowane węzły torusowe typu $((p_n, q_n), (p_1, q_1))$ nie są węzłami Lissajous.
\end{proof}

Podamy teraz pierwszy warunek wystarczający, by być węzłem Lissajous.
Znaleźli go Hoste oraz Zirbel \cite{zirbel2006}:
\index[persons]{Hoste, Jim}%
\index[persons]{Zirbel, Laura}%

\begin{proposition}
\index{niezmiennik!Arfa}%
\index{węzeł!skręcony}%
    Niech $K$ będzie węzłem skręconym.
    Następujące warunki są równoważne:
    \begin{itemize}
        \item węzeł $K$ jest węzłem Lissajous;
        \item niezmiennik Arfa węzła $K$ znika, $\operatorname{Arf} K = 0$.
    \end{itemize}
\end{proposition}

Węzły bilardowe to zamknięte trajektorie kuli, która zostaje wystrzelona z jednej ze ścian sześcianu, odbija się pod takim samym kątem, pod jakim pada na ściany.
\index{węzeł!bilardowy}%
Jones, Przytycki \cite{jones1998} pokazali, że węzły bilardowe to dokładnie węzły Lissajous i~zadali pytanie, czy każdy węzeł można zrealizować jako trajektorię kuli w~jakimś wielościanie.
\index[persons]{Jones, Vaughan}%
\index[persons]{Przytycki, Józef}%

Odpowiedź jest pozytywna.
Koseleff, Pecker \cite{koseleff2014} korzystając z~twierdzenia Manturowa
\index[persons]{Koseleff, Pierre-Vincent}%
\index[persons]{Pecker, Daniel}%
\index{warkocz!toryczny}%
(każdy splot jest domknięciem kwazitorycznego warkocza, patrz komentarz na stronie \pageref{thm:alexander})
pokazują, że każdy węzeł ma diagram, który jest wielokątem gwiaździstym.
Użyte zostało twierdzenie Kroneckera z~1884 roku: jeśli liczby $\theta_0 = 1, \theta_1, \ldots, \theta_k$ są liniowo niezależne nad ciałem $\Q$, to zbiór punktów $(\lfloor n\theta_i \rfloor_{i=0}^k)_{n=0}^\infty$ leży gęsto w kostce jednostkowej.

Lamm, Obermeyer \cite{obermeyer1999} dowiedli w 1999, że węzły bilardowe wewnątrz walca są taśmowe albo okresowe, więc w walcu nie można zrealizować każdego węzła.
\index[persons]{Lamm, Christoph}%
\index[persons]{Obermeyer, Daniel}%
\index{węzeł!okresowy}%
\index{węzeł!taśmowy}%
Lamm postawił hipotezę, że jest to możliwe w eliptycznym walcu.
Pozytywnej odpowiedzi ponownie udzielił niedawno Pecker \cite{pecker2012}.
\index[persons]{Pecker, Daniel}%

\index{węzeł!Lissajous|)}

% Koniec sekcji Węzły Lissajous


\input{50-families/torus}
\input{50-families/satellite}

% TODO: whole file needs a serious refactor

\section{Węzły hiperboliczne}
\index{węzeł!hiperboliczny|(}%
\label{sec:hyperbolic}%
Jak pisaliśmy w~sekcji~\ref{sec:mutant}, słynne węzły Conwaya oraz Kinoshity-Terasakiego odróżnił od siebie po raz pierwszy Riley.
\index[persons]{Riley, Robert}%
\index{węzeł!Conwaya}%
\index{węzeł!Kinoshity-Terasakiego}%
Zbadał paraboliczne reprezentacje ich grup w~skończoną grupę prostą $PSL(2, 7)$, co doprowadziło go do odkrycia struktury hiperbolicznej w~dopełnieniu ósemki \cite{riley1975}.
\index{węzeł!4-1}%
% https://arxiv.org/pdf/2002.00564.pdf
Zainspirowany tym wynikiem Thurston najpierw rozłożył dopełnienie ósemki na dwa idealne wielościany, a~potem znacznie uogólnił swój przykład.
\index[persons]{Thurston, William}%

Autorzy tej książki nie rozumieją węzłów hiperbolicznych!
Przeczytaliśmy przeglądową pracę Kalfagianniego, Futera oraz Purcell \cite{purcell2019}, notatki z~wykładów prowadzone przez samą Purcell i~wreszcie książkę \cite{purcell2020}, jaka z~nich wyrosła.
\index[persons]{Futer, David}%
\index[persons]{Kalfagianni, Efstratia}%
\index[persons]{Purcell, Jessica}%
Wiedzę o~węzłach hiperbolicznych można czerpać także z~artykułu Weeksa \cite{weeks2005}.
\index[persons]{Weeks, Jeff}%
Streścimy to, co zdążyliśmy przeczytać i odsyłamy zaintrygowanego Czytelnika do cytowanych pozycji.

Sekcję o węzłach hiperbolicznych wypada zacząć od przytoczenia ich definicji.
Owszem, zrobimy to, ale bez wyjaśnienia znaczenia użytych tu trudnych słów:

\begin{definition}[hiperboliczny]
    Węzeł $K$, na dopełnieniu którego można zadać zupełną metrykę o~stałej krzywiźnie $-1$ nazywamy hiperbolicznym.
\end{definition}

Jeśli wiemy, czym jest 3-przestrzeń hiperboliczna $\mathbb H^3$, możemy używać równoważnej:

\begin{definition}
    Węzeł $K$ taki, że istnieje dyskretna oraz beztorsyjna grupa izometrii $\Gamma$ izomorficzna z~$\pi_1(S^3 \setminus K)$ taka, że $S^3 \setminus K = \mathbb H^3 / \Gamma$, nazywamy hiperbolicznym.
\end{definition}

Nie pamiętamy kto, gdzie i kiedy, ale ktoś gdzieś i kiedyś opowiedział nam, że Thurston podejrzewał, że każda 3-rozmaitość rozkłada się wzdłuż sfer i~nieściśliwych torusów na części wyposażone w~jedną z~ośmiu kanonicznych geometrii:
\begin{itemize}
\item sferyczną $S^3$, albo euklidesową $E^3$, albo hiperboliczną $H^3$,
\item albo $S^2 \times \R$, albo $H^2 \times \R$,
\item albo uniwersalne nakrycie $SL(2, \R)$,
\item albo geometrię Sol albo geometrię Nil.
\end{itemize}
\index{hipoteza!geometryzacyjna Thurstona}%
Dowód hipotezy geometryzacyjnej dostarczył mniej więcej dwie dekady później Perelman, ale dla nas to jest raczej bez znaczenia.
\index[persons]{Perelman, Grigorij}%
Wystarczy nam starszy wynik samego Thurstona \cite{thurston1982}, który zajął się szczególnym przypadkiem rozmaitości Hakena.
\index{rozmaitość!Hakena}%
Dopełnienia splotów stanowią ich szczególny przypadek, dokładnej definicji nie podajemy umyślnie.
I tak z~przełomowych prac Thurstona z~lat 70. oraz 80. wynika, że dopełnienie splotu jest rozmaitością włóknistą Seiferta, toroidalną albo hiperboliczną.
\index{rozmaitość!włóknista Seiferta}%
\index{rozmaitość!toroidalna}%
\index{rozmaitość!hiperboliczna}%
Innymi słowy, Thurston przedstawił:

\begin{theorem}[trychotomia Thurstona]
\index{twierdzenie!Thurstona}%
\index{węzeł!satelitarny}%
\index{węzeł!torusowy}%
    Każdy węzeł jest satelitarny, torusowy albo hiperboliczny.
\end{theorem}
% luźno związane: https://deltami.edu.pl/2013/01/william-thurston-i-hipoteza-geometryzacyjna/

Thurston pisze, że Riley przypuszczał to samo z pomocą komputera na podstawie dużej liczby przykładów.
\index[persons]{Riley, Robert}%
\index[persons]{Thurston, William}%

\begin{proof}[Niedowód]
    Thurston \cite[s. 360]{thurston1982} dowodzi ogólniejszego faktu: wnętrze zwartej 3-rozmaitości z~niepustym brzegiem $M$ ma strukturę hiperboliczną wtedy i tylko wtedy, gdy rozmaitość $M$ jest homotopijnie atoroidalna, pierwsza oraz nie jest homeomorficzna z $T^2 \times I / (\Z/2)$, gdzie $\Z/2$ odwraca $I$.
    Stąd wynika już trychotomia.
\end{proof}

Węzły hiperboliczne stanowią najliczniejszą i~najmniej zrozumianą rodzinę węzłów.
Sam Nead, użytkownik portalu MathOverflow napisał, że kryterium Thurstona dzięki maszynerii JSJ oraz pracom innych osób można wysłowić algebraicznie.
\index{rozkład Jaco-Shalena-Johannsona}%

\begin{proposition}
    % There is a topological criterion due to Thurston.  Using the JSJ machine (and work of many others) this criterion can also be phrased algebraically.  I'll essay these below.  Please note that the situation is much simpler for knots.  To answer your question most directly, here is the desired reference to Wikipedia.
    % http://en.wikipedia.org/wiki/Hyperbolic_link
    % This page refers to the books of Colin Adams and William Thurston.  Both are excellent.
    % Now, here is Thurston's criterion. (EDIT: exposition improved after reading Bruno Martelli's answer.)
    % Suppose that $L$ is the link and $X$ is the link complement.  Suppose $\pi = \pi_1(X)$. We assume the following properties (and each property assumes the proceeding ones). $\newcommand{\ZZ}{\mathbb{Z}}$
    %  - $L$ is not a split link.  Equivalently, $X$ is contains no essential two-sphere.  Equivalently, $\pi$ is not a free product.
    %  - $L$ is not the unknot. Equivalently, $X$ contains no essential disk. Equivalently, $\pi$ is not $\ZZ$.
    %  - $L$ has no component that is an "undisturbed satellite knot".  Equivalently, $X$ contains no essential torus.
    %  - $L$ is not a torus knot. Equivalently, $X$ contains no essential annulus. These last two topological properties are equivalent to $\pi$ not containing a copy of $\ZZ^2$.
    % Then $X$ admits a hyperbolic structure.
    Niech $L$ będzie splotem, który nie rozszczepia się, nie jest niewęzłem, nie posiada wśród ogniw niezakłóconego węzła satelitarnego\footnote{,,undisturbed satellite knot''} oraz nie jest węzłem torusowym.
\index{splot!rozszczepialny}%
    Wtedy $L$ jest hiperboliczny.
\end{proposition}

\begin{proposition}
    Niech $L$ będzie splotem takim, że jego dopełnienie $S^3 \setminus L$ nie zawiera właściwej 2-sfery, właściwego dysku, właściwego torusa ani właściwego pierścienia.
    Wtedy $L$ jest hiperboliczny.
\end{proposition}

\begin{proposition}
    Niech $L$ będzie splotem takim, że jego grupa $\pi(S^3 \setminus L)$ nie jest produktem wolnym, nie jest izomorficzna z~$\Z$ oraz nie zawiera w~sobie kopii grupy $\Z \oplus \Z$.
    Wtedy $L$ jest hiperboliczny.
\end{proposition}

\begin{proof}[Niedowód]
    Patrz \url{https://mathoverflow.net/a/153327}.
\end{proof}

Czas na podanie jakichś przykładów węzłów hiperbolicznych, za Adamsem \cite{adams2005}.

\begin{proposition}
    Każdy alternujący, pierwszy, oraz nierozszczepialny splot jest albo 2-warkoczem (a zatem, torusowy) albo hiperboliczny.
\index{splot!rozszczepialny}%
\index{węzeł!alternujący}%
\index{węzeł!pierwszy}%
\end{proposition}

\begin{proof}[Niedowód]
    Menasco \cite{menasco1984} pokazał: dopełnienie alternującego węzła nie zawiera nieściśliwych nieperyferyjnych torusów.
\index[persons]{Menasco, William}%
    To w~połączeniu z~unifikacyjnym twierdzeniem Thurstona dla rozmaitości Hakena kończy(nie)dowód.
\index{rozmaitość!Hakena}%
    % zarys dowodu zz MathSciNet
\end{proof}

\begin{proposition}
    Nietrywialne pierwsze prawie alternujące węzły są torusowe albo hiperboliczne.
\index{węzeł!pierwszy}%
\index{węzeł!prawie alternujący}%
\end{proposition}

\begin{proof}[Niedowód]
    Grupa studentów: Brock, Bugbee, Comar, Faigin, Huston, Joseph, Pesikoff pod opieką Adamsa \cite{brock1992}.
\index[persons]{Adams, Colin}%
\index[persons]{Brock, Jeffrey}%
\index[persons]{Bugbee, John}%
\index[persons]{Comar, Timothy}%
\index[persons]{Faigin, Keith}%
\index[persons]{Huston, Amy}%
\index[persons]{Joseph, Anne}%
\index[persons]{Pesikoff, David}%
\end{proof}

\begin{proposition}
    Toroidalnie alternujące węzły pierwsze są torusowe albo hiperboliczne.
    \index{węzeł!pierwszy}
    \index{węzeł!toroidalnie alternujący}
\end{proposition}

Ze wszystkich węzłów pierwszych do 11 skrzyżowań i~pierwszych, nierozszczepialnych splotów do 10 skrzyżowań tylko 3 węzły i~2 sploty nie są toroidalnie alternujące, tak twierdzi Adams \cite{adams2005}.

\begin{proof}[Niedowód]
    Patrz \cite{adamsc1994}.
\end{proof}

\begin{proposition}
\index{splot!Montesinosa}%
    Sploty Montesinosa są prawie zawsze torusowe albo hiperboliczne.
\end{proposition}

\begin{proof}
    Najpierw Boileau, Siebenmann \cite{boileau1980} zidentyfikowali torusowe sploty Montesinosa.
    \index[persons]{Boileau, Michel}%
    \index[persons]{Siebenmann, Laurent}%
    Potem Oertel \cite{oertel1984} dowiódł, że jeśli splot Montesinosa nie jest ani hiperboliczny, ani torusowy, to musi być jednym z wyjątków (z chyba trochę inną notacją niż nasza):
    \begin{itemize}
        \item $K(1/2, 1/2, 21/2, 21/2)$,
        \item $K(2/3, 21/3, 21/3)$,
        \item $K(1/2, 21/4, 21/4)$,
        \item $K(1/2, 21/3, 21/6)$,
        \item lub lustrem tych splotów. \qedhere
    \end{itemize}
\end{proof}

\begin{proposition}
    Mutant węzła hiperbolicznego jest węzłem hiperbolicznym.
    \index{mutant}
\end{proposition}

\begin{proof}[Niedowód]
\index[persons]{Ruberman, Daniel}%
    Ruberman \cite[wniosek 1.4]{ruberman1987}.
\end{proof}

Z kryterium Thurstona mamy prosty wniosek:

\begin{corollary}
    Każdy węzeł hiperboliczny jest pierwszy.
    \index{węzeł!pierwszy}
\end{corollary}

\begin{proof}
    Każdy węzeł złożony jest satelitarny, patrz przykład \ref{swallow_follow_torus}.
\end{proof}

Prawie każdy węzeł pierwszy o~mniej niż 17 skrzyżowaniach jest hiperboliczny, na 32 wyjątki składa się 12 węzłów torusowych oraz 20 satelitów trójlistnika.
\label{page:nonhyperbolic_below_16}%
Te ostatnie mają co najmniej 13 skrzyżowań.
Baza ciągów liczb całkowitych OEIS zawiera informacje na temat liczności poszczególnych typów węzłów.
Analizując ciągi \href{https://oeis.org/A051764}{A051764}, \href{https://oeis.org/A051765}{A051765} oraz \href{https://oeis.org/A052408}{A052408} można dojść do wniosku, że wraz ze wzrostem liczby skrzyżowań, stosunek liczby węzłów hiperbolicznych do wszystkich węzłów dąży do $1$:

\begin{figure}[H]
\renewcommand*{\arraystretch}{1.4}
\footnotesize
\begin{longtable}{lcccccccccccccc}
\hline
    \textbf{rodzaj} & 3 & 4 & 5 & 6 & 7 & 8  & 9  & 10  & 11  & 12   & 13   & 14    & 15     \\ \hline \endhead
    torusowe        & 1 & 0 & 1 & 0 & 1 & 1  & 1  & 1   & 1   & 0    & 1    & 1     & 2      \\
    satelitarne     & 0 & 0 & 0 & 0 & 0 & 0  & 0  & 0   & 0   & 0    & 2    & 2     & 6      \\
    hiperboliczne   & 0 & 1 & 1 & 3 & 6 & 20 & 48 & 164 & 551 & 2176 & 9985 & 46969 & 253285 \\
    \hline
\end{longtable}
\normalsize
\end{figure}

W pracy \cite{malyutin2016} Malutin pokazał jednak, że to przypuszczenie jest sprzeczne z~wieloma innymi starymi hipotezami teorii węzłów:~\ref{con:malyutin1} --~\ref{con:malyutin4}.
\index[persons]{Malutin, Andriej}%

\begin{conjecture}
\label{con:malyutin1}%
    Indeks skrzyżowaniowy jest addytywny względem sumy spójnej.
\index{indeks!skrzyżowaniowy}%
\index{suma spójna}%
\end{conjecture}

(To jest powtórzenie hipotezy~\ref{con:crossing_additive}).

\begin{conjecture}
    Satelita ma większy (w słabszej wersji: nie mniejszy) indeks skrzyżowaniowy niż jego towarzysze.
    \index{węzeł!satelitarny}
\end{conjecture}

Lackenby \cite{lackenby2014} pokazał, że jeśli $K$ jest satelitą z~towarzyszem $L$, to $\crossing K \ge 10^{-13} \crossing L$.
\index[persons]{Lackenby, Marc}%

\begin{conjecture}
%label{con:malyutin3}
    Węzeł złożony ma większy (w słabszej wersji: nie mniejszy) indeks skrzyżowaniowy niż jego składniki.
    \index{węzeł!pierwszy}
\end{conjecture}

Mówimy, że węzeł pierwszy $P$ jest $\lambda$-regularny, jeśli $\crossing K \ge \lambda \cdot \crossing P$ za każdym razem, gdy węzeł $P$ jest składnikiem węzła $K$.
\index{węzeł!regularny}
Zatem hipotezę można wysłowić krótko ,,węzły pierwsze są $1$-regularne''.
Z tego, co pisaliśmy po hipotezie~\ref{con:crossing_additive} wynika, że hipoteza~\ref{con:malyutin1} jest prawdziwa w~klasie węzłów alternujących czy torusowych i~że wszystkie węzły są $1/152$-regularne.

\begin{conjecture}
    \label{con:malyutin4}
    Węzły pierwsze są $2/3$-regularne.
\end{conjecture}

Rozwiązanie zagadki przyniosła praca samego Malutina \cite{malyutin2019} opublikowana latem 2019 roku, przynajmniej dla splotów.
\index[persons]{Malutin, Andriej}%
Pokazał w~niej, że jeśli oznaczymy liczbę splotów pierwszych i~nierozszczepialnych o~$n$ lub mniej skrzyżowaniach przez $P_n$, zaś liczbę hiperbolicznych splotów, także o~$n$ lub mniej skrzyżowaniach, przez $H_n$, prawdziwe będzie oszacowanie
\index{splot!rozszczepialny}
\index{węzeł!pierwszy}
\begin{equation}
    \liminf_{n \to \infty} \frac{H_n}{P_n} < 1 - 10^{-13}.
\end{equation}

Purcell \cite[s. 66]{purcell2020} w formie ćwiczenia proponuje znaleźć rodzinę zupełnych struktur hiperbolicznych na nakłutym torusie oraz czterokrotnie nakłutej sferze.
Podobna elastyczność nie występuje jednak w~przestrzeniach wyższych wymiarów.
Z~twierdzenia o~sztywności, w~wersji algebraicznej:

\begin{theorem}[Mostow-Prasad]
    \index{twierdzenie!o sztywności}
    Niech $\Gamma_1, \Gamma_2$ będą dyskretnymi podgrupami grupy izometrii $\mathbb H^n$ dla $n \ge 3$ takimi, że ilorazy $\mathbb H^n/\Gamma_i$ mają skończone objętości.
    Załóżmy też, że istnieje izomorfizm grup $\varphi \colon \Gamma_1 \to \Gamma_2$.
    Wtedy podgrupy $\Gamma_1, \Gamma_2$ są sprzężone.
\end{theorem}
\index{twierdzenie!Mostowa-Prasada}

albo geometrycznej:

\begin{theorem}[Mostow-Prasad]
    Niech $M_1, M_2$ będą zupełnymi, hiperbolicznymi rozmaitościami o skończonych objętościach.
    Wtedy każdy izomorfizm grup podstawowych $\varphi \colon \pi_1(M_1) \to \pi_1(M_2)$ realizowany jest jednoznacznie przez izometrię.
\end{theorem}

wynika, że jeśli znaleźliśmy jakąś zupełną strukturę hiperboliczną na dopełnieniu splotu, to innych już nie ma.

\begin{proof}[Niedowód]
\index[persons]{Benedetti, Riccardo}%
\index[persons]{Mostow, George}%
\index[persons]{Petronio, Carlo}%
\index[persons]{Prasad, Gopal}%
\index[persons]{Thurston, William}%
Pierwsi byli Mostow \cite{mostow1973} oraz Prasad \cite{prasad1973}.
    Thurston naszkicował rozumowanie w~sekcji 5.9 swoich notatek, na bazie których powstała później książka \cite{thurston1997}.
    Inny tok myślenia można znaleźć w~podręczniku Benedettiego, Petronio \cite[rozdział C]{benedetti1992}.
\end{proof}

Twierdzenie Mostowa-Prasada pozwala nam na wprowadzenie nowych niezmienników splotów hiperbolicznych: wystarczy wziąć dowolny geometryczny niezmiennik dopełnienia węzła.
Najważniejszym z~nich wydaje się być objętość.


\subsection{Objętość hiperboliczna}

\index{objętość|(}%
\begin{definition}[objętość]
    Niech $L$ będzie splotem hiperbolicznym, na dopełnieniu którego zadano zupełną metrykę hiperboliczną.
    Objętość tego dopełnienia nazywamy objętością splotu $L$ i~oznaczamy $\volume L$.
\end{definition}

Objętość jest zawsze skończoną liczbą rzeczywistą.
Dla wygody przyjmuje się czasami, że objętość węzłów torusowych oraz satelitarnych wynosi $0$.
Komputerowy program SnapPea napisany przez Weeksa pozwala na wyznaczenie objętości dowolnego splotu o~rozsądnej ilości skrzyżowań.
\index{SnapPea}

Thurston \cite[s. 365]{thurston1982} zauważył, że tylko skończenie wiele hiperbolicznych 3-rozmaitości może mieć tę samą objętość -- wynika to z~prac Gromowa i~Jørgensena (niestety nie wiemy, o~których pracach mowa).
\index[persons]{Thurston, William}%
W mniej więcej tym samym czasie Wielenberg \cite{wielenberg1981} spostrzegł, że pewne podgrupy klasycznej grupy Picarda działają jako izometrie na górną półprzestrzestrzeń hiperboliczną wymiaru 3 mają podstawowe wielościany, które są takie same jako zbiory, ale różnią się jeśli chodzi o~utożsamienie ze sobą ścian.
\index[persons]{Wielenberg, Norbert}%
\index{grupa Picarda}%
Wynika stąd, że istnieją dowolnie duże kolizje wśród węzłów hiperbolicznych.

Chociaż mutanty mają tę samą objętość hiperboliczną (fakt~\ref{mutants_the_same_volume}), to praktyka pokazuje, że niezmiennik $\volume$ dobrze wspomaga proces tablicowania węzłów.
\index{mutant}%

\begin{example}
    $\volume 4_1 = -6 \int_{0}^{\pi/3} \log |2\sin \theta| \,\mathrm{d}\theta \approx 2.0298832$.
\end{example}

Żaden węzeł hiperboliczny nie ma mniejszej objętości, mówi o tym fakt \ref{prp:eight_smallest_volume}.

\begin{example}
    $\volume 5_2 \approx 2.82812$.
    % https://arxiv.org/pdf/q-alg/9601025.pdf strona 6
\end{example}

W encyklopedii Wolfram Mathworld znajduje się informacja, że $5_2$ oraz pewien węzeł o~dwunastu skrzyżowaniach mają tę samą objętość, prawdopodobnie chodzi tu o~$12n_{242}$, który znany jest także jako $(-2, 3, 7)$-precel.
\index{precel!(-2, 3, 7)}%
\index{węzeł!12n-242}%

\begin{example}
    $\volume 6_1 \approx 3.16396$.
    % https://arxiv.org/pdf/q-alg/9601025.pdf strona 6
\end{example}

\begin{example}
    $\volume 6_2 \approx 4.40083$.
\end{example}

\begin{example}
    $\volume 6_3 \approx 5.69302$.
\end{example}

\begin{example}
\index{para Perko}%
    Niech $K$ będzie jednym z~dwóch węzłów w~parze Perko.
    Wtedy $\volume K \approx 5.63877$.
\end{example}

Praca Futera, Kalfagianni, Purcell \cite{purcell2019} wspomina kilka przyjemnych ograniczeń, jakie musi spełniać objętość.
\index[persons]{Futer, David}%
\index[persons]{Kalfagianni, Efstratia}%
\index[persons]{Purcell, Jessica}%
Aby je przytoczyć, musimy najpierw zdefiniować dwie stałe: $v_4$ oraz $v_8$, objętości idealnego czworościanu (albo rozmaitości Giesekinga, powstałej z~czworościanu przez usunięcie  wierzchołków i~sklejenie ściany 012 z~310 oraz 023 z~32).
 Dopełnienie ósemki jest podwójnym nakryciem tej rozmaitości) oraz ośmiościanu foremnego w~$\mathbb H^3$.
\index{rozmaitość!Giesekinga}%
Mamy
\begin{alignat}{2}
    v_4 & = \int_{0}^{2\pi/3} \log(2 \cos(\theta/2)) \,\mathrm{d}\theta & {}\approx{} & 1.01494\,16064, \\
    % https://en.wikipedia.org/wiki/Gieseking_manifold
    % N[Integrate[Log[2Cos[t/2]], {t, 0, 2Pi/3}], 100]
    v_8 & = 4 \sum_{n=0}^\infty \frac{(-1)^n}{(2n+1)^2} &{}\approx{}& 3.66386\,23767.
    % 4N[Catalan, 100]
\end{alignat}

I tak najpierw Adams \cite{adams1983} pokazał w~swojej rozprawie doktorskiej:
\index[persons]{Adams, Colin}%

\begin{proposition}
    Niech $L$ będzie splotem o $\crossing L \ge 5$ skrzyżowaniach z diagramem $D$, który realizuje indeks skrzyżowaniowy.
    Wtedy
    \begin{equation}
        \volume L \le 4 (\crossing D - 4) v_4.
    \end{equation}
\end{proposition}

A trzy dekady później poprawił swój wynik w~\cite{adams2013}:

\begin{proposition}
    Niech $L$ będzie splotem o $\crossing L \ge 5$ skrzyżowaniach z diagramem $D$, który realizuje indeks skrzyżowaniowy.
    Wtedy
    \begin{equation}
        \volume L \le (\crossing D - 5) v_8 + 4v_4.
    \end{equation}
\end{proposition}

Jego metoda polega na podzieleniu dopełnienia splotu na czterościany i~ośmiościany oraz policzeniu ich.
To, w~połączeniu ze znanymi ograniczeniami na objętość ,,cegiełek'', wystarcza.
Podział na ośmiościany zaproponował Dylan (nie William!) Thurston.
\index[persons]{Thurston, Dylan}%
% wiem to z purcell19

\begin{proposition}
    Zbiór
    \begin{equation}
        \{\volume K: K \textrm{ jest hiperboliczny}\} \subseteq \R
    \end{equation}
    jest dobrze uporządkowany, typu porządkowego $\omega^\omega$.
\end{proposition}

\begin{proof}[Niedowód]
    Zdaniem angielskiej Wikipedii, dowód jest gdzieś w~\cite{neumann1985} (gdzie Neumann, Zagier znajdują eleganckie oszacowanie zmiany objętości po wykonaniu chirurgii Dehna), my tego nie widzimy.
\index[persons]{Neumann, Walter}%
\index[persons]{Zagier, Don}%
    %=% wikipedia - angielski artykuł "hyperbolic volume" 
    Hodgson, Masai \cite{hodgson2013} sugerują, żeby przeczytać notatki Thurstona \cite{thurston2002}.
\index[persons]{Hodgson, Craig}%
\index[persons]{Masai, Hidetoshi}%
    % TODO: https://mathscinet.ams.org/mathscinet-getitem?mr=648524 sugeruje, że to jest tam: "The order type of the set of all volumes of hyperbolic 3-manifolds is ω^ω."
    Jeszcze jedną wzmiankę znaleźliśmy w starszej pracy Thurstona \cite[s. 365]{thurston1982}.
\end{proof}

W dowolnej rodzinie węzłów istnieje element o~najmniejszej objętości.
Przytoczymy teraz przykłady konkretnych rodzin i~najmniejszych węzłów, za Futerem, Kalfagiannim, Purcell \cite[s. 16-17]{purcell2019} oraz Hodgsonem, Masaiem \cite[s. 296]{hodgson2013}.
\index[persons]{Futer, David}%
\index[persons]{Kalfagianni, Efstratia}%
\index[persons]{Purcell, Jessica}%
\index[persons]{Hodgson, Craig}%
\index[persons]{Masai, Hidetoshi}%

\begin{proposition}
\index{ósemka|see {węzeł 4-1}}%
\index{węzeł!4-1}%
\label{prp:eight_smallest_volume}%
    Żaden węzeł nie ma mniejszej objętości hiperbolicznej od ósemki.
\end{proposition}

\begin{proof}[Niedowód]
\index[persons]{Cao, Chun}%
\index[persons]{Meyerhoff, Robert}%
    Cao, Meyerhoff \cite{cao2001} przeanalizowali pakowania horokul w~uniwersalnym nakryciu związanym z~rozmaitościami.
    Doszli do wniosku, że nie ma tam dostatecznieo wolnego miejsca, jeżeli szpic nie jest odpowiedniego rozmiaru.
\index{szpic}%
    Trzykrotnie wspierają się przy tym pomocą komputera, by sprawdzić, że określone warunki są spełnione we wszystkich punktach danej przestrzeni parametrów.
\end{proof}

Nie jesteśmy pewni, jak powinno tłumaczyć się  angielskie \emph{cusp}; w~literaturze spotkaliśmy czasami termin ostrze.
\index{szpic}%
Chcielibyśmy zaproponować słowo szpic.
Rozmaitość szpiczasta (czyli niezwarta, zupełna hiperboliczna rozmaitość ze skończoną objętością Riemanna) byłaby wtedy polskim odpowiednikiem \emph{cusped manifold}.

\begin{proposition}
% DICTIONARY;cusped;szpiczasta;rozmaitość
% DICTIONARY;manifold;rozmaitość;-
\index{splot!Whiteheada}%
    Wśród orientowalnych 3-rozmaitości ze szpicem najmniejszą objętość ma dopełnienie ósemki oraz jego bliźniak, otrzymany przez $(5, 1)$-chirurgię jednego z~ogniw splotu Whiteheada.
\index{szpic}%
% TODO: sformułowanie wygląda jak z "THE MINIMAL VOLUME ORIENTABLE HYPERBOLIC 3-MANIFOLD WITH 4 CUSPS"
\end{proposition}

Klasa rozmaitości wspomniana w~fakcie obejmuje dopełnienia hiperbolicznych węzłów.
Powyższy fakt także został wzięty z~pracy \cite{cao2001}.
Meyerhoff, już bez Cao, nie przestawał pracować nad rozmaitościami o~małych objętościach i~osiem lat później przedstawił z~Gabaiem, Milleyem \cite{meyerhoff2009} bez dowodu:
\index[persons]{Gabai, David}%
\index[persons]{Milley, Peter}%

\begin{proposition}
\index{program SnapPy}%
    Istnieje 10 orientowalnych 3-rozmaitości z~jednym szpicem o~objętości co najwyżej $2.848$: \texttt{m003}, \texttt{m004} (2.02988\ldots), \texttt{m006}, \texttt{m007} (2.56897\ldots), \texttt{m009}, \texttt{m010} (2.66674\ldots), \texttt{m011} (2.78183\ldots), \texttt{m015}, \texttt{m016} oraz \texttt{m017} (2.82812\ldots).
    Nazwy pochodzą ze spisu rozmaitości programu SnapPy.
\end{proposition}

Chociaż panowie obiecali pokazać dowód później, nam nie udało się go odszukać.
Za to udało się rozszyfrować niektóre nazwy.
\texttt{m003} to siostra $4_1$, % https://hal.archives-ouvertes.fr/hal-02867890/document Michel Planat - Quantum computing thanks to Bianchi groups
\texttt{m004} to węzeł $4_1$, % SnapPy - also known as
% m006
% m007
% m009
% m010
% m011
\texttt{m015} to węzeł $5_2$,
\texttt{m016} to węzeł $12n242$, czyli znany nam już $(-2, 3, 7)$-precel,
\index{precel!(-2, 3, 7)}%
\texttt{m017} to siostra $5_2$. % https://arxiv.org/pdf/2107.03275.pdf
% TODO

W tej samej pracy możemy jeszcze znaleźć informację, że:

\begin{proposition}
    Istnieje dokładnie jedna hiperboliczna, domknięta 3-rozmaitość, której objętość jest najmniejsza (pośród wszystkich takich rozmaitości).
\end{proposition}

Chodzi o rozmaitość Weeksa, która powstaje przez wykonanie $(5, 2)$ oraz $(5, 1)$ chirurgii Dehna na dopełnieniu splotu Whiteheada.
\index{rozmaitość!Weeksa}%
\index{chirurgia Dehna}%
\index{splot!Whiteheada}%
Po raz pierwszy odkrył ją Jeffrey Weeks \cite{weeks1985} w~swojej rozprawie doktorskiej oraz niezależnie trzy lata później Matwiejow, Fomenko \cite{fomenko1988}.
\index[persons]{Weeks, Jeffrey}%
\index[persons]{Fomenko, Anatolij (Фомеенко, Анатоолий Тимофеевич)}%
\index[persons]{Matwiejow, Siergiej (Матвеев, Сергей Владимирович)}%
Wiemy z \url{https://oeis.org/A126774}, że
\begin{equation}
    V_{\textrm{Weeks}} = 0.94270 \, 73627 \, 76927 \, 72092 \ldots
    % Formula: Im(dilog(z0)+log(|z0|)*log(1-z0)) where z0 = 0.8774.. + 0.7448..i is the root of z^3-z^2+1 with Im(z)>0. - Herman Jamke (hermanjamke(AT)fastmail.fm), Dec 15 2007
\end{equation}

Następna jest rozmaitość Meyerhoffa, powstała po $(5, 1)$ chirurgii na dopełnieniu ósemki.
\index{rozmaitość!Meyerhoffa}%
Meyerhoff sugerował w~1987, że nie istnieje rozmaitość o mniejszej objętości, ale później okazało się to nieprawdą: 
\begin{equation}
    V_{\textrm{Meyerhoff}} = \frac{ 3 \sqrt{283^3} } {16 \pi^6} \zeta_{x^4-x-1}(2) = 0.98136 \, 88288 \, 92232 \, 08809 \ldots
\end{equation}

\begin{proposition}
    Wśród orientowalnych 3-rozmaitości o~dwóch szpicach najmniejszą objętość mają splot Whiteheada oraz $(-2, 3, 8)$-precel.
\index{szpic}%
\index{splot!Whiteheada}%
\index{precel!(-2, 3, 8)}%
\end{proposition}

% TODO: check cusped manifold in dictionary

Ich objętość wynosi $v_8$.

\begin{proof}[Niedowód]
\index[persons]{Agol, Ian}%
    Ian Agol \cite{agol2010} korzystając z~metod topologicznych dowodzi istnienia \emph{niezbędnej} powierzchni, która zadaje dolne ograniczenie na objętość.
    Tak skutecznie krępuje rozmaitości, które mogą to ograniczenie zrealizować.
\end{proof}

Przypadek trzech szpiców nie jest zbyt dobrze zrozumiany.
\index{szpic}%

\begin{proposition}
    Dopełnienie splotu $8_4^2$ (wg numeracji Rolfsena, czyli L8a13 wg numeracji Thistlethwaite'a) ma najmniejszą objętość wśród orientowalnych 3-rozmaitości o~czterech szpicach.
\index{szpic}%
\end{proposition}

Jego objętość wynosi $2v_8$.

\begin{proof}[Niedowód]
\index[persons]{Yoshida, Kenichi}%
    Rozumowanie Yoshidy \cite{yoshida2013} oparte o~wspomnianą wyżej pracę Agola.
\end{proof}

\index{objętość|)}%



\subsection{Symetrie węzłów hiperbolicznych}
Przytoczone niżej dwa fakty nie są do niczego potrzebne, ale uważamy je za coś uroczego.

\begin{proposition}
    Niech $G$ oznacza grupę izometrii wnętrza dopełnienia węzła hiperbolicznego.
    Wtedy $G$ jest diedralna lub skończona cykliczna.
\end{proposition}

\begin{proof}[Niedowód]
\index[persons]{Riley, Robert}%
\index[persons]{Kodama, Kouzi}%
\index[persons]{Sakuma, Makoto}%
    Pierwszy był Riley \cite[s. 124]{riley1979}, można też zapoznać się z~późniejszą pracą Kodamy, Sakumy \cite{kodama1992}.
    % Kodama - lemat 1.1
\end{proof}

Kawauchi \cite[s. 131]{kawauchi1996} wprowadza jeszcze jedną grupę (grupę symetrii węzła): iloraz grupy PL automorfizmów pary $(S^3, K)$ przez podgrupę elementów, które są otaczająco izotopijne z~odwzorowaniem tożsamościowym.
Okazuje się, że nie wszystkie są skończone:

\begin{proposition}
    Następujące warunki są równoważne -- węzeł $K$ ma skończoną grupę symetrii; węzeł $K$ jest hiperboliczny, torusowy albo kablem węzła torusowego.
\end{proposition}

Kawauchi nie podaje dowodu, ale zaleca zajrzeć do pracy Sakumy.
My zajrzeliśmy i dalej nie mamy pojęcia, jak ten dowód miałby wyglądać.

\index{węzeł!hiperboliczny|)}

% Koniec sekcji Węzły hiperboliczne



\section{Węzły plastrowe i taśmowe}
\label{sec:slice}
Musimy oznajmić z przykrością, że jest to już ostatnia sekcja książki.
Poruszymy tutaj ważne z~punktu widzenia współczesnej teorii węzłów zagadnienia czterowymiarowe: wprowadzamy węzły plastrowe i taśmowe, definiujemy relację zgodności oraz przytaczamy tyle wyników o grupie zgodności, ile tylko jesteśmy w stanie zrozumieć, czyli nie za wiele.
Ze względu na znaczne ubytki wiedzy, ograniczamy się do zreferowania tekstu Kawauchiego \cite[s. 154-169]{kawauchi1996}, miejscami tylko wspominając odsyłacze do literatury, które wydają się nam pomocne.
Wiele wyników, jakie podamy w tej sekcji, pochodzi z artykułu Foxa, Milnora \cite{fox1966}, od którego wszystko się zaczęło.
\index[persons]{Fox, Ralph}%
\index[persons]{Milnor, John}%

Adnotacja tłumacza: być może jesteśmy pierwszymi osobami piszącymi o tych obiektach po polsku, dlatego nie mamy pewności, czy podane tutaj tłumaczenie ,,plastrowy'' przyjmie się.
Inny przymiotnik godny rozważenia to ,,wstęgowy''.

% DICTIONARY;plastrowy;slice;węzeł
\begin{definition}[węzeł topologicznie plastrowy]
\index{węzeł!plastrowy}%
    Węzeł $K$ w sferze $S^3$, który jest brzegiem lokalnie płaskiego dysku $D$ w kuli $B^4$ nazywamy węzłem topologicznie plastrowym. % Kawauchi 155?
    Dysk $D$, kiedy potrzebuje mieć nazwę, też jest dyskiem plastrowym.
\end{definition}

Jeśli nie jest podane inaczej, ,,plastrowy'' to synonim ,,topologicznie plastrowego''.

\begin{definition}[węzeł gładko plastrowy]
    Węzeł $K$ w sferze $S^3$, który jest brzegiem gładkiego dysku $D$ w kuli $B^4$ nazywamy węzłem gładko plastrowym.
\end{definition}

Następujące węzły o~mniej niż jedenastu skrzyżowaniach są plastrowe (topologicznie oraz gładko): $6_1$, $8_{8}$, $8_{9}$, $8_{20}$, $9_{27}$, $9_{41}$, $9_{46}$, $10_{3}$, $10_{22}$, $10_{35}$, $10_{42}$, $10_{48}$, $10_{75}$, $10_{87}$, $10_{99}$, $10_{123}$, $10_{129}$, $10_{137}$, $10_{140}$, $10_{153}$ oraz $10_{155}$.
\index{węzeł!Conwaya}
Wśród pierwszych węzłów do dwunastu skrzyżowań najdłużej opierał się węzeł Conwaya, aż Lisa Piccirillo \cite{piccirillo2020} pokazała, że nie jest gładko plastrowy (ale za to jest topologicznie plastrowy).
\index[persons]{Piccirillo, Lisa}%

\begin{proposition}
    Niech $K$ będzie węzłem.
    Wtedy suma $K \shrap \operatorname{mr} K$ jest węzłem plastrowym.
\end{proposition}

\begin{proof}[Niedowód]
    Kawauchi \cite[s. 155]{kawauchi1996} pisze, że wystarczy wybrać 3-kulę $B \subseteq S^3$ taką, że $K \cap B$ jest trywialnym łukiem w $B$.
    Wtedy 4-kula $\operatorname{cl} (S^3 \setminus B) \times [0,1]$ oraz lokalnie płaski dysk $\operatorname{cl} (K \setminus K \cap B) \times [0,1] $ są świadkami plastrowości węzła $K \shrap \operatorname{mr} K$.
\end{proof}

\begin{proposition}
    Albo wszystkie trzy węzły $K_1, K_2, K_1 \shrap K_2$ są plastrowe, albo co najwyżej jeden z~nich.
\end{proposition}

\begin{proof}
    Kawauchi \cite[s. 155]{kawauchi1996} pisze: załóżmy, że $K_1, K_2$ są plastrowe, z plastrowymi dyskami $D_1, D_2 \subseteq B^4$.
    Wtedy suma brzegowa\footnote{Cokolwiek to jest!} $(B^4, D_1) \natural (B^4, D_2)$ pokazuje, że węzeł $K_1 \shrap K_2$ jest plastrowy.

    Załóżmy teraz, że $K_1, K_3 = K_1 \shrap K_2$ są plastrowe, z plastrowymi dyskami $D_1, D_3 \subseteq B_4$.
    Wybierzmy 3-kule $B_1, B_3$ wewnątrz $S^3$ tak, że domknięcie $K_1 \setminus B_1 \cap K_1$ jest trywialnym łukiem w domknięciu $S^3 \setminus B_1$.
    Wtedy coś tam dalej, ale nie warto tego przepisywać, bo i tak za trudne na tę książkę.
\end{proof}

Pierwszym poważnym wynikiem z dziedziny teorii węzłów plastrowych, pochodzącym jeszcze z pracy \cite{fox1966}, był:

\begin{proposition}[warunek Foxa-Milnora]
\index{warunek!Foxa-Milnora}%
    Niech $K$ będzie węzłem plastrowym.
    Wtedy jego wielomian Alexandera jest postaci $\alexander(t) = f(t) f(1/t)$ dla pewnego wielomianu Laurenta $f(t) \in \Z[t, 1/t]$.
\end{proposition}

\begin{corollary}
    \label{det_slice_square}%
    \index{wyznacznik}
    Wyznacznik węzła plastrowego jest kwadratem.
\end{corollary}

\begin{proof}
    Mamy $\det K = |\alexander_K(-1)| = f(-1) f(-1)$.
\end{proof}

Ten prosty test stwierdza, że 2743 spośród 2977 węzłów o mniej niż 13 skrzyżowaniach nie jest plastrowych.
% podać program tłumaczący, czemu tak jest

\begin{proposition}
% TODO: skąd to stwierdzenie?
\index{sygnatura}%
    Niech $K$ będzie węzłem plastrowym.
    Wtedy $\sigma(K) = 0$.
\end{proposition}

\begin{proof}[Niedowód]
    Można zajrzeć do Lickorisha \cite[s. 90]{lickorish1997} i Murasugiego \cite[twierdzenie 8.8]{murasugi1965}; nie wiadomo, kto pierwszy uznał, że warto tego dowieść.
    %Praca "Infinite Order Amphicheiral Knots". (Charles Livingston, 2001) -- chyba nie?
\end{proof}

Test ten eliminuje kolejne 45 węzłów poniżej 13 skrzyżowań.
% TODO: podać program tłumaczący, czemu tak jest?

\begin{proposition}
    \index{niezmiennik!Arfa}
    Niech $K$ będzie węzłem plastrowym.
    Wtedy $\operatorname{Arf} K = 0$.
\end{proposition}

\begin{proof}
    Składniki dowodu już są, wystarczy je teraz wszystkie wymieszać w~dobrej kolejności.
    Wniosek \ref{det_slice_square} mówi, $\det K$ jest kwadratem, zaś fakt \ref{cor:knot_determinant_odd}, że $\det K$ jest nieparzyste.
    Zatem $\det K \equiv 1 \pmod 8$ i fakt~\ref{prp:arf_murasugi} (warunkek Murasugiego) orzeka $\operatorname{Arf} K = 0$.
\end{proof}

% TODO: podać program tłumaczący, czemu tak jest? tzn. czy coś to eliminuje więcej, jak tak to ile?

Ostatni fakt, jaki podamy we wprowadzeniu, to jedno z niewielu miejsc w całej książce, gdzie dotykamy różnic między kategorią Top oraz PL.

\begin{proposition}
\label{prp:trivial_alexander_implies_slice}%
    Niech $K$ będzie węzłem w kategorii Top.
    Jeżeli jego wielomian Alexandera jest trywialny: $\alexander_K(t) \equiv 1$, to węzeł $K$ jest plastrowy.
\end{proposition}

Twierdzenie to jest bardzo łatwo napotkać przeglądając prezentacje poświęcone węzłom plastrowym, ale nikt nie chce się przyznać, kto jest jego ojcem.
Odpowiedź znaleźliśmy dopiero w artykule ,,The degree of the Alexander polynomial is an upper bound for the topological slice genus'' Petera Fellera!
Miło, że mu się chciało.
% łatwo napotkać czytając o węzłach plastrowych, ale jawnie nikomu się nie chce wskazać dowodu. % Informację, że to jest tw. 1.13b znalazłem wreszcie w https://arxiv.org/pdf/1504.01064.pdf

\begin{proof}
\index[persons]{Freedman, Michael}%
    Freedman \cite[tw. 1.13]{freedman1982}.
\end{proof}

Implikacja~\ref{prp:trivial_alexander_implies_slice} przestaje być prawdziwa po przejściu do kategorii Diff.
Wydawało nam się kiedyś, że Gompf \cite{gompf1986} dobrze tłumaczy tę różnicę przy użyciu twierdzenia Donaldsona.
\index[persons]{Gompf, Robert}%
\index{twierdzenie!Donaldsona}%
(Do tego artykułu odsyła encyklopedia węzłów \cite{adams2021}).
% Encyclopedia of Knot Theory pod redakcją Colin Adams, Erica Flapan, Allison Henrich, Louis H. Kauffman, Lewis D. Ludwig, Sam Nelson, około strony 453 lub 454: "but many knots with Alexander polynomial one are not smoothly slice [9]" i odsyła do \cite{donaldson1983}
% TODO Gompf to nie Donaldson
Dzisiaj wiemy, że nic nie wiemy.


\subsection{Węzły skręcone}
\index{węzeł!skręcony|(}%

Nie potrafiliśmy znaleźć lepszego miejsca dla tej podsekcji, między innymi przez fakt \ref{twist_slice}, który wymaga znajomości przymiotnika ,,plastrowy''.
% DICTIONARY;twist;skręcony;węzeł
Węzły skręcone uważa się za najprostszą (po torusowych) rodzinę węzłów.
Wspomina o~nich bardzo krótko Kawauchi \cite[s. 31]{kawauchi1996}.

\begin{definition}
    Węzeł powstały najpierw przez $n$-krotne półskręcanie domkniętej pętli, a następnie splecienie końców, nazywamy węzłem skręconym.
\end{definition}

Węzły skręcone to dokładnie towarzyszące niewęzłowi w~węzłach satelitarnych, tak zwane whiteheadowskie duble niewęzła.
Wszystkie są odwracalne (ale tylko niewęzeł oraz ósemka są zwierciadlane) i~mają liczbę gordyjską $1$, ponieważ wystarczy rozwiązać skrzyżowanie, które plotło końce.
\index{liczba gordyjska}%
Każdy jest dwumostowy (ćwiczenie u~Rolfsena \cite[s. 114]{rolfsen1976}) i~posiada zerową sygnaturę.
\index{węzeł!dwumostowy}%
\index{sygnatura}%
Dalsze własności węzłów skręconych zależą od $n$, ilości półskrętów.
Indeks skrzyżowaniowy wynosi $n + 2$.

\begin{proposition}
\index{wielomian!Conwaya}%
    Niech $K$ będzie węzłem $n$-skręconym.
    Wtedy
    \begin{equation}
    2 \conway (z) = \begin{cases}
        2 + (n+1) z^{2} & n \mbox{ nieparzyste} \\
        2 - nz^2 & n \mbox{ parzyste}
    \end{cases}
    \end{equation}
\end{proposition}

\begin{proposition}
\index{wielomian!Jonesa}%
    Niech $K$ będzie węzłem $n$-skręconym.
    Wtedy
    \begin{equation}
    (q+1)\jones(q) = \begin{cases}
        1+q^{-2}+q^{-n}-q^{-n-3} & n \mbox{ nieparzyste} \\
        q^3(1+q^{-2}-q^{-n}+q^{-n-3}) & n \mbox{ parzyste}
    \end{cases}
    \end{equation}
\end{proposition}

\begin{proposition}
\index{węzeł!plastrowy}%
\label{twist_slice}%
    Żaden węzeł skręcony poza niewęzłem $0_1$ oraz węzłem dokerskim $6_1$ nie jest plastrowy.
\end{proposition}

\begin{proof}
    Casson, Gordon \cite{casson1986}.
\end{proof}

\index{węzeł!skręcony|)}%

% koniec podsekcji Węzły skręcone




%%% Kawauchi 156:
\subsection{Zgodność}
Zgodność jest relacją równoważności na zbiorze węzłów, która prowadzi do nowej definicji węzłów plastrowych (patrz fakt~\ref{prp:concordant_iff_sum_slice}).
My przytaczamy jej definicję z pracy Gompfa \cite{gompf1986}:

\begin{definition}[zgodność]
\index{zgodność}%
\index{węzeł!zgodny|see {zgodność}}%z
    Dwa węzły $K_0, K_1$ nazywamy (gładko) zgodnymi, jeżeli zbiór
    \begin{equation}
        K_0 \times \{0\} \cup K_1 \times \{1\}
    \end{equation}
    jest brzegiem pewnego pierścienia gładko zanurzonego w $S^3 \times I$.
\end{definition}

Kawauchi \cite[s. 156]{kawauchi1996} pisze \emph{,,Two knots (…) are knot cobordant (or concordant)''}, więc tak jak wielu innych autorów nie odróżnia więc węzłów kobordantnych od zgodnych.
Mamy zamiar zrobić dokładnie to samo: różnica między tymi terminami jest subtelna; węzły zgodne są też kobordantne, ale implikacja w drugą stronę nie zachodzi (wiemy o~tym z~tekstu Blanlœila ,,Cobordism and Concordance of Knots'') chyba, że pracuje się z węzłami sferycznmi, a tak jest w klasycznej teorii węzłów.
\index[persons]{Blanloeil, Vincent}%
% https://www.maths.ed.ac.uk/~v1ranick/papers/blanloeil
% Concordant knots are cobordant, but the converse is not true in general.
% "Cobordism and Concordance of Knots" by Vincent Blanlœil

Dlatego my będziemy zawsze pisać o węzłach zgodnych i nigdy o kobordantnynch.

\begin{proposition}
\label{prp:concordant_iff_sum_slice}%
    Dwa węzły $K_1, K_2$ są zgodne wtedy i tylko wtedy, gdy suma $(\operatorname{mr} K_0) \shrap K_1$ jest plastrowa.
\index{suma spójna}%
\index{węzeł!plastrowy}%
\end{proposition}

\begin{proof}
    Ćwiczenie 12.1.3 w książce Kawauchiego \cite{kawauchi1996}.
\end{proof}

\begin{definition}
    Węzeł zgodny z~niewęzłem nazywamy plastrowym.
\index{węzeł!plastrowy}%
\end{definition}

,,Bycie zgodnym'' jest relacją równoważności, słabszą od ,,bycia izotopijnym'', ale chyba mocniejszą od ,,bycia homotopijnym''.
\index{izotopia}%
\index{homotopia}
% ale mocniejszą od homotopii?
% izotopia: https://encyclopediaofmath.org/wiki/Cobordism_of_knots
% homotopia: https://en.wikipedia.org/wiki/Link_concordance By its nature, link concordance is an equivalence relation. It is weaker than isotopy, and stronger than homotopy: isotopy implies concordance implies homotopy. A link is a slice link if it is concordant to the unlink.
Klasę abstrakcji węzła $K$ oznaczamy przez $[K]$.

\begin{definition}[grupa zgodności]
\index{grupa!zgodności}%
    Niech $C^1$ oznacza iloraz zbioru wszystkich węzłów przez relację zgodności.
    Zbiór $C^1$ wyposażony w~działanie
    \begin{equation}
        [K_1] + [K_2] = [K_1 \shrap K_2]
    \end{equation}
    staje się grupą abelową, nazywaną grupą zgodności.
    Jej elementem neutralnym jest klasa abstrakcji niewęzła.
    Elementem przeciwnym do $[K]$ jest $[\operatorname{mr} K]$.
\end{definition}

%%% Kawauchi 157:

Niech $\Theta$ oznacza rodzinę macierzy Seiferta węzłów (czyli kwadratowych macierzy $V$ o~całkowitych wyrazach takich, że $\det (V - V^T) = 1$).
\index{macierz Seiferta}%
Mówimy, że macierz $V \in \Theta$ jest zerowo kobordantna, jeżeli jest postaci
\index{macierz Seiferta!zerowo kobordantna}%
\begin{equation}
    V = P \begin{pmatrix} 0 & V_{21} \\ V_{12} & V_{22} \end{pmatrix} P^{-1}
\end{equation}
dla pewnej całkowitoliczbowej macierzy $P$ o~wyznaczniku $\pm 1$; takie macierze nazywamy unimodularnie sprzężonymi.
\index{macierz!unimodularnie sprzężona}%
Każda zerowo kobordantna macierz $V \in \Theta$ stanowi macierz Seiferta pewnego plastrowego węzła $K$.
Kawauchi nazywa te węzły algebraicznie plastrowymi i~mówi, że to dokładnie węzły, które ograniczają izotropowe powierzchnie w kuli $B^4$, więc każdy węzeł plastrowy jest algebraicznie plastrowy.

Suma $(-V) \oplus V$ jest zerowo kobordantna dla każdej macierzy $V \in \Theta$.
To (chyba to) inspiruje Kawauchiego do wprowadzenia kolejnej definicji: dwie macierze $V_1, V_2 \in \Theta$ nazywa kobordantnymi, jeżeli $(-V_1) \oplus V_2$ jest zerowo kobordantna.
Kobordyzm stanowi relację równoważności na $\Theta$ -- iloraz $\Theta$ przez tę relację oznacza się $G_-$, jest grupą abelową.

\begin{proposition}
    % Kawauchi 12.2.8
    Odwzorowanie $\psi \colon C^1 \to G_-$ posyłające klasę abstrakcji węzła w klasę abstrakcji jego macierzy Seiferta jest dobrze określonym epimorfizmem.
\end{proposition}

\begin{proof}
    Nie umiemy nic udowodnić, więc wymienimy tylko trzy odsyłacze: z faktu~\ref{prp:cobordant_to_algebraic_is_algebraic} wynika, że odwzorowanie $\psi$ jest dobrze określone, dowód faktu~\ref{prp:signature_additive} pokazuje, że $\psi$ jest homomorfizmem, zaś w \cite[s. 62]{kawauchi1996} można przeczytać, dlaczego jest ,,na''.
\end{proof}

Funkcję $\psi$ rozpatrywał Levine \cite{levine1969} w latach sześćdziesiątych.
\index[persons]{Levine, Jerome}%
Po mniej niż dekadzie Casson, Gordon \cite{casson1978} wskazali nietrywialne elementy jądra.
\index[persons]{Casson, Andrew}%
\index[persons]{Gordon, Cameron}%
% to wyżej wiem z kawauchi98, "Supplementary notes for Chapter 12"
Potem był wynik Jianga \cite{jiang1981}, że jądro nie jest skończenie generowalne, bo zawiera izomorficzną kopię $\Z^\infty$, a~jeszcze później Livingstona \cite{livingston1999}, że zawiera też kopię $(\Z/2\Z)^\infty$.
% to wyżej wiem z https://mathscinet.ams.org/mathscinet-getitem?mr=2179265, pierwsze strony tekstu (nie recenzji)
\index[persons]{Jiang, Boju}%
\index[persons]{Livingston, Charles}%

\begin{proposition}
    $G_- \cong \Z^\infty \oplus (\Z/4\Z)^\infty \oplus (\Z/2\Z)^\infty$.
\end{proposition}

Kawauchi \cite[s. 161]{kawauchi1996} bez uzasadnienia postanawia nie przytoczyć dowodu tego faktu, ale opowiada krótko, jaka jest idea przewodnia i odsyła wprost do pracy Levine'a.
Na dalszych stronach jego pracy przeglądowej pojawiają się jakieś formy kwadratowe oraz uogólnienia wszystkiego do zgodności splotów, ale wracamy nocnym pociągiem i zaraz uśniemy...




\subsection{Węzły taśmowe}
\index{węzeł!taśmowy|(}%
\begin{definition}
    Węzeł $K = f[S^1]$ będący brzegiem osobliwego dysku $f \colon D \to S^3$ posiadającego następującą własność: każda przecinająca siebie składowa jest łukiem $A \subseteq f(D^2)$, dla którego $f^{-1}[A]$ składa się z~dwóch łuków w~$D^2$ (jeden z~nich jest wewnętrzny), nazywamy taśmowym.
\end{definition}

Jak pisze Kawauchi, mamy oczywiste wynikanie:
% TODO: która strona

\begin{proposition}
\index{węzeł!plastrowy}%
    Każdy węzeł taśmowy jest plastrowy.
\end{proposition}

Dawno temu Fox \cite[problem 1.33]{kirby78} zapytał, czy implikacja odwrotna jest prawdziwa:
\index[persons]{Fox, Ralph}%

\begin{conjecture}[slice-ribbon problem]
    \index{hipoteza!plastrowo-taśmowa}
    Czy każdy węzeł plastrowy jest taśmowy?
\end{conjecture}

Wprawdzie Lisca \cite{lisca07} pokazał prawdziwość hipotezy dla węzłów dwumostowych,
\index[persons]{Lisca, Paolo}%
% korzystając ze słynnego tw. Donaldsona: that a definite intersection form of a compact, oriented, simply connected, smooth manifold of dimension 4 is diagonalisable
\index{węzeł!dwumostowy}%
zaś Greene oraz Jabuka \cite{greene11} zrobili to dla precli o trzech pasmach;
\index[persons]{Greene, Joshua}%
\index[persons]{Jabuka, Stanisław}%
\index{precel}%
ale Gompf, Scharlemann i~Thompson \cite{gompf10} zasugerowali potencjalny kontrprzykład.
\index[persons]{Gompf, Robert}%
\index[persons]{Scharlemann, Martin}%
\index[persons]{Thompson, Abigail}%
\index{rozmaitość!szwowa}%
Nie możemy przytoczyć tu tego kontrprzykładu, gdyż korzysta z~rozmaitości szwowych, opisanych w~\cite[s. 53-59]{kawauchi96}.

Teichner myśli\footnote{Patrz \url{https://mathoverflow.net/a/18154}.} o hipotezie plastrowo-taśmowej jako o~życzeniu, które uprościłoby pewne czterowymiarowe problemy, gdyby było prawdziwe.
\index[persons]{Teichner, Peter}%

\index{węzeł!taśmowy|)}

% koniec podsekcji węzły taśmowe




%%% Kawauchi 157:
\subsection{Węzły algebraicznie plastrowe}
Tekst tej podsekcji to w dużej części bełkot, więc nie ma nic złego w zignorowaniu jej tak bardzo, jak tylko można.
W telegraficznym skrócie: węzeł, którego macierz Seiferta jest zerowo kobordantna, nazywamy plastrowym algebraicznie.
Lokalnie płaską, zwartą, zorientowaną, właściwą powierzchnię $S$ w $B^4$ taką, że $K = \partial S$ jest węzłem w $\partial B^4 = S^3$ nazywamy izotropową, jeżeli istnieje lokalnie płaska, zwarta, zorientowana 3-podrozmaitość $M \subseteq B^4$, gdzie $S \subseteq \partial M$ oraz $F = \operatorname{cl} \partial M \setminus S$ jest powierzchnią Seiferta dla $K$ w $S^3$, zaś $S$ jest izotropowa w $M$.

\begin{proposition}
    Węzeł $K$ w~$S^3$ jest algebraicznie plastrowy dokładnie wtedy, gdy ogranicza izotropową powierzchnię $S$ w~kuli $B^4$.
\end{proposition}

\begin{corollary}
    Niech $K$ będzie węzłem plastrowym.
    Wtedy $K$ jest węzłem algebraicznie plastrowym.
\end{corollary}

\begin{proof}
    Kawauchi \cite[s. 158]{kawauchi1996}.
\end{proof}

\begin{proposition}
    \label{prp:cobordant_to_algebraic_is_algebraic}
    Niech $K$ będzie węzłem zgodnym z węzłem algebraicznie plastrowym.
    Wtedy każda macierz Seiferta dowolnej powierzchni Seiferta $K$ jest zerowo kobordantna.
    W szczególności, $K$ jest węzłem algebraicznie plastrowym.
\end{proposition}

\begin{proof}
    Kawauchi \cite[s. 159]{kawauchi1996}.
\end{proof}

% Theorem 1.3[Long 1984].A strongly positive amphicheiral knot is algebraicallyslice.
% Theorem 1.4[Hartley and Kawauchi 1979].If K is strongly positive amphicheiral,the Alexander polynomial1Kis the square of a symmetric polynomial.





\part{Załączniki}
\chapterimage{orange2.jpg} % Chapter heading image
\chapterspaceabove{5.75cm} % Whitespace from the top of the page to the chapter title on chapter pages
\chapterspacebelow{7.25cm} % Amount of vertical whitespace from the top margin to the start of the text on chapter pages

\begin{appendices}
    \renewcommand{\chaptername}{Appendix} % Change the chapter name to Appendix, i.e. "Appendix A: Title", instead of "Chapter A: Title" in the headers
    \chapter{Tablice węzłów pierwszych}
    \input{90-appendix/table_invariants_intro}
    Tabela pierwsza podsumowuje tabelę drugą:
    \input{90-appendix/table_invariants_summary}
    \newpage
    Tabela druga zawiera wartości niezmienników:
    \input{90-appendix/table_invariants}
    
\section{Diagramy węzłów pierwszych do dziesięciu skrzyżowań}
Poniżej znajdują się diagramy węzłów pierwszych, które realizują liczbę gordyjską, jeśli ta nie przekracza dziesięciu.
One także pochodzą ze strony KnotInfo \cite{knotinfo24}, o~której mowa na początku rozdziału.

\begin{comment}
\begin{figure}[H]
    \begin{minipage}[b]{.18\linewidth}
        \centering
        \includegraphics[width=\linewidth]{../data/knots/3_1.png}
        \subcaption{$3_{1}$}
    \end{minipage}
    \begin{minipage}[b]{.18\linewidth}
        \centering
        \includegraphics[width=\linewidth]{../data/knots/4_1.png}
        \subcaption{$4_{1}$}
    \end{minipage}
    \begin{minipage}[b]{.18\linewidth}
        \centering
        \includegraphics[width=\linewidth]{../data/knots/5_1.png}
        \subcaption{$5_{1}$}
    \end{minipage}
    \begin{minipage}[b]{.18\linewidth}
        \centering
        \includegraphics[width=\linewidth]{../data/knots/5_2.png}
        \subcaption{$5_{2}$}
    \end{minipage}
    \begin{minipage}[b]{.18\linewidth}
        \centering
        \includegraphics[width=\linewidth]{../data/knots/6_1.png}
        \subcaption{$6_{1}$}
    \end{minipage}
\end{figure}
\begin{figure}[H]
    \begin{minipage}[b]{.18\linewidth}
        \centering
        \includegraphics[width=\linewidth]{../data/knots/6_2.png}
        \subcaption{$6_{2}$}
    \end{minipage}
    \begin{minipage}[b]{.18\linewidth}
        \centering
        \includegraphics[width=\linewidth]{../data/knots/6_3.png}
        \subcaption{$6_{3}$}
    \end{minipage}
    \begin{minipage}[b]{.18\linewidth}
        \centering
        \includegraphics[width=\linewidth]{../data/knots/7_1.png}
        \subcaption{$7_{1}$}
    \end{minipage}
    \begin{minipage}[b]{.18\linewidth}
        \centering
        \includegraphics[width=\linewidth]{../data/knots/7_2.png}
        \subcaption{$7_{2}$}
    \end{minipage}
    \begin{minipage}[b]{.18\linewidth}
        \centering
        \includegraphics[width=\linewidth]{../data/knots/7_3.png}
        \subcaption{$7_{3}$}
    \end{minipage}
\end{figure}
\begin{figure}[H]
    \begin{minipage}[b]{.18\linewidth}
        \centering
        \includegraphics[width=\linewidth]{../data/knots/7_4.png}
        \subcaption{$7_{4}$}
    \end{minipage}
    \begin{minipage}[b]{.18\linewidth}
        \centering
        \includegraphics[width=\linewidth]{../data/knots/7_5.png}
        \subcaption{$7_{5}$}
    \end{minipage}
    \begin{minipage}[b]{.18\linewidth}
        \centering
        \includegraphics[width=\linewidth]{../data/knots/7_6.png}
        \subcaption{$7_{6}$}
    \end{minipage}
    \begin{minipage}[b]{.18\linewidth}
        \centering
        \includegraphics[width=\linewidth]{../data/knots/7_7.png}
        \subcaption{$7_{7}$}
    \end{minipage}
    \begin{minipage}[b]{.18\linewidth}
        \centering
        \includegraphics[width=\linewidth]{../data/knots/8_1.png}
        \subcaption{$8_{1}$}
    \end{minipage}
\end{figure}
\begin{figure}[H]
    \begin{minipage}[b]{.18\linewidth}
        \centering
        \includegraphics[width=\linewidth]{../data/knots/8_2.png}
        \subcaption{$8_{2}$}
    \end{minipage}
    \begin{minipage}[b]{.18\linewidth}
        \centering
        \includegraphics[width=\linewidth]{../data/knots/8_3.png}
        \subcaption{$8_{3}$}
    \end{minipage}
    \begin{minipage}[b]{.18\linewidth}
        \centering
        \includegraphics[width=\linewidth]{../data/knots/8_4.png}
        \subcaption{$8_{4}$}
    \end{minipage}
    \begin{minipage}[b]{.18\linewidth}
        \centering
        \includegraphics[width=\linewidth]{../data/knots/8_5.png}
        \subcaption{$8_{5}$}
    \end{minipage}
    \begin{minipage}[b]{.18\linewidth}
        \centering
        \includegraphics[width=\linewidth]{../data/knots/8_6.png}
        \subcaption{$8_{6}$}
    \end{minipage}
\end{figure}
\begin{figure}[H]
    \begin{minipage}[b]{.18\linewidth}
        \centering
        \includegraphics[width=\linewidth]{../data/knots/8_7.png}
        \subcaption{$8_{7}$}
    \end{minipage}
    \begin{minipage}[b]{.18\linewidth}
        \centering
        \includegraphics[width=\linewidth]{../data/knots/8_8.png}
        \subcaption{$8_{8}$}
    \end{minipage}
    \begin{minipage}[b]{.18\linewidth}
        \centering
        \includegraphics[width=\linewidth]{../data/knots/8_9.png}
        \subcaption{$8_{9}$}
    \end{minipage}
    \begin{minipage}[b]{.18\linewidth}
        \centering
        \includegraphics[width=\linewidth]{../data/knots/8_10.png}
        \subcaption{$8_{10}$}
    \end{minipage}
    \begin{minipage}[b]{.18\linewidth}
        \centering
        \includegraphics[width=\linewidth]{../data/knots/8_11.png}
        \subcaption{$8_{11}$}
    \end{minipage}
\end{figure}
\begin{figure}[H]
    \begin{minipage}[b]{.18\linewidth}
        \centering
        \includegraphics[width=\linewidth]{../data/knots/8_12.png}
        \subcaption{$8_{12}$}
    \end{minipage}
    \begin{minipage}[b]{.18\linewidth}
        \centering
        \includegraphics[width=\linewidth]{../data/knots/8_13.png}
        \subcaption{$8_{13}$}
    \end{minipage}
    \begin{minipage}[b]{.18\linewidth}
        \centering
        \includegraphics[width=\linewidth]{../data/knots/8_14.png}
        \subcaption{$8_{14}$}
    \end{minipage}
    \begin{minipage}[b]{.18\linewidth}
        \centering
        \includegraphics[width=\linewidth]{../data/knots/8_15.png}
        \subcaption{$8_{15}$}
    \end{minipage}
    \begin{minipage}[b]{.18\linewidth}
        \centering
        \includegraphics[width=\linewidth]{../data/knots/8_16.png}
        \subcaption{$8_{16}$}
    \end{minipage}
\end{figure}
\begin{figure}[H]
    \begin{minipage}[b]{.18\linewidth}
        \centering
        \includegraphics[width=\linewidth]{../data/knots/8_17.png}
        \subcaption{$8_{17}$}
    \end{minipage}
    \begin{minipage}[b]{.18\linewidth}
        \centering
        \includegraphics[width=\linewidth]{../data/knots/8_18.png}
        \subcaption{$8_{18}$}
    \end{minipage}
    \begin{minipage}[b]{.18\linewidth}
        \centering
        \includegraphics[width=\linewidth]{../data/knots/8_19.png}
        \subcaption{$8_{19}$}
    \end{minipage}
    \begin{minipage}[b]{.18\linewidth}
        \centering
        \includegraphics[width=\linewidth]{../data/knots/8_20.png}
        \subcaption{$8_{20}$}
    \end{minipage}
    \begin{minipage}[b]{.18\linewidth}
        \centering
        \includegraphics[width=\linewidth]{../data/knots/8_21.png}
        \subcaption{$8_{21}$}
    \end{minipage}
\end{figure}
\begin{figure}[H]
    \begin{minipage}[b]{.18\linewidth}
        \centering
        \includegraphics[width=\linewidth]{../data/knots/9_1.png}
        \subcaption{$9_{1}$}
    \end{minipage}
    \begin{minipage}[b]{.18\linewidth}
        \centering
        \includegraphics[width=\linewidth]{../data/knots/9_2.png}
        \subcaption{$9_{2}$}
    \end{minipage}
    \begin{minipage}[b]{.18\linewidth}
        \centering
        \includegraphics[width=\linewidth]{../data/knots/9_3.png}
        \subcaption{$9_{3}$}
    \end{minipage}
    \begin{minipage}[b]{.18\linewidth}
        \centering
        \includegraphics[width=\linewidth]{../data/knots/9_4.png}
        \subcaption{$9_{4}$}
    \end{minipage}
    \begin{minipage}[b]{.18\linewidth}
        \centering
        \includegraphics[width=\linewidth]{../data/knots/9_5.png}
        \subcaption{$9_{5}$}
    \end{minipage}
\end{figure}
\begin{figure}[H]
    \begin{minipage}[b]{.18\linewidth}
        \centering
        \includegraphics[width=\linewidth]{../data/knots/9_6.png}
        \subcaption{$9_{6}$}
    \end{minipage}
    \begin{minipage}[b]{.18\linewidth}
        \centering
        \includegraphics[width=\linewidth]{../data/knots/9_7.png}
        \subcaption{$9_{7}$}
    \end{minipage}
    \begin{minipage}[b]{.18\linewidth}
        \centering
        \includegraphics[width=\linewidth]{../data/knots/9_8.png}
        \subcaption{$9_{8}$}
    \end{minipage}
    \begin{minipage}[b]{.18\linewidth}
        \centering
        \includegraphics[width=\linewidth]{../data/knots/9_9.png}
        \subcaption{$9_{9}$}
    \end{minipage}
    \begin{minipage}[b]{.18\linewidth}
        \centering
        \includegraphics[width=\linewidth]{../data/knots/9_10.png}
        \subcaption{$9_{10}$}
    \end{minipage}
\end{figure}
\begin{figure}[H]
    \begin{minipage}[b]{.18\linewidth}
        \centering
        \includegraphics[width=\linewidth]{../data/knots/9_11.png}
        \subcaption{$9_{11}$}
    \end{minipage}
    \begin{minipage}[b]{.18\linewidth}
        \centering
        \includegraphics[width=\linewidth]{../data/knots/9_12.png}
        \subcaption{$9_{12}$}
    \end{minipage}
    \begin{minipage}[b]{.18\linewidth}
        \centering
        \includegraphics[width=\linewidth]{../data/knots/9_13.png}
        \subcaption{$9_{13}$}
    \end{minipage}
    \begin{minipage}[b]{.18\linewidth}
        \centering
        \includegraphics[width=\linewidth]{../data/knots/9_14.png}
        \subcaption{$9_{14}$}
    \end{minipage}
    \begin{minipage}[b]{.18\linewidth}
        \centering
        \includegraphics[width=\linewidth]{../data/knots/9_15.png}
        \subcaption{$9_{15}$}
    \end{minipage}
\end{figure}
\begin{figure}[H]
    \begin{minipage}[b]{.18\linewidth}
        \centering
        \includegraphics[width=\linewidth]{../data/knots/9_16.png}
        \subcaption{$9_{16}$}
    \end{minipage}
    \begin{minipage}[b]{.18\linewidth}
        \centering
        \includegraphics[width=\linewidth]{../data/knots/9_17.png}
        \subcaption{$9_{17}$}
    \end{minipage}
    \begin{minipage}[b]{.18\linewidth}
        \centering
        \includegraphics[width=\linewidth]{../data/knots/9_18.png}
        \subcaption{$9_{18}$}
    \end{minipage}
    \begin{minipage}[b]{.18\linewidth}
        \centering
        \includegraphics[width=\linewidth]{../data/knots/9_19.png}
        \subcaption{$9_{19}$}
    \end{minipage}
    \begin{minipage}[b]{.18\linewidth}
        \centering
        \includegraphics[width=\linewidth]{../data/knots/9_20.png}
        \subcaption{$9_{20}$}
    \end{minipage}
\end{figure}
\begin{figure}[H]
    \begin{minipage}[b]{.18\linewidth}
        \centering
        \includegraphics[width=\linewidth]{../data/knots/9_21.png}
        \subcaption{$9_{21}$}
    \end{minipage}
    \begin{minipage}[b]{.18\linewidth}
        \centering
        \includegraphics[width=\linewidth]{../data/knots/9_22.png}
        \subcaption{$9_{22}$}
    \end{minipage}
    \begin{minipage}[b]{.18\linewidth}
        \centering
        \includegraphics[width=\linewidth]{../data/knots/9_23.png}
        \subcaption{$9_{23}$}
    \end{minipage}
    \begin{minipage}[b]{.18\linewidth}
        \centering
        \includegraphics[width=\linewidth]{../data/knots/9_24.png}
        \subcaption{$9_{24}$}
    \end{minipage}
    \begin{minipage}[b]{.18\linewidth}
        \centering
        \includegraphics[width=\linewidth]{../data/knots/9_25.png}
        \subcaption{$9_{25}$}
    \end{minipage}
\end{figure}
\begin{figure}[H]
    \begin{minipage}[b]{.18\linewidth}
        \centering
        \includegraphics[width=\linewidth]{../data/knots/9_26.png}
        \subcaption{$9_{26}$}
    \end{minipage}
    \begin{minipage}[b]{.18\linewidth}
        \centering
        \includegraphics[width=\linewidth]{../data/knots/9_27.png}
        \subcaption{$9_{27}$}
    \end{minipage}
    \begin{minipage}[b]{.18\linewidth}
        \centering
        \includegraphics[width=\linewidth]{../data/knots/9_28.png}
        \subcaption{$9_{28}$}
    \end{minipage}
    \begin{minipage}[b]{.18\linewidth}
        \centering
        \includegraphics[width=\linewidth]{../data/knots/9_29.png}
        \subcaption{$9_{29}$}
    \end{minipage}
    \begin{minipage}[b]{.18\linewidth}
        \centering
        \includegraphics[width=\linewidth]{../data/knots/9_30.png}
        \subcaption{$9_{30}$}
    \end{minipage}
\end{figure}
\begin{figure}[H]
    \begin{minipage}[b]{.18\linewidth}
        \centering
        \includegraphics[width=\linewidth]{../data/knots/9_31.png}
        \subcaption{$9_{31}$}
    \end{minipage}
    \begin{minipage}[b]{.18\linewidth}
        \centering
        \includegraphics[width=\linewidth]{../data/knots/9_32.png}
        \subcaption{$9_{32}$}
    \end{minipage}
    \begin{minipage}[b]{.18\linewidth}
        \centering
        \includegraphics[width=\linewidth]{../data/knots/9_33.png}
        \subcaption{$9_{33}$}
    \end{minipage}
    \begin{minipage}[b]{.18\linewidth}
        \centering
        \includegraphics[width=\linewidth]{../data/knots/9_34.png}
        \subcaption{$9_{34}$}
    \end{minipage}
    \begin{minipage}[b]{.18\linewidth}
        \centering
        \includegraphics[width=\linewidth]{../data/knots/9_35.png}
        \subcaption{$9_{35}$}
    \end{minipage}
\end{figure}
\begin{figure}[H]
    \begin{minipage}[b]{.18\linewidth}
        \centering
        \includegraphics[width=\linewidth]{../data/knots/9_36.png}
        \subcaption{$9_{36}$}
    \end{minipage}
    \begin{minipage}[b]{.18\linewidth}
        \centering
        \includegraphics[width=\linewidth]{../data/knots/9_37.png}
        \subcaption{$9_{37}$}
    \end{minipage}
    \begin{minipage}[b]{.18\linewidth}
        \centering
        \includegraphics[width=\linewidth]{../data/knots/9_38.png}
        \subcaption{$9_{38}$}
    \end{minipage}
    \begin{minipage}[b]{.18\linewidth}
        \centering
        \includegraphics[width=\linewidth]{../data/knots/9_39.png}
        \subcaption{$9_{39}$}
    \end{minipage}
    \begin{minipage}[b]{.18\linewidth}
        \centering
        \includegraphics[width=\linewidth]{../data/knots/9_40.png}
        \subcaption{$9_{40}$}
    \end{minipage}
\end{figure}
\begin{figure}[H]
    \begin{minipage}[b]{.18\linewidth}
        \centering
        \includegraphics[width=\linewidth]{../data/knots/9_41.png}
        \subcaption{$9_{41}$}
    \end{minipage}
    \begin{minipage}[b]{.18\linewidth}
        \centering
        \includegraphics[width=\linewidth]{../data/knots/9_42.png}
        \subcaption{$9_{42}$}
    \end{minipage}
    \begin{minipage}[b]{.18\linewidth}
        \centering
        \includegraphics[width=\linewidth]{../data/knots/9_43.png}
        \subcaption{$9_{43}$}
    \end{minipage}
    \begin{minipage}[b]{.18\linewidth}
        \centering
        \includegraphics[width=\linewidth]{../data/knots/9_44.png}
        \subcaption{$9_{44}$}
    \end{minipage}
    \begin{minipage}[b]{.18\linewidth}
        \centering
        \includegraphics[width=\linewidth]{../data/knots/9_45.png}
        \subcaption{$9_{45}$}
    \end{minipage}
\end{figure}
\begin{figure}[H]
    \begin{minipage}[b]{.18\linewidth}
        \centering
        \includegraphics[width=\linewidth]{../data/knots/9_46.png}
        \subcaption{$9_{46}$}
    \end{minipage}
    \begin{minipage}[b]{.18\linewidth}
        \centering
        \includegraphics[width=\linewidth]{../data/knots/9_47.png}
        \subcaption{$9_{47}$}
    \end{minipage}
    \begin{minipage}[b]{.18\linewidth}
        \centering
        \includegraphics[width=\linewidth]{../data/knots/9_48.png}
        \subcaption{$9_{48}$}
    \end{minipage}
    \begin{minipage}[b]{.18\linewidth}
        \centering
        \includegraphics[width=\linewidth]{../data/knots/9_49.png}
        \subcaption{$9_{49}$}
    \end{minipage}
    \begin{minipage}[b]{.18\linewidth}
        \centering
        \includegraphics[width=\linewidth]{../data/knots/10_1.png}
        \subcaption{$10_{1}$}
    \end{minipage}
\end{figure}
\begin{figure}[H]
    \begin{minipage}[b]{.18\linewidth}
        \centering
        \includegraphics[width=\linewidth]{../data/knots/10_2.png}
        \subcaption{$10_{2}$}
    \end{minipage}
    \begin{minipage}[b]{.18\linewidth}
        \centering
        \includegraphics[width=\linewidth]{../data/knots/10_3.png}
        \subcaption{$10_{3}$}
    \end{minipage}
    \begin{minipage}[b]{.18\linewidth}
        \centering
        \includegraphics[width=\linewidth]{../data/knots/10_4.png}
        \subcaption{$10_{4}$}
    \end{minipage}
    \begin{minipage}[b]{.18\linewidth}
        \centering
        \includegraphics[width=\linewidth]{../data/knots/10_5.png}
        \subcaption{$10_{5}$}
    \end{minipage}
    \begin{minipage}[b]{.18\linewidth}
        \centering
        \includegraphics[width=\linewidth]{../data/knots/10_6.png}
        \subcaption{$10_{6}$}
    \end{minipage}
\end{figure}
\begin{figure}[H]
    \begin{minipage}[b]{.18\linewidth}
        \centering
        \includegraphics[width=\linewidth]{../data/knots/10_7.png}
        \subcaption{$10_{7}$}
    \end{minipage}
    \begin{minipage}[b]{.18\linewidth}
        \centering
        \includegraphics[width=\linewidth]{../data/knots/10_8.png}
        \subcaption{$10_{8}$}
    \end{minipage}
    \begin{minipage}[b]{.18\linewidth}
        \centering
        \includegraphics[width=\linewidth]{../data/knots/10_9.png}
        \subcaption{$10_{9}$}
    \end{minipage}
    \begin{minipage}[b]{.18\linewidth}
        \centering
        \includegraphics[width=\linewidth]{../data/knots/10_10.png}
        \subcaption{$10_{10}$}
    \end{minipage}
    \begin{minipage}[b]{.18\linewidth}
        \centering
        \includegraphics[width=\linewidth]{../data/knots/10_11.png}
        \subcaption{$10_{11}$}
    \end{minipage}
\end{figure}
\begin{figure}[H]
    \begin{minipage}[b]{.18\linewidth}
        \centering
        \includegraphics[width=\linewidth]{../data/knots/10_12.png}
        \subcaption{$10_{12}$}
    \end{minipage}
    \begin{minipage}[b]{.18\linewidth}
        \centering
        \includegraphics[width=\linewidth]{../data/knots/10_13.png}
        \subcaption{$10_{13}$}
    \end{minipage}
    \begin{minipage}[b]{.18\linewidth}
        \centering
        \includegraphics[width=\linewidth]{../data/knots/10_14.png}
        \subcaption{$10_{14}$}
    \end{minipage}
    \begin{minipage}[b]{.18\linewidth}
        \centering
        \includegraphics[width=\linewidth]{../data/knots/10_15.png}
        \subcaption{$10_{15}$}
    \end{minipage}
    \begin{minipage}[b]{.18\linewidth}
        \centering
        \includegraphics[width=\linewidth]{../data/knots/10_16.png}
        \subcaption{$10_{16}$}
    \end{minipage}
\end{figure}
\begin{figure}[H]
    \begin{minipage}[b]{.18\linewidth}
        \centering
        \includegraphics[width=\linewidth]{../data/knots/10_17.png}
        \subcaption{$10_{17}$}
    \end{minipage}
    \begin{minipage}[b]{.18\linewidth}
        \centering
        \includegraphics[width=\linewidth]{../data/knots/10_18.png}
        \subcaption{$10_{18}$}
    \end{minipage}
    \begin{minipage}[b]{.18\linewidth}
        \centering
        \includegraphics[width=\linewidth]{../data/knots/10_19.png}
        \subcaption{$10_{19}$}
    \end{minipage}
    \begin{minipage}[b]{.18\linewidth}
        \centering
        \includegraphics[width=\linewidth]{../data/knots/10_20.png}
        \subcaption{$10_{20}$}
    \end{minipage}
    \begin{minipage}[b]{.18\linewidth}
        \centering
        \includegraphics[width=\linewidth]{../data/knots/10_21.png}
        \subcaption{$10_{21}$}
    \end{minipage}
\end{figure}
\begin{figure}[H]
    \begin{minipage}[b]{.18\linewidth}
        \centering
        \includegraphics[width=\linewidth]{../data/knots/10_22.png}
        \subcaption{$10_{22}$}
    \end{minipage}
    \begin{minipage}[b]{.18\linewidth}
        \centering
        \includegraphics[width=\linewidth]{../data/knots/10_23.png}
        \subcaption{$10_{23}$}
    \end{minipage}
    \begin{minipage}[b]{.18\linewidth}
        \centering
        \includegraphics[width=\linewidth]{../data/knots/10_24.png}
        \subcaption{$10_{24}$}
    \end{minipage}
    \begin{minipage}[b]{.18\linewidth}
        \centering
        \includegraphics[width=\linewidth]{../data/knots/10_25.png}
        \subcaption{$10_{25}$}
    \end{minipage}
    \begin{minipage}[b]{.18\linewidth}
        \centering
        \includegraphics[width=\linewidth]{../data/knots/10_26.png}
        \subcaption{$10_{26}$}
    \end{minipage}
\end{figure}
\begin{figure}[H]
    \begin{minipage}[b]{.18\linewidth}
        \centering
        \includegraphics[width=\linewidth]{../data/knots/10_27.png}
        \subcaption{$10_{27}$}
    \end{minipage}
    \begin{minipage}[b]{.18\linewidth}
        \centering
        \includegraphics[width=\linewidth]{../data/knots/10_28.png}
        \subcaption{$10_{28}$}
    \end{minipage}
    \begin{minipage}[b]{.18\linewidth}
        \centering
        \includegraphics[width=\linewidth]{../data/knots/10_29.png}
        \subcaption{$10_{29}$}
    \end{minipage}
    \begin{minipage}[b]{.18\linewidth}
        \centering
        \includegraphics[width=\linewidth]{../data/knots/10_30.png}
        \subcaption{$10_{30}$}
    \end{minipage}
    \begin{minipage}[b]{.18\linewidth}
        \centering
        \includegraphics[width=\linewidth]{../data/knots/10_31.png}
        \subcaption{$10_{31}$}
    \end{minipage}
\end{figure}
\begin{figure}[H]
    \begin{minipage}[b]{.18\linewidth}
        \centering
        \includegraphics[width=\linewidth]{../data/knots/10_32.png}
        \subcaption{$10_{32}$}
    \end{minipage}
    \begin{minipage}[b]{.18\linewidth}
        \centering
        \includegraphics[width=\linewidth]{../data/knots/10_33.png}
        \subcaption{$10_{33}$}
    \end{minipage}
    \begin{minipage}[b]{.18\linewidth}
        \centering
        \includegraphics[width=\linewidth]{../data/knots/10_34.png}
        \subcaption{$10_{34}$}
    \end{minipage}
    \begin{minipage}[b]{.18\linewidth}
        \centering
        \includegraphics[width=\linewidth]{../data/knots/10_35.png}
        \subcaption{$10_{35}$}
    \end{minipage}
    \begin{minipage}[b]{.18\linewidth}
        \centering
        \includegraphics[width=\linewidth]{../data/knots/10_36.png}
        \subcaption{$10_{36}$}
    \end{minipage}
\end{figure}
\begin{figure}[H]
    \begin{minipage}[b]{.18\linewidth}
        \centering
        \includegraphics[width=\linewidth]{../data/knots/10_37.png}
        \subcaption{$10_{37}$}
    \end{minipage}
    \begin{minipage}[b]{.18\linewidth}
        \centering
        \includegraphics[width=\linewidth]{../data/knots/10_38.png}
        \subcaption{$10_{38}$}
    \end{minipage}
    \begin{minipage}[b]{.18\linewidth}
        \centering
        \includegraphics[width=\linewidth]{../data/knots/10_39.png}
        \subcaption{$10_{39}$}
    \end{minipage}
    \begin{minipage}[b]{.18\linewidth}
        \centering
        \includegraphics[width=\linewidth]{../data/knots/10_40.png}
        \subcaption{$10_{40}$}
    \end{minipage}
    \begin{minipage}[b]{.18\linewidth}
        \centering
        \includegraphics[width=\linewidth]{../data/knots/10_41.png}
        \subcaption{$10_{41}$}
    \end{minipage}
\end{figure}
\begin{figure}[H]
    \begin{minipage}[b]{.18\linewidth}
        \centering
        \includegraphics[width=\linewidth]{../data/knots/10_42.png}
        \subcaption{$10_{42}$}
    \end{minipage}
    \begin{minipage}[b]{.18\linewidth}
        \centering
        \includegraphics[width=\linewidth]{../data/knots/10_43.png}
        \subcaption{$10_{43}$}
    \end{minipage}
    \begin{minipage}[b]{.18\linewidth}
        \centering
        \includegraphics[width=\linewidth]{../data/knots/10_44.png}
        \subcaption{$10_{44}$}
    \end{minipage}
    \begin{minipage}[b]{.18\linewidth}
        \centering
        \includegraphics[width=\linewidth]{../data/knots/10_45.png}
        \subcaption{$10_{45}$}
    \end{minipage}
    \begin{minipage}[b]{.18\linewidth}
        \centering
        \includegraphics[width=\linewidth]{../data/knots/10_46.png}
        \subcaption{$10_{46}$}
    \end{minipage}
\end{figure}
\begin{figure}[H]
    \begin{minipage}[b]{.18\linewidth}
        \centering
        \includegraphics[width=\linewidth]{../data/knots/10_47.png}
        \subcaption{$10_{47}$}
    \end{minipage}
    \begin{minipage}[b]{.18\linewidth}
        \centering
        \includegraphics[width=\linewidth]{../data/knots/10_48.png}
        \subcaption{$10_{48}$}
    \end{minipage}
    \begin{minipage}[b]{.18\linewidth}
        \centering
        \includegraphics[width=\linewidth]{../data/knots/10_49.png}
        \subcaption{$10_{49}$}
    \end{minipage}
    \begin{minipage}[b]{.18\linewidth}
        \centering
        \includegraphics[width=\linewidth]{../data/knots/10_50.png}
        \subcaption{$10_{50}$}
    \end{minipage}
    \begin{minipage}[b]{.18\linewidth}
        \centering
        \includegraphics[width=\linewidth]{../data/knots/10_51.png}
        \subcaption{$10_{51}$}
    \end{minipage}
\end{figure}
\begin{figure}[H]
    \begin{minipage}[b]{.18\linewidth}
        \centering
        \includegraphics[width=\linewidth]{../data/knots/10_52.png}
        \subcaption{$10_{52}$}
    \end{minipage}
    \begin{minipage}[b]{.18\linewidth}
        \centering
        \includegraphics[width=\linewidth]{../data/knots/10_53.png}
        \subcaption{$10_{53}$}
    \end{minipage}
    \begin{minipage}[b]{.18\linewidth}
        \centering
        \includegraphics[width=\linewidth]{../data/knots/10_54.png}
        \subcaption{$10_{54}$}
    \end{minipage}
    \begin{minipage}[b]{.18\linewidth}
        \centering
        \includegraphics[width=\linewidth]{../data/knots/10_55.png}
        \subcaption{$10_{55}$}
    \end{minipage}
    \begin{minipage}[b]{.18\linewidth}
        \centering
        \includegraphics[width=\linewidth]{../data/knots/10_56.png}
        \subcaption{$10_{56}$}
    \end{minipage}
\end{figure}
\begin{figure}[H]
    \begin{minipage}[b]{.18\linewidth}
        \centering
        \includegraphics[width=\linewidth]{../data/knots/10_57.png}
        \subcaption{$10_{57}$}
    \end{minipage}
    \begin{minipage}[b]{.18\linewidth}
        \centering
        \includegraphics[width=\linewidth]{../data/knots/10_58.png}
        \subcaption{$10_{58}$}
    \end{minipage}
    \begin{minipage}[b]{.18\linewidth}
        \centering
        \includegraphics[width=\linewidth]{../data/knots/10_59.png}
        \subcaption{$10_{59}$}
    \end{minipage}
    \begin{minipage}[b]{.18\linewidth}
        \centering
        \includegraphics[width=\linewidth]{../data/knots/10_60.png}
        \subcaption{$10_{60}$}
    \end{minipage}
    \begin{minipage}[b]{.18\linewidth}
        \centering
        \includegraphics[width=\linewidth]{../data/knots/10_61.png}
        \subcaption{$10_{61}$}
    \end{minipage}
\end{figure}
\begin{figure}[H]
    \begin{minipage}[b]{.18\linewidth}
        \centering
        \includegraphics[width=\linewidth]{../data/knots/10_62.png}
        \subcaption{$10_{62}$}
    \end{minipage}
    \begin{minipage}[b]{.18\linewidth}
        \centering
        \includegraphics[width=\linewidth]{../data/knots/10_63.png}
        \subcaption{$10_{63}$}
    \end{minipage}
    \begin{minipage}[b]{.18\linewidth}
        \centering
        \includegraphics[width=\linewidth]{../data/knots/10_64.png}
        \subcaption{$10_{64}$}
    \end{minipage}
    \begin{minipage}[b]{.18\linewidth}
        \centering
        \includegraphics[width=\linewidth]{../data/knots/10_65.png}
        \subcaption{$10_{65}$}
    \end{minipage}
    \begin{minipage}[b]{.18\linewidth}
        \centering
        \includegraphics[width=\linewidth]{../data/knots/10_66.png}
        \subcaption{$10_{66}$}
    \end{minipage}
\end{figure}
\begin{figure}[H]
    \begin{minipage}[b]{.18\linewidth}
        \centering
        \includegraphics[width=\linewidth]{../data/knots/10_67.png}
        \subcaption{$10_{67}$}
    \end{minipage}
    \begin{minipage}[b]{.18\linewidth}
        \centering
        \includegraphics[width=\linewidth]{../data/knots/10_68.png}
        \subcaption{$10_{68}$}
    \end{minipage}
    \begin{minipage}[b]{.18\linewidth}
        \centering
        \includegraphics[width=\linewidth]{../data/knots/10_69.png}
        \subcaption{$10_{69}$}
    \end{minipage}
    \begin{minipage}[b]{.18\linewidth}
        \centering
        \includegraphics[width=\linewidth]{../data/knots/10_70.png}
        \subcaption{$10_{70}$}
    \end{minipage}
    \begin{minipage}[b]{.18\linewidth}
        \centering
        \includegraphics[width=\linewidth]{../data/knots/10_71.png}
        \subcaption{$10_{71}$}
    \end{minipage}
\end{figure}
\begin{figure}[H]
    \begin{minipage}[b]{.18\linewidth}
        \centering
        \includegraphics[width=\linewidth]{../data/knots/10_72.png}
        \subcaption{$10_{72}$}
    \end{minipage}
    \begin{minipage}[b]{.18\linewidth}
        \centering
        \includegraphics[width=\linewidth]{../data/knots/10_73.png}
        \subcaption{$10_{73}$}
    \end{minipage}
    \begin{minipage}[b]{.18\linewidth}
        \centering
        \includegraphics[width=\linewidth]{../data/knots/10_74.png}
        \subcaption{$10_{74}$}
    \end{minipage}
    \begin{minipage}[b]{.18\linewidth}
        \centering
        \includegraphics[width=\linewidth]{../data/knots/10_75.png}
        \subcaption{$10_{75}$}
    \end{minipage}
    \begin{minipage}[b]{.18\linewidth}
        \centering
        \includegraphics[width=\linewidth]{../data/knots/10_76.png}
        \subcaption{$10_{76}$}
    \end{minipage}
\end{figure}
\begin{figure}[H]
    \begin{minipage}[b]{.18\linewidth}
        \centering
        \includegraphics[width=\linewidth]{../data/knots/10_77.png}
        \subcaption{$10_{77}$}
    \end{minipage}
    \begin{minipage}[b]{.18\linewidth}
        \centering
        \includegraphics[width=\linewidth]{../data/knots/10_78.png}
        \subcaption{$10_{78}$}
    \end{minipage}
    \begin{minipage}[b]{.18\linewidth}
        \centering
        \includegraphics[width=\linewidth]{../data/knots/10_79.png}
        \subcaption{$10_{79}$}
    \end{minipage}
    \begin{minipage}[b]{.18\linewidth}
        \centering
        \includegraphics[width=\linewidth]{../data/knots/10_80.png}
        \subcaption{$10_{80}$}
    \end{minipage}
    \begin{minipage}[b]{.18\linewidth}
        \centering
        \includegraphics[width=\linewidth]{../data/knots/10_81.png}
        \subcaption{$10_{81}$}
    \end{minipage}
\end{figure}
\begin{figure}[H]
    \begin{minipage}[b]{.18\linewidth}
        \centering
        \includegraphics[width=\linewidth]{../data/knots/10_82.png}
        \subcaption{$10_{82}$}
    \end{minipage}
    \begin{minipage}[b]{.18\linewidth}
        \centering
        \includegraphics[width=\linewidth]{../data/knots/10_83.png}
        \subcaption{$10_{83}$}
    \end{minipage}
    \begin{minipage}[b]{.18\linewidth}
        \centering
        \includegraphics[width=\linewidth]{../data/knots/10_84.png}
        \subcaption{$10_{84}$}
    \end{minipage}
    \begin{minipage}[b]{.18\linewidth}
        \centering
        \includegraphics[width=\linewidth]{../data/knots/10_85.png}
        \subcaption{$10_{85}$}
    \end{minipage}
    \begin{minipage}[b]{.18\linewidth}
        \centering
        \includegraphics[width=\linewidth]{../data/knots/10_86.png}
        \subcaption{$10_{86}$}
    \end{minipage}
\end{figure}
\begin{figure}[H]
    \begin{minipage}[b]{.18\linewidth}
        \centering
        \includegraphics[width=\linewidth]{../data/knots/10_87.png}
        \subcaption{$10_{87}$}
    \end{minipage}
    \begin{minipage}[b]{.18\linewidth}
        \centering
        \includegraphics[width=\linewidth]{../data/knots/10_88.png}
        \subcaption{$10_{88}$}
    \end{minipage}
    \begin{minipage}[b]{.18\linewidth}
        \centering
        \includegraphics[width=\linewidth]{../data/knots/10_89.png}
        \subcaption{$10_{89}$}
    \end{minipage}
    \begin{minipage}[b]{.18\linewidth}
        \centering
        \includegraphics[width=\linewidth]{../data/knots/10_90.png}
        \subcaption{$10_{90}$}
    \end{minipage}
    \begin{minipage}[b]{.18\linewidth}
        \centering
        \includegraphics[width=\linewidth]{../data/knots/10_91.png}
        \subcaption{$10_{91}$}
    \end{minipage}
\end{figure}
\begin{figure}[H]
    \begin{minipage}[b]{.18\linewidth}
        \centering
        \includegraphics[width=\linewidth]{../data/knots/10_92.png}
        \subcaption{$10_{92}$}
    \end{minipage}
    \begin{minipage}[b]{.18\linewidth}
        \centering
        \includegraphics[width=\linewidth]{../data/knots/10_93.png}
        \subcaption{$10_{93}$}
    \end{minipage}
    \begin{minipage}[b]{.18\linewidth}
        \centering
        \includegraphics[width=\linewidth]{../data/knots/10_94.png}
        \subcaption{$10_{94}$}
    \end{minipage}
    \begin{minipage}[b]{.18\linewidth}
        \centering
        \includegraphics[width=\linewidth]{../data/knots/10_95.png}
        \subcaption{$10_{95}$}
    \end{minipage}
    \begin{minipage}[b]{.18\linewidth}
        \centering
        \includegraphics[width=\linewidth]{../data/knots/10_96.png}
        \subcaption{$10_{96}$}
    \end{minipage}
\end{figure}
\begin{figure}[H]
    \begin{minipage}[b]{.18\linewidth}
        \centering
        \includegraphics[width=\linewidth]{../data/knots/10_97.png}
        \subcaption{$10_{97}$}
    \end{minipage}
    \begin{minipage}[b]{.18\linewidth}
        \centering
        \includegraphics[width=\linewidth]{../data/knots/10_98.png}
        \subcaption{$10_{98}$}
    \end{minipage}
    \begin{minipage}[b]{.18\linewidth}
        \centering
        \includegraphics[width=\linewidth]{../data/knots/10_99.png}
        \subcaption{$10_{99}$}
    \end{minipage}
    \begin{minipage}[b]{.18\linewidth}
        \centering
        \includegraphics[width=\linewidth]{../data/knots/10_100.png}
        \subcaption{$10_{100}$}
    \end{minipage}
    \begin{minipage}[b]{.18\linewidth}
        \centering
        \includegraphics[width=\linewidth]{../data/knots/10_101.png}
        \subcaption{$10_{101}$}
    \end{minipage}
\end{figure}
\begin{figure}[H]
    \begin{minipage}[b]{.18\linewidth}
        \centering
        \includegraphics[width=\linewidth]{../data/knots/10_102.png}
        \subcaption{$10_{102}$}
    \end{minipage}
    \begin{minipage}[b]{.18\linewidth}
        \centering
        \includegraphics[width=\linewidth]{../data/knots/10_103.png}
        \subcaption{$10_{103}$}
    \end{minipage}
    \begin{minipage}[b]{.18\linewidth}
        \centering
        \includegraphics[width=\linewidth]{../data/knots/10_104.png}
        \subcaption{$10_{104}$}
    \end{minipage}
    \begin{minipage}[b]{.18\linewidth}
        \centering
        \includegraphics[width=\linewidth]{../data/knots/10_105.png}
        \subcaption{$10_{105}$}
    \end{minipage}
    \begin{minipage}[b]{.18\linewidth}
        \centering
        \includegraphics[width=\linewidth]{../data/knots/10_106.png}
        \subcaption{$10_{106}$}
    \end{minipage}
\end{figure}
\begin{figure}[H]
    \begin{minipage}[b]{.18\linewidth}
        \centering
        \includegraphics[width=\linewidth]{../data/knots/10_107.png}
        \subcaption{$10_{107}$}
    \end{minipage}
    \begin{minipage}[b]{.18\linewidth}
        \centering
        \includegraphics[width=\linewidth]{../data/knots/10_108.png}
        \subcaption{$10_{108}$}
    \end{minipage}
    \begin{minipage}[b]{.18\linewidth}
        \centering
        \includegraphics[width=\linewidth]{../data/knots/10_109.png}
        \subcaption{$10_{109}$}
    \end{minipage}
    \begin{minipage}[b]{.18\linewidth}
        \centering
        \includegraphics[width=\linewidth]{../data/knots/10_110.png}
        \subcaption{$10_{110}$}
    \end{minipage}
    \begin{minipage}[b]{.18\linewidth}
        \centering
        \includegraphics[width=\linewidth]{../data/knots/10_111.png}
        \subcaption{$10_{111}$}
    \end{minipage}
\end{figure}
\begin{figure}[H]
    \begin{minipage}[b]{.18\linewidth}
        \centering
        \includegraphics[width=\linewidth]{../data/knots/10_112.png}
        \subcaption{$10_{112}$}
    \end{minipage}
    \begin{minipage}[b]{.18\linewidth}
        \centering
        \includegraphics[width=\linewidth]{../data/knots/10_113.png}
        \subcaption{$10_{113}$}
    \end{minipage}
    \begin{minipage}[b]{.18\linewidth}
        \centering
        \includegraphics[width=\linewidth]{../data/knots/10_114.png}
        \subcaption{$10_{114}$}
    \end{minipage}
    \begin{minipage}[b]{.18\linewidth}
        \centering
        \includegraphics[width=\linewidth]{../data/knots/10_115.png}
        \subcaption{$10_{115}$}
    \end{minipage}
    \begin{minipage}[b]{.18\linewidth}
        \centering
        \includegraphics[width=\linewidth]{../data/knots/10_116.png}
        \subcaption{$10_{116}$}
    \end{minipage}
\end{figure}
\begin{figure}[H]
    \begin{minipage}[b]{.18\linewidth}
        \centering
        \includegraphics[width=\linewidth]{../data/knots/10_117.png}
        \subcaption{$10_{117}$}
    \end{minipage}
    \begin{minipage}[b]{.18\linewidth}
        \centering
        \includegraphics[width=\linewidth]{../data/knots/10_118.png}
        \subcaption{$10_{118}$}
    \end{minipage}
    \begin{minipage}[b]{.18\linewidth}
        \centering
        \includegraphics[width=\linewidth]{../data/knots/10_119.png}
        \subcaption{$10_{119}$}
    \end{minipage}
    \begin{minipage}[b]{.18\linewidth}
        \centering
        \includegraphics[width=\linewidth]{../data/knots/10_120.png}
        \subcaption{$10_{120}$}
    \end{minipage}
    \begin{minipage}[b]{.18\linewidth}
        \centering
        \includegraphics[width=\linewidth]{../data/knots/10_121.png}
        \subcaption{$10_{121}$}
    \end{minipage}
\end{figure}
\begin{figure}[H]
    \begin{minipage}[b]{.18\linewidth}
        \centering
        \includegraphics[width=\linewidth]{../data/knots/10_122.png}
        \subcaption{$10_{122}$}
    \end{minipage}
    \begin{minipage}[b]{.18\linewidth}
        \centering
        \includegraphics[width=\linewidth]{../data/knots/10_123.png}
        \subcaption{$10_{123}$}
    \end{minipage}
    \begin{minipage}[b]{.18\linewidth}
        \centering
        \includegraphics[width=\linewidth]{../data/knots/10_124.png}
        \subcaption{$10_{124}$}
    \end{minipage}
    \begin{minipage}[b]{.18\linewidth}
        \centering
        \includegraphics[width=\linewidth]{../data/knots/10_125.png}
        \subcaption{$10_{125}$}
    \end{minipage}
    \begin{minipage}[b]{.18\linewidth}
        \centering
        \includegraphics[width=\linewidth]{../data/knots/10_126.png}
        \subcaption{$10_{126}$}
    \end{minipage}
\end{figure}
\begin{figure}[H]
    \begin{minipage}[b]{.18\linewidth}
        \centering
        \includegraphics[width=\linewidth]{../data/knots/10_127.png}
        \subcaption{$10_{127}$}
    \end{minipage}
    \begin{minipage}[b]{.18\linewidth}
        \centering
        \includegraphics[width=\linewidth]{../data/knots/10_128.png}
        \subcaption{$10_{128}$}
    \end{minipage}
    \begin{minipage}[b]{.18\linewidth}
        \centering
        \includegraphics[width=\linewidth]{../data/knots/10_129.png}
        \subcaption{$10_{129}$}
    \end{minipage}
    \begin{minipage}[b]{.18\linewidth}
        \centering
        \includegraphics[width=\linewidth]{../data/knots/10_130.png}
        \subcaption{$10_{130}$}
    \end{minipage}
    \begin{minipage}[b]{.18\linewidth}
        \centering
        \includegraphics[width=\linewidth]{../data/knots/10_131.png}
        \subcaption{$10_{131}$}
    \end{minipage}
\end{figure}
\begin{figure}[H]
    \begin{minipage}[b]{.18\linewidth}
        \centering
        \includegraphics[width=\linewidth]{../data/knots/10_132.png}
        \subcaption{$10_{132}$}
    \end{minipage}
    \begin{minipage}[b]{.18\linewidth}
        \centering
        \includegraphics[width=\linewidth]{../data/knots/10_133.png}
        \subcaption{$10_{133}$}
    \end{minipage}
    \begin{minipage}[b]{.18\linewidth}
        \centering
        \includegraphics[width=\linewidth]{../data/knots/10_134.png}
        \subcaption{$10_{134}$}
    \end{minipage}
    \begin{minipage}[b]{.18\linewidth}
        \centering
        \includegraphics[width=\linewidth]{../data/knots/10_135.png}
        \subcaption{$10_{135}$}
    \end{minipage}
    \begin{minipage}[b]{.18\linewidth}
        \centering
        \includegraphics[width=\linewidth]{../data/knots/10_136.png}
        \subcaption{$10_{136}$}
    \end{minipage}
\end{figure}
\begin{figure}[H]
    \begin{minipage}[b]{.18\linewidth}
        \centering
        \includegraphics[width=\linewidth]{../data/knots/10_137.png}
        \subcaption{$10_{137}$}
    \end{minipage}
    \begin{minipage}[b]{.18\linewidth}
        \centering
        \includegraphics[width=\linewidth]{../data/knots/10_138.png}
        \subcaption{$10_{138}$}
    \end{minipage}
    \begin{minipage}[b]{.18\linewidth}
        \centering
        \includegraphics[width=\linewidth]{../data/knots/10_139.png}
        \subcaption{$10_{139}$}
    \end{minipage}
    \begin{minipage}[b]{.18\linewidth}
        \centering
        \includegraphics[width=\linewidth]{../data/knots/10_140.png}
        \subcaption{$10_{140}$}
    \end{minipage}
    \begin{minipage}[b]{.18\linewidth}
        \centering
        \includegraphics[width=\linewidth]{../data/knots/10_141.png}
        \subcaption{$10_{141}$}
    \end{minipage}
\end{figure}
\begin{figure}[H]
    \begin{minipage}[b]{.18\linewidth}
        \centering
        \includegraphics[width=\linewidth]{../data/knots/10_142.png}
        \subcaption{$10_{142}$}
    \end{minipage}
    \begin{minipage}[b]{.18\linewidth}
        \centering
        \includegraphics[width=\linewidth]{../data/knots/10_143.png}
        \subcaption{$10_{143}$}
    \end{minipage}
    \begin{minipage}[b]{.18\linewidth}
        \centering
        \includegraphics[width=\linewidth]{../data/knots/10_144.png}
        \subcaption{$10_{144}$}
    \end{minipage}
    \begin{minipage}[b]{.18\linewidth}
        \centering
        \includegraphics[width=\linewidth]{../data/knots/10_145.png}
        \subcaption{$10_{145}$}
    \end{minipage}
    \begin{minipage}[b]{.18\linewidth}
        \centering
        \includegraphics[width=\linewidth]{../data/knots/10_146.png}
        \subcaption{$10_{146}$}
    \end{minipage}
\end{figure}
\begin{figure}[H]
    \begin{minipage}[b]{.18\linewidth}
        \centering
        \includegraphics[width=\linewidth]{../data/knots/10_147.png}
        \subcaption{$10_{147}$}
    \end{minipage}
    \begin{minipage}[b]{.18\linewidth}
        \centering
        \includegraphics[width=\linewidth]{../data/knots/10_148.png}
        \subcaption{$10_{148}$}
    \end{minipage}
    \begin{minipage}[b]{.18\linewidth}
        \centering
        \includegraphics[width=\linewidth]{../data/knots/10_149.png}
        \subcaption{$10_{149}$}
    \end{minipage}
    \begin{minipage}[b]{.18\linewidth}
        \centering
        \includegraphics[width=\linewidth]{../data/knots/10_150.png}
        \subcaption{$10_{150}$}
    \end{minipage}
    \begin{minipage}[b]{.18\linewidth}
        \centering
        \includegraphics[width=\linewidth]{../data/knots/10_151.png}
        \subcaption{$10_{151}$}
    \end{minipage}
\end{figure}
\begin{figure}[H]
    \begin{minipage}[b]{.18\linewidth}
        \centering
        \includegraphics[width=\linewidth]{../data/knots/10_152.png}
        \subcaption{$10_{152}$}
    \end{minipage}
    \begin{minipage}[b]{.18\linewidth}
        \centering
        \includegraphics[width=\linewidth]{../data/knots/10_153.png}
        \subcaption{$10_{153}$}
    \end{minipage}
    \begin{minipage}[b]{.18\linewidth}
        \centering
        \includegraphics[width=\linewidth]{../data/knots/10_154.png}
        \subcaption{$10_{154}$}
    \end{minipage}
    \begin{minipage}[b]{.18\linewidth}
        \centering
        \includegraphics[width=\linewidth]{../data/knots/10_155.png}
        \subcaption{$10_{155}$}
    \end{minipage}
    \begin{minipage}[b]{.18\linewidth}
        \centering
        \includegraphics[width=\linewidth]{../data/knots/10_156.png}
        \subcaption{$10_{156}$}
    \end{minipage}
\end{figure}
\begin{figure}[H]
    \begin{minipage}[b]{.18\linewidth}
        \centering
        \includegraphics[width=\linewidth]{../data/knots/10_157.png}
        \subcaption{$10_{157}$}
    \end{minipage}
    \begin{minipage}[b]{.18\linewidth}
        \centering
        \includegraphics[width=\linewidth]{../data/knots/10_158.png}
        \subcaption{$10_{158}$}
    \end{minipage}
    \begin{minipage}[b]{.18\linewidth}
        \centering
        \includegraphics[width=\linewidth]{../data/knots/10_159.png}
        \subcaption{$10_{159}$}
    \end{minipage}
    \begin{minipage}[b]{.18\linewidth}
        \centering
        \includegraphics[width=\linewidth]{../data/knots/10_160.png}
        \subcaption{$10_{160}$}
    \end{minipage}
    \begin{minipage}[b]{.18\linewidth}
        \centering
        \includegraphics[width=\linewidth]{../data/knots/10_161.png}
        \subcaption{$10_{161}$}
    \end{minipage}
\end{figure}
\begin{figure}[H]
    \begin{minipage}[b]{.18\linewidth}
        \centering
        \includegraphics[width=\linewidth]{../data/knots/10_162.png}
        \subcaption{$10_{162}$}
    \end{minipage}
    \begin{minipage}[b]{.18\linewidth}
        \centering
        \includegraphics[width=\linewidth]{../data/knots/10_163.png}
        \subcaption{$10_{163}$}
    \end{minipage}
    \begin{minipage}[b]{.18\linewidth}
        \centering
        \includegraphics[width=\linewidth]{../data/knots/10_164.png}
        \subcaption{$10_{164}$}
    \end{minipage}
    \begin{minipage}[b]{.18\linewidth}
        \centering
        \includegraphics[width=\linewidth]{../data/knots/10_165.png}
        \subcaption{$10_{165}$}
    \end{minipage}
\end{figure}
\end{comment}



    \chapter{Tablice splotów}
    
\section{Diagramy splotów pierwszych do dziewięciu skrzyżowań}
% Poniżej znajdują się diagramy węzłów pierwszych, które realizują liczbę gordyjską, jeśli ta nie przekracza dziesięciu.
% One także pochodzą ze strony KnotInfo \cite{knotinfo2024}, o~której mowa na początku rozdziału.
Coś o LinkInfo...
% TODO

\begin{comment}
\begin{figure}[H]
    \begin{minipage}[b]{.18\linewidth}
        \centering
        \includegraphics[width=\linewidth]{../data/links/2_2_1.png}
        \subcaption{$2^2_{1} = L_2a_{1}$}
    \end{minipage}
    \begin{minipage}[b]{.18\linewidth}
        \centering
        \includegraphics[width=\linewidth]{../data/links/4_2_1.png}
        \subcaption{$4^2_{1} = L_4a_{1}$}
    \end{minipage}
    \begin{minipage}[b]{.18\linewidth}
        \centering
        \includegraphics[width=\linewidth]{../data/links/5_2_1.png}
        \subcaption{$5^2_{1} = L_5a_{1}$}
    \end{minipage}
    \begin{minipage}[b]{.18\linewidth}
        \centering
        \includegraphics[width=\linewidth]{../data/links/6_2_1.png}
        \subcaption{$6^2_{1} = L_6a_{3}$}
    \end{minipage}
    \begin{minipage}[b]{.18\linewidth}
        \centering
        \includegraphics[width=\linewidth]{../data/links/6_2_2.png}
        \subcaption{$6^2_{2} = L_6a_{2}$}
    \end{minipage}
\end{figure}
\begin{figure}[H]
    \begin{minipage}[b]{.18\linewidth}
        \centering
        \includegraphics[width=\linewidth]{../data/links/6_2_3.png}
        \subcaption{$6^2_{3} = L_6a_{1}$}
    \end{minipage}
    \begin{minipage}[b]{.18\linewidth}
        \centering
        \includegraphics[width=\linewidth]{../data/links/6_3_1.png}
        \subcaption{$6^3_{1} = L_6a_{5}$}
    \end{minipage}
    \begin{minipage}[b]{.18\linewidth}
        \centering
        \includegraphics[width=\linewidth]{../data/links/6_3_2.png}
        \subcaption{$6^3_{2} = L_6a_{4}$}
    \end{minipage}
    \begin{minipage}[b]{.18\linewidth}
        \centering
        \includegraphics[width=\linewidth]{../data/links/6_3_3.png}
        \subcaption{$6^3_{3} = L_6n_{1}$}
    \end{minipage}
    \begin{minipage}[b]{.18\linewidth}
        \centering
        \includegraphics[width=\linewidth]{../data/links/7_2_1.png}
        \subcaption{$7^2_{1} = L_7a_{6}$}
    \end{minipage}
\end{figure}
\begin{figure}[H]
    \begin{minipage}[b]{.18\linewidth}
        \centering
        \includegraphics[width=\linewidth]{../data/links/7_2_2.png}
        \subcaption{$7^2_{2} = L_7a_{5}$}
    \end{minipage}
    \begin{minipage}[b]{.18\linewidth}
        \centering
        \includegraphics[width=\linewidth]{../data/links/7_2_3.png}
        \subcaption{$7^2_{3} = L_7a_{4}$}
    \end{minipage}
    \begin{minipage}[b]{.18\linewidth}
        \centering
        \includegraphics[width=\linewidth]{../data/links/7_2_4.png}
        \subcaption{$7^2_{4} = L_7a_{3}$}
    \end{minipage}
    \begin{minipage}[b]{.18\linewidth}
        \centering
        \includegraphics[width=\linewidth]{../data/links/7_2_5.png}
        \subcaption{$7^2_{5} = L_7a_{2}$}
    \end{minipage}
    \begin{minipage}[b]{.18\linewidth}
        \centering
        \includegraphics[width=\linewidth]{../data/links/7_2_6.png}
        \subcaption{$7^2_{6} = L_7a_{1}$}
    \end{minipage}
\end{figure}
\begin{figure}[H]
    \begin{minipage}[b]{.18\linewidth}
        \centering
        \includegraphics[width=\linewidth]{../data/links/7_2_7.png}
        \subcaption{$7^2_{7} = L_7n_{1}$}
    \end{minipage}
    \begin{minipage}[b]{.18\linewidth}
        \centering
        \includegraphics[width=\linewidth]{../data/links/7_2_8.png}
        \subcaption{$7^2_{8} = L_7n_{2}$}
    \end{minipage}
    \begin{minipage}[b]{.18\linewidth}
        \centering
        \includegraphics[width=\linewidth]{../data/links/7_3_1.png}
        \subcaption{$7^3_{1} = L_7a_{7}$}
    \end{minipage}
    \begin{minipage}[b]{.18\linewidth}
        \centering
        \includegraphics[width=\linewidth]{../data/links/8_2_1.png}
        \subcaption{$8^2_{1} = L_8a_{14}$}
    \end{minipage}
    \begin{minipage}[b]{.18\linewidth}
        \centering
        \includegraphics[width=\linewidth]{../data/links/8_2_2.png}
        \subcaption{$8^2_{2} = L_8a_{12}$}
    \end{minipage}
\end{figure}
\begin{figure}[H]
    \begin{minipage}[b]{.18\linewidth}
        \centering
        \includegraphics[width=\linewidth]{../data/links/8_2_3.png}
        \subcaption{$8^2_{3} = L_8a_{11}$}
    \end{minipage}
    \begin{minipage}[b]{.18\linewidth}
        \centering
        \includegraphics[width=\linewidth]{../data/links/8_2_4.png}
        \subcaption{$8^2_{4} = L_8a_{13}$}
    \end{minipage}
    \begin{minipage}[b]{.18\linewidth}
        \centering
        \includegraphics[width=\linewidth]{../data/links/8_2_5.png}
        \subcaption{$8^2_{5} = L_8a_{10}$}
    \end{minipage}
    \begin{minipage}[b]{.18\linewidth}
        \centering
        \includegraphics[width=\linewidth]{../data/links/8_2_6.png}
        \subcaption{$8^2_{6} = L_8a_{6}$}
    \end{minipage}
    \begin{minipage}[b]{.18\linewidth}
        \centering
        \includegraphics[width=\linewidth]{../data/links/8_2_7.png}
        \subcaption{$8^2_{7} = L_8a_{8}$}
    \end{minipage}
\end{figure}
\begin{figure}[H]
    \begin{minipage}[b]{.18\linewidth}
        \centering
        \includegraphics[width=\linewidth]{../data/links/8_2_8.png}
        \subcaption{$8^2_{8} = L_8a_{9}$}
    \end{minipage}
    \begin{minipage}[b]{.18\linewidth}
        \centering
        \includegraphics[width=\linewidth]{../data/links/8_2_9.png}
        \subcaption{$8^2_{9} = L_8a_{3}$}
    \end{minipage}
    \begin{minipage}[b]{.18\linewidth}
        \centering
        \includegraphics[width=\linewidth]{../data/links/8_2_10.png}
        \subcaption{$8^2_{10} = L_8a_{2}$}
    \end{minipage}
    \begin{minipage}[b]{.18\linewidth}
        \centering
        \includegraphics[width=\linewidth]{../data/links/8_2_11.png}
        \subcaption{$8^2_{11} = L_8a_{5}$}
    \end{minipage}
    \begin{minipage}[b]{.18\linewidth}
        \centering
        \includegraphics[width=\linewidth]{../data/links/8_2_12.png}
        \subcaption{$8^2_{12} = L_8a_{4}$}
    \end{minipage}
\end{figure}
\begin{figure}[H]
    \begin{minipage}[b]{.18\linewidth}
        \centering
        \includegraphics[width=\linewidth]{../data/links/8_2_13.png}
        \subcaption{$8^2_{13} = L_8a_{1}$}
    \end{minipage}
    \begin{minipage}[b]{.18\linewidth}
        \centering
        \includegraphics[width=\linewidth]{../data/links/8_2_14.png}
        \subcaption{$8^2_{14} = L_8a_{7}$}
    \end{minipage}
    \begin{minipage}[b]{.18\linewidth}
        \centering
        \includegraphics[width=\linewidth]{../data/links/8_2_15.png}
        \subcaption{$8^2_{15} = L_8n_{2}$}
    \end{minipage}
    \begin{minipage}[b]{.18\linewidth}
        \centering
        \includegraphics[width=\linewidth]{../data/links/8_2_16.png}
        \subcaption{$8^2_{16} = L_8n_{1}$}
    \end{minipage}
    \begin{minipage}[b]{.18\linewidth}
        \centering
        \includegraphics[width=\linewidth]{../data/links/8_3_1.png}
        \subcaption{$8^3_{1} = L_8a_{18}$}
    \end{minipage}
\end{figure}
\begin{figure}[H]
    \begin{minipage}[b]{.18\linewidth}
        \centering
        \includegraphics[width=\linewidth]{../data/links/8_3_2.png}
        \subcaption{$8^3_{2} = L_8a_{17}$}
    \end{minipage}
    \begin{minipage}[b]{.18\linewidth}
        \centering
        \includegraphics[width=\linewidth]{../data/links/8_3_3.png}
        \subcaption{$8^3_{3} = L_8a_{15}$}
    \end{minipage}
    \begin{minipage}[b]{.18\linewidth}
        \centering
        \includegraphics[width=\linewidth]{../data/links/8_3_4.png}
        \subcaption{$8^3_{4} = L_8a_{20}$}
    \end{minipage}
    \begin{minipage}[b]{.18\linewidth}
        \centering
        \includegraphics[width=\linewidth]{../data/links/8_3_5.png}
        \subcaption{$8^3_{5} = L_8a_{16}$}
    \end{minipage}
    \begin{minipage}[b]{.18\linewidth}
        \centering
        \includegraphics[width=\linewidth]{../data/links/8_3_6.png}
        \subcaption{$8^3_{6} = L_8a_{19}$}
    \end{minipage}
\end{figure}
\begin{figure}[H]
    \begin{minipage}[b]{.18\linewidth}
        \centering
        \includegraphics[width=\linewidth]{../data/links/8_3_7.png}
        \subcaption{$8^3_{7} = L_8n_{3}$}
    \end{minipage}
    \begin{minipage}[b]{.18\linewidth}
        \centering
        \includegraphics[width=\linewidth]{../data/links/8_3_8.png}
        \subcaption{$8^3_{8} = L_8n_{4}$}
    \end{minipage}
    \begin{minipage}[b]{.18\linewidth}
        \centering
        \includegraphics[width=\linewidth]{../data/links/8_3_9.png}
        \subcaption{$8^3_{9} = L_8n_{5}$}
    \end{minipage}
    \begin{minipage}[b]{.18\linewidth}
        \centering
        \includegraphics[width=\linewidth]{../data/links/8_3_10.png}
        \subcaption{$8^3_{10} = L_8n_{6}$}
    \end{minipage}
    \begin{minipage}[b]{.18\linewidth}
        \centering
        \includegraphics[width=\linewidth]{../data/links/8_4_1.png}
        \subcaption{$8^4_{1} = L_8a_{21}$}
    \end{minipage}
\end{figure}
\begin{figure}[H]
    \begin{minipage}[b]{.18\linewidth}
        \centering
        \includegraphics[width=\linewidth]{../data/links/8_4_2.png}
        \subcaption{$8^4_{2} = L_8n_{7}$}
    \end{minipage}
    \begin{minipage}[b]{.18\linewidth}
        \centering
        \includegraphics[width=\linewidth]{../data/links/8_4_3.png}
        \subcaption{$8^4_{3} = L_8n_{8}$}
    \end{minipage}
    \begin{minipage}[b]{.18\linewidth}
        \centering
        \includegraphics[width=\linewidth]{../data/links/9_2_1.png}
        \subcaption{$9^2_{1} = L_9a_{36}$}
    \end{minipage}
    \begin{minipage}[b]{.18\linewidth}
        \centering
        \includegraphics[width=\linewidth]{../data/links/9_2_2.png}
        \subcaption{$9^2_{2} = L_9a_{39}$}
    \end{minipage}
    \begin{minipage}[b]{.18\linewidth}
        \centering
        \includegraphics[width=\linewidth]{../data/links/9_2_3.png}
        \subcaption{$9^2_{3} = L_9a_{30}$}
    \end{minipage}
\end{figure}
\begin{figure}[H]
    \begin{minipage}[b]{.18\linewidth}
        \centering
        \includegraphics[width=\linewidth]{../data/links/9_2_4.png}
        \subcaption{$9^2_{4} = L_9a_{40}$}
    \end{minipage}
    \begin{minipage}[b]{.18\linewidth}
        \centering
        \includegraphics[width=\linewidth]{../data/links/9_2_5.png}
        \subcaption{$9^2_{5} = L_9a_{38}$}
    \end{minipage}
    \begin{minipage}[b]{.18\linewidth}
        \centering
        \includegraphics[width=\linewidth]{../data/links/9_2_6.png}
        \subcaption{$9^2_{6} = L_9a_{34}$}
    \end{minipage}
    \begin{minipage}[b]{.18\linewidth}
        \centering
        \includegraphics[width=\linewidth]{../data/links/9_2_7.png}
        \subcaption{$9^2_{7} = L_9a_{37}$}
    \end{minipage}
    \begin{minipage}[b]{.18\linewidth}
        \centering
        \includegraphics[width=\linewidth]{../data/links/9_2_8.png}
        \subcaption{$9^2_{8} = L_9a_{25}$}
    \end{minipage}
\end{figure}
\begin{figure}[H]
    \begin{minipage}[b]{.18\linewidth}
        \centering
        \includegraphics[width=\linewidth]{../data/links/9_2_9.png}
        \subcaption{$9^2_{9} = L_9a_{35}$}
    \end{minipage}
    \begin{minipage}[b]{.18\linewidth}
        \centering
        \includegraphics[width=\linewidth]{../data/links/9_2_10.png}
        \subcaption{$9^2_{10} = L_9a_{18}$}
    \end{minipage}
    \begin{minipage}[b]{.18\linewidth}
        \centering
        \includegraphics[width=\linewidth]{../data/links/9_2_11.png}
        \subcaption{$9^2_{11} = L_9a_{26}$}
    \end{minipage}
    \begin{minipage}[b]{.18\linewidth}
        \centering
        \includegraphics[width=\linewidth]{../data/links/9_2_12.png}
        \subcaption{$9^2_{12} = L_9a_{27}$}
    \end{minipage}
    \begin{minipage}[b]{.18\linewidth}
        \centering
        \includegraphics[width=\linewidth]{../data/links/9_2_13.png}
        \subcaption{$9^2_{13} = L_9a_{14}$}
    \end{minipage}
\end{figure}
\begin{figure}[H]
    \begin{minipage}[b]{.18\linewidth}
        \centering
        \includegraphics[width=\linewidth]{../data/links/9_2_14.png}
        \subcaption{$9^2_{14} = L_9a_{12}$}
    \end{minipage}
    \begin{minipage}[b]{.18\linewidth}
        \centering
        \includegraphics[width=\linewidth]{../data/links/9_2_15.png}
        \subcaption{$9^2_{15} = L_9a_{15}$}
    \end{minipage}
    \begin{minipage}[b]{.18\linewidth}
        \centering
        \includegraphics[width=\linewidth]{../data/links/9_2_16.png}
        \subcaption{$9^2_{16} = L_9a_{13}$}
    \end{minipage}
    \begin{minipage}[b]{.18\linewidth}
        \centering
        \includegraphics[width=\linewidth]{../data/links/9_2_17.png}
        \subcaption{$9^2_{17} = L_9a_{7}$}
    \end{minipage}
    \begin{minipage}[b]{.18\linewidth}
        \centering
        \includegraphics[width=\linewidth]{../data/links/9_2_18.png}
        \subcaption{$9^2_{18} = L_9a_{4}$}
    \end{minipage}
\end{figure}
\begin{figure}[H]
    \begin{minipage}[b]{.18\linewidth}
        \centering
        \includegraphics[width=\linewidth]{../data/links/9_2_19.png}
        \subcaption{$9^2_{19} = L_9a_{29}$}
    \end{minipage}
    \begin{minipage}[b]{.18\linewidth}
        \centering
        \includegraphics[width=\linewidth]{../data/links/9_2_20.png}
        \subcaption{$9^2_{20} = L_9a_{28}$}
    \end{minipage}
    \begin{minipage}[b]{.18\linewidth}
        \centering
        \includegraphics[width=\linewidth]{../data/links/9_2_21.png}
        \subcaption{$9^2_{21} = L_9a_{24}$}
    \end{minipage}
    \begin{minipage}[b]{.18\linewidth}
        \centering
        \includegraphics[width=\linewidth]{../data/links/9_2_22.png}
        \subcaption{$9^2_{22} = L_9a_{23}$}
    \end{minipage}
    \begin{minipage}[b]{.18\linewidth}
        \centering
        \includegraphics[width=\linewidth]{../data/links/9_2_23.png}
        \subcaption{$9^2_{23} = L_9a_{41}$}
    \end{minipage}
\end{figure}
\begin{figure}[H]
    \begin{minipage}[b]{.18\linewidth}
        \centering
        \includegraphics[width=\linewidth]{../data/links/9_2_24.png}
        \subcaption{$9^2_{24} = L_9a_{33}$}
    \end{minipage}
    \begin{minipage}[b]{.18\linewidth}
        \centering
        \includegraphics[width=\linewidth]{../data/links/9_2_25.png}
        \subcaption{$9^2_{25} = L_9a_{8}$}
    \end{minipage}
    \begin{minipage}[b]{.18\linewidth}
        \centering
        \includegraphics[width=\linewidth]{../data/links/9_2_26.png}
        \subcaption{$9^2_{26} = L_9a_{11}$}
    \end{minipage}
    \begin{minipage}[b]{.18\linewidth}
        \centering
        \includegraphics[width=\linewidth]{../data/links/9_2_27.png}
        \subcaption{$9^2_{27} = L_9a_{17}$}
    \end{minipage}
    \begin{minipage}[b]{.18\linewidth}
        \centering
        \includegraphics[width=\linewidth]{../data/links/9_2_28.png}
        \subcaption{$9^2_{28} = L_9a_{16}$}
    \end{minipage}
\end{figure}
\begin{figure}[H]
    \begin{minipage}[b]{.18\linewidth}
        \centering
        \includegraphics[width=\linewidth]{../data/links/9_2_29.png}
        \subcaption{$9^2_{29} = L_9a_{6}$}
    \end{minipage}
    \begin{minipage}[b]{.18\linewidth}
        \centering
        \includegraphics[width=\linewidth]{../data/links/9_2_30.png}
        \subcaption{$9^2_{30} = L_9a_{5}$}
    \end{minipage}
    \begin{minipage}[b]{.18\linewidth}
        \centering
        \includegraphics[width=\linewidth]{../data/links/9_2_31.png}
        \subcaption{$9^2_{31} = L_9a_{2}$}
    \end{minipage}
    \begin{minipage}[b]{.18\linewidth}
        \centering
        \includegraphics[width=\linewidth]{../data/links/9_2_32.png}
        \subcaption{$9^2_{32} = L_9a_{1}$}
    \end{minipage}
    \begin{minipage}[b]{.18\linewidth}
        \centering
        \includegraphics[width=\linewidth]{../data/links/9_2_33.png}
        \subcaption{$9^2_{33} = L_9a_{3}$}
    \end{minipage}
\end{figure}
\begin{figure}[H]
    \begin{minipage}[b]{.18\linewidth}
        \centering
        \includegraphics[width=\linewidth]{../data/links/9_2_34.png}
        \subcaption{$9^2_{34} = L_9a_{21}$}
    \end{minipage}
    \begin{minipage}[b]{.18\linewidth}
        \centering
        \includegraphics[width=\linewidth]{../data/links/9_2_35.png}
        \subcaption{$9^2_{35} = L_9a_{22}$}
    \end{minipage}
    \begin{minipage}[b]{.18\linewidth}
        \centering
        \includegraphics[width=\linewidth]{../data/links/9_2_36.png}
        \subcaption{$9^2_{36} = L_9a_{10}$}
    \end{minipage}
    \begin{minipage}[b]{.18\linewidth}
        \centering
        \includegraphics[width=\linewidth]{../data/links/9_2_37.png}
        \subcaption{$9^2_{37} = L_9a_{9}$}
    \end{minipage}
    \begin{minipage}[b]{.18\linewidth}
        \centering
        \includegraphics[width=\linewidth]{../data/links/9_2_38.png}
        \subcaption{$9^2_{38} = L_9a_{19}$}
    \end{minipage}
\end{figure}
\begin{figure}[H]
    \begin{minipage}[b]{.18\linewidth}
        \centering
        \includegraphics[width=\linewidth]{../data/links/9_2_39.png}
        \subcaption{$9^2_{39} = L_9a_{31}$}
    \end{minipage}
    \begin{minipage}[b]{.18\linewidth}
        \centering
        \includegraphics[width=\linewidth]{../data/links/9_2_40.png}
        \subcaption{$9^2_{40} = L_9a_{32}$}
    \end{minipage}
    \begin{minipage}[b]{.18\linewidth}
        \centering
        \includegraphics[width=\linewidth]{../data/links/9_2_41.png}
        \subcaption{$9^2_{41} = L_9a_{42}$}
    \end{minipage}
    \begin{minipage}[b]{.18\linewidth}
        \centering
        \includegraphics[width=\linewidth]{../data/links/9_2_42.png}
        \subcaption{$9^2_{42} = L_9a_{20}$}
    \end{minipage}
    \begin{minipage}[b]{.18\linewidth}
        \centering
        \includegraphics[width=\linewidth]{../data/links/9_2_43.png}
        \subcaption{$9^2_{43} = L_9n_{4}$}
    \end{minipage}
\end{figure}
\begin{figure}[H]
    \begin{minipage}[b]{.18\linewidth}
        \centering
        \includegraphics[width=\linewidth]{../data/links/9_2_44.png}
        \subcaption{$9^2_{44} = L_9n_{5}$}
    \end{minipage}
    \begin{minipage}[b]{.18\linewidth}
        \centering
        \includegraphics[width=\linewidth]{../data/links/9_2_45.png}
        \subcaption{$9^2_{45} = L_9n_{1}$}
    \end{minipage}
    \begin{minipage}[b]{.18\linewidth}
        \centering
        \includegraphics[width=\linewidth]{../data/links/9_2_46.png}
        \subcaption{$9^2_{46} = L_9n_{2}$}
    \end{minipage}
    \begin{minipage}[b]{.18\linewidth}
        \centering
        \includegraphics[width=\linewidth]{../data/links/9_2_47.png}
        \subcaption{$9^2_{47} = L_9n_{3}$}
    \end{minipage}
    \begin{minipage}[b]{.18\linewidth}
        \centering
        \includegraphics[width=\linewidth]{../data/links/9_2_48.png}
        \subcaption{$9^2_{48} = L_9n_{7}$}
    \end{minipage}
\end{figure}
\begin{figure}[H]
    \begin{minipage}[b]{.18\linewidth}
        \centering
        \includegraphics[width=\linewidth]{../data/links/9_2_49.png}
        \subcaption{$9^2_{49} = L_9n_{15}$}
    \end{minipage}
    \begin{minipage}[b]{.18\linewidth}
        \centering
        \includegraphics[width=\linewidth]{../data/links/9_2_50.png}
        \subcaption{$9^2_{50} = L_9n_{14}$}
    \end{minipage}
    \begin{minipage}[b]{.18\linewidth}
        \centering
        \includegraphics[width=\linewidth]{../data/links/9_2_51.png}
        \subcaption{$9^2_{51} = L_9n_{16}$}
    \end{minipage}
    \begin{minipage}[b]{.18\linewidth}
        \centering
        \includegraphics[width=\linewidth]{../data/links/9_2_52.png}
        \subcaption{$9^2_{52} = L_9n_{17}$}
    \end{minipage}
    \begin{minipage}[b]{.18\linewidth}
        \centering
        \includegraphics[width=\linewidth]{../data/links/9_2_53.png}
        \subcaption{$9^2_{53} = L_9n_{18}$}
    \end{minipage}
\end{figure}
\begin{figure}[H]
    \begin{minipage}[b]{.18\linewidth}
        \centering
        \includegraphics[width=\linewidth]{../data/links/9_2_54.png}
        \subcaption{$9^2_{54} = L_9n_{13}$}
    \end{minipage}
    \begin{minipage}[b]{.18\linewidth}
        \centering
        \includegraphics[width=\linewidth]{../data/links/9_2_55.png}
        \subcaption{$9^2_{55} = L_9n_{6}$}
    \end{minipage}
    \begin{minipage}[b]{.18\linewidth}
        \centering
        \includegraphics[width=\linewidth]{../data/links/9_2_56.png}
        \subcaption{$9^2_{56} = L_9n_{8}$}
    \end{minipage}
    \begin{minipage}[b]{.18\linewidth}
        \centering
        \includegraphics[width=\linewidth]{../data/links/9_2_57.png}
        \subcaption{$9^2_{57} = L_9n_{11}$}
    \end{minipage}
    \begin{minipage}[b]{.18\linewidth}
        \centering
        \includegraphics[width=\linewidth]{../data/links/9_2_58.png}
        \subcaption{$9^2_{58} = L_9n_{10}$}
    \end{minipage}
\end{figure}
\begin{figure}[H]
    \begin{minipage}[b]{.18\linewidth}
        \centering
        \includegraphics[width=\linewidth]{../data/links/9_2_59.png}
        \subcaption{$9^2_{59} = L_9n_{12}$}
    \end{minipage}
    \begin{minipage}[b]{.18\linewidth}
        \centering
        \includegraphics[width=\linewidth]{../data/links/9_2_60.png}
        \subcaption{$9^2_{60} = L_9n_{9}$}
    \end{minipage}
    \begin{minipage}[b]{.18\linewidth}
        \centering
        \includegraphics[width=\linewidth]{../data/links/9_2_61.png}
        \subcaption{$9^2_{61} = L_9n_{19}$}
    \end{minipage}
    \begin{minipage}[b]{.18\linewidth}
        \centering
        \includegraphics[width=\linewidth]{../data/links/9_3_1.png}
        \subcaption{$9^3_{1} = L_9a_{50}$}
    \end{minipage}
    \begin{minipage}[b]{.18\linewidth}
        \centering
        \includegraphics[width=\linewidth]{../data/links/9_3_2.png}
        \subcaption{$9^3_{2} = L_9a_{47}$}
    \end{minipage}
\end{figure}
\begin{figure}[H]
    \begin{minipage}[b]{.18\linewidth}
        \centering
        \includegraphics[width=\linewidth]{../data/links/9_3_3.png}
        \subcaption{$9^3_{3} = L_9a_{44}$}
    \end{minipage}
    \begin{minipage}[b]{.18\linewidth}
        \centering
        \includegraphics[width=\linewidth]{../data/links/9_3_4.png}
        \subcaption{$9^3_{4} = L_9a_{43}$}
    \end{minipage}
    \begin{minipage}[b]{.18\linewidth}
        \centering
        \includegraphics[width=\linewidth]{../data/links/9_3_5.png}
        \subcaption{$9^3_{5} = L_9a_{48}$}
    \end{minipage}
    \begin{minipage}[b]{.18\linewidth}
        \centering
        \includegraphics[width=\linewidth]{../data/links/9_3_6.png}
        \subcaption{$9^3_{6} = L_9a_{49}$}
    \end{minipage}
    \begin{minipage}[b]{.18\linewidth}
        \centering
        \includegraphics[width=\linewidth]{../data/links/9_3_7.png}
        \subcaption{$9^3_{7} = L_9a_{45}$}
    \end{minipage}
\end{figure}
\begin{figure}[H]
    \begin{minipage}[b]{.18\linewidth}
        \centering
        \includegraphics[width=\linewidth]{../data/links/9_3_8.png}
        \subcaption{$9^3_{8} = L_9a_{52}$}
    \end{minipage}
    \begin{minipage}[b]{.18\linewidth}
        \centering
        \includegraphics[width=\linewidth]{../data/links/9_3_9.png}
        \subcaption{$9^3_{9} = L_9a_{54}$}
    \end{minipage}
    \begin{minipage}[b]{.18\linewidth}
        \centering
        \includegraphics[width=\linewidth]{../data/links/9_3_10.png}
        \subcaption{$9^3_{10} = L_9a_{46}$}
    \end{minipage}
    \begin{minipage}[b]{.18\linewidth}
        \centering
        \includegraphics[width=\linewidth]{../data/links/9_3_11.png}
        \subcaption{$9^3_{11} = L_9a_{51}$}
    \end{minipage}
    \begin{minipage}[b]{.18\linewidth}
        \centering
        \includegraphics[width=\linewidth]{../data/links/9_3_12.png}
        \subcaption{$9^3_{12} = L_9a_{53}$}
    \end{minipage}
\end{figure}
\begin{figure}[H]
    \begin{minipage}[b]{.18\linewidth}
        \centering
        \includegraphics[width=\linewidth]{../data/links/9_3_13.png}
        \subcaption{$9^3_{13} = L_9n_{23}$}
    \end{minipage}
    \begin{minipage}[b]{.18\linewidth}
        \centering
        \includegraphics[width=\linewidth]{../data/links/9_3_14.png}
        \subcaption{$9^3_{14} = L_9n_{24}$}
    \end{minipage}
    \begin{minipage}[b]{.18\linewidth}
        \centering
        \includegraphics[width=\linewidth]{../data/links/9_3_15.png}
        \subcaption{$9^3_{15} = L_9n_{22}$}
    \end{minipage}
    \begin{minipage}[b]{.18\linewidth}
        \centering
        \includegraphics[width=\linewidth]{../data/links/9_3_16.png}
        \subcaption{$9^3_{16} = L_9n_{20}$}
    \end{minipage}
    \begin{minipage}[b]{.18\linewidth}
        \centering
        \includegraphics[width=\linewidth]{../data/links/9_3_17.png}
        \subcaption{$9^3_{17} = L_9n_{21}$}
    \end{minipage}
\end{figure}
\begin{figure}[H]
    \begin{minipage}[b]{.18\linewidth}
        \centering
        \includegraphics[width=\linewidth]{../data/links/9_3_18.png}
        \subcaption{$9^3_{18} = L_9n_{25}$}
    \end{minipage}
    \begin{minipage}[b]{.18\linewidth}
        \centering
        \includegraphics[width=\linewidth]{../data/links/9_3_19.png}
        \subcaption{$9^3_{19} = L_9n_{26}$}
    \end{minipage}
    \begin{minipage}[b]{.18\linewidth}
        \centering
        \includegraphics[width=\linewidth]{../data/links/9_3_20.png}
        \subcaption{$9^3_{20} = L_9n_{28}$}
    \end{minipage}
    \begin{minipage}[b]{.18\linewidth}
        \centering
        \includegraphics[width=\linewidth]{../data/links/9_3_21.png}
        \subcaption{$9^3_{21} = L_9n_{27}$}
    \end{minipage}
    \begin{minipage}[b]{.18\linewidth}
        \centering
        \includegraphics[width=\linewidth]{../data/links/9_4_1.png}
        \subcaption{$9^4_{1} = L_9a_{55}$}
    \end{minipage}
\end{figure}

\end{comment}



    \chapter{Notacja, użyte symbole}
    Przez $\N, \Z, \Q, \R, \C$ oznaczamy kolejno zbiór liczb naturalnych (z zerem lub bez, jak kto lubi), całkowitych, wymiernych, rzeczywistych i wreszcie zespolonych.
$S^n$ oznacza $n$-sferę (i tak $S^1$ jest okręgiem, zaś $S^2$ powierzchnią trójwymiarowej kuli); $D^n$ to $n$-dysk.
Włożenie $A$ w $B$ to $A \hookrightarrow B$, ich izotopia to $A \cong B$.
Kratka ($\shrap$) oznacza sumę spójną.
Suma rozłączna $\sqcup$ tym różni się od teoriomnogościowej $\cup$, że składniki muszą być rozłączne. Obraz funkcji $f \colon X \to Y$ staramy się oznaczać $f[X]$, nie $f(X)$.
Lustro i rewers węzła $K$ oznaczamy kolejno $\operatorname{m} K$ i $\operatorname{r} K$.

Dla wielomianowych niezmienników stosujemy następujące symbole: $\alexander$, $\conway$, $\jones$, $P$, $F$, $Q$ dla wielomianu Alexandera, Conwaya, Jonesa, HOMFLY-PT, Kauffmana, BLM/Ho.
Wiele innych niezmienników ma krótkie, jednoliterowe oznaczenia.
I tak $\crossing, \braid, \bridge, \unknotting, \linking, \stick$ to kolejno: liczba/indeks skrzyżowaniowy, warkoczowy, mostowy, gordyjska, zaczepienia, patykowa.
$\sigma$ oznacza sygnaturę (Levine'a-Tristrama), $\det$ wyznacznik, $g, \chi$ genus i charakterystykę Eulera, $\operatorname{wr}$ spin, $\operatorname{lk}$ indeks zaczepienia, 
Rzadko, naprawdę rzadko, potrzebujemy czegoś dla liczby $p$-kolorowań, wybór pada zawsze na $\tau_p$.

% $$PSL(2, 7)$ to rzutowa specjalna grupa liniowa nad ciałem $F_7$.



    \chapter{Słownik angielsko-polski}
    \begin{compactitem}
    \input{90-appendix/dictionary}
    \end{compactitem}
\end{appendices}

\input{include/foot}
\end{document}