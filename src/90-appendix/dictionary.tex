\item \textbf{2-przejście}: 2-pass move
\item \textbf{algebra chińskich znaków}: algebra of Chinese characters
\item \textbf{band sum}: suma paskowa
\item \textbf{cień}: shadow
\item \textbf{diagram}: diagram
(\emph{cięciw}: chord, \emph{cięty}: sliced)
\item \textbf{długość sznurowa}: ropelength
\item \textbf{genus}: genus
(\emph{kanoniczny}: canonical, \emph{wolny}: free)
\item \textbf{indeks skrzyżowaniowy}: crossing number
\item \textbf{indeks zaczepienia}: linking number
\item \textbf{izotopia}: isotopy
(\emph{otaczająca}: ambient)
\item \textbf{klamra Kauffmana}: Kauffman bracket
\item \textbf{liczba gordyjska}: unknotting number
\item \textbf{liczba mostowa}: bridge number
\item \textbf{liczba patykowa}: stick number
\item \textbf{niewęzeł}: unknot
\item \textbf{niezmiennik}: invariant
(\emph{... skończonego typu}: finite type ...)
\item \textbf{nieściśliwy}: incompressible
\item \textbf{obramowanie}: framing
\item \textbf{ogniwo splotu}: component
\item \textbf{okres}: period
\item \textbf{pięciolistnik}: cinquefoil knot
\item \textbf{pochodna Foxa}: Fox derivative
\item \textbf{południk}: meridian
\item \textbf{przesmyk}: isthmus
\item \textbf{rozmaitość}: manifold
(\emph{szpiczasta}: cusped, \emph{szwowa}: sutured)
\item \textbf{rozwłókniony, włóknisty}: fibered
\item \textbf{ruch Reidemeistera/Turajewa/...}: Reidemeister/Turaev/... move
\item \textbf{równoleżnik}: latitude
\item \textbf{skrzyżowanie}: crossing
(\emph{znak}: sign)
\item \textbf{spin}: writhe
\item \textbf{splot}: link
(\emph{rozszczepialny}: splittable, \emph{sznurkowy}: string)
\item \textbf{suma niespójna}: distant union
\item \textbf{suma spójna}: connected sum
\item \textbf{sygnatura}: signature
\item \textbf{szwindel Mazura}: Mazur swindle
\item \textbf{tablica rzeczywistości}: actuality table
\item \textbf{tablica węzłów}: knot table
\item \textbf{trójlistnik}: trefoil knot
\item \textbf{układ ciężarów}: weight system
\item \textbf{warkocz}: braid
(\emph{czysty}: pure, \emph{domknięcie ...}: closure of ..., \emph{pasmo ...}: strand)
\item \textbf{wielomian Jonesa}: Jones polynomial
(\emph{powiększony}: augmented)
\item \textbf{wygładzenie}: smoothing
\item \textbf{węzeł}: knot
(\emph{adekwatny}: adequate, \emph{alternujący}: alternating, \emph{dziki}: wild, \emph{długi}: long, \emph{ilorazowy}: quotient, \emph{lustro/lustrzany}: mirror, \emph{obramowany}: frame, \emph{odwracalny}: reversible, \emph{okresowy}: periodic, \emph{osobliwy}: singular, \emph{pierwszy}: prime, \emph{poskromiony}: tame, \emph{preclowy}: pretzel, \emph{rewers/odwrotny}: reverse, \emph{skrętny/chiralny}: chiral, \emph{wirtualny}: virtual, \emph{zespawany}: welded, \emph{zorientowany}: oriented, \emph{zwierciadlany}: achiral/amphicheiral, \emph{złożony}: composite)
\item \textbf{węzeł babski}: granny knot
\item \textbf{węzeł dokerski}: stevedore knot
\item \textbf{ósemka}: figure-eight
