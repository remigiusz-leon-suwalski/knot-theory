
\section{Wartości niezmienników}

Ta sekcja zawiera dwie tabele, zacznijmy od opisu drugiej.
Przedstawia ona węzły pierwsze o~co najwyżej dwunastu skrzyżowaniach oraz wartości ich niezmienników (całkowitoliczbowych lub wielomianowych).
Zgodnie z~oznaczeniami przyjętymi w reszcie książki, $\unknotting, \braid, \bridge$ to kolejno liczba gordyjska, warkoczowa i~mostowa.
Zapis $2..3$ mówi, że dokładna wartość nie jest znana i leży w przedziale $[2,3]$.
Jeśli liczba mostowa wynosi dokładnie $2$, zamiast niej podajemy nieskracalny ułamek $p/q$, który koduje węzeł.
Dalej, $\det$ jest wyznacznikiem, $\sigma$ sygnaturą.
Wielomian Conwaya $\conway(z)$ dla oszczędności miejsca podajemy jako ciąg współczynników, na przykład $1-1$ jest skrótem od $1-z^2$.
Ostatnia kolumna mówi, czy węzeł alternuje.

Pierwsza tabela stanowi podsumowanie drugiej: mówi, ile różnych wartości przybiera dany niezmiennik wśród węzłów o~danej liczbie skrzyżowań.
Jak widać, wielomian Conwaya radzi sobie najlepiej, ale nie doskonale.
Dane pochodzą z~portalu KnotInfo \cite{knotinfo22} (założonego w 2004 roku przez Charlesa Livingstona, do którego dołączył wkrótce Jae Choon Cha. W 2019 roku w~rozwoju strony zaczęli brać udział jeszcze Allison Moore oraz Eric Ost), gdzie znaleźć można opisy wszystkich niezmienników z~tabeli oraz wiele więcej danych, gorąco zachęcamy do odwiedzin tej strony internetowej.
\index[persons]{Livingston, Charles}%
\index[persons]{Moore, Allison}%
\index[persons]{Ost, Eric}%
