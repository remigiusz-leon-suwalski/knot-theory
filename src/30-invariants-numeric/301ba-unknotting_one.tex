
\subsubsection{Sploty 1-gordyjskie}
Sploty o liczbie gordyjskiej 1 zasługują na szczególną uwagę.

\begin{proposition}
\index{węzeł!wymierny}%
    Niech $L$ będzie wymiernym splotem 1-gordyjskim.
    Wtedy na minimalnym diagramie $L$ jedno ze skrzyżowań jest rozwiązujące.
\end{proposition}

\begin{proof}[Niedowód]
\index[persons]{Kanenobu, Taizo}%
\index[persons]{Murakami, Hitoshi}%
\index[persons]{Kohn, Peter}%
    Kanenobu, Murakami dla węzłów \cite{kanenobumurakami1986}, po chwili Kohn dla splotów \cite{kohn1991}.
\end{proof}

\begin{proposition}
\label{prp:unknotting_one_prime}%
    Węzły $1$-gordyjskie są pierwsze.
\end{proposition}

Podejrzewał to Hilmar Wendt w~1937 roku, kiedy policzył liczbę gordyjską węzła babskiego używając homologii rozgałęzionego nakrycia cyklicznego \cite{wendt1937}.
\index[persons]{Wendt, Hilmar}%

\begin{proof}[Niedowód]
    W pracy \cite{scharlemann1985} z~1985 roku Scharlemann podał dość zawiłe uzasadnienie, w~które zamieszane były grafy planarne.
\index[persons]{Scharlemann, Martin}%
    Obecnie znamy prostsze dowody: najpierw Zhang \cite{zhang1991} zauważył, że wynika to z~prac Lickorisha \cite{lickorish1985}, Gordona, Lueckego \cite{luecke1987}, Kima, Tollefsona \cite{tollefson1980}.
\index[persons]{Zhang, Xingru}%
\index[persons]{Lickorish, William}%
\index[persons]{Gordon, Cameron}%
\index[persons]{Luecke, John}%
\index[persons]{Kim, Paik Kee}%
\index[persons]{Tollefson, Jeffrey}%
    Potem Lackenby napisał \cite{lackenby1997}: stamtąd wiemy, że Scharlemann powtórzył dowód, tym razem wykorzystując rozmaitości szwowe.
\index{rozmaitość!szwowa}%
\index[persons]{Lackenby, Marc}%
\end{proof}

% Wynik Scharlemanna był kilkakrotnie uogólniany, najpierw przez Kobayashiego \cite{kobayashit1989}, potem przez Eudavego-Muñoza \cite{eudave1995}.
% \index[persons]{Kobayashi, Tsuyoshi}%
% \index[persons]{Eudave-Muñoz, Mario}%
% zakomentowałem, bo nie umiem sam siebie przekonać, jak tamte uogólniają tamto

Scharlemann pokazał w \cite[wniosek 1.6]{scharlemann1998}, że liczba gordyjska jest podaddytywna, to znaczy zachodzi $\unknotting(K_1 \shrap K_2) \le \unknotting(K_1) + \unknotting(K_2)$.
Stąd oraz z faktu \ref{prp:unknotting_one_prime} wynika, że suma dwóch $1$-gordyjskich węzłów jest $2$-gordyjska, ale od początku teorii węzłów podejrzewano dużo więcej, że liczba gordyjska jest addytywna:

\begin{conjecture}
\index{hipoteza!o liczbie gordyjskiej}%
\label{conjecture_unknotting_additive}%
    Niech $K_1, K_2$ będą węzłami.
    Wtedy $\unknotting (K_1 \shrap K_2) = \unknotting(K_1) + \unknotting(K_2)$.
\end{conjecture}

Niech $K$ będzie 1-gordyjskim węzłem o genusie 1.
Wtedy $K$ jest dublem pewnego węzła (Scharlemann, Thompson \cite{thompson1988}, Kobayashi \cite{kobayashitsuyoshi1989}).
\index[persons]{Kobayashi, Tsuyoshi}%
\index[persons]{Scharlemann, Martin}%
\index[persons]{Thompson, Abigail}%
Dużo później Coward, Lackenby dowiedli w~\cite{coward2011}, że z~dokładnością do pewnej relacji równoważności, tylko jedna zmiana skrzyżowania rozwiązuje węzeł $K$; chyba że ten jest ósemką -- wtedy takie zmiany są dwie.
\index[persons]{Coward, Alexander}%
\index[persons]{Lackenby, Marc}%

