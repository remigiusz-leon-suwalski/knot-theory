
% DICTIONARY;unknotting number;liczba gordyjska;-
\section{Liczba gordyjska}
\index{liczba!gordyjska|(}%

\begin{definition}
    Niech $L$ będzie splotem.
    Minimalną liczbę skrzyżowań, które trzeba odwrócić na pewnym jego diagramie, by dostać niewęzeł, nazywamy liczbą gordyjską i~oznaczamy $\unknotting L$.
\end{definition}

Zgodnie z ,,klasyczną'' definicją, między odwracaniem kolejnych skrzyżowań mamy prawo wykonać izotopie otaczające; natomiast zgodnie ze ,,standardową'' definicją, takie izotopie są zabronione.
Obie definicje są równoważne: tłumaczy to książka Adamsa \cite[s. 58]{adams1994}.
(Książka nie tłumaczy, czemu te dwie definicje zostały określone akurat takimi przymiotnikami.)

\begin{lemma}
\label{lem:unknotting_well_defined}%
    W dowolnym rzucie splotu można odwrócić pewne skrzyżowania tak, by uzyskać diagram niesplotu.
\end{lemma}

(To jest \cite[ćwiczenie E 1.6, s. 15]{burde2014}.)

\begin{proof}
    Bez straty ogólności załóźmy, że diagram przedstawia węzeł.
    Ustalmy zatem diagram węzła i~wybierzmy jakiś początkowy punkt na nim, różny od skrzyżowania wraz z~kierunkiem, wzdłuż którego będziemy przemierzać węzeł.
    Za każdym razem, kiedy odwiedzamy nowe skrzyżowanie, zmieniamy je w~razie potrzeby na takie, przez które przemieszczamy się wzdłuż górnego łuku.
    Skrzyżowań już odwiedzonych nie zmieniamy wcale.

    Teraz wyobraźmy sobie nasz nowy węzeł w~trójwymiarowej przestrzeni $\mathbb R^3$, przy czym oś $z$ skierowana jest z~płaszczyzny, w~której leży diagram, w~naszą stronę.
    Umieśćmy początkowy punkt tak, by jego trzecią współrzędną była $z = 1$.

    Przemierzając węzeł, zmniejszamy stopniowo tę współrzędną, aż osiągniemy wartość $0$ tuż przed punktem, z~którego wyruszyliśmy.
    Połączmy obydwa punkty (początkowy oraz ten, w~którym osiągamy współrzędną $z = 0$) pionowym odcinkiem.
    Zauważmy, że kiedy patrzymy na węzeł w~kierunku osi $z$, nie widzimy żadnych skrzyżowań.

    Oznacza to, że nasza procedura przekształciła początkowy diagram w~diagram niewęzła, co należało okazać.
\end{proof}

Nakanishi \cite{nakanishi1983} znalazł 2-gordyjski diagram 1-gordyjskiego węzła $6_2$, a po trzynastu latach udowodnił, że każdy nietrywialny węzeł ma diagram, który nie jest 1-gordyjski \cite{nakanishi1996}.
\index[persons]{Nakanishi, Yasutaka}%
Jego wyniki uogólnia praca Taniyamy \cite{taniyama2009}: dla każdego nietrywialnego splotu istnieje diagram wymagający odwrócenia dowolnie wielu skrzyżowań.
\index[persons]{Taniyama, Kouki}%

\begin{proposition}
    Niech $L$ będzie nietrywialnym splotem.
    Dla każdej liczby naturalnej $n \in \N$ istnieje diagram $D$ splotu $L$ taki, że $\unknotting D \ge n$.
\end{proposition}

Pokazany jest tam jeszcze jeden godny uwagi fakt.
Jeśli odwrócenie pewnych skrzyżowań daje niewęzeł, to odwrócenie pozostałych także.
Zatem dla splotów $L$ o $n$ skrzyżowaniach mamy $2 \crossing L \le n$.
Nie jest to zbyt pomocne, daje rozstrzygnięcie pięć razy dla pierwszych węzłów do 12 skrzyżowań: $3_{1}$, $5_{1}$, $7_{1}$, $9_{1}$, $11a_{367}$.
Ale...

\begin{proposition}
    Jeśli liczba gordyjska diagramu $D$ węzła $K$ wynosi $\frac 12 (\crossing D - 1)$, to węzeł jest $(2,p)$-torusowy albo wygląda jak diagram niewęzła po pierwszym ruchu Reidemeistera.
\end{proposition}

W powyższym stwierdzeniu nie można zastąpić słowa ,,węzeł'' przez ,,splot''.

\input{30-invariants-numeric/301ba-unknotting_one}


\subsubsection{Znane wartości}
Cha, Livingston \cite{cha2018} podają, że znamy liczby gordyjskie wszystkich węzłów pierwszych do dziesięciu skrzyżowań poza dziewięcioma wyjątkami: $10_{11}$, $10_{47}$, $10_{51}$, $10_{54}$, $10_{61}$, $10_{76}$, $10_{77}$, $10_{79}$, $10_{100}$ (gdzie nie mamy pewności, czy $\unknotting = 2$, czy $\unknotting = 3$).
\index[persons]{Cha, Jae}%
\index[persons]{Livingston, Charles}%
Kto pierwszy znalazł liczbę gordyjską którego węzła ostaraliśmy się bezbłędnie przepisać z bazy danych KnotInfo\footnote{Patrz \url{https://knotinfo.math.indiana.edu/descriptions/unknotting_number.html}}.
Według KnotInfo oprócz węzłów torusowych, tych wymienionych poniżej oraz 1-gordyjskich, do 10 skrzyżowań mamy jeszcze 2 węzły o~siedmiu skrzyżowaniach, 3 o~ośmiu, 15 o~dziewięciu i~68 o~dziesięciu, których liczba gordyjska zdaje się należeć do folkloru matematycznego.

{
    \setlength{\intextsep}{4pt plus 2pt minus 2pt}
\begin{table}[H]
    \raggedright
    \footnotesize
    \centering
    \begin{tabular}{l|p{100mm}} \toprule
    rok & węzły i odkrywcy ich liczb gordyjskich \\ \midrule
    1982 & $7_{4}$ (Lickorish \cite{lickorish1985}) \\
    1986 & $8_{4}, 8_{6}, 8_{8}, 8_{12}, 9_{5}, 9_{8}, 9_{15}, 9_{17}, 9_{31}$ (Kanenobu, Murakami \cite{kanenobumurakami1986}) \\
    1989 & $9_{25}$ (Kobayashi \cite{kobayashi1989}) \\
    1994 & $10_{8}$ (Adams \cite[s. 62]{adams1994}?) \\
    1998 & $10_{65}, 10_{69}, 10_{89}, 10_{97}, 10_{108}, 10_{163}, 10_{165}$ (Miyazawa \cite{miyazawa1998}), $10_{154}, 10_{161}$ (Tanaka \cite{tanaka1998}) \\
    1999 & $10_{67}$ (Traczyk \cite{traczyk1999}) \\
    2000 & $8_{16}$ (Murakami, Yasuhara \cite{yasuhara2000}) \\
    2002 & $10_{139}, 10_{145}, 10_{152}, 10_{154}, 10_{161}$ (Gibson, Ishikawa \cite{ishikawa2002}) \\
    2004 & $8_{18}, 9_{37}, 9_{40}, 9_{46}, 9_{48}, 9_{49}, 10_{86}, 10_{103}, 10_{105}, 10_{106}, 10_{109}, 10_{121}, 10_{131}$ (Stojmenow \cite{stoimenow2004}; ostatni węzeł zdaje się być jedynym 1-gordyjskim na tej liście!) \\
    2005 & $9_{29}$, $10_{79}$, $10_{81}$, $10_{87}$, $10_{90}$, $10_{93}$, $10_{94}$, $10_{96}$, $10_{148}$, $10_{151}$, $10_{153}$ (Gordon, Luecke \cite{gordon2006}), $8_{10}$, $10_{48}$, $10_{52}$, $10_{54}$, $10_{57}$, $10_{58}$, $10_{64}$, $10_{68}$, $10_{70}$, $10_{77}$, $10_{110}$, $10_{112}$, $10_{116}$, $10_{117}$, $10_{125}$, $10_{126}$, $10_{130}$, $10_{135}$, $10_{138}$, $10_{158}$, $10_{162}$ (Ozsváth, Szabó \cite{szabo2005}), $10_{83}$ (Nakanishi \cite{nakanishi2005}) \\
    2008 & $9_{10}, 9_{13}, 9_{35}, 9_{38}, 10_{53}, 10_{101}, 10_{120}$ (Owens \cite{owens2008}) \\
    \bottomrule
    \hline
    \end{tabular}
% 50 za dużo -< aż do przykład NB
% 25 trochę za dużo  \vspace{-25pt} ^ to samo
% 5 za mało
\end{table}
}
\index[persons]{Lickorish, William}%
\index[persons]{Kanenobu, Taizo}%
\index[persons]{Murakami, Hitoshi}%
\index[persons]{Kobayashi, Tsuyoshi}%
\index[persons]{Adams, Colin}%
\index[persons]{Miyazawa, Yasuyuki}%
\index[persons]{Tanaka, Toshifumi}%
\index[persons]{Traczyk, Paweł}%
\index[persons]{Yasuhara, Akira}%
\index[persons]{Gibson, William}%
\index[persons]{Ishikawa, Masaharu}%
\index[persons]{Stojmenow, Aleksander}%
\index[persons]{Gordon, Cameron}%
\index[persons]{Luecke, John}%
\index[persons]{Nakanishi, Yasutaka}%
\index[persons]{Szabó, Zoltán}%
\index[persons]{Ozsváth, Peter}%
\index[persons]{Owens, Brendan}%
\normalsize




\subsubsection{Dolne ograniczenia liczby gordyjskiej}
Dokładna wartość liczby gordyjskiej jest znana tylko dla niektórych klas węzłów, na przykład torusowych (fakt \ref{prp:torus_unknotting_number}) albo skręconych.
\index{węzeł!torusowy}%
\index{węzeł!skręcony}%

Borodzik oraz Friedl podali niedawno całkiem mocne ograniczenia na liczbę gordyjską w~pracach \cite{borodzik2014} i~\cite{borodzik2015}.
\index[persons]{Borodzik, Maciej}%
\index[persons]{Friedl, Stefan}%
Ich narzędziem jest parowanie Blanchfielda.
\index{parowanie Blanchfielda}%
Poprawiają tam starsze estymaty wynikające z~sygnatury Levine'a-Tristrama, indeksu Nakanishiego oraz przeszkody Lickorisha.
\index{indeks Nakanishiego}%
\index{przeszkoda Lickorisha}%
\index{sygnatura!Levine'a-Tristrama}%
% DICTIONARY;Lickorish obstruction;przeszkoda Lickorisha;-
Wśród pierwszych węzłów o~co najwyżej 12 skrzyżowaniach dwadzieścia pięć ma liczbę gordyjską równą co najmniej trzy, trudno było uzasadnić to innymi metodami.



\input{30-invariants-numeric/301bd-bleiler}

\input{30-invariants-numeric/301be-metric}

\subsubsection{Inne operacje rozwiązujące węzły}

Shimizu w pracy \cite{shimizu2014} rozpatruje różne operacje, które rozwiązują węzły lub sploty.
\index[persons]{Shimizu, Ayaka}%
Nie będziemy się nimi zajmować, podamy tylko przykład: zamiana pod- i nadskrzyżowań wokół obszaru na diagramie rozwiązuje węzły, ale nie sploty; kontrprzykładem jest splot Hopfa.
\index{splot!Hopfa}%
(Ciąg dalszy tamtej pracy razem z Oshikirim, Tamurą \cite{oshikiri2024}).
\index[persons]{Tamura, Junya}%
\index[persons]{Oshikiri, Tokio}%
Patrz też, co pisze Kawauchi w \cite[s. 141-154]{kawauchi1996}.

Mieliśmy też:

\begin{conjecture}
    Dowolny splot można rozwiązać wykonując ciąg 3-ruchów (zastępując dwie równoległe nici przez trzy półskręty lub odwrotnie).
\end{conjecture}

Ze zbioru problemów Kirby'ego \cite{kirby1978} wiemy, że Nakanishi zastanawiał się nad tym w 1981 roku.
\index[persons]{Nakanishi, Yasutaka}%
Nie to samo, ale podobne pytanie zadał wcześniej Montesinos w związku z nakryciami i~dlatego Kirby nazwał problem hipotezą Nakanishiego-Montesinosa.
\index[persons]{Montesinos, José}%
Conway zauważył, że hipoteza jest prawdziwa dla węzłów algebraicznych.
\index[persons]{Conway, John}%
Coxeter rozprawił się z nią dla prawie wszystkich splotów o~indeksie warkoczowym mniejszym niż $6$ oraz indeksie mostowym mniejszym niż $4$.
\index[persons]{Coxeter, Harold}%
Nakanishi w 1994 pokazał splot zbudowany z pierścieni Boromeuszy wobec którego podejrzewał, że jest kontrprzykładem.
\index{pierścienie Boromeuszy}%
Żeby zdobyć więcej informacji o postępie prac nad hipotezą, musieliśmy sięgnąć po artykuł Przytyckiego, Dąbkowskiego \cite{dabkowski2002}.
\index[persons]{Przytycki, Józef}%
\index[persons]{Dąbkowski, Mieczysław}%
Chen w~1999 zasugerował inny kontrprzykład, domknięcie 5-warkocza $(\sigma_1\sigma_2\sigma_3\sigma_4)^{10}$.
\index[persons]{Chen, Qi}%
Artykuł \cite{dabkowski2002} dowodzi, że te dwa sploty istotnie obalają hipotezę.
Używa się w~nim nieprzemiennej wersji $n$-kolorowań Foxa, tak zwanej $n$-tej grupy Burnside'a splotu.
\index{grupa!Burnside'a}%
\index{kolorowanie}%

Nakanishi w 1979, a więc zanim ogłosił powyższą hipotezę, miał wrażenie, że $4$-ruchy rozwiązują wszystkie sploty.
Najpierw sprawdzono, że jest prawdziwa dla wszystkich dwu- i trzymostowych węzłów, a także węzłów do 12 skrzyżowań, ale potem Askitas ogłosił, że pewien węzeł o 16 skrzyżowaniach obala ją.
\index[persons]{Askitas, Nikos}%
Później pojawili się inni podejrzani, ale nie wiemy, czy naprawdę są kontrprzykładami.

% znalezione przypadkiem w MR3143585
% 1979 Nakanishi: hipoteza że 4-ruch jest rozwiązujący
% dowody: 2-mostowe i 3-mostowe węzły, wszystkie do 12 skrzyżowań

\index{liczba!gordyjska|)}%

% Koniec podsekcji Liczba gordyjska

