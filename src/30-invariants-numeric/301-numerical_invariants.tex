
Wspomnieliśmy na s. \pageref{page_first_invariant}, że pytanie, czy dwa dane diagramy splotów przedstawiają ten sam czy różne sploty, bywa trudne.
Napomknęliśmy też, że świadkiem równości dwóch splotów jest ciąg ruchów Reidemeistera, natomiast niezmienników używa się do wykazania różności.
Czym jest niezmiennik?
To pewna wielkość, która nie ulega zmianie (w naszym przypadku: podczas izotopii otaczającej przestrzeni, w której jest zanurzony splot).
Jeśli dany niezmiennik przyjmuje różne wartości dla dwóch splotów, to nie mogą być równoważne.

Jak pisze Przytycki w~,,Dwustu latach teorii węzłów'', kiedy badamy nowy niezmiennik, powinniśmy zadać sobie trzy pytania.
\index[persons]{Przytycki, Józef}%
Dobrze byłoby o nich nie zapominać.
Oto pytania Przytyckiego:
\begin{enumerate}
    \item czy łatwo wyznaczyć wartość niezmiennika?
    \item czy w zbiorze wartości niezmiennika łatwo odróżnia się elementy?
    \item czy niezmiennik odróżnia wiele splotów?
\end{enumerate}

Poznaliśmy dotychczas dwa niezmienniki: liczbę ogniw (którą łatwo wyznaczyć i której wartości -- liczby naturalne -- łatwo odróżniać; ale nie odróżnia wielu splotów) oraz topologię dopełnienia splotu (tym razem odróżnia wszystkie węzły pierwsze, ale trudno wyznaczyć jej ,,wartość'').
Tutaj przedstawiamy kilka więcej; przede wszystkim te, które nie wymagają mocnej znajomości reszty książki.
Niektóre z~nich są miarą złożoności splotów zgodnie z~następującym przepisem: niech $f$ będzie pewną funkcją określoną dla dowolnego diagramu splotu.
Wtedy odwzorowanie
\begin{equation}
    f(L) = \min \{f(D) : D \text{ jest diagramem splotu } L\}
\end{equation}
stanowi niezmiennik splotów.
Dowód jest trywialny i~pozostawiamy go jako ćwiczenie dla Czytelnika.
Im większa wartość funkcji $f$, tym bardziej skomplikowany splot.

Później poznamy inne niezmienniki, oprócz opisanych poniżej miarą złożoności jest też liczba warkoczowa (definicja \ref{def:braid_number}), ale nie wyznacznik (definicja \ref{def:determinant}) czy sygnatura (definicja \ref{def:signature}).
Przekonamy się też, że istnieją użyteczne niezmienniki, które są wielomianami albo innymi obiektami algebraicznymi.


% DICTIONARY;crossing number;indeks skrzyżowaniowy;-
\section{Indeks skrzyżowaniowy}
Opisując notację Alexandera-Briggsa w podsekcji \ref{alexander_briggs_notation} wspomnieliśmy już raz o indeksie skrzyżowaniowym.

\index{indeks skrzyżowaniowy|(}%
\begin{definition}
    Niech $L$ będzie splotem.
    Minimalną liczbę skrzyżowań, jakie muszą się pojawić na diagramie przedstawiającym splot $L$ nazywamy indeksem skrzyżowaniowym i oznaczamy krótko $\crossing L$.
\end{definition}

Pytanie, czy indeks skrzyżowaniowy jest addytywny, to jeden z najstarszych problemów teorii węzłów.

\begin{conjecture}
\index{hipoteza!o indeksie skrzyżowaniowym}%
\index{suma spójna}%
\label{con:crossing_additive}%
    Niech $K_1$ oraz $K_2$ będą węzłami.
    Wtedy $\crossing K_1 + \crossing K_2 = \crossing K_1 \shrap K_2$.
\end{conjecture}

Oto częściowe odpowiedzi.
Jeśli $K_1, K_2$ są alternującymi węzłami o~odpowiednio $c_1, c_2$ skrzyżowaniach, to istnieje alternujący diagram ich sumy $K_1 \shrap K_2$ o~$c_1 + c_2$ skrzyżowaniach.
\index{węzeł!alternujący}%
Kauffman \cite[twierdzenie 2.10]{kauffman1987}, Murasugi \cite[wniosek 6]{murasugi1987} oraz Thistlethwaite \cite[wniosek 1]{thistlethwaite1987} pokazali niezależnie, że diagram ten jest minimalny.
\index[persons]{Kauffman, Louis}%
\index[persons]{Murasugi, Kunio}%
\index[persons]{Thistlethwaite, Morwen}%

% DICTIONARY;adequate;adekwatny;węzeł
Thistlethwaite rozszerzył wynik do tak zwanych węzłów adekwatnych: sam \cite{thistlethwaite1988} albo razem z Lickorishem \cite{lickorish1988} (tak przynajmniej sugeruje Malutin \cite[s. 3]{malyutin2016}, nie widzimy tego).
\index[persons]{Lickorish, William}%
\index[persons]{Malutin, Andriej}%
\index{węzeł!adekwatny}%
Mając diagram węzła $K$, można wygładzić wszystkie skrzyżowania dodatnio i dostać splot złożony z rozłącznych okręgów.
Jeżeli zmiana dowolnego wygładzenia na ujemne sprawia, że liczba ogniw splotu zmniejsza się, diagram nazywamy dodatnio adekwatnym.
Węzeł dodatnio adekwatny to taki, który posiada jakiś dodatnio adekwatny diagram.
Analogicznie definiuje się ujemną adekwatność.
Węzeł, który jest dodatnio oraz ujemnie adekwatny, nazywamy krótko adekwatnym.

Na początku XX wieku Diao \cite{diao2004} oraz Gruber \cite{gruber2003} niezależnie udowodnili hipotezę \ref{con:crossing_additive} dla pewnej szerokiej klasy węzłów, obejmującej wszystkie węzły torusowe, wiele węzłów alternujących oraz jeszcze inne obiekty, których nie chcemy opisywać.
% diao04 -> tw. 3.8
\index[persons]{Diao, Yuanan}%
\index[persons]{Gruber, Hermann}%
\index{węzeł!torusowy}%

Lackenby \cite{lackenby2009} pokazał, że dla pewnej stałej $N \le 152$ zachodzi
\index[persons]{Lackenby, Marc}
\begin{equation}
    \frac 1 N \sum_{i=1}^n \crossing{K_i} \le \crossing \left(\bigshrap_{i=1}^n K_i\right) \le \sum_{i=1}^n \crossing{K_i}.
\end{equation}
(Tylko pierwsza nierówność jest ciekawa).
Jego argumentu wykorzystującego powierzchnie normalne nie można poprawić tak, by otrzymać stałą $N = 1$.
Jednocześnie od 2009 roku nie widać postępu nad hipotezą.

\index{indeks skrzyżowaniowy|)}%

% Koniec podsekcji Indeks skrzyżowaniowy




% DICTIONARY;unknotting number;liczba gordyjska;-
\subsection{Liczba gordyjska}
\index{liczba gordyjska|(}%

\begin{definition}
    Niech $L$ będzie splotem.
    Minimalną liczbę skrzyżowań, które trzeba odwrócić na pewnym jego diagramie, by dostać niewęzeł, nazywamy liczbą gordyjską i~oznaczamy $\unknotting L$.
\end{definition}

Zgodnie z ,,klasyczną'' definicją, między odwracaniem kolejnych skrzyżowań mamy prawo wykonać izotopie otaczające; natomiast zgodnie ze ,,standardową'' definicją, takie izotopie są zabronione.
Obie definicje są równoważne: tłumaczy to książka Adamsa \cite[s. 58]{adams1994}.

\begin{lemma}
\label{lem:unknotting_well_defined}%
    W dowolnym rzucie splotu można odwrócić pewne skrzyżowania tak, by uzyskać diagram niesplotu.
\end{lemma}

\begin{proof}
    Bez straty ogólności załóźmy, że diagram przedstawia węzeł.
    Ustalmy zatem diagram węzła i~wybierzmy jakiś początkowy punkt na nim, różny od skrzyżowania wraz z~kierunkiem, wzdłuż którego będziemy przemierzać węzeł.
    Za każdym razem, kiedy odwiedzamy nowe skrzyżowanie, zmieniamy je w~razie potrzeby na takie, przez które przemieszczamy się wzdłuż górnego łuku.
    Skrzyżowań już odwiedzonych nie zmieniamy wcale.

    Teraz wyobraźmy sobie nasz nowy węzeł w~trójwymiarowej przestrzeni $\mathbb R^3$, przy czym oś $z$ skierowana jest z~płaszczyzny, w~której leży diagram, w~naszą stronę.
    Umieśćmy początkowy punkt tak, by jego trzecią współrzędną była $z = 1$.

    Przemierzając węzeł, zmniejszamy stopniowo tę współrzędną, aż osiągniemy wartość $0$ tuż przed punktem, z~którego wyruszyliśmy.
    Połączmy obydwa punkty (początkowy oraz ten, w~którym osiągamy współrzędną $z = 0$) pionowym odcinkiem.
    Zauważmy, że kiedy patrzymy na węzeł w~kierunku osi $z$, nie widzimy żadnych skrzyżowań.

    Oznacza to, że nasza procedura przekształciła początkowy diagram w~diagram niewęzła, co należało okazać.
\end{proof}

Nakanishi \cite{nakanishi1983} znalazł 2-gordyjski diagram 1-gordyjskiego węzła $6_2$, a po trzynastu latach udowodnił, że każdy nietrywialny węzeł ma diagram, który nie jest 1-gordyjski \cite{nakanishi1996}.
\index[persons]{Nakanishi, Yasutaka}%
Jego wyniki uogólnia praca Taniyamy \cite{taniyama2009}: dla każdego nietrywialnego splotu istnieje diagram wymagający odwrócenia dowolnie wielu skrzyżowań.
\index[persons]{Taniyama, Kouki}%

\begin{proposition}
    Dla każdego $n \in \N$ istnieje diagram $D$ nietrywialnego splotu $L$ taki, że $\unknotting D \ge n$.
\end{proposition}

Pokazany jest tam jeszcze jeden godny uwagi fakt.

\begin{proposition}
    Jeśli liczba gordyjska diagramu $D$ wynosi $\frac 12 (\crossing D - 1)$, co jest maksymalną możliwą wartością zgodnie z~naszym prostym ograniczeniem, to węzeł jest $(2,p)$-torusowy albo wygląda jak diagram niewęzła po pierwszym ruchu Reidemeistera.
\end{proposition}


\subsubsection{Sploty 1-gordyjskie}
Sploty o liczbie gordyjskiej 1 zasługują na szczególną uwagę.

\begin{proposition}
\index{węzeł!wymierny}%
    Niech $L$ będzie wymiernym splotem 1-gordyjskim.
    Wtedy na minimalnym diagramie $L$ jedno ze skrzyżowań jest rozwiązujące.
\end{proposition}

\begin{proof}[Niedowód]
\index[persons]{Kanenobu, Taizo}%
\index[persons]{Murakami, Hitoshi}%
\index[persons]{Kohn, Peter}%
    Kanenobu, Murakami dla węzłów \cite{kanenobumurakami1986}, po chwili Kohn dla splotów \cite{kohn1991}.
\end{proof}

\begin{proposition}
\label{prp:unknotting_one_prime}%
    Węzły $1$-gordyjskie są pierwsze.
\end{proposition}

Podejrzewał to Hilmar Wendt w~1937 roku, kiedy policzył liczbę gordyjską węzła babskiego używając homologii rozgałęzionego nakrycia cyklicznego \cite{wendt1937}.
\index[persons]{Wendt, Hilmar}%

\begin{proof}[Niedowód]
    W pracy \cite{scharlemann1985} z~1985 roku Scharlemann podał dość zawiłe uzasadnienie, w~które zamieszane były grafy planarne.
\index[persons]{Scharlemann, Martin}%
    Obecnie znamy prostsze dowody: najpierw Zhang \cite{zhang1991} zauważył, że wynika to z~prac Lickorisha \cite{lickorish1985}, Gordona, Lueckego \cite{luecke1987}, Kima, Tollefsona \cite{tollefson1980}.
\index[persons]{Zhang, Xingru}%
\index[persons]{Lickorish, William}%
\index[persons]{Gordon, Cameron}%
\index[persons]{Luecke, John}%
\index[persons]{Kim, Paik Kee}%
\index[persons]{Tollefson, Jeffrey}%
    Potem Lackenby napisał \cite{lackenby1997}: stamtąd wiemy, że Scharlemann powtórzył dowód, tym razem wykorzystując rozmaitości szwowe.
\index{rozmaitość!szwowa}%
\index[persons]{Lackenby, Marc}%
\end{proof}

% Wynik Scharlemanna był kilkakrotnie uogólniany, najpierw przez Kobayashiego \cite{kobayashit1989}, potem przez Eudavego-Muñoza \cite{eudave1995}.
% \index[persons]{Kobayashi, Tsuyoshi}%
% \index[persons]{Eudave-Muñoz, Mario}%
% zakomentowałem, bo nie umiem sam siebie przekonać, jak tamte uogólniają tamto

Scharlemann pokazał w \cite[wniosek 1.6]{scharlemann1998}, że liczba gordyjska jest podaddytywna, to znaczy zachodzi $\unknotting(K_1 \shrap K_2) \le \unknotting(K_1) + \unknotting(K_2)$.
Stąd oraz z faktu \ref{prp:unknotting_one_prime} wynika, że suma dwóch $1$-gordyjskich węzłów jest $2$-gordyjska, ale od początku teorii węzłów podejrzewano dużo więcej, że liczba gordyjska jest addytywna:

\begin{conjecture}
\index{hipoteza!o liczbie gordyjskiej}%
\label{conjecture_unknotting_additive}%
    Niech $K_1, K_2$ będą węzłami.
    Wtedy $\unknotting (K_1 \shrap K_2) = \unknotting(K_1) + \unknotting(K_2)$.
\end{conjecture}

Niech $K$ będzie 1-gordyjskim węzłem o genusie 1.
Wtedy $K$ jest dublem pewnego węzła (Scharlemann, Thompson \cite{thompson1988}, Kobayashi \cite{kobayashitsuyoshi1989}).
\index[persons]{Kobayashi, Tsuyoshi}%
\index[persons]{Scharlemann, Martin}%
\index[persons]{Thompson, Abigail}%
Dużo później Coward, Lackenby dowiedli w~\cite{coward2011}, że z~dokładnością do pewnej relacji równoważności, tylko jedna zmiana skrzyżowania rozwiązuje węzeł $K$; chyba że ten jest ósemką -- wtedy takie zmiany są dwie.
\index[persons]{Coward, Alexander}%
\index[persons]{Lackenby, Marc}%




\subsubsection{Dolne ograniczenia liczby gordyjskiej}
Dokładna wartość liczby gordyjskiej jest znana tylko dla niektórych klas węzłów, na przykład torusowych (fakt \ref{prp:torus_unknotting_number}) albo skręconych.
\index{węzeł!torusowy}%
\index{węzeł!skręcony}%

Jeśli odwrócenie pewnych skrzyżowań daje niewęzeł, to odwrócenie pozostałych także.
To daje proste liczby gordyjskiej: $2 \unknotting K \le \crossing K$.
Nie jest zbyt pomocne, daje rozstrzygnięcie pięć razy dla pierwszych węzłów do 12 skrzyżowań: $3_{1}$, $5_{1}$, $7_{1}$, $9_{1}$, $11a_{367}$.

Borodzik oraz Friedl podali niedawno całkiem mocne ograniczenia na liczbę gordyjską w~pracach \cite{borodzik2014} i~\cite{borodzik2015}.
\index[persons]{Borodzik, Maciej}%
\index[persons]{Friedl, Stefan}%
Ich narzędziem jest parowanie Blanchfielda.
\index{parowanie Blanchfielda}%
Poprawiają tam starsze estymaty wynikające z~sygnatury Levine'a-Tristrama, indeksu Nakanishiego\footnote{Indeks Nakanishiego nie jest omawiany w tej książce, patrz \cite{nakanishi1981}.} oraz przeszkody Lickorisha\footnote{Przeszkoda Lickorisha nie jest omawiana w tej książce, patrz \cite{cochran1986}, \cite{lickorish1985}; są jeszcze przeszkody Murakamiego \cite{murakami1990} oraz Jabuki \cite{jabuka2010}.}.
\index{indeks Nakanishiego}%
\index{przeszkoda Lickorisha}%
\index{sygnatura!Levine'a-Tristrama}%
% DICTIONARY;Lickorish obstruction;przeszkoda Lickorisha;-
Wśród pierwszych węzłów o~co najwyżej 12 skrzyżowaniach dwadzieścia pięć ma liczbę gordyjską równą co najmniej trzy, trudno było uzasadnić to innymi metodami.




\subsubsection{Znane wartości}
Dotychczas wyznaczono liczbę gordyjską prawie wszystkich węzłów pierwszych o~co najwyżej dziesięciu skrzyżowaniach.
Cha, Livingston \cite{cha2018} podają następującą listę wyjątków:
\index[persons]{Cha, Jae}%
\index[persons]{Livingston, Charles}%
$10_{11}$, $10_{47}$, $10_{51}$, $10_{54}$, $10_{61}$, $10_{76}$, $10_{77}$, $10_{79}$, $10_{100}$ (stan na rok 2018).
Poniżej podajemy za stroną internetową KnotInfo\footnote{Patrz \url{https://knotinfo.math.indiana.edu/descriptions/unknotting_number.html}. Pomijając węzły torusowe, skopiowane przez nas na liście oraz 1-gordyjskie, do dziesięciu skrzyżowań zostawia to: 2 węzły o~siedmiu skrzyżowaniach, 3 o~ośmiu, 15 o~dziewięciu i~68 o~dziesięciu.} listę odkrywców liczb gordyjskich węzłów do 10 skrzyżowań.
KnotInfo wymienia więcej, bo węzły do 12 skrzyżowań.

\begin{compactitem}
\item Lickorish \cite{lickorish1985}: $7_{4}$.
\index[persons]{Lickorish, William}%
\item Kanenobu, Murakami \cite{kanenobumurakami1986}: $8_{3}$, $8_{4}$, $8_{6}$, $8_{8}$, $8_{12}$, $9_{5}$, $9_{8}$, $9_{15}$, $9_{17}$, $9_{31}$.
\index[persons]{Kanenobu, Taizo}%
\index[persons]{Murakami, Hitoshi}%
\item Szabó \cite{szabo2005}: $8_{10}$, $10_{48}$, $10_{52}$, $10_{54}$ ($\unknotting \neq 1$), $10_{57}$, $10_{58}$, $10_{64}$, $10_{68}$, $10_{70}$, $10_{77}$ ($\unknotting \neq 1$), $10_{110}$, $10_{112}$, $10_{116}$, $10_{117}$, $10_{125}$, $10_{126}$, $10_{130}$, $10_{135}$, $10_{138}$, $10_{158}$, $10_{162}$.
\index[persons]{Szabó, Zoltán}%
\item Murakami, Yasuhara \cite{yasuhara2000}, Stojmenow \cite{stoimenow2004}: $8_{16}$.
\index[persons]{Yasuhara, Akira}%
\index[persons]{Stojmenow, Aleksander}%
\item Stojmenow \cite{stoimenow2004}: $8_{18}$, $9_{37}$, $9_{40}$, $9_{46}$, $9_{48}$, $9_{49}$, $10_{103}$.
\item Owens \cite{owens2008}: $9_{10}$, $9_{13}$, $9_{35}$, $9_{38}$, $10_{53}$, $10_{101}$, $10_{120}$.
\index[persons]{Owens, Brendan}%
\item Kanenobu, Murakami \cite{kanenobumurakami1986}, Stojmenow \cite{stoimenow2004}: $9_{15}$, $9_{17}$.
\item Kobayashi \cite{kobayashi1989}: $9_{25}$.
\index[persons]{Kobayashi, Tsuyoshi}%
\item Gordon, Luecke \cite{gordon2006}, Szabó \cite{szabo2005}: $9_{29}$, $10_{81}$, $10_{87}$, $10_{90}$, $10_{93}$, $10_{94}$, $10_{96}$.
\index[persons]{Gordon, Cameron}%
\index[persons]{Luecke, John}%
\item Adams? \cite[s. 62]{adams1994}: $10_{8}$.
\index[persons]{Adams, Colin}%
\item Miyazawa \cite{miyazawa1998}: $10_{65}$, $10_{69}$, $10_{89}$, $10_{108}$, $10_{163}$, $10_{165}$.
\index[persons]{Miyazawa, Yasuyuki}%
\item Traczyk \cite{traczyk1999}, Szabó \cite{szabo2005}: $10_{67}$.
\index[persons]{Traczyk, Paweł}%
\item Szabó \cite{szabo2005}, ($\unknotting \neq 1$ Gordon, Luecke \cite{gordon2006}): $10_{79}$.
\item Gordon, Luecke \cite{gordon2006}, Szabó \cite{szabo2005}, Nakanishi \cite{nakanishi2005}: $10_{83}$.
\index[persons]{Nakanishi, Yasutaka}%
\item Stojmenow \cite{stoimenow2004}, Szabó \cite{szabo2005}, Gordon, Luecke \cite{gordon2006}: $10_{86}$.
\item Miyazawa \cite{miyazawa1998}, Nakanishi \cite{nakanishi2005}: $10_{97}$.
\item Szabó \cite{szabo2005}, Stojmenow \cite{stoimenow2004}, Nakanishi \cite{nakanishi2005}: $10_{105}$, $10_{106}$, $10_{109}$, $10_{121}$.
\item Stojmenow \cite{stoimenow2004}, Szabó \cite{szabo2005}: $10_{131}$ (jedyny 1-gordyjski na tej liście!).
\item Gibson, Ishikawa \cite{ishikawa2002}: $10_{139}$, $10_{145}$, $10_{152}$.
\index[persons]{Gibson, William}%
\index[persons]{Ishikawa, Masaharu}%
\item Gordon, Luecke \cite{gordon2006}, Szabó \cite{szabo2005}: $10_{148}$, $10_{151}$.
\item Gordon, Luecke \cite{gordon2006}: $10_{153}$.
\item Stojmenow \cite{stoimenow2003}, Gibson, Ishikawa \cite{ishikawa2002}: $10_{154}$.
\item Gibson, Ishikawa \cite{ishikawa2002}: $10_{161}$.
\end{compactitem}

% dummy commit, 2022-04-17




\subsubsection{Przykład Nakanishiego-Bleilera. Hipoteza Bernharda-Jablana}
Najpierw Nakanishi \cite{nakanishi1983}, a potem Bleiler \cite{bleiler1984} odkryli fascynujący przykład wymiernego węzła $10_8$, który jest $2$-gordyjski, ale świadkiem tego nie może być żaden diagram mininalny, ponieważ, co jeszcze bardziej fascynujące, węzeł ten posiada tylko jeden diagram o~dziesięciu skrzyżowaniach oraz liczbie gordyjskiej 3.
\index[persons]{Bleiler, Steven}%
\index[persons]{Nakanishi, Yasutaka}%
\index{węzeł!10-8}%
Wynika stąd, że liczba $\unknotting$ nie musi być osiągana przez diagram minimalny, wbrew powszechnym jeszcze w latach 70 przypuszczeniom.
Praca \cite{bernhard1994} zawiera indukcyjny dowód faktu, że żaden minimalny diagram węzła oznaczanego w notacji Conwaya przez $C(2m+1, 1, 2m)$ nie daje się rozwiązać w $m$ ruchach, ale pewne nieminimalne diagramy dają się.
Przypadek $m = 2$ odpowiada węzłowi $10_8$.

Przykład Bleilera pokazuje, że do szukania liczby gordyjskiej potrzeba wyrafinowanego algorytmu.
Ponieważ odwrócenie jednego ze skrzyżowań na minimalnym diagramie węzła $10_8$ daje $1$-gordyjski węzeł $4_1, 5_1, 6_1$ lub $6_2$, możemy liczyć, że każdy diagram minimalny ma skrzyżowanie, którego odwrócenie zmniejsza liczbę gordyjską.
Dlatego w~latach 90. Bernhard \cite{bernhard1994} i Jablan \cite{jablan1998} postawili hipotezę:

\begin{conjecture}[Bernharda-Jablana]
\index[persons]{Bernhard, James}%
\index[persons]{Jablan, Slavik}%
\index{hipoteza!Bernharda-Jablana}%
\label{con:bernhard_jablan}%
    Niech $K$ będzie węzłem z diagramem $D$, który realizuje liczbę gordyjską $\unknotting K$.
    Istnieje wtedy skrzyżowanie, którego odwrócenie daje nowy diagram $D'$ nowego węzła $K'$ o~mniejszej liczbie gordyjskiej: $1 + \unknotting D' = \unknotting D$.
\end{conjecture}

Przypuszczenie to sprawdzono dla węzłów do jedenastu skrzyżowań oraz splotów o dwóch ogniwach do dziewięciu skrzyżowań (Kohn w \cite{kohn1993}?).
\index[persons]{Kohn, Peter}%
Gdyby hipoteza~\ref{con:bernhard_jablan} była prawdziwa dla wszystkich węzłów, mielibyśmy prosty sposób na wyznaczenie liczby $\unknotting K$: weźmy skończenie wiele diagramów minimalnych dla węzła $K$, na każdym z~nich odwracajmy skrzyżowania i~rekursywnie szukajmy liczb gordyjskich prostszych węzłów.
Najmniejsza spośród nich różni się wtedy o~jeden od liczby $\unknotting K$.

Brittenham, Hermiller w artykule \cite{brittenham2021} twierdzą, że hipoteza jest fałszywa.
Kontrprzykład został znaleziony komputerowo, z pomocą programu SnapPy.
\index{program SnapPy}%
\index[persons]{Brittenham, Mark}%
\index[persons]{Hermiller, Susan}%

\begin{example}[Brittenham, Hermiller]
\index{węzeł!12n-288}%
\index{węzeł!12n-491}%
\index{węzeł!12n-501}%
\index{węzeł!13n-3370}%
    Hipoteza Bernharda-Jablana jest fałszywa dla co najmniej jednego spośród czterech węzłów: $12n_{288}$, $12n_{491}$, $12n_{501}$, $13n_{3'370}$.
\end{example}

Bleiler \cite{bleiler1984} postawił problem: czy jeden węzeł może mieć kilka diagramów minimalnych, z~których tylko niektóre są świadkiem $1$-gordyjskości?
Rozwiązanie przyszło z Japonii: według Kanenobu, Murakamiego \cite{kanenobumurakami1986} dzieje się tak m.in. dla węzła $8_{14}$.
\index{węzeł!8-14}%
\index[persons]{Kanenobu, Taizo}%
\index[persons]{Murakami, Hitoshi}%
Stojmenow w~pracy \cite{stoimenow2001} pełnej różnych przykładów wskazał dodatkowo węzły $14_{36'750}$ oraz $14_{36'760}$.
\index{węzeł!14-36750}%
\index{węzeł!14-36760}%
\index[persons]{Stojmenow, Aleksander}%




\subsubsection{Liczba gordyjska jako metryka}
Mając dane dwa węzły $K_0, K_1$, rozpatrzmy wszystkie homotopie
\begin{equation}
    f \colon [0,1] \times S^1 \to \R^3
\end{equation}
takie, że wszystkie funkcje $f_t$ są zanurzeniami z co najwyżej jednym punktem podwójnym.
Zażądajmy dodatkowo, by styczne do krótkich łuków, które przecinają się w tym punkcie, były od siebie różne.
Odległością gordyjską między węzłami $K_0, K_1$ jest minimalna liczba podwójnych punktów, jakie posiada homotopia $f$.
Twierdzenie C~z~pracy Gambaudo, Ghysa \cite{gambaudo2005} głosi, że przestrzeń wszystkich węzłów wyposażona w taką metrykę zawiera prawie idealną kopię przestrzeni euklidesowej dowolnego wymiaru.
\index[persons]{Gambaudo, Jean-Marc}%
\index[persons]{Ghys, Étienne}%
Dokładniej:

\begin{proposition}
    Dla każdej liczby całkowitej $n \ge 1$ istnieje funkcja $\xi: \Z^n \to \mathcal{K}$, dodatnie stałe $A, B, C$ i norma $\|\cdot\|$ na przestrzeni $\R^n$ takie, że spełniona jest podwójna nierówność
    \begin{equation}
        A\|x-y\| - B \le d(\xi(x), \xi(y)) \le C\|x-y\|.
    \end{equation}
\end{proposition}

To nie jest koronne twierdzenie tamże, tylko efekt uboczny pracy nad głównym wynikiem: autorzy definiują $\omega$-sygnaturę domknięcia warkocza, a~że sklejenie dwóch 4-rozmaitości z~narożnikami nie odpowiada dodaniu ich sygnatur, to ich funkcja nie jest homomorfizmem.
\index{warkocz}%
Wspomniany jest wzór Novikowa-Walla, który wyraża różnicę pewnych defektów jako indeks Masłowa i (to jest główne twierdzenie) różnica ta pokrywa się z kocyklem Meyera reprezentacji Burau-Squiera, cokolwiek to znaczy.
Pojawia się również jakaś funkcja Rademachera.
\index{funkcja Rademachera}%
\index{indeks Masłowa}%
\index{kocykl Meyera}%
\index{reprezentacja Burau-Squiera}%
\index{wzór Novkowa-Walla}%

Grupy warkoczowe poznamy w sekcji~\ref{sec:braid}.



\subsubsection{Inne operacje rozwiązujące węzły}

Shimizu w pracy \cite{shimizu2014} rozpatruje różne operacje, które rozwiązują węzły lub sploty.
\index[persons]{Shimizu, Ayaka}%
Nie będziemy się nimi zajmować, podamy tylko przykład: zamiana pod- i nadskrzyżowań wokół obszaru na diagramie rozwiązuje węzły, ale nie sploty; kontrprzykładem jest splot Hopfa.
\index{splot!Hopfa}%
Patrz też, co pisze Kawauchi w \cite[s. 141-154]{kawauchi1996}.
% TODO: co pisze?

Mieliśmy też:

\begin{conjecture}
    Dowolny splot można rozwiązać wykonując ciąg 3-ruchów (zastępując dwie równoległe nici przez trzy półskręty lub odwrotnie).
\end{conjecture}

Ze zbioru problemów Kirby'ego \cite{kirby1978} wiemy, że Nakanishi zastanawiał się nad tym w 1981 roku.
\index[persons]{Nakanishi, Yasutaka}%
Nie to samo, ale podobne pytanie zadał wcześniej Montesinos w związku z nakryciami i~dlatego Kirby nazwał problem hipotezą Nakanishiego-Montesinosa.
\index[persons]{Montesinos, José}%
Conway zauważył, że hipoteza jest prawdziwa dla węzłów algebraicznych.
\index[persons]{Conway, John}%
Coxeter rozprawił się z nią dla prawie wszystkich splotów o~indeksie warkoczowym mniejszym niż $6$ oraz indeksie mostowym mniejszym niż $4$.
\index[persons]{Coxeter, Harold}%
Nakanishi w 1994 pokazał splot zbudowany z pierścieni Boromeuszy wobec którego podejrzewał, że jest kontrprzkyładem.
\index{pierścienie Boromeuszy}%
Żeby zdobyć więcej informacji o postępie prac nad hipotezą, musieliśmy sięgnąć po artykuł Przytyckiego, Dąbkowskiego \cite{dabkowski2002}.
\index[persons]{Przytycki, Józef}%
\index[persons]{Dąbkowski, Mieczysław}%
Chen w~1999 zasugerował inny kontrprzykład, domknięcie 5-warkocza $(\sigma_1\sigma_2\sigma_3\sigma_4)^{10}$.
\index[persons]{Chen, Qi}%
Artykuł \cite{dabkowski2002} dowodzi, że te dwa sploty istotnie obalają hipotezę.
Używa się w~nim nieprzemiennej wersji $n$-kolorowań Foxa, tak zwanej $n$-tej grupy Burnside'a splotu.
\index{grupa Burnside'a}%
\index{kolorowanie}%

Nakanishi w 1979, a więc zanim ogłosił powyższą hipotezę, miał wrażenie, że $4$-ruchy rozwiązują wszystkie sploty.
Najpierw sprawdzono, że jest prawdziwa dla wszystkich dwu- i trzymostowych węzłów, a także węzłów do 12 skrzyżowań, ale potem Askitas ogłosił, że pewien węzeł o 16 skrzyżowaniach obala ją.
\index[persons]{Askitas, Nikos}%
Później pojawili się inni podejrzani, ale nie wiemy, czy naprawdę są kontrprzykładami.

% znalezione przypadkiem w MR3143585
% 1979 Nakanishi: hipoteza że 4-ruch jest rozwiązujący
% dowody: 2-mostowe i 3-mostowe węzły, wszystkie do 12 skrzyżowań

\index{liczba gordyjska|)}%

% Koniec podsekcji Liczba gordyjska




% DICTIONARY;bridge number;liczba mostowa;-
\subsection{Liczba mostowa}
\index{liczba mostowa|(}%

Wprowadzona w~1954 przez Schuberta \cite{schubert1954}.
\index[persons]{Schubert, Horst}%

\begin{definition}
    Niech $D$ będzie diagramem węzła $K$.
    Liczbę mostów, czyli długich łuków, które biegną tylko przez nadskrzyżowania, nazywamy liczbą mostową diagramu $D$.
    Minimalną liczbę mostową wśród wszystkich diagramów $D$ węzła $K$, $\bridge K$, nazywamy liczbą mostową węzła $K$.
\end{definition}

% źródło: https://knotinfo.math.indiana.edu/descriptions/bridge_index.html
W 2012 roku Musick \cite{musick2012} znalazł liczbę mostową węzłów pierwszych o 11 skrzyżowaniach: węzły, które nie są ani wymierne, ani Montesinosa, są trzymostowe.
\index[persons]{Musick, Chad}%
Po ośmiu latach pchnięto granicę wiedzy do 12 skrzyżowań, dokonał tego międzynarodowy zespół: Blair, Kjuczukowa, Velazquez, Villanueva \cite{blair2020}.
\index[persons]{Blair, Ryan}%
\index[persons]{Kjuchukova, Alexandra}%
\index[persons]{Velazquez, Roman}%
\index[persons]{Villanueva, Paul}%

Jedynym węzłem jednomostowym jest niewęzeł.
Kolejne w~hierarchii skomplikowania, czyli dwumostowe, to domknięcia wymiernych supłów (piszemy o~nich w podsekcji \ref{sub:twobridge}).
Fukuhama, Ozawa, Teragaito \cite{fukuhama1999} pokazali, że trzymostowe węzły genusu jeden są preclami.
\index[persons]{Fukuhama, Satoshi}%
\index[persons]{Ozawa, Makoto}%
\index[persons]{Teragaito, Masakazu}%
\index{genus}%
\index{precel}%
\index{węzeł!trzymostowy}%
Hilden, Montesinos, Tejada, Toro \cite{hilden2012} klasyfikują wszystkie węzły trzymostowe przy użyciu tak zwanej reprezentacji motylkowej, podobną do wyniku Schuberta z~sekcji \ref{sub:twobridge}.
\index[persons]{Hilden, Hugh}%
\index[persons]{Montesinos, José}%
\index[persons]{Tejada, Débora}%
\index[persons]{Toro, Margarita}%
\index{liczba mostowa!węzeł trzymostowy}%
\index{węzeł!trzymostowy!see {liczba mostowa}}%
\index{reprezentacja!motylkowa}%

Węzły $n$-mostowe rozkładają się na sumę dwóch wymiernych $n$-supłów.
% źródło: Wiki Bridge_number, TODO: znaleźć lepsze źródło?
\index{supeł!wymierny}%

\begin{proposition}
\label{prp:bridge_additive}%
    Niech $K_1, K_2$ będą węzłami.
    Wtedy $\bridge (K_1) + \bridge(K_2) = \bridge(K_1 \# K_2) + 1$.
\end{proposition}

\begin{proof}[Niedowód]
\index[persons]{Schubert, Horst}%
\index[persons]{Schultens, Jennifer}%
    Schubert \cite[s. 279]{schubert1954} pokazał to blisko pół wieku temu.
    Nowszy dowód pochodzi od Schultens \cite{schultens2003}, skorzystała z~foliacji na brzegu węzła towarzyszącego satelitarnemu.
    Ale dokładniejszy opis powyższych prac wykraczałby poza zakres tego opracowania, więc zostanie pominięty.
\end{proof}

Murasugi wspomina w rozdziale 4.3 podręcznika \cite{murasugi1996} następującą hipotezę, nie podaje jednak wcale, skąd się wzięła:
\index[persons]{Murasugi, Kunio}%

\begin{conjecture}[mostowo-skrzyźowaniowa]
\index{hipoteza!mostowo-skrzyżowaniowa}%
    Niech $K$ będzie węzłem.
    Wtedy $\crossing K \ge 3 \bridge K - 3$, przy czym równość zachodzi dokładnie dla niewęzła, trójlistnika i~sumy spójnej trójlistników.
\end{conjecture}

Należy więc uzupełnić brakujące informacje.
Murasugi \cite{murasugi1988} przypuszcza, że dla splotów o~$\mu$~ogniwach zachodzi nierównosć $\crossing L + \mu - 1 \ge 3 \bridge L - 3$, przedstawia jednocześnie dowód jej szczególnego przypadku, dla alternujących splotów algebraicznych.
\index[persons]{Murasugi, Kunio}%
Hipoteza Murasugiego stanowi uogólnienie dużo starszego problemu pochodzącego od Foxa \cite{fox1950}, który zapytał, czy nierówność jest prawdziwa dla węzłów, gdy $\mu = 1$.
\index[persons]{Fox, Ralph}%

\begin{proposition}
\index{liczba gordyjska}%
\label{no_relation_bridge_unknotting}%
    Nie istnieje bezpośredni związek między liczbą mostową oraz gordyjską.
\end{proposition}

Wiemy o tym z książki Livingstona \cite[s. 146]{livingston1993}.

\begin{proof}
    Niech $K_n$ będzie węzłem $(2, 2n+1)$-torusowym.
    Wtedy $K_n$ jest dwumostowy i~jego liczba gordyjska wynosi $n$, to jest dokładnie treść hipotezy Milnora (patrz fakt \ref{prp:torus_unknotting_number}).

    Livingston pisze, że Schubert \cite[s. 281]{schubert1954} udowodnił następujący wynik.
    Jeśli węzeł $K'$ jest dublem węzła $K$, to $\bridge K' = 2 \bridge K$ chyba, że $K$ jest niewęzłem (Livingston mówi tylko, że jest jakiś wyjątek i~każe go znaleźć).
    Zatem wielokrotne dublowanie prowadzi do węzłów o~dowolnie dużej liczbie mostowej.
    Duble są 1-gordyjskie: rysunek 7.12 w książce Livingstona \cite[s. 146]{livingston1993} świetnie to pokazuje.
\end{proof}

% Podobnie nie ma zależności między liczbą mostową oraz genusem.
% \index{genus}%
% TODO: zamiast tego, z każdego miejsca dolinkować znowu do \subsection{Podsumowanie}

\index{liczba mostowa|)}%

% Koniec podsekcji Liczba mostowa




\subsection{Spin}
\index{spin|(}%

Niektórzy, na przykład Przytycki, używają określenia ,,liczba Taita'', ale nam się ono nie podoba, więc proponujemy nasze, lepsze.
\index{liczba Taita}%

% DICTIONARY;writhe;spin;-
\begin{definition}[spin]
    Niech $D$ będzie diagramem zorientowanego splotu $L$.
    Wtedy sumę wszystkich znaków skrzyżowań diagramu:
    \begin{equation}
        \writhe D = \sum_c \operatorname{sign} c,
    \end{equation}
    nazywamy jego spinem (diagramu, nie splotu!).
\end{definition}

Co ważne, spin nie jest niezmiennikiem splotów ani węzłów.
Para Perko przedstawia ten sam węzeł z~minimalną liczbą skrzyżowań i~spinem równym siedem lub dziewięć.
\index{para Perko}%
Dzięki temu przez wiele lat nie została dostrzeżona.
Spin jest za to niezmiennikiem węzłów alternujących, mówi o~tym druga hipoteza Taita.
\index{hipoteza!Taita}%

\begin{lemma}
\label{lem:writhe_reidemeister}%
    Spin nie zależy od orientacji diagramu.
    Tylko I ruch Reidemeistera zmienia spin:
\begin{comment}
    \begin{equation}
        \writhe \left(\MedLarReidemeisterOneLeft\right) =
        \writhe \left(\MedLarReidemeisterOneStraight\right) - 1.
    \end{equation}
\end{comment}
    Pozostałe ruchy nie mają na niego wpływu.
\end{lemma}

\index{spin|)}%

% Koniec sekcji Spin



% DICTIONARY;linking number;indeks zaczepienia;-
\section{Indeks zaczepienia}
\index{indeks zaczepienia|(}%
Około 1833 roku Gauß wyraził indeks zaczepienia dwóch węzłów jako pewną (nieciekawą dla nas) całkę, co czyni go najstarszym niezmiennikiem splotów.
\index[persons]{Gauß, Carl}%
Po przeczytaniu \cite{colberg2013} wydaje nam się, że osiągnął to korzystając z praw fizyki: prawa Ampère'a i Biota-Savarta.
My żyjemy w~XXI wieku, wystarczy nam diagramatyczna definicja.
(Ale patrz też: \cite[s. 11]{kawauchi1996}.)
% Erin Colberg - A brief history of knot theory
% TODO: czemu patrz też?

% DICTIONARY;sign;znak;skrzyżowanie
\begin{definition}[znak]
\index{znak skrzyżowania}%
    Liczbę $\pm 1$ przypisaną do skrzyżowania zgodnie z regułą:
\begin{comment}
    \setlength{\intextsep}{4pt plus 2pt minus 2pt}
    \begin{figure}[H]
        \begin{minipage}[b]{.48\linewidth}
            \[
                \sign \left( \MedLarPlusCrossingArrows \right) = +1
            \]
        \end{minipage}
        \begin{minipage}[b]{.48\linewidth}
            \[
                \sign \left( \MedLarMinusCrossingArrows \right) = -1
            \]
        \end{minipage}

    \end{figure}
\end{comment}
\noindent
    nazywamy znakiem skrzyżowania.
\end{definition}

Skrzyżowania dodatnie to takie, w których obrócenie dolnego łuku w prawo daje górny łuk, dlatego czasem nazywa się je także praworęcznymi.
Oczywiście skrzyżowania ujemne nazywamy wtedy leworęcznymi.
\index{skrzyżowanie!dodatnie i ujemne}%
\index{skrzyżowanie!lewo- i prawoskrętne}%

% DICTIONARY;smoothing;wygładzenie;-
\begin{definition}[wygładzenie]
\index{skrzyżowanie!... wygładzenie}%
    Diagramy powstałe przez zmianę biegu łuków danego skrzyżowania zgodnie z poniższymy rysunkami:
\begin{comment}
    {\setlength{\intextsep}{4pt plus 2pt minus 2pt}
    \begin{figure}[H]
        \setlength{\intextsep}{4pt plus 2pt minus 2pt}
        \begin{minipage}[b]{.48\linewidth}
            \[
                \MedLarAlphaSmoothing
            \]
            \subcaption{wygładzenie dodatnie}
        \end{minipage}
        \begin{minipage}[b]{.48\linewidth}
            \[
                \MedLarBetaSmoothing
            \]
            \subcaption{wygładzenie ujemne}
        \end{minipage}
    \end{figure}
    }
\end{comment}
\noindent
    nazywamy wygładzeniem.
    Jeżeli nie zaznaczono inaczej, wygładzamy zgodnie ze znakiem skrzyżowania.
\end{definition}

\begin{definition}[indeks zaczepienia]
    Niech $L = K_1 \sqcup K_2$ będzie splotem o dwóch ogniwach, zaś $D$ jego diagramem.
    Wielkość
    \begin{equation}
        \linking(K_1, K_2) = \frac 12 \sum_i \sign c_i,
    \end{equation}
    gdzie sumowanie rozciąga się na wszystkie skrzyżowania, na których spotykają się łuki różnych ogniw, nazywamy indeksem zaczepienia węzłów $K_1, K_2$.
    Ogólniej, jeśli dany jest splot $L = K_1 \sqcup \ldots \sqcup K_n$ posiadający $n$ ogniw, to jego indeks zaczepienia wyznacza wzór
    \begin{equation}
        \linking(L) = \sum_{i < j} \linking(K_i, K_j).
    \end{equation}
\end{definition}

Zauważmy, że indeks zaczepienia splotu Hopfa wynosi $1$, natomiast splotu Whiteheada $0$.
\index{splot!Hopfa}%
\index{splot!Whiteheada}%
Są zatem różne.
W obydwu przypadkach indeks zaczepienia jest liczbą całkowitą.
Istotnie, na mocy twierdzenia Jordana $\linking$ jest funkcją o całkowitych wartościach.

\begin{proposition}
    Indeks zaczepienia jest dobrze określonym niezmiennikiem zorientowanych splotów.
\end{proposition}

\begin{proof}
    Wielkość $\linking L$ jest sumą znaków pewnych skrzyżowań, zatem na mocy twierdzenia Reidemeistera wystarczy sprawdzić, jaki jest wpływ ruchów Reidemeistera na te składniki:
\begin{comment}
{\setlength{\intextsep}{4pt plus 2pt minus 2pt}
\begin{figure}[H]
\centering
    %
    \begin{minipage}[b]{.3\linewidth}
        \[
            \MedLarReidemeisterOneLeft \cong \MedLarReidemeisterOneStraight
        \]
        \subcaption{ruch $R_1$}
    \end{minipage}
    %
    \begin{minipage}[b]{.3\linewidth}
        \[
            \MedLarReidemeisterTwoLinkingA \cong \MedLarReidemeisterTwoB
        \]
        \subcaption{ruch $R_2$}
    \end{minipage}
    %
    \begin{minipage}[b]{.35\linewidth}
        \[
            \MedLarReidemeisterThreeLinkingA \cong \MedLarReidemeisterThreeLinkingB
        \]
        \subcaption{ruch $R_3$}
    \end{minipage}
\end{figure}
}
\end{comment}
\noindent
    Ὅπερ ἔδει δεῖξαι...
\end{proof}

\index{indeks zaczepienia|)}%

% koniec podsekcji Indeks zaczepienia



\subsection{Liczba patykowa}
\index{liczba patykowa|(}%

% DICTIONARY;stick number;liczba patykowa;-

\begin{definition}
    Minimalną liczbę odcinków w~łamanej, która przedstawia węzeł $K$, nazywamy jego liczbą patykową i~oznaczamy $\stick(K)$.
\end{definition}

Wielkość tę wprowadził do matematyki Randell \cite{randell1998} i~znalazł dokładną jej wartość dla niewęzła (3), trójlistnika (6), ósemki (7) i $5_1, 5_2, 6_1, 6_2, 6_3$ (8).
\index[persons]{Randell, Richard}%
Negami \cite{negami1991} pokazał trzy lata później przy użyciu teorii grafów, że dla nietrywialnych węzłów prawdziwe są nierówności
\index[persons]{Negami, Seiya}%
\begin{equation}
    \frac{5+\sqrt{9 + 8 \crossing K}}{2} \le \stick K \le 2 \crossing K.
\end{equation}

Trójlistnik to jedyny węzeł realizujący górne ograniczenie.
% Huh, Oh 2011 z trójlistnikiem?
Z~pracy Elrifaia \cite{elrifai2006} (a wiemy o~niej z~\cite{huh2011}) wynika, że dla węzłów o~co najwyżej 26 skrzyżowaniach, dolne ograniczenie jest ostre: można pisać $<$ w~miejsce $\le$.
\index[persons]{Elrifai, Elsayed}%

W encyklopedii Wolfram Mathworld przeczytaliśmy, że dokładnie jeden węzeł pierwszy do 10 skrzyżowań ma liczbę patykową równą 14: jest to $10_{84}$, dokładnie pięć ma liczbę patykową równą 13: $10_{39}$, $10_{64}$, $10_{73}$, $10_{76}$, $10_{80}$, liczba patykowych pozostałych nie przekracza 12.
Niestety Wolfram nie podaje źródeł tych rewelacji.

Jin oraz Kim w 1993 ograniczyli liczby patykowe dla węzłów torusowych korzystając z~liczby supermostowej.
\index[persons]{Jin, Gyo}%
\index[persons]{Kim, Hyoung-Seok}%
Wkrótce wynik został poprawiony przez samego Jina, w pracy \cite{jin1997} znalazł dokładne wartości dla niektórych węzłów.
I~tak, jeśli $2 \le p < q < 2p$, to $\stick T_{p,q} = 2q$ oraz $\stick T_{p, 2p} = 4p-1$.
Ten sam wynik, choć dla węższego zakresu parametrów, odkryli Adams, Brennan, Greilsheimer, Woo \cite{greilsheimer1997}.
\index[persons]{Adams, Colin}%
\index[persons]{Brennan, Bevin}%
\index[persons]{Greilsheimer, Deborah}%
\index[persons]{Woo, Alexander}%
\index{suma spójna}%
Autorzy niezależnie od siebie znaleźli proste oszacowanie z~góry dla liczby patykowej sumy spójnej:
\begin{equation}
    \stick(K_1 \shrap K_2) \le \stick(K_1) + \stick(K_2) - 3.
\end{equation}

Koniec dekady przyniósł jeszcze jedną pracę McCabe z~nierównością $\stick(K) \le 3 + \crossing (K)$ dla węzłów dwumostowych (\cite{mccabe1998}) oraz odkrycie Calvo \cite{calvo2001}: jeśli ograniczymy się do łamanych o co najwyżej siedmiu odcinkach, ósemka przestaje być odwracalna.
\index[persons]{McCabe, Cynthia}%
\index[persons]{Calvo, Jorge}%

Na początku XXI wieku nierówności Negamiego poprawiono, z dołu dokonał tego Calvo w~\cite{calvo2001}, z góry natomiast Huh, Oh w \cite{huh2011}.
\index[persons]{Calvo, Jorge}%
\index[persons]{Huh, Youngsik}%
\index[persons]{Oh, Seungsang}%
% huh11: simple and self-contained
Górne ograniczenie można zmniejszyć o $3/2$, jeżeli $K$ jest niealternującym węzłem pierwszym.
\begin{equation}
    \frac{7+\sqrt{1 + 8 \crossing K}}{2} \le \stick K \le \frac{3}{2} (1 + \crossing K).
\end{equation}

Liczba patykowa nie pojawia się już nigdzie w następnych rozdziałach.

% https://knotinfo.math.indiana.edu/descriptions/polygon_index.html wspomina jeszcze <[6] Shonkwilker, C., "All prime knots through 10 crossings have superbridge Index <= 5. Arxiv preprint.>

\index{liczba patykowa|)}%

% Koniec podsekcji Liczba patykowa




\section{Długość sznurowa}
\index{długość sznurowa|(}%
% DICTIONARY;ropelength;długość sznurowa;-
Matematyczne węzły nie mają grubości i można je dowolnie rozciągać.
Długość sznurowa, najsłabiej poznany niezmiennik numeryczny, pochodzi z~fizycznej teorii węzłów, która bierze pod uwagę obiekty wykonane z~nieelastycznych materiałów.

\begin{definition}
    Niech $L$ będzie splotem o długości $l$, który posiada rurowe otoczenie bez samoprzecięć z~przekrojem poprzecznym o~promieniu $\tau$ (mówimy, że $L$ ma grubość $\tau$).
    Iloraz
    \begin{equation}
        \ropelength L = \frac l \tau
    \end{equation}
    nazywamy długością sznurową splotu.
\end{definition}

Przez wiele lat zastanawiano się: czy można zawiązać węzeł ze sznura o~długości jednej stopy i~promieniu jednego cala?
Lub równoważnie:

\begin{conjecture}
    Czy istnieje nietrywialny węzęł $K$ taki, że $\ropelength K \le 12$?
\end{conjecture}

Na początku XXI wieku wiedzieliśmy dzięki Cantarelli, Kusnerowi, Sullivanowi \cite{cantarella2002}, że najkrótszy węzeł ma długość co najmniej $(2 + \sqrt 2)\pi \approx 10.726$, krótko po tym Diao \cite[s. 14]{diao2003} udzielił negatywnej odpowiedzi na pytanie.
\index[persons]{Cantarella, Jason}%
\index[persons]{Kusner, Robert}%
\index[persons]{Sullivan, John}%
\index[persons]{Diao, Yuanan}%

\begin{example}
    Rozumowanie Denne, Diao, Sullivana \cite{denne2006} oparte o~czterosieczne pokazuje, że długość sznurowa nietrywialnego węzła wynosi co najmniej $15.66$.
\index[persons]{Denne, Elizabeth}
\index[persons]{Sullivan, John}
\index[persons]{Diao, Yuanan}
Ale eksperymenty komputerowe pokazują, że długość trójlistnika nie przekracza $16.372$, więc oszacowanie jest dość ostre.
\end{example}

Prowadzono obszerne poszukiwania na temat zależności między długością sznurową i~innymi niezmiennikami.
Mamy na przykład:

\begin{proposition}
    Istnieją stałe $c_1, c_2$ takie, że $c_1 \crossing^{3/4} K \le \ropelength K \le c_2 \crossing^{3/2} K$.
\end{proposition}

Udowodniono, że dolnym ograniczeniem na czynnik $c_1$ jest $(4\pi/11)^{3/4} \approx 1.105$ (gdyż tak pisze Cantarella i~inni w~\cite[tw. 23]{cantarella2002}).
Wiemy też dzięki Klotzowi i Maldonado \cite{klotz2021}, że stała $c_1$ nie może przekraczać $12.64$, w przeciwnym razie nierówność nie zachodziłaby dla węzła torusowego $T_{3,5}$.
\index[persons]{Klotz, Alexander}%
\index[persons]{Maldonado, Matthew}%
Dowód górnego ograniczenia opiera się na cyklach Hamiltona w~grafach zanurzonych w~kratach liczbowych (Yu \cite{yu2004}).

Obydwa ograniczenia poprawiono.
Klotz, Maldonado \cite{klotz2021} piszą, że Diao dostał:

\begin{proposition}
    Niech $K$ będzie węzłem.
    Wtedy
    \begin{equation}
        \frac 12 \left(17.334 + \sqrt{17.334^2 + 64 \pi \crossing K}\right) \le \ropelength K.
    \end{equation}
\end{proposition}

To oszacowanie jest lepsze dla węzłów do 1850 skrzyżowań (chociaż nigdy nie widzieliśmy takiego potwora na własne oczy).

% Ograniczenie to realizowane jest przez pewne węzły torusowe oraz sploty Hopfa. - wikipedia o c_1

\begin{proposition}
    Niech $K$ będzie węzłem.
    Wtedy $\ropelength K = O(\crossing K \cdot \log^5(\crossing K))$.
\end{proposition}

\begin{proof}[Niedowód]
    Świeży wynik Diao, Ernsta, Pora, Zieglera \cite{diao2019}, którego dowód też wykorzystuje kraty liczbowe.
\end{proof}

Jakościowy wynik znaleźliśmy znowu w~\cite{klotz2021}: zachodzi $\ropelength L \le a_u \crossing K \log^5 \crossing K$ (nie tylko dla węzłów, ale też splotów), przy czym stała $a_u$ musi być większa od $8\pi/\log^5 2 \approx 78.5$, by dobrze ograniczała splot Hopfa.

Przywołajmy pracę \cite{klotz2021} jeszcze jeden, ostatni raz.
Badania satelitów do 42 skrzyżowań pokazały, że satelita jest zazwyczaj trzy razy dłuższy niż jego towarzysz.
% tak piszą w recenzji na MathSciNecie.

Długość sznurowa nie pojawia się w~dalszych rozdziałach.

\index{długość sznurowa|)}%

% Koniec podsekcji Długość sznurowa



\subsection{Podsumowanie}
(Podana dalej lista może być niezrozumiała przy pierwszym czytaniu).
Między niektórymi niezmiennikami nie ma bezpośredniego związku:
\begin{enumerate}
    \item między liczbą gordyjską i~mostową (fakt \ref{no_relation_bridge_unknotting}),
    \item między liczbą mostową i~genusem (wzmianka po fakcie \ref{no_relation_bridge_unknotting}),
    \item między defektami modulo różne liczby pierwsze (fakt \ref{no_relation_defects}),
    \item między liczbą mostową i~wielomianem Alexandera (fakt \ref{no_relation_bridge_alexander}),
    \item między wielomianem Jonesa i~Alexandera (paragraf przed faktem \ref{homfly_stronger}),
    \item między liczbą mostową i~sygnaturą (wniosek \ref{no_relation_signature_bridge}),
    \item między liczbą gordyjską i~wielomianem Alexandera (dowód faktu \ref{balanced_iff_four_conditions}).
\end{enumerate}

Ale między niektórymi innymi takie związki są:
\begin{enumerate}
    \item między indeksem skrzyżowaniowym i genusem % TODO: \cite{livingston1993}, 142
    \item między liczbą gordyjską i sygnaturą % TODO: \cite{livingston1993}, 142
\end{enumerate}

% Koniec sekcji Niezmienniki liczbowe

