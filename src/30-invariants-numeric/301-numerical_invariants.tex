
Wspomnieliśmy na s. \pageref{page_first_invariant}, że pytanie, czy dwa dane diagramy splotów przedstawiają ten sam czy różne sploty, bywa trudne.
Napomknęliśmy też, że świadkiem równości dwóch splotów jest ciąg ruchów Reidemeistera, natomiast niezmienników używa się do wykazania różności.
Czym jest niezmiennik?
To pewna wielkość, która nie ulega zmianie (w naszym przypadku: podczas izotopii otaczającej przestrzeni, w której jest zanurzony splot).
Jeśli dany niezmiennik przyjmuje różne wartości dla dwóch splotów, to nie mogą być równoważne.

Jak pisze Przytycki w~,,Dwustu latach teorii węzłów'', kiedy badamy nowy niezmiennik, powinniśmy zadać sobie trzy pytania.
\index[persons]{Przytycki, Józef}%
Dobrze byłoby o nich nie zapominać.
Oto pytania Przytyckiego:
\begin{enumerate}
    \item czy łatwo wyznaczyć wartość niezmiennika?
    \item czy w zbiorze wartości niezmiennika łatwo odróżnia się elementy?
    \item czy niezmiennik odróżnia wiele splotów?
\end{enumerate}

Poznaliśmy dotychczas dwa niezmienniki: liczbę ogniw (którą łatwo wyznaczyć i której wartości -- liczby naturalne -- łatwo odróżniać; ale nie odróżnia wielu splotów) oraz topologię dopełnienia splotu (tym razem odróżnia wszystkie węzły pierwsze, ale trudno wyznaczyć jej ,,wartość'').
Tutaj przedstawiamy kilka więcej; przede wszystkim te, które nie wymagają mocnej znajomości reszty książki.
Niektóre z~nich są miarą złożoności splotów zgodnie z~następującym przepisem: niech $f$ będzie pewną funkcją określoną dla dowolnego diagramu splotu.
Wtedy odwzorowanie
\begin{equation}
    f(L) = \min \{f(D) : D \text{ jest diagramem splotu } L\}
\end{equation}
stanowi niezmiennik splotów.
Dowód jest trywialny i~pozostawiamy go jako ćwiczenie dla Czytelnika.
Im większa wartość funkcji $f$, tym bardziej skomplikowany splot.

Później poznamy inne niezmienniki, oprócz opisanych poniżej miarą złożoności jest też liczba warkoczowa (definicja \ref{def:braid_number}), ale nie wyznacznik (definicja \ref{def:determinant}) czy sygnatura (definicja \ref{def:signature}).
Przekonamy się też, że istnieją użyteczne niezmienniki, które są wielomianami albo innymi obiektami algebraicznymi.

\input{30-invariants-numeric/301a-crossing}


% DICTIONARY;unknotting number;liczba gordyjska;-
\section{Liczba gordyjska}
\index{liczba!gordyjska|(}%

\begin{definition}
    Niech $L$ będzie splotem.
    Minimalną liczbę skrzyżowań, które trzeba odwrócić na pewnym jego diagramie, by dostać niewęzeł, nazywamy liczbą gordyjską i~oznaczamy $\unknotting L$.
\end{definition}

Zgodnie z ,,klasyczną'' definicją, między odwracaniem kolejnych skrzyżowań mamy prawo wykonać izotopie otaczające; natomiast zgodnie ze ,,standardową'' definicją, takie izotopie są zabronione.
Obie definicje są równoważne: tłumaczy to książka Adamsa \cite[s. 58]{adams1994}.
(Książka nie tłumaczy, czemu te dwie definicje zostały określone akurat takimi przymiotnikami.)

\begin{lemma}
\label{lem:unknotting_well_defined}%
    W dowolnym rzucie splotu można odwrócić pewne skrzyżowania tak, by uzyskać diagram niesplotu.
\end{lemma}

(To jest \cite[ćwiczenie E 1.6, s. 15]{burde2014}.)

\begin{proof}
    Bez straty ogólności załóźmy, że diagram przedstawia węzeł.
    Ustalmy zatem diagram węzła i~wybierzmy jakiś początkowy punkt na nim, różny od skrzyżowania wraz z~kierunkiem, wzdłuż którego będziemy przemierzać węzeł.
    Za każdym razem, kiedy odwiedzamy nowe skrzyżowanie, zmieniamy je w~razie potrzeby na takie, przez które przemieszczamy się wzdłuż górnego łuku.
    Skrzyżowań już odwiedzonych nie zmieniamy wcale.

    Teraz wyobraźmy sobie nasz nowy węzeł w~trójwymiarowej przestrzeni $\mathbb R^3$, przy czym oś $z$ skierowana jest z~płaszczyzny, w~której leży diagram, w~naszą stronę.
    Umieśćmy początkowy punkt tak, by jego trzecią współrzędną była $z = 1$.

    Przemierzając węzeł, zmniejszamy stopniowo tę współrzędną, aż osiągniemy wartość $0$ tuż przed punktem, z~którego wyruszyliśmy.
    Połączmy obydwa punkty (początkowy oraz ten, w~którym osiągamy współrzędną $z = 0$) pionowym odcinkiem.
    Zauważmy, że kiedy patrzymy na węzeł w~kierunku osi $z$, nie widzimy żadnych skrzyżowań.

    Oznacza to, że nasza procedura przekształciła początkowy diagram w~diagram niewęzła, co należało okazać.
\end{proof}

Nakanishi \cite{nakanishi1983} znalazł 2-gordyjski diagram 1-gordyjskiego węzła $6_2$, a po trzynastu latach udowodnił, że każdy nietrywialny węzeł ma diagram, który nie jest 1-gordyjski \cite{nakanishi1996}.
\index[persons]{Nakanishi, Yasutaka}%
Jego wyniki uogólnia praca Taniyamy \cite{taniyama2009}: dla każdego nietrywialnego splotu istnieje diagram wymagający odwrócenia dowolnie wielu skrzyżowań.
\index[persons]{Taniyama, Kouki}%

\begin{proposition}
    Niech $L$ będzie nietrywialnym splotem.
    Dla każdej liczby naturalnej $n \in \N$ istnieje diagram $D$ splotu $L$ taki, że $\unknotting D \ge n$.
\end{proposition}

Pokazany jest tam jeszcze jeden godny uwagi fakt.
Jeśli odwrócenie pewnych skrzyżowań daje niewęzeł, to odwrócenie pozostałych także.
Zatem dla splotów $L$ o $n$ skrzyżowaniach mamy $2 \crossing L \le n$.
Nie jest to zbyt pomocne, daje rozstrzygnięcie pięć razy dla pierwszych węzłów do 12 skrzyżowań: $3_{1}$, $5_{1}$, $7_{1}$, $9_{1}$, $11a_{367}$.
Ale...

\begin{proposition}
    Jeśli liczba gordyjska diagramu $D$ węzła $K$ wynosi $\frac 12 (\crossing D - 1)$, to węzeł jest $(2,p)$-torusowy albo wygląda jak diagram niewęzła po pierwszym ruchu Reidemeistera.
\end{proposition}

W powyższym stwierdzeniu nie można zastąpić słowa ,,węzeł'' przez ,,splot''.

\input{30-invariants-numeric/301ba-unknotting_one}


\subsubsection{Znane wartości}
Cha, Livingston \cite{cha2018} podają, że znamy liczby gordyjskie wszystkich węzłów pierwszych do dziesięciu skrzyżowań poza dziewięcioma wyjątkami: $10_{11}$, $10_{47}$, $10_{51}$, $10_{54}$, $10_{61}$, $10_{76}$, $10_{77}$, $10_{79}$, $10_{100}$ (gdzie nie mamy pewności, czy $\unknotting = 2$, czy $\unknotting = 3$).
\index[persons]{Cha, Jae}%
\index[persons]{Livingston, Charles}%
Kto pierwszy znalazł liczbę gordyjską którego węzła ostaraliśmy się bezbłędnie przepisać z bazy danych KnotInfo\footnote{Patrz \url{https://knotinfo.math.indiana.edu/descriptions/unknotting_number.html}}.
Według KnotInfo oprócz węzłów torusowych, tych wymienionych poniżej oraz 1-gordyjskich, do 10 skrzyżowań mamy jeszcze 2 węzły o~siedmiu skrzyżowaniach, 3 o~ośmiu, 15 o~dziewięciu i~68 o~dziesięciu, których liczba gordyjska zdaje się należeć do folkloru matematycznego.

{
    \setlength{\intextsep}{4pt plus 2pt minus 2pt}
\begin{table}[H]
    \raggedright
    \footnotesize
    \centering
    \begin{tabular}{l|p{100mm}} \toprule
    rok & węzły i odkrywcy ich liczb gordyjskich \\ \midrule
    1982 & $7_{4}$ (Lickorish \cite{lickorish1985}) \\
    1986 & $8_{4}, 8_{6}, 8_{8}, 8_{12}, 9_{5}, 9_{8}, 9_{15}, 9_{17}, 9_{31}$ (Kanenobu, Murakami \cite{kanenobumurakami1986}) \\
    1989 & $9_{25}$ (Kobayashi \cite{kobayashi1989}) \\
    1994 & $10_{8}$ (Adams \cite[s. 62]{adams1994}?) \\
    1998 & $10_{65}, 10_{69}, 10_{89}, 10_{97}, 10_{108}, 10_{163}, 10_{165}$ (Miyazawa \cite{miyazawa1998}), $10_{154}, 10_{161}$ (Tanaka \cite{tanaka1998}) \\
    1999 & $10_{67}$ (Traczyk \cite{traczyk1999}) \\
    2000 & $8_{16}$ (Murakami, Yasuhara \cite{yasuhara2000}) \\
    2002 & $10_{139}, 10_{145}, 10_{152}, 10_{154}, 10_{161}$ (Gibson, Ishikawa \cite{ishikawa2002}) \\
    2004 & $8_{18}, 9_{37}, 9_{40}, 9_{46}, 9_{48}, 9_{49}, 10_{86}, 10_{103}, 10_{105}, 10_{106}, 10_{109}, 10_{121}, 10_{131}$ (Stojmenow \cite{stoimenow2004}; ostatni węzeł zdaje się być jedynym 1-gordyjskim na tej liście!) \\
    2005 & $9_{29}$, $10_{79}$, $10_{81}$, $10_{87}$, $10_{90}$, $10_{93}$, $10_{94}$, $10_{96}$, $10_{148}$, $10_{151}$, $10_{153}$ (Gordon, Luecke \cite{gordon2006}), $8_{10}$, $10_{48}$, $10_{52}$, $10_{54}$, $10_{57}$, $10_{58}$, $10_{64}$, $10_{68}$, $10_{70}$, $10_{77}$, $10_{110}$, $10_{112}$, $10_{116}$, $10_{117}$, $10_{125}$, $10_{126}$, $10_{130}$, $10_{135}$, $10_{138}$, $10_{158}$, $10_{162}$ (Ozsváth, Szabó \cite{szabo2005}), $10_{83}$ (Nakanishi \cite{nakanishi2005}) \\
    2008 & $9_{10}, 9_{13}, 9_{35}, 9_{38}, 10_{53}, 10_{101}, 10_{120}$ (Owens \cite{owens2008}) \\
    \bottomrule
    \hline
    \end{tabular}
% 50 za dużo -< aż do przykład NB
% 25 trochę za dużo  \vspace{-25pt} ^ to samo
% 5 za mało
\end{table}
}
\index[persons]{Lickorish, William}%
\index[persons]{Kanenobu, Taizo}%
\index[persons]{Murakami, Hitoshi}%
\index[persons]{Kobayashi, Tsuyoshi}%
\index[persons]{Adams, Colin}%
\index[persons]{Miyazawa, Yasuyuki}%
\index[persons]{Tanaka, Toshifumi}%
\index[persons]{Traczyk, Paweł}%
\index[persons]{Yasuhara, Akira}%
\index[persons]{Gibson, William}%
\index[persons]{Ishikawa, Masaharu}%
\index[persons]{Stojmenow, Aleksander}%
\index[persons]{Gordon, Cameron}%
\index[persons]{Luecke, John}%
\index[persons]{Nakanishi, Yasutaka}%
\index[persons]{Szabó, Zoltán}%
\index[persons]{Ozsváth, Peter}%
\index[persons]{Owens, Brendan}%
\normalsize




\subsubsection{Dolne ograniczenia liczby gordyjskiej}
Dokładna wartość liczby gordyjskiej jest znana tylko dla niektórych klas węzłów, na przykład torusowych (fakt \ref{prp:torus_unknotting_number}) albo skręconych.
\index{węzeł!torusowy}%
\index{węzeł!skręcony}%

Borodzik oraz Friedl podali niedawno całkiem mocne ograniczenia na liczbę gordyjską w~pracach \cite{borodzik2014} i~\cite{borodzik2015}.
\index[persons]{Borodzik, Maciej}%
\index[persons]{Friedl, Stefan}%
Ich narzędziem jest parowanie Blanchfielda.
\index{parowanie Blanchfielda}%
Poprawiają tam starsze estymaty wynikające z~sygnatury Levine'a-Tristrama, indeksu Nakanishiego oraz przeszkody Lickorisha.
\index{indeks Nakanishiego}%
\index{przeszkoda Lickorisha}%
\index{sygnatura!Levine'a-Tristrama}%
% DICTIONARY;Lickorish obstruction;przeszkoda Lickorisha;-
Wśród pierwszych węzłów o~co najwyżej 12 skrzyżowaniach dwadzieścia pięć ma liczbę gordyjską równą co najmniej trzy, trudno było uzasadnić to innymi metodami.



\input{30-invariants-numeric/301bd-bleiler}

\input{30-invariants-numeric/301be-metric}

\subsubsection{Inne operacje rozwiązujące węzły}

Shimizu w pracy \cite{shimizu2014} rozpatruje różne operacje, które rozwiązują węzły lub sploty.
\index[persons]{Shimizu, Ayaka}%
Nie będziemy się nimi zajmować, podamy tylko przykład: zamiana pod- i nadskrzyżowań wokół obszaru na diagramie rozwiązuje węzły, ale nie sploty; kontrprzykładem jest splot Hopfa.
\index{splot!Hopfa}%
(Ciąg dalszy tamtej pracy razem z Oshikirim, Tamurą \cite{oshikiri2024}).
\index[persons]{Tamura, Junya}%
\index[persons]{Oshikiri, Tokio}%
Patrz też, co pisze Kawauchi w \cite[s. 141-154]{kawauchi1996}.

Mieliśmy też:

\begin{conjecture}
    Dowolny splot można rozwiązać wykonując ciąg 3-ruchów (zastępując dwie równoległe nici przez trzy półskręty lub odwrotnie).
\end{conjecture}

Ze zbioru problemów Kirby'ego \cite{kirby1978} wiemy, że Nakanishi zastanawiał się nad tym w 1981 roku.
\index[persons]{Nakanishi, Yasutaka}%
Nie to samo, ale podobne pytanie zadał wcześniej Montesinos w związku z nakryciami i~dlatego Kirby nazwał problem hipotezą Nakanishiego-Montesinosa.
\index[persons]{Montesinos, José}%
Conway zauważył, że hipoteza jest prawdziwa dla węzłów algebraicznych.
\index[persons]{Conway, John}%
Coxeter rozprawił się z nią dla prawie wszystkich splotów o~indeksie warkoczowym mniejszym niż $6$ oraz indeksie mostowym mniejszym niż $4$.
\index[persons]{Coxeter, Harold}%
Nakanishi w 1994 pokazał splot zbudowany z pierścieni Boromeuszy wobec którego podejrzewał, że jest kontrprzykładem.
\index{pierścienie Boromeuszy}%
Żeby zdobyć więcej informacji o postępie prac nad hipotezą, musieliśmy sięgnąć po artykuł Przytyckiego, Dąbkowskiego \cite{dabkowski2002}.
\index[persons]{Przytycki, Józef}%
\index[persons]{Dąbkowski, Mieczysław}%
Chen w~1999 zasugerował inny kontrprzykład, domknięcie 5-warkocza $(\sigma_1\sigma_2\sigma_3\sigma_4)^{10}$.
\index[persons]{Chen, Qi}%
Artykuł \cite{dabkowski2002} dowodzi, że te dwa sploty istotnie obalają hipotezę.
Używa się w~nim nieprzemiennej wersji $n$-kolorowań Foxa, tak zwanej $n$-tej grupy Burnside'a splotu.
\index{grupa!Burnside'a}%
\index{kolorowanie}%

Nakanishi w 1979, a więc zanim ogłosił powyższą hipotezę, miał wrażenie, że $4$-ruchy rozwiązują wszystkie sploty.
Najpierw sprawdzono, że jest prawdziwa dla wszystkich dwu- i trzymostowych węzłów, a także węzłów do 12 skrzyżowań, ale potem Askitas ogłosił, że pewien węzeł o 16 skrzyżowaniach obala ją.
\index[persons]{Askitas, Nikos}%
Później pojawili się inni podejrzani, ale nie wiemy, czy naprawdę są kontrprzykładami.

% znalezione przypadkiem w MR3143585
% 1979 Nakanishi: hipoteza że 4-ruch jest rozwiązujący
% dowody: 2-mostowe i 3-mostowe węzły, wszystkie do 12 skrzyżowań

\index{liczba!gordyjska|)}%

% Koniec podsekcji Liczba gordyjska



\input{30-invariants-numeric/301c-bridge}


\subsection{Spin}
\index{spin|(}%

Niektórzy, na przykład Przytycki, używają określenia ,,liczba Taita'', ale nam się ono nie podoba, więc proponujemy nasze, lepsze.
\index{liczba!Taita}%

% DICTIONARY;writhe;spin;-
\begin{definition}[spin]
    Niech $D$ będzie diagramem zorientowanego splotu $L$.
    Wtedy sumę wszystkich znaków skrzyżowań diagramu:
    \begin{equation}
        \writhe D = \sum_c \operatorname{sign} c,
    \end{equation}
    nazywamy jego spinem (diagramu, nie splotu!).
\end{definition}

Co ważne, spin nie jest niezmiennikiem splotów ani węzłów.
Para Perko przedstawia ten sam węzeł z~minimalną liczbą skrzyżowań i~spinem równym siedem lub dziewięć.
\index{para Perko}%
Dzięki temu przez wiele lat nie została dostrzeżona.
Spin jest za to niezmiennikiem węzłów alternujących, mówi o~tym druga hipoteza Taita.
\index{hipoteza!Taita}%

\begin{lemma}
\label{lem:writhe_reidemeister}%
    Spin nie zależy od orientacji diagramu.
    Tylko I ruch Reidemeistera zmienia spin:
\begin{comment}
    \begin{equation}
        \writhe \left(\MedLarReidemeisterOneLeft\right) =
        \writhe \left(\MedLarReidemeisterOneStraight\right) - 1.
    \end{equation}
\end{comment}
    Pozostałe ruchy nie mają na niego wpływu.
\end{lemma}

\index{spin|)}%

% Koniec sekcji Spin




% DICTIONARY;linking number;indeks zaczepienia;-
\section{Indeks zaczepienia}
\index{indeks!zaczepienia|(}%
Około 1833 roku Gauß wyraził indeks zaczepienia dwóch węzłów jako pewną (nieciekawą dla nas) całkę, co czyni go najstarszym niezmiennikiem splotów.
\index[persons]{Gauß, Carl}%
Po przeczytaniu \cite{colberg2013} wydaje nam się, że osiągnął to korzystając z praw fizyki: prawa Ampère'a i Biota-Savarta.
My żyjemy w~XXI wieku, wystarczy nam diagramatyczna definicja.
(Ale patrz też: \cite[s. 11]{kawauchi1996}.)
% Erin Colberg - A brief history of knot theory
% TODO: czemu patrz też?

% DICTIONARY;sign;znak;skrzyżowanie
\begin{definition}[znak]
\index{znak skrzyżowania}%
    Liczbę $\pm 1$ przypisaną do skrzyżowania zgodnie z regułą:
\begin{comment}
    \setlength{\intextsep}{4pt plus 2pt minus 2pt}
    \begin{figure}[H]
        \begin{minipage}[b]{.48\linewidth}
            \[
                \sign \left( \MedLarPlusCrossingArrows \right) = +1
            \]
        \end{minipage}
        \begin{minipage}[b]{.48\linewidth}
            \[
                \sign \left( \MedLarMinusCrossingArrows \right) = -1
            \]
        \end{minipage}

    \end{figure}
\end{comment}
\noindent
    nazywamy znakiem skrzyżowania.
\end{definition}

Skrzyżowania dodatnie to takie, w których obrócenie dolnego łuku w prawo daje górny łuk, dlatego czasem nazywa się je także praworęcznymi.
Oczywiście skrzyżowania ujemne nazywamy wtedy leworęcznymi.
\index{skrzyżowanie!dodatnie i ujemne}%
\index{skrzyżowanie!lewo- i prawoskrętne}%

% DICTIONARY;smoothing;wygładzenie;-
\begin{definition}[wygładzenie]
\index{skrzyżowanie!... wygładzenie}%
    Diagramy powstałe przez zmianę biegu łuków danego skrzyżowania zgodnie z poniższymy rysunkami:
\begin{comment}
    {\setlength{\intextsep}{4pt plus 2pt minus 2pt}
    \begin{figure}[H]
        \setlength{\intextsep}{4pt plus 2pt minus 2pt}
        \begin{minipage}[b]{.48\linewidth}
            \[
                \MedLarAlphaSmoothing
            \]
            \subcaption{wygładzenie dodatnie}
        \end{minipage}
        \begin{minipage}[b]{.48\linewidth}
            \[
                \MedLarBetaSmoothing
            \]
            \subcaption{wygładzenie ujemne}
        \end{minipage}
    \end{figure}
    }
\end{comment}
\noindent
    nazywamy wygładzeniem.
    Jeżeli nie zaznaczono inaczej, wygładzamy zgodnie ze znakiem skrzyżowania.
\end{definition}

\begin{definition}[indeks zaczepienia]
    Niech $L = K_1 \sqcup K_2$ będzie splotem o dwóch ogniwach, zaś $D$ jego diagramem.
    Wielkość
    \begin{equation}
        \linking(K_1, K_2) = \frac 12 \sum_i \sign c_i,
    \end{equation}
    gdzie sumowanie rozciąga się na wszystkie skrzyżowania, na których spotykają się łuki różnych ogniw, nazywamy indeksem zaczepienia węzłów $K_1, K_2$.
    Ogólniej, jeśli dany jest splot $L = K_1 \sqcup \ldots \sqcup K_n$ posiadający $n$ ogniw, to jego indeks zaczepienia wyznacza wzór
    \begin{equation}
        \linking(L) = \sum_{i < j} \linking(K_i, K_j).
    \end{equation}
\end{definition}

Zauważmy, że indeks zaczepienia splotu Hopfa wynosi $1$, natomiast splotu Whiteheada $0$.
\index{splot!Hopfa}%
\index{splot!Whiteheada}%
Są zatem różne.
W obydwu przypadkach indeks zaczepienia jest liczbą całkowitą.
Istotnie, na mocy twierdzenia Jordana $\linking$ jest funkcją o całkowitych wartościach.

\begin{proposition}
    Indeks zaczepienia jest dobrze określonym niezmiennikiem zorientowanych splotów.
\end{proposition}

\begin{proof}
    Wielkość $\linking L$ jest sumą znaków pewnych skrzyżowań, zatem na mocy twierdzenia Reidemeistera wystarczy sprawdzić, jaki jest wpływ ruchów Reidemeistera na te składniki:
\begin{comment}
{\setlength{\intextsep}{4pt plus 2pt minus 2pt}
\begin{figure}[H]
\centering
    %
    \begin{minipage}[b]{.3\linewidth}
        \[
            \MedLarReidemeisterOneLeft \cong \MedLarReidemeisterOneStraight
        \]
        \subcaption{ruch $R_1$}
    \end{minipage}
    %
    \begin{minipage}[b]{.3\linewidth}
        \[
            \MedLarReidemeisterTwoLinkingA \cong \MedLarReidemeisterTwoB
        \]
        \subcaption{ruch $R_2$}
    \end{minipage}
    %
    \begin{minipage}[b]{.35\linewidth}
        \[
            \MedLarReidemeisterThreeLinkingA \cong \MedLarReidemeisterThreeLinkingB
        \]
        \subcaption{ruch $R_3$}
    \end{minipage}
\end{figure}
}
\end{comment}
\noindent
    Ὅπερ ἔδει δεῖξαι...
\end{proof}

\index{indeks!zaczepienia|)}%

% koniec podsekcji Indeks zaczepienia




\section{Liczba patykowa}
\index{liczba patykowa|(}%

% DICTIONARY;stick number;liczba patykowa;-

\begin{definition}
    Niech $L$ będzie splotem.
    Minimalną liczbę odcinków łamanej, która przedstawia splot $L$, nazywamy jego liczbą patykową i~oznaczamy $\stick(L)$.
\end{definition}

Liczbę patykową $\stick$ do matematyki wprowadził Randell \cite{randell1994}, gdzie znalazł od razu jej wartość dla niewęzła (3), trójlistnika (6), ósemki (7) i pozostałych węzłów pierwszych do sześciu skrzyżowań: $5_1, 5_2, 6_1, 6_2, 6_3$ (8).
\index[persons]{Randell, Richard}%
Według Cromwella \cite{cromwell2004}, z węzłami od siedmiu do dziewięciu częściowo rozprawiali się Randell \cite{randell1994}, Calvo \cite{calvo1998} i~Negami \cite{negami1991}.
\index[persons]{Randell, Richard}%
\index[persons]{Calvo, Jorge}%
\index[persons]{Negami, Seiya}%
Niedawno dołączył do nich Shonkwiler \cite{shonkwiler2022}.
\index[persons]{Shonkwiler, Clayton}%
Baza danych KnotInfo podpowiada, jaki jest obecny stan wiedzy.

Jest siedem węzłów z $\stick K = 8$: wymienione wcześniej, $8_{19}$ i $8_{20}$;
dwadzieścia pięć gdzie $\stick K = 9$; oprócz tego $\stick 10_{124} = 10$.
O pozostałych węzłach pierwszych do 10 skrzyżowań wciąż brakuje nam danych.
Zachodzi $9 \le \stick 10_{37}, \stick 10_{76} \le 12$, dla pozostałych węzłów mamy lepsze oszacowania.
88 węzłów spełnia $9 \le \stick K \le 11$, zaś 124 innych $9 \le \stick K \le 10$.

Encyklopedia Wolfram Mathworld plecie farmazony, że dokładnie jeden węzeł pierwszy do 10 skrzyżowań ma liczbę patykową równą 14: jest to $10_{84}$, dokładnie pięć ma liczbę patykową równą 13: $10_{39}$, $10_{64}$, $10_{73}$, $10_{76}$, $10_{80}$, a liczba patykowych pozostałych nie przekracza 12.
Niestety Wolfram nie podaje źródeł tych rewelacji.
Ugh!

Negami \cite{negami1991} pokazał przy użyciu teorii grafów, że dla nietrywialnych węzłów prawdziwe są nierówności
\index[persons]{Negami, Seiya}%
\begin{equation}
    \frac{5+\sqrt{9 + 8 \crossing K}}{2} \le \stick K \le 2 \crossing K.
\end{equation}
Trójlistnik to jedyny węzeł realizujący górne ograniczenie.
% Huh, Oh 2011 z trójlistnikiem?
Z~pracy Elrifaia \cite{elrifai2006} (a wiemy o~niej z~\cite{huh2011}) wynika, że dla węzłów o~co najwyżej 26 skrzyżowaniach, dolne ograniczenie jest ostre: można pisać $<$ w~miejsce $\le$.
\index[persons]{Elrifai, Elsayed}%

Jin oraz Kim w 1993 ograniczyli liczby patykowe dla węzłów torusowych korzystając z~liczby supermostowej.
\index[persons]{Jin, Gyo}%
\index[persons]{Kim, Hyoung-Seok}%
Wkrótce wynik został poprawiony przez samego Jina, w pracy \cite{jin1997} znalazł dokładne wartości dla niektórych węzłów.
I~tak, jeśli $2 \le p < q < 2p$, to $\stick T_{p,q} = 2q$ oraz $\stick T_{p, 2p} = 4p-1$.
Ten sam wynik, choć dla węższego zakresu parametrów, odkryli Adams, Brennan, Greilsheimer, Woo \cite{greilsheimer1997}.
\index[persons]{Adams, Colin}%
\index[persons]{Brennan, Bevin}%
\index[persons]{Greilsheimer, Deborah}%
\index[persons]{Woo, Alexander}%
\index{suma spójna}%
Autorzy niezależnie od siebie znaleźli proste oszacowanie z~góry dla liczby patykowej sumy spójnej:
\label{stick_bounded_factors}%
\begin{equation}
    \stick(K_1 \shrap K_2) \le \stick(K_1) + \stick(K_2) - 3.
\end{equation}

Koniec dekady przyniósł jeszcze jedną pracę McCabe z~nierównością $\stick(K) \le 3 + \crossing (K)$ dla węzłów dwumostowych (\cite{mccabe1998}) oraz odkrycie Calvo \cite{calvo2001}: jeśli ograniczymy się do łamanych o co najwyżej siedmiu odcinkach, ósemka przestaje być odwracalna.
\index[persons]{McCabe, Cynthia}%
\index[persons]{Calvo, Jorge}%

Na początku XXI wieku nierówności Negamiego poprawiono, z dołu dokonał tego Calvo \cite{calvo2001}, z góry natomiast Huh, Oh \cite{huh2011}.
\index[persons]{Calvo, Jorge}%
\index[persons]{Huh, Youngsik}%
\index[persons]{Oh, Seungsang}%
% huh11: simple and self-contained
Górne ograniczenie można zmniejszyć o $3/2$, jeżeli $K$ jest niealternującym węzłem pierwszym.
\begin{equation}
    \frac{7+\sqrt{1 + 8 \crossing K}}{2} \le \stick K \le \frac{3}{2} (1 + \crossing K).
\end{equation}

Liczba patykowa nie pojawia się już nigdzie w następnych rozdziałach.

% https://knotinfo.math.indiana.edu/descriptions/polygon_index.html wspomina jeszcze <[6] Shonkwilker, C., "All prime knots through 10 crossings have superbridge Index <= 5. Arxiv preprint.>

\index{liczba patykowa|)}%

% Koniec podsekcji Liczba patykowa



\input{30-invariants-numeric/301g-ropelength}

\subsection{Podsumowanie}
(Podana dalej lista może być niezrozumiała przy pierwszym czytaniu).
Między niektórymi niezmiennikami nie ma bezpośredniego związku:
\begin{enumerate}
    \item między liczbą gordyjską i~mostową (fakt \ref{no_relation_bridge_unknotting}),
    \item między liczbą mostową i~genusem (wzmianka po fakcie \ref{no_relation_bridge_unknotting}),
    \item między defektami modulo różne liczby pierwsze (fakt \ref{no_relation_defects}),
    \item między liczbą mostową i~wielomianem Alexandera (fakt \ref{no_relation_bridge_alexander}),
    \item między wielomianem Jonesa i~Alexandera (paragraf przed faktem \ref{homfly_stronger}),
    \item między liczbą mostową i~sygnaturą (wniosek \ref{no_relation_signature_bridge}),
    \item między liczbą gordyjską i~wielomianem Alexandera (dowód faktu \ref{balanced_iff_four_conditions}).
\end{enumerate}

Ale między niektórymi innymi takie związki są:
\begin{enumerate}
    \item między indeksem skrzyżowaniowym i genusem % TODO: \cite{livingston1993}, 142
    \item między liczbą gordyjską i sygnaturą % TODO: \cite{livingston1993}, 142
\end{enumerate}

% Koniec sekcji Niezmienniki liczbowe

