
\subsection{Spin}
\index{spin|(}%

Niektórzy, na przykład Przytycki, używają określenia ,,liczba Taita'', ale nam się ono nie podoba, więc proponujemy nasze, lepsze.
\index{liczba Taita}%

% DICTIONARY;writhe;spin;-
\begin{definition}[spin]
    Niech $D$ będzie diagramem zorientowanego splotu.
    Wielkość
    \begin{equation}
        \writhe D = \sum_c \operatorname{sign} c,
    \end{equation}
    gdzie sumowanie przebiega po wszystkich skrzyżowaniach diagramu $D$, nazywamy spinem.
\end{definition}

Co ważne, spin nie jest niezmiennikiem splotów ani węzłów.
Para Perko przedstawia ten sam węzeł z~minimalną liczbą skrzyżowań i~spinem równym siedem lub dziewięć.
\index{para Perko}%
Dzięki temu przez wiele lat nie została dostrzeżona.
Spin jest za to niezmiennikiem węzłów alternujących, mówi o~tym druga hipoteza Taita.
\index{hipoteza!Taita}%

\begin{lemma}
\label{lem:writhe_reidemeister}%
    Spin nie zależy od orientacji diagramu.
    Tylko I ruch Reidemeistera zmienia spin:
\begin{comment}
    \begin{equation}
        \writhe \left(\MediumReidemeisterOneLeft\right) =
        \writhe \left(\MediumReidemeisterOneStraight\right) - 1.
    \end{equation}
\end{comment}
    Pozostałe ruchy nie mają na niego wpływu.
\end{lemma}

\index{spin|)}%

% Koniec sekcji Spin

