
\subsection{Objętość hiperboliczna}

\index{objętość!hipeboliczna|(}%
\begin{definition}[objętość]
    Niech $L$ będzie splotem hiperbolicznym, na dopełnieniu którego zadano zupełną metrykę hiperboliczną.
    Objętość tego dopełnienia nazywamy objętością splotu $L$ i~oznaczamy $\volume L$.
\end{definition}

Objętość jest zawsze skończoną liczbą rzeczywistą.
Dla wygody przyjmuje się czasami, że objętość węzłów torusowych oraz satelitarnych wynosi $0$.
Komputerowy program SnapPea napisany przez Weeksa pozwala na wyznaczenie objętości dowolnego splotu o~rozsądnej ilości skrzyżowań.
\index{SnapPea}

Thurston \cite[s. 365]{thurston1982} zauważył, że tylko skończenie wiele hiperbolicznych 3-rozmaitości może mieć tę samą objętość -- wynika to z~prac Gromowa i~Jørgensena (niestety nie wiemy, o~których pracach mowa).
\index[persons]{Thurston, William}%
W mniej więcej tym samym czasie Wielenberg \cite{wielenberg1981} spostrzegł, że pewne podgrupy klasycznej grupy Picarda działają jako izometrie na górną półprzestrzestrzeń hiperboliczną wymiaru 3 mają podstawowe wielościany, które są takie same jako zbiory, ale różnią się jeśli chodzi o~utożsamienie ze sobą ścian.
\index[persons]{Wielenberg, Norbert}%
\index{grupa Picarda}%
Wynika stąd, że istnieją dowolnie duże kolizje wśród węzłów hiperbolicznych.

Chociaż mutanty mają tę samą objętość hiperboliczną (fakt~\ref{mutants_the_same_volume}), to praktyka pokazuje, że niezmiennik $\volume$ dobrze wspomaga proces tablicowania węzłów.
\index{mutant}%

\begin{example}
    $\volume 4_1 = -6 \int_{0}^{\pi/3} \log |2\sin \theta| \,\mathrm{d}\theta \approx 2.0298832$.
\end{example}

Żaden węzeł hiperboliczny nie ma mniejszej objętości, mówi o tym fakt \ref{prp:eight_smallest_volume}.

\begin{example}
    $\volume 5_2 \approx 2.82812$.
    % https://arxiv.org/pdf/q-alg/9601025.pdf strona 6
\end{example}

W encyklopedii Wolfram Mathworld znajduje się informacja, że $5_2$ oraz pewien węzeł o~dwunastu skrzyżowaniach mają tę samą objętość, prawdopodobnie chodzi tu o~$12n_{242}$, który znany jest także jako $(-2, 3, 7)$-precel.
\index{precel!(-2, 3, 7)}%
\index{węzeł!12n-242}%

\begin{example}
    $\volume 6_1 \approx 3.16396$.
    % https://arxiv.org/pdf/q-alg/9601025.pdf strona 6
\end{example}

\begin{example}
    $\volume 6_2 \approx 4.40083$.
\end{example}

\begin{example}
    $\volume 6_3 \approx 5.69302$.
\end{example}

\begin{example}
\index{para Perko}%
    Niech $K$ będzie jednym z~dwóch węzłów w~parze Perko.
    Wtedy $\volume K \approx 5.63877$.
\end{example}

Praca Futera, Kalfagianni, Purcell \cite{purcell2019} wspomina kilka przyjemnych ograniczeń, jakie musi spełniać objętość.
\index[persons]{Futer, David}%
\index[persons]{Kalfagianni, Efstratia}%
\index[persons]{Purcell, Jessica}%
Aby je przytoczyć, musimy najpierw zdefiniować dwie stałe: $v_4$ oraz $v_8$, objętości idealnego czworościanu (albo rozmaitości Giesekinga, powstałej z~czworościanu przez usunięcie  wierzchołków i~sklejenie ściany 012 z~310 oraz 023 z~32).
 Dopełnienie ósemki jest podwójnym nakryciem tej rozmaitości) oraz ośmiościanu foremnego w~$\mathbb H^3$.
\index{rozmaitość!Giesekinga}%
Mamy
\begin{alignat}{2}
    v_4 & = \int_{0}^{2\pi/3} \log(2 \cos(\theta/2)) \,\mathrm{d}\theta & {}\approx{} & 1.01494\,16064, \\
    % https://en.wikipedia.org/wiki/Gieseking_manifold
    % N[Integrate[Log[2Cos[t/2]], {t, 0, 2Pi/3}], 100]
    v_8 & = 4 \sum_{n=0}^\infty \frac{(-1)^n}{(2n+1)^2} &{}\approx{}& 3.66386\,23767.
    % 4N[Catalan, 100]
\end{alignat}

I tak najpierw Adams \cite{adams1983} pokazał w~swojej rozprawie doktorskiej:
\index[persons]{Adams, Colin}%

\begin{proposition}
    Niech $L$ będzie splotem o $\crossing L \ge 5$ skrzyżowaniach z diagramem $D$, który realizuje indeks skrzyżowaniowy.
    Wtedy
    \begin{equation}
        \volume L \le 4 (\crossing D - 4) v_4.
    \end{equation}
\end{proposition}

A trzy dekady później poprawił swój wynik w~\cite{adams2013}:

\begin{proposition}
    Niech $L$ będzie splotem o $\crossing L \ge 5$ skrzyżowaniach z diagramem $D$, który realizuje indeks skrzyżowaniowy.
    Wtedy
    \begin{equation}
        \volume L \le (\crossing D - 5) v_8 + 4v_4.
    \end{equation}
\end{proposition}

Jego metoda polega na podzieleniu dopełnienia splotu na czterościany i~ośmiościany oraz policzeniu ich.
To, w~połączeniu ze znanymi ograniczeniami na objętość ,,cegiełek'', wystarcza.
Podział na ośmiościany zaproponował Dylan (nie William!) Thurston.
\index[persons]{Thurston, Dylan}%
% wiem to z purcell19

\begin{proposition}
    Zbiór
    \begin{equation}
        \{\volume K: K \textrm{ jest hiperboliczny}\} \subseteq \R
    \end{equation}
    jest dobrze uporządkowany, typu porządkowego $\omega^\omega$.
\end{proposition}

\begin{proof}[Niedowód]
    Zdaniem angielskiej Wikipedii, dowód jest gdzieś w~\cite{neumann1985} (gdzie Neumann, Zagier znajdują eleganckie oszacowanie zmiany objętości po wykonaniu chirurgii Dehna), my tego nie widzimy.
\index[persons]{Neumann, Walter}%
\index[persons]{Zagier, Don}%
    %=% wikipedia - angielski artykuł "hyperbolic volume" 
    Hodgson, Masai \cite{hodgson2013} sugerują, żeby przeczytać notatki Thurstona \cite{thurston2002}.
\index[persons]{Hodgson, Craig}%
\index[persons]{Masai, Hidetoshi}%
    % TODO: https://mathscinet.ams.org/mathscinet-getitem?mr=648524 sugeruje, że to jest tam: "The order type of the set of all volumes of hyperbolic 3-manifolds is ω^ω."
    Jeszcze jedną wzmiankę znaleźliśmy w starszej pracy Thurstona \cite[s. 365]{thurston1982}.
\end{proof}

W dowolnej rodzinie węzłów istnieje element o~najmniejszej objętości.
Przytoczymy teraz przykłady konkretnych rodzin i~najmniejszych węzłów, za Futerem, Kalfagiannim, Purcell \cite[s. 16-17]{purcell2019} oraz Hodgsonem, Masaiem \cite[s. 296]{hodgson2013}.
\index[persons]{Futer, David}%
\index[persons]{Kalfagianni, Efstratia}%
\index[persons]{Purcell, Jessica}%
\index[persons]{Hodgson, Craig}%
\index[persons]{Masai, Hidetoshi}%

\begin{proposition}
\index{ósemka|see {węzeł 4-1}}%
\index{węzeł!4-1}%
\label{prp:eight_smallest_volume}%
    Żaden węzeł nie ma mniejszej objętości hiperbolicznej od ósemki.
\end{proposition}

\begin{proof}[Niedowód]
\index[persons]{Cao, Chun}%
\index[persons]{Meyerhoff, Robert}%
    Cao, Meyerhoff \cite{cao2001} przeanalizowali pakowania horokul w~uniwersalnym nakryciu związanym z~rozmaitościami.
    Doszli do wniosku, że nie ma tam dostatecznieo wolnego miejsca, jeżeli szpic nie jest odpowiedniego rozmiaru.
\index{szpic}%
    Trzykrotnie wspierają się przy tym pomocą komputera, by sprawdzić, że określone warunki są spełnione we wszystkich punktach danej przestrzeni parametrów.
\end{proof}

Nie jesteśmy pewni, jak powinno tłumaczyć się  angielskie \emph{cusp}; w~literaturze spotkaliśmy czasami termin ostrze.
\index{szpic}%
Chcielibyśmy zaproponować słowo szpic.
Rozmaitość szpiczasta (czyli niezwarta, zupełna hiperboliczna rozmaitość ze skończoną objętością Riemanna) byłaby wtedy polskim odpowiednikiem \emph{cusped manifold}.

\begin{proposition}
% DICTIONARY;cusped;szpiczasta;rozmaitość
% DICTIONARY;manifold;rozmaitość;-
\index{splot!Whiteheada}%
    Wśród orientowalnych 3-rozmaitości ze szpicem najmniejszą objętość ma dopełnienie ósemki oraz jego bliźniak, otrzymany przez $(5, 1)$-chirurgię jednego z~ogniw splotu Whiteheada.
\index{szpic}%
% TODO: sformułowanie wygląda jak z "THE MINIMAL VOLUME ORIENTABLE HYPERBOLIC 3-MANIFOLD WITH 4 CUSPS"
\end{proposition}

Klasa rozmaitości wspomniana w~fakcie obejmuje dopełnienia hiperbolicznych węzłów.
Powyższy fakt także został wzięty z~pracy \cite{cao2001}.
Meyerhoff, już bez Cao, nie przestawał pracować nad rozmaitościami o~małych objętościach i~osiem lat później przedstawił z~Gabaiem, Milleyem \cite{meyerhoff2009} bez dowodu:
\index[persons]{Gabai, David}%
\index[persons]{Milley, Peter}%

\begin{proposition}
\index{program SnapPy}%
    Istnieje 10 orientowalnych 3-rozmaitości z~jednym szpicem o~objętości co najwyżej $2.848$: \texttt{m003}, \texttt{m004} (2.02988\ldots), \texttt{m006}, \texttt{m007} (2.56897\ldots), \texttt{m009}, \texttt{m010} (2.66674\ldots), \texttt{m011} (2.78183\ldots), \texttt{m015}, \texttt{m016} oraz \texttt{m017} (2.82812\ldots).
    Nazwy pochodzą ze spisu rozmaitości programu SnapPy.
\end{proposition}

Chociaż panowie obiecali pokazać dowód później, nam nie udało się go odszukać.
Za to udało się rozszyfrować niektóre nazwy.
\texttt{m003} to siostra $4_1$, % https://hal.archives-ouvertes.fr/hal-02867890/document Michel Planat - Quantum computing thanks to Bianchi groups
\texttt{m004} to węzeł $4_1$, % SnapPy - also known as
% m006
% m007
% m009
% m010
% m011
\texttt{m015} to węzeł $5_2$,
\texttt{m016} to węzeł $12n242$, czyli znany nam już $(-2, 3, 7)$-precel,
\index{precel!(-2, 3, 7)}%
\texttt{m017} to siostra $5_2$. % https://arxiv.org/pdf/2107.03275.pdf
% TODO

W tej samej pracy możemy jeszcze znaleźć informację, że:

\begin{proposition}
    Istnieje dokładnie jedna hiperboliczna, domknięta 3-rozmaitość, której objętość jest najmniejsza (pośród wszystkich takich rozmaitości).
\end{proposition}

Chodzi o rozmaitość Weeksa, która powstaje przez wykonanie $(5, 2)$ oraz $(5, 1)$ chirurgii Dehna na dopełnieniu splotu Whiteheada.
\index{rozmaitość!Weeksa}%
\index{chirurgia Dehna}%
\index{splot!Whiteheada}%
Po raz pierwszy odkrył ją Jeffrey Weeks \cite{weeks1985} w~swojej rozprawie doktorskiej oraz niezależnie trzy lata później Matwiejow, Fomenko \cite{fomenko1988}.
\index[persons]{Weeks, Jeffrey}%
\index[persons]{Fomenko, Anatolij (Фомеенко, Анатоолий Тимофеевич)}%
\index[persons]{Matwiejow, Siergiej (Матвеев, Сергей Владимирович)}%
Wiemy z \url{https://oeis.org/A126774}, że
\begin{equation}
    V_{\textrm{Weeks}} = 0.94270 \, 73627 \, 76927 \, 72092 \ldots
    % Formula: Im(dilog(z0)+log(|z0|)*log(1-z0)) where z0 = 0.8774.. + 0.7448..i is the root of z^3-z^2+1 with Im(z)>0. - Herman Jamke (hermanjamke(AT)fastmail.fm), Dec 15 2007
\end{equation}

Następna jest rozmaitość Meyerhoffa, powstała po $(5, 1)$ chirurgii na dopełnieniu ósemki.
\index{rozmaitość!Meyerhoffa}%
Meyerhoff sugerował w~1987, że nie istnieje rozmaitość o mniejszej objętości, ale później okazało się to nieprawdą: 
\begin{equation}
    V_{\textrm{Meyerhoff}} = \frac{ 3 \sqrt{283^3} } {16 \pi^6} \zeta_{x^4-x-1}(2) = 0.98136 \, 88288 \, 92232 \, 08809 \ldots
\end{equation}

\begin{proposition}
    Wśród orientowalnych 3-rozmaitości o~dwóch szpicach najmniejszą objętość mają splot Whiteheada oraz $(-2, 3, 8)$-precel.
\index{szpic}%
\index{splot!Whiteheada}%
\index{precel!(-2, 3, 8)}%
\end{proposition}

% TODO: check cusped manifold in dictionary

Ich objętość wynosi $v_8$.

\begin{proof}[Niedowód]
\index[persons]{Agol, Ian}%
    Ian Agol \cite{agol2010} korzystając z~metod topologicznych dowodzi istnienia \emph{niezbędnej} powierzchni, która zadaje dolne ograniczenie na objętość.
    Tak skutecznie krępuje rozmaitości, które mogą to ograniczenie zrealizować.
\end{proof}

Przypadek trzech szpiców nie jest zbyt dobrze zrozumiany.
\index{szpic}%

\begin{proposition}
    Dopełnienie splotu $8_4^2$ (wg numeracji Rolfsena, czyli L8a13 wg numeracji Thistlethwaite'a) ma najmniejszą objętość wśród orientowalnych 3-rozmaitości o~czterech szpicach.
\index{szpic}%
\end{proposition}

Jego objętość wynosi $2v_8$.

\begin{proof}[Niedowód]
\index[persons]{Yoshida, Kenichi}%
    Rozumowanie Yoshidy \cite{yoshida2013} oparte o~wspomnianą wyżej pracę Agola.
\end{proof}

\index{objętość!hipeboliczna|)}%

