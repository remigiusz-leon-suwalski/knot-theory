
%%% Kawauchi 156:
\subsection{Zgodność}
Wprowadzimy teraz relację równoważności na zbiorze węzłów, która prowadzi przez fakt \ref{prp:cobordant_iff_sum_slice} do alternatywnej definicji węzłów plastrowych.

\begin{definition}[zgodność]
    \index{kobordyzm}
    \index{zgodność}
    % Dwa węzły $K_0, K_1$ takie, że istnieje lokalnie płaski, zorientowany, właściwy pierścień $C$ taki, że $C \cap S^3 \times i = K_i \times i$, nazywamy zgodnymi.
    % Dwa sploty $K, L \subseteq S^n$ nazywamy zgodnymi (z angielskiego \emph{concordant}), jeśli istnieje włożenie $f \colon K \times [0,1] \to S^n \times [0,1]$ spełniające dwa warunki: $f(K \times 0) = K \times 0$ oraz $f(K \times 1) = L \times 1$.
    Dwa węzły $K_0, K_1$ nazywamy (gładko) zgodnymi, jeżeli istnieje gładko zanurzony pierścień w $S^3 \times I$, którego brzegiem jest zbiór $K_0 \times 0 \cup K_1 \times 1$.
        % z \cite{gompf86}
\end{definition}

W języku angielskim przez zgodność rozumie się zazwyczaj \emph{concordance}, rzadziej termin \emph{cobordism}.

\begin{proposition}
    \label{prp:cobordant_iff_sum_slice}
    Dwa węzły $K_1, K_2$ są zgodne wtedy i tylko wtedy, gdy suma $mr K_0 \shrap K_1$ jest plastrowa.
\end{proposition}

\begin{proof}
    Ćwiczenie 12.1.3 w \cite{kawauchi96}.
\end{proof}

\begin{definition}
    Węzeł zgodny z~niewęzłem nazywamy plastrowym.
\end{definition}

,,Bycie zgodnym'' jest relacją równoważności, słabszą od bycia izotopijnym.
% ale mocniejszą od homotopii?
% izotopia: https://encyclopediaofmath.org/wiki/Cobordism_of_knots
Klasę abstrakcji węzła $K$ oznaczamy przez $[K]$.

\begin{definition}[grupa zgodności]
    \index{grupa!zgodności}
    Niech $C^1$ oznacza iloraz zbioru wszystkich węzłów przez relację zgodności.
    Zbiór $C^1$ wyposażony w~działanie
    \begin{equation}
        [K_1] + [K_2] = [K_1 \shrap K_2]
    \end{equation}
    staje się grupą abelową, nazywaną grupą zgodności.
    Jej elementem eneutralnym jest klasa abstrakcji niewęzła.
    Elementem przeciwnym do $[K]$ jest $[mr K]$.
\end{definition}

%%% Kawauchi 157:

Niech $\Theta$ oznacza rodzinę macierzy Seiferta, kwadratowych macierzy $V$ o całkowitych wyrazach takich, że $\det (V - V^T) = 1$.
Mówimy, macierz $V \in \Theta$ jest zerowo kobordantna, jeśli istnieje całkowitoliczbowa macierz $P$ o~wyznaczniku równym $\pm 1$, że
\begin{equation}
    V = P \begin{pmatrix} 0 & V_{21} \\ V_{12} & V_{22} \end{pmatrix} P^{-1}
\end{equation}
\index{macierz!unimodularnie sprzężona}
Takie macierze nazywamy unimodularnie sprzężonymi.

\begin{proposition}
    Niech $V \in \Theta$ będzie macierzą zerowo kobordantną.
    Wtedy istnieje plastrowy węzeł $K$, którego macierzą Seiferta jest $V$.
\end{proposition}

\begin{proof}
    Fakt 12.2.1 w \cite{kawauchi96}.
\end{proof}

Przez analogię, o dwóch macierzach $V_1, V_2 \in \Theta$ mówimy, że są kobordantne, jeżeli $(-V_1) \oplus V_2$ jest zerowo kobordantna.
Kobordyzm jest znowu relacją równoważności, iloraz $\Theta$ przez nią oznaczamy przez $G_-$, a elementy tego ilorazu jako $[V]$.
Wyposażony w działanie $[V_1] + [V_2] = [V_1 \oplus V_2]$ staje się grupą abelową.

\begin{proposition}
    % Kawauchi 12.2.8
    Odwzorowanie $\psi \colon C^1 \to G_-$ posyłające klasę abstrakcji węzła w klasę abstrakcji jego macierzy Seiferta jest dobrze określonym epimorfizmem.
\end{proposition}

\begin{proof}
    Funkcja $\psi$ jest dobrze określona na mocy faktu \ref{prp:cobordant_to_algebraic_is_algebraic}, jest homomorfizmem jak wynika z dowodu faktu \ref{prp:signature_additive}.
    To, że jest ,,na'', jest wnioskiem z \cite[s. 62]{kawauchi96}
\end{proof}

Funkcję $\psi$ rozpatrywał Levine \cite{levine69}.
\index[persons]{Levine, Jerome}%
Casson, Gordon pokazali w latach 70., że jej jądro jest niepuste \cite{gordon78}.
\index[persons]{Gordon, Cameron}%
\index[persons]{Casson, ?}%

\begin{proposition}
    % Dowód tego faktu jest pominięty nawet w pracy Kawauchiego
    $G_- \cong \Z^\infty \oplus (\Z/4\Z)^\infty \oplus (\Z/2\Z)^\infty$.
    % TODO: (Livingston: A SURVEY OF CLASSICAL KNOT CONCORDANCE)
    % The application of abelian knot invariants (those determined by the cohomology of abelian covers or, equivalently, by the Seifert form) to concordance culminated in 1969 with Levine’s classification of higher dimensional knot concordance, [62, 63], which applied in the classical dimension to give a surjective homomorphism $\varphi \colon C \to \Z^\infty \oplus \Z_2^\infty \oplus \Z_4^\infty$.
    % In 1975 Casson and Gordon [8, 9] proved that Levine’s homomorphism is not an isomorphism, constructing nontrivial elements in the kernel, and Jiang expanded on this to show that the kernel contains a subgroup isomorphic to $\Z_2^\infty$.
\end{proposition}

