
\section{Supły}
\index{supeł|(}%
\label{sec:tangle}%
Na przełomie lat sześćdziesiątych i~siedemdziesiątych Conway szukał sposobu na zbudowanie kompletnej tablicy węzłów.
Niezmienniki znane w~tym czasie nie były dostatecznie mocne, by sprostać temu wyzwaniu.
Conway wprowadził pojęcie supła i~chociaż wszystkich węzłów nie można z~nich uzyskać, teoria została pchnięta do przodu.
Supły stanowią budulec splotów takich jak na przykład precle z~definicji~\ref{def:pretzel}.

Sekcja oparta jest na podręczniku Murasugiego \cite{murasugi1996} i~pracach Conwaya \cite{conway1970}, Kauffmana, Goldmana \cite{kauffman1997}, Kauffmana, Lambropoulou \cite{kauffman2004}, a~także Schuberta \cite{schubert1956}, ale nie przeglądowej książce Kawauchiego \cite[s. 34-36]{kawauchi1996}.
\index[persons]{Conway, John}%
\index[persons]{Goldman, Jay}%
\index[persons]{Kauffman, Louis}%
\index[persons]{Lambropoulou, Sofia}%
\index[persons]{Schubert, Horst}%

Supły występują także w polskojęzycznym artykule Janiak-Osajcy, Pogody \cite{janiak2004}, ale ten zawiera nieprzyjemną pułapkę: wprowadza notację sprzeczną z~powszechnie akceptowaną.
\index[persons]{Janiak-Osajca, Agnieszka}%
\index[persons]{Pogoda, Zdzisław}%

% DICTIONARY;tangle;supeł;-
\begin{definition}[supeł]
    \label{def:tangle}
    Zawarty w~kole fragment diagramu splotu o~dwóch łukach wyjściowych oraz dwóch wejściowych, nazywamy supłem.
\end{definition}

% z: AMPHICHEIRALS ACCORDING TO TAIT AND HASEMAN
Słowo ,,supeł'' zaproponowała Haseman, już ona rysowała supły wewnątrz pomocniczego okręgu, który tnie diagram w~czterech punktach.
\index[persons]{Haseman, Mary}%

Istnieją dwa rodzaje supłów:
\begin{comment}
\begin{figure}[H]
    \centering
    \begin{minipage}[b]{.48\linewidth}
        \[\LargeTangleAlternatingYes\]
        \subcaption{supeł naprzemienny}
    \end{minipage}
    \begin{minipage}[b]{.48\linewidth}
        \centering
        \[\LargeTangleAlternatingNo\]
        \subcaption{supeł sąsiądujący}
    \end{minipage}
\end{figure}
\end{comment}

Podobnie jak dla węzłów, pojawia się naturalne pytanie o~równoważność dwóch supłów.
Jest tak wtedy, gdy istnieje homeomorfizm kuli na siebie, który przekształca jeden supeł na drugi, ale nie rusza sfery otaczającej.
Dla diagramów odpowiada to ruchom Reidemeistera, nie mamy jednak prawa opuszczać kuli zawierającej supeł.

Dużo dokładniej mówi o tym Turajew \cite{turaev1990}:
\index[persons]{Turajew, Władymir (Тураев, Владимир Георгиевич)}%

\begin{proposition}
    Oznaczmy przez OTa kategorię zorientowanych supłów.
    Jej obiektami są skończone ciągi złożone z~$\pm 1$, razem z~ciągiem pustym.
    Morfizm ciągu $\varepsilon = (\varepsilon_1, \ldots, \varepsilon_k)$ w ciąg $\nu = (\nu_1, \ldots, \nu_l)$ jest klasą izotopii zorientowanego $(k, l)$-supła $L$ tak, że źródłem $L$ jest $\varepsilon$, zaś celem $\nu$.
    Na przykład supły $\curvearrowright$, $\curvearrowleft$ oraz $X_+$ opisane są przez morfizmy $\varnothing \to (-1, 1)$, $\varnothing \to (1, -1)$ oraz $(1, 1) \to (1, 1)$.
    Składanie morfizmów odpowiada mnożeniu supłów.

    Wprowadźmy iloczyn tensorowy $\otimes$.
    Iloczynem obiektów $\varepsilon, \nu$ (które znaczą to, co wcześniej) jest obiekt $(\varepsilon_1, \ldots, \varepsilon_k, \nu_1, \ldots, \nu_l)$, zaś iloczyn tensorowy morfizmów będzie iloczynem tensorowym splotów i łatwo widać, że $(OTa, \otimes, \varnothing)$ jest ściśle monoidalną kategorią (cokolwiek to znaczy).

    Zdefiniujmy cztery słowa:
    \begin{align}
        A & = (\downarrow \, \downarrow \, \curvearrowright) \circ (\downarrow \, \downarrow \, \uparrow \, \curvearrowright \, \downarrow) \circ (\downarrow \, \downarrow X_\pm \downarrow \, \downarrow) \circ (\downarrow \inversedcurvearrowright \, \uparrow \, \downarrow \, \downarrow) \circ (\inversedcurvearrowright \downarrow \, \downarrow) \\
        B & = (\curvearrowleft \, \downarrow \, \downarrow) \circ (\downarrow \, \curvearrowleft \, \uparrow \, \downarrow \, \downarrow) \circ (\downarrow \, \downarrow X_\pm \downarrow \, \downarrow) \circ (\downarrow \, \downarrow \, \uparrow  \inversedcurvearrowleft \downarrow) \circ (\downarrow \, \downarrow \inversedcurvearrowleft) \\
        T & = (\curvearrowleft \, \uparrow \, \downarrow) \circ (\downarrow X_- \downarrow) \circ (\downarrow \, \uparrow \inversedcurvearrowleft) \\
        Y & = (\uparrow \, \downarrow \, \curvearrowright) \circ (\downarrow X_+ \downarrow) \circ (\inversedcurvearrowright \uparrow \, \downarrow)
    \end{align}
    Kategorię OTa można przedstawić przez morfizmy $\inversedcurvearrowright, \inversedcurvearrowleft, \curvearrowright, \curvearrowleft, X_+, X_-$ (generatory) oraz relacje:
    \begin{align}
        (\curvearrowright \, \uparrow) \circ (\uparrow \inversedcurvearrowright) = & \uparrow \, = (\uparrow \, \curvearrowleft) \circ (\inversedcurvearrowleft \uparrow) \\
        (\curvearrowleft \, \downarrow) \circ (\downarrow \inversedcurvearrowleft) = & \downarrow \, = (\downarrow \, \curvearrowright) \circ (\inversedcurvearrowright \downarrow) \\
        A & = B \\
        X_+ \circ X_- & = X_- \circ X_+ = \, \uparrow \, \uparrow \\
        (X_+ \uparrow) \circ (\uparrow X_-) \circ (X_+ \uparrow) & = (\uparrow X_+) \circ (X_+ \uparrow) \circ (\uparrow X_+) \\
        (\uparrow \, \curvearrowright) \circ (X_\pm \downarrow) \circ (\uparrow \inversedcurvearrowleft) & = \, \uparrow \\
        Y \circ T = \, \downarrow \, \uparrow, & \quad T \circ Y = \, \uparrow \, \downarrow
    \end{align}
\end{proposition}

\begin{proof}
    Dowód twierdzenia oraz graficzne przedstawienie relacji z kategorii OTa zawiera praca Turajewa \cite{turaev1990}.
    Wszystkie relacje odpowiadają ruchom Reidemeistera.
\index{ruch!Reidemeistera}%
    Trzecie od końca równanie to geometryczny wariant równania Yanga-Baxtera.
\index{równanie Yanga-Baxtera}%
% TODO: przerysować... do kodu

Patrz też \cite[s. 29-30]{duzhin2012} (Czmutow, Dużin, Mostovoy przygotowali tam śliczne rysunki) albo \cite[s. 31]{schieber2018} (gdzie Schieber przedstawił ruchy Reidemeistera i~cięte diagramy).
\index[persons]{Czmutow, Siergiej (Чмутов, Сергей Владимирович)}%
\index[persons]{Dużin, Siergiej (Дужин, Сергей Васильевич)}%
\index[persons]{Mostovoy, Jacob}%
\index[persons]{Schieber, Nathaniel}%
% sliced diagrams
% DICTIONARY;sliced;cięty;diagram
\end{proof}

Wszystkich supłów jest bardzo dużo, więc ograniczymy się do końca rozdziału do pewnej ich regularnej rodziny.
Oto cztery podstawowe supły:
\begin{comment}
\begin{figure}[H]
    \centering
    \begin{minipage}[b]{.23\linewidth}
        \[
            \LargeTangleBasicZero
        \]
        \subcaption{$(0)$}
    \end{minipage}
    \begin{minipage}[b]{.23\linewidth}
        \[
            \LargeTangleBasicInfinity
        \]
        \subcaption{$(\infty) = (0, 0)$}
    \end{minipage}
    \begin{minipage}[b]{.23\linewidth}
        \[
            \LargeTangleBasicMinus
        \]
        \subcaption{$(-1)$}
    \end{minipage}
    \begin{minipage}[b]{.23\linewidth}
        \[
            \LargeTangleBasicPlus
        \]
        \subcaption{$(+1)$}
    \end{minipage}
\end{figure}
\end{comment}

\begin{definition}
    Supły powstające z~$(0)$ lub $(\infty)$ przez homeomorfizm kuli na siebie permutujący wejścia i~wyjścia nazywamy wymiernymi.
\end{definition}

Pokażemy teraz, jak zamienić dowolny skończony ciąg liczb całkowitych w~supeł, jako że jest to prostsze od procesu odwrotnego.
Nazwijmy jednak najpierw dwa rodzaje skrętów:
\begin{comment}
\begin{figure}[H]
    \centering
    \begin{minipage}[b]{.48\linewidth}
        \[\LargeTwistsRight\]
        \subcaption{skręty prawe}
    \end{minipage}
    \begin{minipage}[b]{.48\linewidth}
        \centering
        \[\LargeTwistsLeft\]
        \subcaption{skręty lewe}
    \end{minipage}
\end{figure}
\end{comment}

Mając ciąg $(a_1, a_2, \ldots, a_n)$ wykonujemy naprzemiennie obroty półsferą dolną (SW--SE, takie nazywamy pionowymi) oraz prawą (SW--NW, a takie poziomymi) tak, by ostatni był obrót poziomy.
Oto reguła zgodnie z którą wybieramy kierunek obrotów.
Podczas pionowych obrotów, prawy skręt jest dodatni, zaś lewy ujemny.
Podczas poziomych, zamieniamy znaki: prawy odpowiada ujemnym wyrazom ciągu, lewy dodatnim.
Wreszcie, jeżeli $n$ jest nieparzyste, zaczynamy od supła $T(0)$, w przeciwnym razie od supła $T(0, 0)$.

Różnym ciągom mogą odpowiadać te same supły, na przykład $T(-2, 3, 3) = T(3, -2)$, więc notacja nie jest jednoznaczna, ale to nic złego.
Każdemu supłowi przypiszmy pewną liczbę wymierną, według przepisu:
\begin{equation}
    T(a_1, a_2, \ldots, a_n) \mapsto a_n + \frac{1}{\ldots + 1/a_1} = \frac \alpha \beta.
\end{equation}

\begin{proposition}
    Istnieje bijekcja między supłami wymiernymi oraz ułamkami łańcuchowymi.
\end{proposition}

\begin{proof}[Niedowód]
    Praca Conwaya \cite[s. 331-332]{conway1970}.
\end{proof}

\begin{proposition}[ćwiczenie 9.2.6 w \cite{murasugi1996}]
    \label{prp:continued_fractions}
    Niech $T(a_1, a_2, \ldots, a_n)$ będzie supłem różnym od $0$ oraz $\infty$.
    Wtedy bez straty ogólności można założyć, że wszystkie liczby $a_i$ są tego samego znaku.
\end{proposition}

Z każdym supłem $T$ związane jest jego odbicie $\overline T$, obraz wyjściowego przez symetrię względem prostej $y = -x$.
Mając dwa supły obok siebie, można dokonać ich sklejenia wzdłuż połówek kul, w~których leżą:
\begin{comment}
\begin{figure}[H]
    \centering
    \begin{minipage}[b]{.23\linewidth}
        \[
            \LargeTangleSummandA
        \]
        \subcaption{jakiś supeł}
    \end{minipage}
    \begin{minipage}[b]{.23\linewidth}
        \centering
        \[
            \LargeTangleSummandB
        \]
        \subcaption{jakiś inny supeł}
    \end{minipage}
    \begin{minipage}[b]{.48\linewidth}
        \centering
        \[
            \LargeTangleSumAB
        \]
        \subcaption{suma tych supłów}
    \end{minipage}
\end{figure}
\end{comment}

Oznaczmy tak otrzymany splot przez $T_1 + T_2$.
Niektórzy definiują dalsze działania, jak produkt: $T_1 \cdot T_2 = \overline T_1 + T_2$ czy rozgałęzienie, $\overline T_1 + \overline T_2$.
Rodzina supłów wymiernych jest zamknięta na branie produktów, ale nie sum.
Wprowadzamy więc następującą, ogólniejszą definicję.
Supeł będący skończoną sumą supłów wymiernych, ich luster, odbić lub odbić luster nazywamy algebraicznym.

Conway korzystając ze skończonej listy ,,wielościanów podstawowych'' (pewnych grafów planarnych) był w stanie zakodować wszystkie węzły o~małej liczbie skrzyżowań.
Ale ponieważ notacja ta nie jest uniwersalna -- im więcej skrzyżowań, tym więcej wielościanów potrzeba, by opisać wszystkie węzły -- nie opiszemy, jak działa.

Przez zszycie par łuków wejściowych (lub wyjściowych) zamieniamy supły w~węzły:
\begin{figure}[H]
    \centering
    \begin{minipage}[b]{.3\linewidth}
        \centering
        \LargeTangleFraction
        \subcaption{supeł $T$}
    \end{minipage}
    \begin{minipage}[b]{.3\linewidth}
        \centering
        \LargeTangleFractionNumerator
        \subcaption{licznik, $N(T)$}
    \end{minipage}
    \begin{minipage}[b]{.3\linewidth}
        \centering
        \LargeTangleFractionDenominator
        \subcaption{mianownik, $D(T)$}
    \end{minipage}
\end{figure}

% DICTIONARY;... numerator;licznik ...;supeł
% DICTIONARY;... denominator;mianownik ...;supeł
Oznaczenia $N(T)$ oraz $D(T)$ pochodzą od angielskich słów \emph{numerator}, \emph{denominator}.
Być może nie jest jasne, dlaczego terminy stosowane zazwyczaj do opisu ułamków stosujemy wobec diagramów splotów.
Nazewnictwo nie jest przypadkowe.

\begin{proposition}
\index{ułamek supła}%
    Ułamek supła zadany wzorem
    \begin{equation}
        F(A) = \frac{\conway_{N(A)}(z)}{\conway_{D(A)}(z)}
    \end{equation}
    spełnia zależność $F(A+B) = F(A) + F(B)$.
\end{proposition}

\begin{proof}
    Praca \cite{conway1970} Conwaya.
\end{proof}

Praca \cite{conway1970} zawiera jeszcze jeden ciekawy rezultat, uogólniony przez Lickorisha i~Milletta w~\cite[fakt 12]{lickorish1987}.
Używamy tu wersji wielomianu HOMFLY o zmiennych $l, m$.

\begin{proposition}
    Niech $A, B$ będą supłami, zaś $T_n$ (odpowiednio: $T_d$)  wielomianem HOMFLY licznika (mianownika) supła $T$.
    Wtedy
    \begin{equation}
        (\mu^2 - 1)(A+B)_n = \mu(A_nB_n + A_dB_d) - (A_nB_d + A_dB_n),
    \end{equation}
    gdzie $\mu = -(l + 1/l)/m$. % oraz
    %\begin{equation}
        %(A+B)_d = A_dB_d.
    %\end{equation}
    % TODO: tego nie potrafię znaleźć w pracy Lickorisha
\end{proposition}

\input{50-families/tangle-bridge}

\input{50-families/tangle-mutants}

\index{supeł|)}%

