\section{Węzły plastrowe i taśmowe} % (fold)
\label{sec:slice}
Węzły plastrowe i taśmowe oraz pojęcie kobordyzmu, które wkrótce opiszemy, należą do świata 4-wymiarowej teorii węzłów.
Nie zapoznamy się z nią bliżej oraz nie podamy naszego ulubionego odniesienia do tego tematu w~literaturze, ponieważ sami nie rozumiemy go zbyt dobrze.
Wszystko zaczęło się od artykułu \cite{fox66} Foxa, Milnora.

\begin{definition}[płaski]
    Niech $D \subseteq B^4$ będzie dyskiem posiadającym otoczenie $N$, kopię zbioru $D \times I
    ^2$, która przecina sferę $S^3$ dokładnie w $\partial D \times I^2$.
    Mówimy wtedy, że dysk $D$ jest płaski.
\end{definition}

\begin{definition}[węzeł plastrowy]
    \index{węzeł!plastrowy}
    Niech $K \subseteq S^3$ będzie takim węzłem, że w kuli $B^4$ istnieje płaski dysk $D$ taki, że $K = \partial D = D \cap S^3$.
\end{definition}

Następujące węzły o~mniej niż jedenastu skrzyżowaniach są plastrowe: $6_1$, $8_{20}$, $8_{8}$, $8_{9}$, $9_{27}$, $9_{41}$, $9_{46}$, $10_{22}$, $10_{35}$, $10_{3}$, $10_{42}$, $10_{48}$, $10_{75}$, $10_{87}$, $10_{99}$, $10_{123}$, $10_{129}$, $10_{137}$, $10_{140}$, $10_{153}$, $10_{155}$, $10_{155}$.
Wśród pierwszych węzłów do dwunastu skrzyżowań najdłużej opierał się węzeł Conwaya, aż Piccirillo pokazała w~\cite{piccirillo20}, że nie jest plastrowy.

\begin{proposition}
    Niech $K$ będzie węzłem.
    Wtedy $K \shrap mr K$ jest węzłem plastrowym.
\end{proposition}

\begin{proof}[Niedowód]
    Pierwszy był Fox z Milnorem \cite{fox66}, patrz także lemat 12.1.2.2 w \cite{kawauchi96}.
\end{proof}

Istnieje konkurencyjna definicja węzłów plastrowych.
Dwa sploty $K, L \subseteq S^n$ nazywamy zgodnymi (z angielskiego \emph{concordant}), jeśli istnieje włożenie $f \colon K \times [0,1] \to S^n \times [0,1]$ spełniające dwa warunki: $f(K \times 0) = K \times 0$ oraz $f(K \times 1) = L \times 1$.

\begin{definition}
    Węzeł zgodny z~niewęzłem nazywamy plastrowym.
\end{definition}

Zgodność jest relacją równoważności, słabszą od izotopii, ale mocniejszą od homotopii.
W zbiorze jej klas abstrakcji, oznaczanym przez $C^1$, można zadać strukturę grupy abelowej.
%izomorficznej z~$\Z^\infty \oplus (\Z/2)^\infty \oplus (\Z/4)^\infty$.
\index{grupa!zgodności}
Działanie dane jest wzorem $[K] + [L] = [K \shrap L]$; niewęzeł stanowi element neutralny.
Elementem odwrotnym do $[K]$ jest $[mrK]$.

% TODO: \textbf{Livingston: A SURVEY OF CLASSICAL KNOT CONCORDANCE}. The application of abelian knot invariants (those determined by the cohomology of abelian covers or, equivalently, by the Seifert form) to concordance culminated in 1969 with Levine’s classification of higher dimensional knot concordance, [62, 63], which applied in the classical dimension to give a surjective homomorphism $\varphi \colon C \to \Z^\infty \oplus \Z_2^\infty \oplus \Z_4^\infty$. In 1975 Casson and Gordon [8, 9] proved that Levine’s homomorphism is not an isomorphism, constructing nontrivial elements in the kernel, and Jiang expanded on this to show that the kernel contains a subgroup isomorphic to $\Z_2^\infty$. More significant, Freedman proved that all knots with trivial Alexander polynomial are in fact slice in the topological locally flat category.

\begin{proposition}
    Albo wszystkie trzy węzły $K, L, K \shrap L$ są plastrowe, albo co najwyżej jeden z~nich.
\end{proposition}

\begin{proof}[Niedowód]
    Lemat 12.1.2.3 w \cite{kawauchi96}.
\end{proof}

Pierwszym poważnym wynikiem z dziedziny teorii węzłów plastrowych, pochodzącym jeszcze z pracy \cite{fox66}, był:

\begin{proposition}[warunek Foxa-Milnora]
    \index{warunek!Foxa-Milnora}
    Niech $K$ będzie węzłem plastrowym.
    Wtedy jego wielomian Alexandera jest postaci $\alexander(t) = f(t) f(1/t)$ dla pewnego wielomianu Laurenta $f \in \Z[t, 1/t]$.
\end{proposition}

\begin{corollary}
    Wyznacznik węzła plastrowego jest kwadratem.
\end{corollary}

\begin{proof}
    Mamy $\det K = |\alexander(-1)| = f(-1) f(-1)$.
\end{proof}

Ten prosty test stwierdza, że 2743 spośród 2977 węzłów o mniej niż 13 skrzyżowaniach nie jest plastrowych.

\begin{proposition}
    Niech $K$ będzie węzłem plastrowym.
    Wtedy $\operatorname{Arf} K = 0$.
\end{proposition}

\begin{proof}
    Ustalmy węzeł $K$, wiemy już, że jego wyznacznik jest kwadratem, a na mocy faktu \ref{cor:knot_determinant_odd} także tyle, że jest liczbą nieparzystą.
    Wynika stąd przystawanie $\det K \equiv 1 \mod 8$, które w~połączeniu z warunkiem Murasugiego (fakt \ref{prp:arf_murasugi}) daje $\operatorname{Arf} K = 0$.
\end{proof}

\begin{proposition}
    \label{prp:slice_signature}
    Niech $K$ będzie węzłem plastrowym.
    Wtedy $\sigma(K) = 0$.
\end{proposition}

\begin{proof}[Szkic dowodu]
    Ustalmy odwzorowanie $f$, które jest niesingularne, symetryczne i~dwuliniowe, z~przestrzeni $V$ o~wymiarze $2n$ oraz wyznaczoną przez nie formę kwadratową.
    Jeśli znika ona na podprzestrzeni wymiaru $n$, to ma zerową sygnaturę.
    % TODO: \textbf{(dowód znaleziony w~podręczniku Lickorisha). Patrz też twierdzenie 8.8 z~artykułu \cite{murasugi65}. Praca "Infinite Order Amphicheiral Knots". (Charles Livingston, 2001) -- chyba nie?}
\end{proof}

Test ten eliminuje kolejne 45 węzłów poniżej 13 skrzyżowań.

\begin{definition}
    \index{węzeł!taśmowy}
    Węzeł $K = f(S^1)$ będący brzegiem singularnego dysku $f \colon D \to S^3$ posiadającego następującą własność: każda przecinająca siebie składowa jest łukiem $A \subseteq f(D^2)$, dla którego $f^{-1}(A)$ składa się z~dwóch łuków w~$D^2$ (jeden z~nich jest wewnętrzny), nazywamy taśmowym.
\end{definition}

Jak pisze Kawauchi, mamy oczywiste wynikanie:

\begin{proposition}
    Każdy węzeł taśmowy jest plastrowy.
\end{proposition}

Dawno temu Fox zapytał, czy implikacja odwrotna także jest prawdziwa:

\begin{conjecture}[slice-ribbon problem]
    \index{hipoteza!plastrowo-taśmowa}
    Czy każdy węzeł plastrowy jest taśmowy?
\end{conjecture}

Nie wiemy do dzisiaj.
Lisca pokazał prawdziwość hipotezy dla węzłów 2-mostowych \cite{lisca07}, Greene oraz Jabuka zrobili to dla precli o trzech pasmach w \cite{greene11}.
P. Teichner myśli o niej jako o~życzeniu, które uprościłoby pewne 4-wymiarowe problemy, gdyby było prawdziwe, ale Gompf, Scharlemann i Thompson zasugerowali w~\cite{gompf10} potencjalny kontrprzykład.

\begin{proposition}
    Każda macierz Seiferta $V$ (całkowitoliczbowa, kwadratowa, taka że $\det (V - V^t) = 1$), która jest unimodularnie sprzężona: istnieje całkowitoliczbowa macierz $P$ o~wyznaczniku równym $\pm 1$, że
    \begin{equation}
        V = P \begin{pmatrix} 0 & V_{21} \\ V_{12} & V_{22} \end{pmatrix} P^{-1}
    \end{equation}
    stanowi macierz Seiferta pewnego węzła plastrowego.
\end{proposition}

Takie węzły nazywamy plastrowymi algebraicznie.
Węzeł $K$ w~$S^3$ jest algebraicznie plastrowy dokładnie wtedy, gdy ogranicza izotropiczną powierzchnię w~kuli $B^4$.
Więcej informacji w~podręczniku Kawauchiego.

% Theorem 1.3[Long 1984].A strongly positive amphicheiral knot is algebraicallyslice.
% Theorem 1.4[Hartley and Kawauchi 1979].If K is strongly positive amphicheiral,the Alexander polynomial1Kis the square of a symmetric polynomial.

\subsection{Węzły skręcone}
\begin{definition}
    \index{węzeł!skręcony}
    \label{def:twist_knot}
    Węzeł powstały przez $n$-krotne półskręcanie domkniętej pętli oraz splecienie końców nazywamy węzłem skręconym.
\end{definition}

Węzły skręcone to dokładnie towarzyszące niewęzłowi w~węzłach satelitarnych, tak zwane whiteheadowskie duble niewęzła.
Wszystkie są odwracalne (ale tylko niewęzeł oraz ósemka są amfichiralne) i~mają liczbę gordyjską $1$, ponieważ wystarczy rozwiązać skrzyżowanie, które plotło końce.
Każdy jest $2$-mostowy i~posiada zerową sygnaturę.
Dalsze własności węzłów skręconych zależą od $n$, ilości półskrętów.
Indeks skrzyżowaniowy wynosi $n + 2$.

\begin{proposition}
    Wielomianowymi niezmiennikami węzłów skręconych są:
    \begin{align*}
    (q+1)\jones(q) & = \begin{cases}
        1+q^{-2}+q^{-n}-q^{-n-3} & n \mbox{ nieparzyste} \\
        q^{3}+q-q^{3-n}+q^{-n} & n \mbox{ parzyste}
    \end{cases} \\
    2 \conway (z) & = \begin{cases}
        (n+1) z^{2} + 2 & n \mbox{ nieparzyste} \\
        2 - nz^2 & n \mbox{ parzyste}
    \end{cases}
    \end{align*}
\end{proposition}

\begin{proposition}
    Niewęzeł oraz węzeł dokerski $6_1$ są jedynymi skręconymi węzłami plastrowymi.
\end{proposition}

\begin{proof}
    \cite{casson86}.
\end{proof}

% Koniec sekcji Węzły plastrowe
