
\section{Węzły plastrowe i taśmowe}
\label{sec:slice}
Musimy oznajmić z przykrością, że jest to już ostatnia sekcja książki.
Poruszymy tutaj ważne z~punktu widzenia współczesnej teorii węzłów zagadnienia czterowymiarowe: wprowadzamy węzły plastrowe i taśmowe, definiujemy relację zgodności oraz przytaczamy tyle wyników o grupie zgodności, ile tylko jesteśmy w stanie zrozumieć, czyli nie za wiele.
Ze względu na znaczne ubytki wiedzy, ograniczamy się do zreferowania tekstu Kawauchiego \cite[s. 154-169]{kawauchi1996}, miejscami tylko wspominając odsyłacze do literatury, które wydają się nam pomocne.
Wiele wyników, jakie podamy w tej sekcji, pochodzi z artykułu Foxa, Milnora \cite{fox1966}, od którego wszystko się zaczęło.
\index[persons]{Fox, Ralph}%
\index[persons]{Milnor, John}%


Adnotacja tłumacza: być może jesteśmy pierwszymi osobami piszącymi o tych obiektach po polsku, dlatego nie mamy pewności, czy podane tutaj tłumaczenie ,,plastrowy'' przyjmie się.
Inny przymiotnik godny rozważenia to ,,wstęgowy''.

% DICTIONARY;plastrowy;slice;węzeł
\begin{definition}[węzeł topologicznie plastrowy]
\index{węzeł!plastrowy}%
    Węzeł $K$ w sferze $S^3$, który jest brzegiem lokalnie płaskiego dysku $D$ w kuli $B^4$ nazywamy węzłem topologicznie plastrowym. % Kawauchi 155?
    Dysk $D$, kiedy potrzebuje mieć nazwę, też jest dyskiem plastrowym.
\end{definition}

Jeśli nie jest podane inaczej, ,,plastrowy'' to synonim ,,topologicznie plastrowego''.

\begin{definition}[węzeł gładko plastrowy]
    Węzeł $K$ w sferze $S^3$, który jest brzegiem gładkiego dysku $D$ w kuli $B^4$ nazywamy węzłem gładko plastrowym.
\end{definition}

Następujące węzły o~mniej niż jedenastu skrzyżowaniach są plastrowe (topologicznie oraz gładko): $6_1$, $8_{8}$, $8_{9}$, $8_{20}$, $9_{27}$, $9_{41}$, $9_{46}$, $10_{3}$, $10_{22}$, $10_{35}$, $10_{42}$, $10_{48}$, $10_{75}$, $10_{87}$, $10_{99}$, $10_{123}$, $10_{129}$, $10_{137}$, $10_{140}$, $10_{153}$ oraz $10_{155}$.
\index{węzeł!Conwaya}
Wśród pierwszych węzłów do dwunastu skrzyżowań najdłużej opierał się węzeł Conwaya, aż Lisa Piccirillo \cite{piccirillo2020} pokazała, że nie jest gładko plastrowy (ale za to jest topologicznie plastrowy).
\index[persons]{Piccirillo, Lisa}%

\begin{proposition}
    Niech $K$ będzie węzłem.
    Wtedy suma $K \shrap \operatorname{mr} K$ jest węzłem plastrowym.
\end{proposition}

\begin{proof}[Niedowód]
    Kawauchi \cite[s. 155]{kawauchi1996} pisze, że wystarczy wybrać 3-kulę $B \subseteq S^3$ taką, że $K \cap B$ jest trywialnym łukiem w $B$.
    Wtedy 4-kula $\operatorname{cl} (S^3 \setminus B) \times [0,1]$ oraz lokalnie płaski dysk $\operatorname{cl} (K \setminus K \cap B) \times [0,1] $ są świadkami plastrowości węzła $K \shrap \operatorname{mr} K$.
\end{proof}

\begin{proposition}
    Albo wszystkie trzy węzły $K_1, K_2, K_1 \shrap K_2$ są plastrowe, albo co najwyżej jeden z~nich.
\end{proposition}

\begin{proof}
    Kawauchi \cite[s. 155]{kawauchi1996} pisze: załóżmy, że $K_1, K_2$ są plastrowe, z plastrowymi dyskami $D_1, D_2 \subseteq B^4$.
    Wtedy suma brzegowa\footnote{Cokolwiek to jest!} $(B^4, D_1) \natural (B^4, D_2)$ pokazuje, że węzeł $K_1 \shrap K_2$ jest plastrowy.

    Załóżmy teraz, że $K_1, K_3 = K_1 \shrap K_2$ są plastrowe, z plastrowymi dyskami $D_1, D_3 \subseteq B_4$.
    Wybierzmy 3-kule $B_1, B_3$ wewnątrz $S^3$ tak, że domknięcie $K_1 \setminus B_1 \cap K_1$ jest trywialnym łukiem w domknięciu $S^3 \setminus B_1$.
    Wtedy coś tam dalej, ale nie warto tego przepisywać, bo i tak za trudne na tę książkę.
\end{proof}

Pierwszym poważnym wynikiem z dziedziny teorii węzłów plastrowych, pochodzącym jeszcze z pracy \cite{fox1966}, był:

\begin{proposition}[warunek Foxa-Milnora]
\index{warunek!Foxa-Milnora}%
    Niech $K$ będzie węzłem plastrowym.
    Wtedy jego wielomian Alexandera jest postaci $\alexander(t) = f(t) f(1/t)$ dla pewnego wielomianu Laurenta $f(t) \in \Z[t, 1/t]$.
\end{proposition}

\begin{corollary}
    \label{det_slice_square}%
    \index{wyznacznik}
    Wyznacznik węzła plastrowego jest kwadratem.
\end{corollary}

\begin{proof}
    Mamy $\det K = |\alexander_K(-1)| = f(-1) f(-1)$.
\end{proof}

Ten prosty test stwierdza, że 2743 spośród 2977 węzłów o mniej niż 13 skrzyżowaniach nie jest plastrowych.
% podać program tłumaczący, czemu tak jest

\begin{proposition}
% TODO: skąd to stwierdzenie?
\index{sygnatura}%
    Niech $K$ będzie węzłem plastrowym.
    Wtedy $\sigma(K) = 0$.
\end{proposition}

\begin{proof}[Niedowód]
    Można zajrzeć do Lickorisha \cite[s. 90]{lickorish1997} i Murasugiego \cite[twierdzenie 8.8]{murasugi1965}; nie wiadomo, kto pierwszy uznał, że warto tego dowieść.
    %Praca "Infinite Order Amphicheiral Knots". (Charles Livingston, 2001) -- chyba nie?
\end{proof}

Test ten eliminuje kolejne 45 węzłów poniżej 13 skrzyżowań.
% TODO: podać program tłumaczący, czemu tak jest?

\begin{proposition}
    \index{niezmiennik!Arfa}
    Niech $K$ będzie węzłem plastrowym.
    Wtedy $\operatorname{Arf} K = 0$.
\end{proposition}

\begin{proof}
    Składniki dowodu już są, wystarczy je teraz wszystkie wymieszać w~dobrej kolejności.
    Wniosek \ref{det_slice_square} mówi, $\det K$ jest kwadratem, zaś fakt \ref{cor:knot_determinant_odd}, że $\det K$ jest nieparzyste.
    Zatem $\det K \equiv 1 \pmod 8$ i fakt~\ref{prp:arf_murasugi} (warunkek Murasugiego) orzeka $\operatorname{Arf} K = 0$.
\end{proof}

% TODO: podać program tłumaczący, czemu tak jest? tzn. czy coś to eliminuje więcej, jak tak to ile?

Ostatni fakt, jaki podamy we wprowadzeniu, to jedno z niewielu miejsc w całej książce, gdzie dotykamy różnic między kategorią Top oraz PL.

\begin{proposition}
\label{prp:trivial_alexander_implies_slice}%
    Niech $K$ będzie węzłem w kategorii Top.
    Jeżeli jego wielomian Alexandera jest trywialny: $\alexander_K(t) \equiv 1$, to węzeł $K$ jest plastrowy.
\end{proposition}

Twierdzenie to jest bardzo łatwo napotkać przeglądając prezentacje poświęcone węzłom plastrowym, ale nikt nie chce się przyznać, kto jest jego ojcem.
Odpowiedź znaleźliśmy dopiero w artykule ,,The degree of the Alexander polynomial is an upper bound for the topological slice genus'' Petera Fellera!
Miło, że mu się chciało.
% łatwo napotkać czytając o węzłach plastrowych, ale jawnie nikomu się nie chce wskazać dowodu. % Informację, że to jest tw. 1.13b znalazłem wreszcie w https://arxiv.org/pdf/1504.01064.pdf

\begin{proof}
\index[persons]{Freedman, Michael}%
    Freedman \cite[tw. 1.13]{freedman1982}.
\end{proof}

Implikacja~\ref{prp:trivial_alexander_implies_slice} przestaje być prawdziwa po przejściu do kategorii Diff.
Wydawało nam się kiedyś, że Gompf \cite{gompf1986} dobrze tłumaczy tę różnicę przy użyciu twierdzenia Donaldsona.
\index[persons]{Gompf, Robert}%
\index{twierdzenie Donaldsona}%
(Do tego artykułu odsyła encyklopedia węzłów \cite{adams2021}).
% Encyclopedia of Knot Theory pod redakcją Colin Adams, Erica Flapan, Allison Henrich, Louis H. Kauffman, Lewis D. Ludwig, Sam Nelson, około strony 453 lub 454: "but many knots with Alexander polynomial one are not smoothly slice [9]" i odsyła do \cite{donaldson1983}
% TODO Gompf to nie Donaldson
Dzisiaj wiemy, że nic nie wiemy.

\subsection{Węzły skręcone}
\begin{definition}
    \index{węzeł!skręcony}
    \label{def:twist_knot}
    Węzeł powstały przez $n$-krotne półskręcanie domkniętej pętli oraz splecienie końców nazywamy węzłem skręconym.
\end{definition}

Węzły skręcone to dokładnie towarzyszące niewęzłowi w~węzłach satelitarnych, tak zwane whiteheadowskie duble niewęzła.
Wszystkie są odwracalne (ale tylko niewęzeł oraz ósemka są amfichiralne) i~mają liczbę gordyjską $1$, ponieważ wystarczy rozwiązać skrzyżowanie, które plotło końce.
Każdy jest $2$-mostowy i~posiada zerową sygnaturę.
Dalsze własności węzłów skręconych zależą od $n$, ilości półskrętów.
Indeks skrzyżowaniowy wynosi $n + 2$.

\begin{proposition}
    Wielomianowymi niezmiennikami węzłów skręconych są:
    \begin{align*}
    (q+1)\jones(q) & = \begin{cases}
        1+q^{-2}+q^{-n}-q^{-n-3} & n \mbox{ nieparzyste} \\
        q^{3}+q-q^{3-n}+q^{-n} & n \mbox{ parzyste}
    \end{cases} \\
    2 \conway (z) & = \begin{cases}
        (n+1) z^{2} + 2 & n \mbox{ nieparzyste} \\
        2 - nz^2 & n \mbox{ parzyste}
    \end{cases}
    \end{align*}
\end{proposition}

\begin{proposition}
    Niewęzeł oraz węzeł dokerski $6_1$ są jedynymi skręconymi węzłami plastrowymi.
\end{proposition}

\begin{proof}
    \cite{casson86}.
\end{proof}



%%% Kawauchi 156:
\subsection{Zgodność}
Wprowadzimy teraz relację równoważności na zbiorze węzłów, która prowadzi przez fakt \ref{prp:cobordant_iff_sum_slice} do alternatywnej definicji węzłów plastrowych.

\begin{definition}[zgodność]
    \index{kobordyzm}
    \index{zgodność}
    % Dwa węzły $K_0, K_1$ takie, że istnieje lokalnie płaski, zorientowany, właściwy pierścień $C$ taki, że $C \cap S^3 \times i = K_i \times i$, nazywamy zgodnymi.
    % Dwa sploty $K, L \subseteq S^n$ nazywamy zgodnymi (z angielskiego \emph{concordant}), jeśli istnieje włożenie $f \colon K \times [0,1] \to S^n \times [0,1]$ spełniające dwa warunki: $f(K \times 0) = K \times 0$ oraz $f(K \times 1) = L \times 1$.
    Dwa węzły $K_0, K_1$ nazywamy (gładko) zgodnymi, jeżeli istnieje gładko zanurzony pierścień w $S^3 \times I$, którego brzegiem jest zbiór $K_0 \times 0 \cup K_1 \times 1$.
        % z \cite{gompf86}
\end{definition}

W języku angielskim przez zgodność rozumie się zazwyczaj \emph{concordance}, rzadziej termin \emph{cobordism}.

\begin{proposition}
    \label{prp:cobordant_iff_sum_slice}
    Dwa węzły $K_1, K_2$ są zgodne wtedy i tylko wtedy, gdy suma $mr K_0 \shrap K_1$ jest plastrowa.
\end{proposition}

\begin{proof}
    Ćwiczenie 12.1.3 w \cite{kawauchi96}.
\end{proof}

\begin{definition}
    Węzeł zgodny z~niewęzłem nazywamy plastrowym.
\end{definition}

,,Bycie zgodnym'' jest relacją równoważności, słabszą od bycia izotopijnym.
% ale mocniejszą od homotopii?
% izotopia: https://encyclopediaofmath.org/wiki/Cobordism_of_knots
Klasę abstrakcji węzła $K$ oznaczamy przez $[K]$.

\begin{definition}[grupa zgodności]
    \index{grupa!zgodności}
    Niech $C^1$ oznacza iloraz zbioru wszystkich węzłów przez relację zgodności.
    Zbiór $C^1$ wyposażony w~działanie
    \begin{equation}
        [K_1] + [K_2] = [K_1 \shrap K_2]
    \end{equation}
    staje się grupą abelową, nazywaną grupą zgodności.
    Jej elementem eneutralnym jest klasa abstrakcji niewęzła.
    Elementem przeciwnym do $[K]$ jest $[mr K]$.
\end{definition}

%%% Kawauchi 157:

Niech $\Theta$ oznacza rodzinę macierzy Seiferta, kwadratowych macierzy $V$ o całkowitych wyrazach takich, że $\det (V - V^T) = 1$.
Mówimy, macierz $V \in \Theta$ jest zerowo kobordantna, jeśli istnieje całkowitoliczbowa macierz $P$ o~wyznaczniku równym $\pm 1$, że
\begin{equation}
    V = P \begin{pmatrix} 0 & V_{21} \\ V_{12} & V_{22} \end{pmatrix} P^{-1}
\end{equation}
\index{macierz!unimodularnie sprzężona}
Takie macierze nazywamy unimodularnie sprzężonymi.

\begin{proposition}
    Niech $V \in \Theta$ będzie macierzą zerowo kobordantną.
    Wtedy istnieje plastrowy węzeł $K$, którego macierzą Seiferta jest $V$.
\end{proposition}

\begin{proof}
    Fakt 12.2.1 w \cite{kawauchi96}.
\end{proof}

Przez analogię, o dwóch macierzach $V_1, V_2 \in \Theta$ mówimy, że są kobordantne, jeżeli $(-V_1) \oplus V_2$ jest zerowo kobordantna.
Kobordyzm jest znowu relacją równoważności, iloraz $\Theta$ przez nią oznaczamy przez $G_-$, a elementy tego ilorazu jako $[V]$.
Wyposażony w działanie $[V_1] + [V_2] = [V_1 \oplus V_2]$ staje się grupą abelową.

\begin{proposition}
    % Kawauchi 12.2.8
    Odwzorowanie $\psi \colon C^1 \to G_-$ posyłające klasę abstrakcji węzła w klasę abstrakcji jego macierzy Seiferta jest dobrze określonym epimorfizmem.
\end{proposition}

\begin{proof}
    Funkcja $\psi$ jest dobrze określona na mocy faktu \ref{prp:cobordant_to_algebraic_is_algebraic}, jest homomorfizmem jak wynika z dowodu faktu \ref{prp:signature_additive}.
    To, że jest ,,na'', jest wnioskiem z \cite[s. 62]{kawauchi96}
\end{proof}

Funkcję $\psi$ rozpatrywał Levine \cite{levine69}.
\index{człowiek!Levine, Jerome}%
Casson, Gordon pokazali w latach 70., że jej jądro jest niepuste \cite{gordon78}.
\index{człowiek!Gordon, Cameron}%
\index{człowiek!Casson, ?}%

\begin{proposition}
    % Dowód tego faktu jest pominięty nawet w pracy Kawauchiego
    $G_- \cong \Z^\infty \oplus (\Z/4\Z)^\infty \oplus (\Z/2\Z)^\infty$.
    % TODO: (Livingston: A SURVEY OF CLASSICAL KNOT CONCORDANCE)
    % The application of abelian knot invariants (those determined by the cohomology of abelian covers or, equivalently, by the Seifert form) to concordance culminated in 1969 with Levine’s classification of higher dimensional knot concordance, [62, 63], which applied in the classical dimension to give a surjective homomorphism $\varphi \colon C \to \Z^\infty \oplus \Z_2^\infty \oplus \Z_4^\infty$.
    % In 1975 Casson and Gordon [8, 9] proved that Levine’s homomorphism is not an isomorphism, constructing nontrivial elements in the kernel, and Jiang expanded on this to show that the kernel contains a subgroup isomorphic to $\Z_2^\infty$.
\end{proposition}




\subsection{Węzły taśmowe}
\index{węzeł!taśmowy|(}
\begin{definition}
    Węzeł $K = f(S^1)$ będący brzegiem osobliwego dysku $f \colon D \to S^3$ posiadającego następującą własność: każda przecinająca siebie składowa jest łukiem $A \subseteq f(D^2)$, dla którego $f^{-1}(A)$ składa się z~dwóch łuków w~$D^2$ (jeden z~nich jest wewnętrzny), nazywamy taśmowym.
\end{definition}

Jak pisze Kawauchi, mamy oczywiste wynikanie:

\begin{proposition}
\index{węzeł!plastrowy}%
    Każdy węzeł taśmowy jest plastrowy.
\end{proposition}

Dawno temu Fox zapytał, czy implikacja odwrotna jest prawdziwa (\cite[problem 1.33]{kirby78}):
\index[persons]{Fox, Ralph}%

\begin{conjecture}[slice-ribbon problem]
    \index{hipoteza!plastrowo-taśmowa}
    Czy każdy węzeł plastrowy jest taśmowy?
\end{conjecture}

Wprawdzie Lisca pokazał prawdziwość hipotezy dla węzłów dwumostowych \cite{lisca07},
\index[persons]{Lisca, ?}%
% korzystając ze słynnego tw. Donaldsona: that a definite intersection form of a compact, oriented, simply connected, smooth manifold of dimension 4 is diagonalisable
\index{węzeł!dwumostowy}%
zaś Greene oraz Jabuka zrobili to dla precli o trzech pasmach w~\cite{greene11};
\index[persons]{Greene, ?}%
\index[persons]{Jabuka, ?}%
\index{precel}%
ale Gompf, Scharlemann i~Thompson zasugerowali w~\cite{gompf10} potencjalny kontrprzykład.
\index[persons]{Gompf, ?}%
\index[persons]{Scharlemann, ?}%
\index[persons]{Thompson, ?}%
\index{rozmaitość szwowa}%
Nie możemy przytoczyć tego kontrprzykładu, gdyż korzysta z~rozmaitości szwowych, opisanych w~\cite[s. 53-59]{kawauchi96}.

Teichner myśli\footnote{Patrz \url{https://mathoverflow.net/a/18154}.} o hipotezie plastrowo-taśmowej jako o~życzeniu, które uprościłoby pewne czterowymiarowe problemy, gdyby było prawdziwe.
\index[persons]{Teichner, ?}%

\index{węzeł!taśmowy|)}

% koniec podsekcji węzły taśmowe




%%% Kawauchi 157:
\subsection{Węzły algebraicznie plastrowe}
Węzeł, którego macierz Seiferta jest zerowo kobordantna, nazywamy plastrowym algebraicznie.
Lokalnie płaską, zwartą, zorientowaną, właściwą powierzchnię $S$ w $B^4$ taką, że $K = \partial S$ jest węzłem w $\partial B^4 = S^3$ nazywamy izotropową, jeżeli istnieje lokalnie płaska, zwarta, zorientowana 3-podrozmaitość $M \subseteq B^4$, gdzie $S \subseteq \partial M$ oraz $F = \operatorname{cl} \partial M \setminus S$ jest powierzchnią Seiferta dla $K$ w $S^3$, zaś $S$ jest izotropowa w $M$.

\begin{proposition}
    Węzeł $K$ w~$S^3$ jest algebraicznie plastrowy dokładnie wtedy, gdy ogranicza izotropową powierzchnię $S$ w~kuli $B^4$.
\end{proposition}

\begin{corollary}
    Niech $K$ będzie węzłem plastrowym.
    Wtedy $K$ jest węzłem algebraicznie plastrowym.
\end{corollary}

\begin{proof}
    Kawauchi \cite[s. 158]{kawauchi1996}.
\end{proof}

\begin{proposition}
    \label{prp:cobordant_to_algebraic_is_algebraic}
    Niech $K$ będzie węzłem zgodnym z węzłem algebraicznie plastrowym.
    Wtedy każda macierz Seiferta dowolnej powierzchni Seiferta $K$ jest zerowo kobordantna.
    W szczególności, $K$ jest węzłem algebraicznie plastrowym.
\end{proposition}

\begin{proof}
    Kawauchi \cite[s. 159]{kawauchi1996}.
\end{proof}

% Theorem 1.3[Long 1984].A strongly positive amphicheiral knot is algebraicallyslice.
% Theorem 1.4[Hartley and Kawauchi 1979].If K is strongly positive amphicheiral,the Alexander polynomial1Kis the square of a symmetric polynomial.





