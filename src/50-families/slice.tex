
\section{Węzły plastrowe i taśmowe}
\label{sec:slice}
Musimy oznajmić z przykrością, że jest to już ostatnia sekcja książki.
Poruszymy tutaj ważne z~punktu widzenia współczesnej teorii węzłów zagadnienia czterowymiarowe: wprowadzamy węzły plastrowe i taśmowe, definiujemy relację zgodności oraz przytaczamy tyle wyników o grupie zgodności, ile tylko jesteśmy w stanie zrozumieć, czyli nie za wiele.
Ze względu na znaczne ubytki wiedzy, ograniczamy się do zreferowania tekstu Kawauchiego \cite[s. 154-169]{kawauchi1996}, miejscami tylko wspominając odsyłacze do literatury, które wydają się nam pomocne.
Wiele wyników, jakie podamy w tej sekcji, pochodzi z artykułu Foxa, Milnora \cite{fox1966}, od którego wszystko się zaczęło.
\index[persons]{Fox, Ralph}%
\index[persons]{Milnor, John}%

Adnotacja tłumacza: być może jesteśmy pierwszymi osobami piszącymi o tych obiektach po polsku, dlatego nie mamy pewności, czy podane tutaj tłumaczenie ,,plastrowy'' przyjmie się.
Inny przymiotnik godny rozważenia to ,,wstęgowy''.

% DICTIONARY;plastrowy;slice;węzeł
\begin{definition}[węzeł topologicznie plastrowy]
\index{węzeł!plastrowy}%
    Węzeł $K$ w sferze $S^3$, który jest brzegiem lokalnie płaskiego dysku $D$ w kuli $B^4$ nazywamy węzłem topologicznie plastrowym. % Kawauchi 155?
    Dysk $D$, kiedy potrzebuje mieć nazwę, też jest dyskiem plastrowym.
\end{definition}

Jeśli nie jest podane inaczej, ,,plastrowy'' to synonim ,,topologicznie plastrowego''.

\begin{definition}[węzeł gładko plastrowy]
    Węzeł $K$ w sferze $S^3$, który jest brzegiem gładkiego dysku $D$ w kuli $B^4$ nazywamy węzłem gładko plastrowym.
\end{definition}

Następujące węzły o~mniej niż jedenastu skrzyżowaniach są plastrowe (topologicznie oraz gładko): $6_1$, $8_{8}$, $8_{9}$, $8_{20}$, $9_{27}$, $9_{41}$, $9_{46}$, $10_{3}$, $10_{22}$, $10_{35}$, $10_{42}$, $10_{48}$, $10_{75}$, $10_{87}$, $10_{99}$, $10_{123}$, $10_{129}$, $10_{137}$, $10_{140}$, $10_{153}$ oraz $10_{155}$.
\index{węzeł!Conwaya}
Wśród pierwszych węzłów do dwunastu skrzyżowań najdłużej opierał się węzeł Conwaya, aż Lisa Piccirillo \cite{piccirillo2020} pokazała, że nie jest gładko plastrowy (ale za to jest topologicznie plastrowy).
\index[persons]{Piccirillo, Lisa}%

\begin{proposition}
    Niech $K$ będzie węzłem.
    Wtedy suma $K \shrap \operatorname{mr} K$ jest węzłem plastrowym.
\end{proposition}

\begin{proof}[Niedowód]
    Kawauchi \cite[s. 155]{kawauchi1996} pisze, że wystarczy wybrać 3-kulę $B \subseteq S^3$ taką, że $K \cap B$ jest trywialnym łukiem w $B$.
    Wtedy 4-kula $\operatorname{cl} (S^3 \setminus B) \times [0,1]$ oraz lokalnie płaski dysk $\operatorname{cl} (K \setminus K \cap B) \times [0,1] $ są świadkami plastrowości węzła $K \shrap \operatorname{mr} K$.
\end{proof}

\begin{proposition}
    Albo wszystkie trzy węzły $K_1, K_2, K_1 \shrap K_2$ są plastrowe, albo co najwyżej jeden z~nich.
\end{proposition}

\begin{proof}
    Kawauchi \cite[s. 155]{kawauchi1996} pisze: załóżmy, że $K_1, K_2$ są plastrowe, z plastrowymi dyskami $D_1, D_2 \subseteq B^4$.
    Wtedy suma brzegowa\footnote{Cokolwiek to jest!} $(B^4, D_1) \natural (B^4, D_2)$ pokazuje, że węzeł $K_1 \shrap K_2$ jest plastrowy.

    Załóżmy teraz, że $K_1, K_3 = K_1 \shrap K_2$ są plastrowe, z plastrowymi dyskami $D_1, D_3 \subseteq B_4$.
    Wybierzmy 3-kule $B_1, B_3$ wewnątrz $S^3$ tak, że domknięcie $K_1 \setminus B_1 \cap K_1$ jest trywialnym łukiem w domknięciu $S^3 \setminus B_1$.
    Wtedy coś tam dalej, ale nie warto tego przepisywać, bo i tak za trudne na tę książkę.
\end{proof}

Pierwszym poważnym wynikiem z dziedziny teorii węzłów plastrowych, pochodzącym jeszcze z pracy \cite{fox1966}, był:

\begin{proposition}[warunek Foxa-Milnora]
\index{warunek!Foxa-Milnora}%
    Niech $K$ będzie węzłem plastrowym.
    Wtedy jego wielomian Alexandera jest postaci $\alexander(t) = f(t) f(1/t)$ dla pewnego wielomianu Laurenta $f(t) \in \Z[t, 1/t]$.
\end{proposition}

\begin{corollary}
    \label{det_slice_square}%
    \index{wyznacznik}
    Wyznacznik węzła plastrowego jest kwadratem.
\end{corollary}

\begin{proof}
    Mamy $\det K = |\alexander_K(-1)| = f(-1) f(-1)$.
\end{proof}

Ten prosty test stwierdza, że 2743 spośród 2977 węzłów o mniej niż 13 skrzyżowaniach nie jest plastrowych.
% podać program tłumaczący, czemu tak jest

\begin{proposition}
% TODO: skąd to stwierdzenie?
\index{sygnatura}%
    Niech $K$ będzie węzłem plastrowym.
    Wtedy $\sigma(K) = 0$.
\end{proposition}

\begin{proof}[Niedowód]
    Można zajrzeć do Lickorisha \cite[s. 90]{lickorish1997} i Murasugiego \cite[twierdzenie 8.8]{murasugi1965}; nie wiadomo, kto pierwszy uznał, że warto tego dowieść.
    %Praca "Infinite Order Amphicheiral Knots". (Charles Livingston, 2001) -- chyba nie?
\end{proof}

Test ten eliminuje kolejne 45 węzłów poniżej 13 skrzyżowań.
% TODO: podać program tłumaczący, czemu tak jest?

\begin{proposition}
    \index{niezmiennik!Arfa}
    Niech $K$ będzie węzłem plastrowym.
    Wtedy $\operatorname{Arf} K = 0$.
\end{proposition}

\begin{proof}
    Składniki dowodu już są, wystarczy je teraz wszystkie wymieszać w~dobrej kolejności.
    Wniosek \ref{det_slice_square} mówi, $\det K$ jest kwadratem, zaś fakt \ref{cor:knot_determinant_odd}, że $\det K$ jest nieparzyste.
    Zatem $\det K \equiv 1 \pmod 8$ i fakt~\ref{prp:arf_murasugi} (warunkek Murasugiego) orzeka $\operatorname{Arf} K = 0$.
\end{proof}

% TODO: podać program tłumaczący, czemu tak jest? tzn. czy coś to eliminuje więcej, jak tak to ile?

Ostatni fakt, jaki podamy we wprowadzeniu, to jedno z niewielu miejsc w całej książce, gdzie dotykamy różnic między kategorią Top oraz PL.

\begin{proposition}
\label{prp:trivial_alexander_implies_slice}%
    Niech $K$ będzie węzłem w kategorii Top.
    Jeżeli jego wielomian Alexandera jest trywialny: $\alexander_K(t) \equiv 1$, to węzeł $K$ jest plastrowy.
\end{proposition}

Twierdzenie to jest bardzo łatwo napotkać przeglądając prezentacje poświęcone węzłom plastrowym, ale nikt nie chce się przyznać, kto jest jego ojcem.
Odpowiedź znaleźliśmy dopiero w artykule ,,The degree of the Alexander polynomial is an upper bound for the topological slice genus'' Petera Fellera!
Miło, że mu się chciało.
% łatwo napotkać czytając o węzłach plastrowych, ale jawnie nikomu się nie chce wskazać dowodu. % Informację, że to jest tw. 1.13b znalazłem wreszcie w https://arxiv.org/pdf/1504.01064.pdf

\begin{proof}
\index[persons]{Freedman, Michael}%
    Freedman \cite[tw. 1.13]{freedman1982}.
\end{proof}

Implikacja~\ref{prp:trivial_alexander_implies_slice} przestaje być prawdziwa po przejściu do kategorii Diff.
Wydawało nam się kiedyś, że Gompf \cite{gompf1986} dobrze tłumaczy tę różnicę przy użyciu twierdzenia Donaldsona.
\index[persons]{Gompf, Robert}%
\index{twierdzenie!Donaldsona}%
(Do tego artykułu odsyła encyklopedia węzłów \cite{adams2021}).
% Encyclopedia of Knot Theory pod redakcją Colin Adams, Erica Flapan, Allison Henrich, Louis H. Kauffman, Lewis D. Ludwig, Sam Nelson, około strony 453 lub 454: "but many knots with Alexander polynomial one are not smoothly slice [9]" i odsyła do \cite{donaldson1983}
% TODO Gompf to nie Donaldson
Dzisiaj wiemy, że nic nie wiemy.


\subsection{Węzły skręcone}
\index{węzeł!skręcony|(}%

Nie potrafiliśmy znaleźć lepszego miejsca dla tej podsekcji, między innymi przez fakt \ref{twist_slice}, który wymaga znajomości przymiotnika ,,plastrowy''.
% DICTIONARY;twist;skręcony;węzeł
Węzły skręcone uważa się za najprostszą (po torusowych) rodzinę węzłów.
Wspomina o~nich bardzo krótko Kawauchi \cite[s. 31]{kawauchi1996}.

\begin{definition}
    Węzeł powstały najpierw przez $n$-krotne półskręcanie domkniętej pętli, a następnie splecienie końców, nazywamy węzłem skręconym.
\end{definition}

Węzły skręcone to dokładnie towarzyszące niewęzłowi w~węzłach satelitarnych, tak zwane whiteheadowskie duble niewęzła.
Wszystkie są odwracalne (ale tylko niewęzeł oraz ósemka są zwierciadlane) i~mają liczbę gordyjską $1$, ponieważ wystarczy rozwiązać skrzyżowanie, które plotło końce.
\index{liczba gordyjska}%
Każdy jest dwumostowy (ćwiczenie u~Rolfsena \cite[s. 114]{rolfsen1976}) i~posiada zerową sygnaturę.
\index{węzeł!dwumostowy}%
\index{sygnatura}%
Dalsze własności węzłów skręconych zależą od $n$, ilości półskrętów.
Indeks skrzyżowaniowy wynosi $n + 2$.

\begin{proposition}
\index{wielomian!Conwaya}%
    Niech $K$ będzie węzłem $n$-skręconym.
    Wtedy
    \begin{equation}
    2 \conway (z) = \begin{cases}
        2 + (n+1) z^{2} & n \mbox{ nieparzyste} \\
        2 - nz^2 & n \mbox{ parzyste}
    \end{cases}
    \end{equation}
\end{proposition}

\begin{proposition}
\index{wielomian!Jonesa}%
    Niech $K$ będzie węzłem $n$-skręconym.
    Wtedy
    \begin{equation}
    (q+1)\jones(q) = \begin{cases}
        1+q^{-2}+q^{-n}-q^{-n-3} & n \mbox{ nieparzyste} \\
        q^3(1+q^{-2}-q^{-n}+q^{-n-3}) & n \mbox{ parzyste}
    \end{cases}
    \end{equation}
\end{proposition}

\begin{proposition}
\index{węzeł!plastrowy}%
\label{twist_slice}%
    Żaden węzeł skręcony poza niewęzłem $0_1$ oraz węzłem dokerskim $6_1$ nie jest plastrowy.
\end{proposition}

\begin{proof}
    Casson, Gordon \cite{casson1986}.
\end{proof}

\index{węzeł!skręcony|)}%

% koniec podsekcji Węzły skręcone




%%% Kawauchi 156:
\subsection{Zgodność}
Zgodność jest relacją równoważności na zbiorze węzłów, która prowadzi do nowej definicji węzłów plastrowych (patrz fakt~\ref{prp:concordant_iff_sum_slice}).
My przytaczamy jej definicję z pracy Gompfa \cite{gompf1986}:

\begin{definition}[zgodność]
\index{zgodność}%
\index{węzeł!zgodny|see {zgodność}}%z
    Dwa węzły $K_0, K_1$ nazywamy (gładko) zgodnymi, jeżeli zbiór
    \begin{equation}
        K_0 \times \{0\} \cup K_1 \times \{1\}
    \end{equation}
    jest brzegiem pewnego pierścienia gładko zanurzonego w $S^3 \times I$.
\end{definition}

Kawauchi \cite[s. 156]{kawauchi1996} pisze \emph{,,Two knots (…) are knot cobordant (or concordant)''}, więc tak jak wielu innych autorów nie odróżnia więc węzłów kobordantnych od zgodnych.
Mamy zamiar zrobić dokładnie to samo: różnica między tymi terminami jest subtelna; węzły zgodne są też kobordantne, ale implikacja w drugą stronę nie zachodzi (wiemy o~tym z~tekstu Blanlœila ,,Cobordism and Concordance of Knots'') chyba, że pracuje się z węzłami sferycznmi, a tak jest w klasycznej teorii węzłów.
\index[persons]{Blanloeil, Vincent}%
% https://www.maths.ed.ac.uk/~v1ranick/papers/blanloeil
% Concordant knots are cobordant, but the converse is not true in general.
% "Cobordism and Concordance of Knots" by Vincent Blanlœil

Dlatego my będziemy zawsze pisać o węzłach zgodnych i nigdy o kobordantnynch.

\begin{proposition}
\label{prp:concordant_iff_sum_slice}%
    Dwa węzły $K_1, K_2$ są zgodne wtedy i tylko wtedy, gdy suma $(\operatorname{mr} K_0) \shrap K_1$ jest plastrowa.
\index{suma spójna}%
\index{węzeł!plastrowy}%
\end{proposition}

\begin{proof}
    Ćwiczenie 12.1.3 w książce Kawauchiego \cite{kawauchi1996}.
\end{proof}

\begin{definition}
    Węzeł zgodny z~niewęzłem nazywamy plastrowym.
\index{węzeł!plastrowy}%
\end{definition}

,,Bycie zgodnym'' jest relacją równoważności, słabszą od ,,bycia izotopijnym'', ale chyba mocniejszą od ,,bycia homotopijnym''.
\index{izotopia}%
\index{homotopia}
% ale mocniejszą od homotopii?
% izotopia: https://encyclopediaofmath.org/wiki/Cobordism_of_knots
% homotopia: https://en.wikipedia.org/wiki/Link_concordance By its nature, link concordance is an equivalence relation. It is weaker than isotopy, and stronger than homotopy: isotopy implies concordance implies homotopy. A link is a slice link if it is concordant to the unlink.
Klasę abstrakcji węzła $K$ oznaczamy przez $[K]$.

\begin{definition}[grupa zgodności]
\index{grupa!zgodności}%
    Niech $C^1$ oznacza iloraz zbioru wszystkich węzłów przez relację zgodności.
    Zbiór $C^1$ wyposażony w~działanie
    \begin{equation}
        [K_1] + [K_2] = [K_1 \shrap K_2]
    \end{equation}
    staje się grupą abelową, nazywaną grupą zgodności.
    Jej elementem neutralnym jest klasa abstrakcji niewęzła.
    Elementem przeciwnym do $[K]$ jest $[\operatorname{mr} K]$.
\end{definition}

%%% Kawauchi 157:

Niech $\Theta$ oznacza rodzinę macierzy Seiferta węzłów (czyli kwadratowych macierzy $V$ o~całkowitych wyrazach takich, że $\det (V - V^T) = 1$).
\index{macierz Seiferta}%
Mówimy, że macierz $V \in \Theta$ jest zerowo kobordantna, jeżeli jest postaci
\index{macierz Seiferta!zerowo kobordantna}%
\begin{equation}
    V = P \begin{pmatrix} 0 & V_{21} \\ V_{12} & V_{22} \end{pmatrix} P^{-1}
\end{equation}
dla pewnej całkowitoliczbowej macierzy $P$ o~wyznaczniku $\pm 1$; takie macierze nazywamy unimodularnie sprzężonymi.
\index{macierz!unimodularnie sprzężona}%
Każda zerowo kobordantna macierz $V \in \Theta$ stanowi macierz Seiferta pewnego plastrowego węzła $K$.
Kawauchi nazywa te węzły algebraicznie plastrowymi i~mówi, że to dokładnie węzły, które ograniczają izotropowe powierzchnie w kuli $B^4$, więc każdy węzeł plastrowy jest algebraicznie plastrowy.

Suma $(-V) \oplus V$ jest zerowo kobordantna dla każdej macierzy $V \in \Theta$.
To (chyba to) inspiruje Kawauchiego do wprowadzenia kolejnej definicji: dwie macierze $V_1, V_2 \in \Theta$ nazywa kobordantnymi, jeżeli $(-V_1) \oplus V_2$ jest zerowo kobordantna.
Kobordyzm stanowi relację równoważności na $\Theta$ -- iloraz $\Theta$ przez tę relację oznacza się $G_-$, jest grupą abelową.

\begin{proposition}
    % Kawauchi 12.2.8
    Odwzorowanie $\psi \colon C^1 \to G_-$ posyłające klasę abstrakcji węzła w klasę abstrakcji jego macierzy Seiferta jest dobrze określonym epimorfizmem.
\end{proposition}

\begin{proof}
    Nie umiemy nic udowodnić, więc wymienimy tylko trzy odsyłacze: z faktu~\ref{prp:cobordant_to_algebraic_is_algebraic} wynika, że odwzorowanie $\psi$ jest dobrze określone, dowód faktu~\ref{prp:signature_additive} pokazuje, że $\psi$ jest homomorfizmem, zaś w \cite[s. 62]{kawauchi1996} można przeczytać, dlaczego jest ,,na''.
\end{proof}

Funkcję $\psi$ rozpatrywał Levine \cite{levine1969} w latach sześćdziesiątych.
\index[persons]{Levine, Jerome}%
Po mniej niż dekadzie Casson, Gordon \cite{casson1978} wskazali nietrywialne elementy jądra.
\index[persons]{Casson, Andrew}%
\index[persons]{Gordon, Cameron}%
% to wyżej wiem z kawauchi98, "Supplementary notes for Chapter 12"
Potem był wynik Jianga \cite{jiang1981}, że jądro nie jest skończenie generowalne, bo zawiera izomorficzną kopię $\Z^\infty$, a~jeszcze później Livingstona \cite{livingston1999}, że zawiera też kopię $(\Z/2\Z)^\infty$.
% to wyżej wiem z https://mathscinet.ams.org/mathscinet-getitem?mr=2179265, pierwsze strony tekstu (nie recenzji)
\index[persons]{Jiang, Boju}%
\index[persons]{Livingston, Charles}%

\begin{proposition}
    $G_- \cong \Z^\infty \oplus (\Z/4\Z)^\infty \oplus (\Z/2\Z)^\infty$.
\end{proposition}

Kawauchi \cite[s. 161]{kawauchi1996} bez uzasadnienia postanawia nie przytoczyć dowodu tego faktu, ale opowiada krótko, jaka jest idea przewodnia i odsyła wprost do pracy Levine'a.
Na dalszych stronach jego pracy przeglądowej pojawiają się jakieś formy kwadratowe oraz uogólnienia wszystkiego do zgodności splotów, ale wracamy nocnym pociągiem i zaraz uśniemy...




\subsection{Węzły taśmowe}
\index{węzeł!taśmowy|(}%
\begin{definition}
    Węzeł $K = f[S^1]$ będący brzegiem osobliwego dysku $f \colon D \to S^3$ posiadającego następującą własność: każda przecinająca siebie składowa jest łukiem $A \subseteq f(D^2)$, dla którego $f^{-1}[A]$ składa się z~dwóch łuków w~$D^2$ (jeden z~nich jest wewnętrzny), nazywamy taśmowym.
\end{definition}

Jak pisze Kawauchi, mamy oczywiste wynikanie:
% TODO: która strona

\begin{proposition}
\index{węzeł!plastrowy}%
    Każdy węzeł taśmowy jest plastrowy.
\end{proposition}

Dawno temu Fox \cite[problem 1.33]{kirby78} zapytał, czy implikacja odwrotna jest prawdziwa:
\index[persons]{Fox, Ralph}%

\begin{conjecture}[slice-ribbon problem]
    \index{hipoteza!plastrowo-taśmowa}
    Czy każdy węzeł plastrowy jest taśmowy?
\end{conjecture}

Wprawdzie Lisca \cite{lisca07} pokazał prawdziwość hipotezy dla węzłów dwumostowych,
\index[persons]{Lisca, Paolo}%
% korzystając ze słynnego tw. Donaldsona: that a definite intersection form of a compact, oriented, simply connected, smooth manifold of dimension 4 is diagonalisable
\index{węzeł!dwumostowy}%
zaś Greene oraz Jabuka \cite{greene11} zrobili to dla precli o trzech pasmach;
\index[persons]{Greene, Joshua}%
\index[persons]{Jabuka, Stanisław}%
\index{precel}%
ale Gompf, Scharlemann i~Thompson \cite{gompf10} zasugerowali potencjalny kontrprzykład.
\index[persons]{Gompf, Robert}%
\index[persons]{Scharlemann, Martin}%
\index[persons]{Thompson, Abigail}%
\index{rozmaitość!szwowa}%
Nie możemy przytoczyć tu tego kontrprzykładu, gdyż korzysta z~rozmaitości szwowych, opisanych w~\cite[s. 53-59]{kawauchi96}.

Teichner myśli\footnote{Patrz \url{https://mathoverflow.net/a/18154}.} o hipotezie plastrowo-taśmowej jako o~życzeniu, które uprościłoby pewne czterowymiarowe problemy, gdyby było prawdziwe.
\index[persons]{Teichner, Peter}%

\index{węzeł!taśmowy|)}

% koniec podsekcji węzły taśmowe




%%% Kawauchi 157:
\subsection{Węzły algebraicznie plastrowe}
Tekst tej podsekcji to w dużej części bełkot, więc nie ma nic złego w zignorowaniu jej tak bardzo, jak tylko można.
W telegraficznym skrócie: węzeł, którego macierz Seiferta jest zerowo kobordantna, nazywamy plastrowym algebraicznie.
Lokalnie płaską, zwartą, zorientowaną, właściwą powierzchnię $S$ w $B^4$ taką, że $K = \partial S$ jest węzłem w $\partial B^4 = S^3$ nazywamy izotropową, jeżeli istnieje lokalnie płaska, zwarta, zorientowana 3-podrozmaitość $M \subseteq B^4$, gdzie $S \subseteq \partial M$ oraz $F = \operatorname{cl} \partial M \setminus S$ jest powierzchnią Seiferta dla $K$ w $S^3$, zaś $S$ jest izotropowa w $M$.

\begin{proposition}
    Węzeł $K$ w~$S^3$ jest algebraicznie plastrowy dokładnie wtedy, gdy ogranicza izotropową powierzchnię $S$ w~kuli $B^4$.
\end{proposition}

\begin{corollary}
    Niech $K$ będzie węzłem plastrowym.
    Wtedy $K$ jest węzłem algebraicznie plastrowym.
\end{corollary}

\begin{proof}
    Kawauchi \cite[s. 158]{kawauchi1996}.
\end{proof}

\begin{proposition}
    \label{prp:cobordant_to_algebraic_is_algebraic}
    Niech $K$ będzie węzłem zgodnym z węzłem algebraicznie plastrowym.
    Wtedy każda macierz Seiferta dowolnej powierzchni Seiferta $K$ jest zerowo kobordantna.
    W szczególności, $K$ jest węzłem algebraicznie plastrowym.
\end{proposition}

\begin{proof}
    Kawauchi \cite[s. 159]{kawauchi1996}.
\end{proof}

% Theorem 1.3[Long 1984].A strongly positive amphicheiral knot is algebraicallyslice.
% Theorem 1.4[Hartley and Kawauchi 1979].If K is strongly positive amphicheiral,the Alexander polynomial1Kis the square of a symmetric polynomial.



