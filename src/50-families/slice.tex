\section{Węzły plastrowe i taśmowe}
\label{sec:slice}
Węzły plastrowe i taśmowe oraz pojęcie kobordyzmu, które wkrótce opiszemy, należą do świata 4-wymiarowej teorii węzłów.
Nie zapoznamy się z nią bliżej oraz nie podamy naszego ulubionego odniesienia do tego tematu w~literaturze, ponieważ sami nie rozumiemy go zbyt dobrze.
Wszystko zaczęło się od artykułu \cite{fox66} Foxa, Milnora.

%%% Kawauchi 155:

\begin{definition}[płaski dysk]
    Niech $D \subseteq B^4$ będzie dyskiem posiadającym otoczenie $N$, kopię zbioru $D \times I
    ^2$, która przecina sferę $S^3$ dokładnie w $\partial D \times I^2$.
    Mówimy wtedy, że dysk $D$ jest płaski.
\end{definition}

\begin{tobedone}
    Płaski? Lokalnie płaski?
    \cite[s. 155]{kawauchi96}
\end{tobedone}

\begin{definition}[węzeł plastrowy]
    \index{węzeł!plastrowy}
    Niech $K \subseteq S^3$ będzie takim węzłem, że w kuli $B^4$ istnieje płaski dysk $D$ taki, że $K = \partial D = D \cap S^3$.
    Wtedy $K$ nazywamy węzłem plastrowym.
\end{definition}

Następujące węzły o~mniej niż jedenastu skrzyżowaniach są plastrowe: $6_1$, $8_{8}$, $8_{9}$, $8_{20}$, $9_{27}$, $9_{41}$, $9_{46}$, $10_{3}$, $10_{22}$, $10_{35}$, $10_{42}$, $10_{48}$, $10_{75}$, $10_{87}$, $10_{99}$, $10_{123}$, $10_{129}$, $10_{137}$, $10_{140}$, $10_{153}$ oraz $10_{155}$.
\index{węzeł!Conwaya}
Wśród pierwszych węzłów do dwunastu skrzyżowań najdłużej opierał się węzeł Conwaya, aż Piccirillo pokazała w~\cite{piccirillo20}, że nie jest plastrowy.

\begin{proposition}
    Niech $K$ będzie węzłem.
    Wtedy $K \shrap mr K$ jest węzłem plastrowym.
\end{proposition}

\begin{proof}[Niedowód]
    Pierwszy był Fox z Milnorem \cite{fox66}, patrz także lemat 12.1.2.2 w \cite{kawauchi96}.
\end{proof}

\begin{proposition}
    Albo wszystkie trzy węzły $K_1, K_2, K_1 \shrap K_2$ są plastrowe, albo co najwyżej jeden z~nich.
\end{proposition}

\begin{proof}[Niedowód]
    Lemat 12.1.2.3 w \cite{kawauchi96}.
\end{proof}

Pierwszym poważnym wynikiem z dziedziny teorii węzłów plastrowych, pochodzącym jeszcze z pracy \cite{fox66}, był:

\begin{proposition}[warunek Foxa-Milnora]
    \index{warunek!Foxa-Milnora}
    Niech $K$ będzie węzłem plastrowym.
    Wtedy jego wielomian Alexandera jest postaci $\alexander(t) = f(t) f(1/t)$ dla pewnego wielomianu Laurenta $f \in \Z[t, 1/t]$.
\end{proposition}

\begin{corollary}
    Wyznacznik węzła plastrowego jest kwadratem.
\end{corollary}

\begin{proof}
    Mamy $\det K = |\alexander(-1)| = f(-1) f(-1)$.
\end{proof}

Ten prosty test stwierdza, że 2743 spośród 2977 węzłów o mniej niż 13 skrzyżowaniach nie jest plastrowych.

\begin{proposition}
    \label{prp:slice_signature}
    Niech $K$ będzie węzłem plastrowym.
    Wtedy $\sigma(K) = 0$.
\end{proposition}

\begin{tobedone}[Szkic dowodu]
    Ustalmy odwzorowanie $f$, które jest niesingularne, symetryczne i~dwuliniowe, z~przestrzeni $V$ o~wymiarze $2n$ oraz wyznaczoną przez nie formę kwadratową.
    Jeśli znika ona na podprzestrzeni wymiaru $n$, to ma zerową sygnaturę.
    dowód znaleziony w~podręczniku Lickorisha.
    Patrz też twierdzenie 8.8 z~artykułu \cite{murasugi65}.
    Praca "Infinite Order Amphicheiral Knots". (Charles Livingston, 2001) -- chyba nie?
\end{tobedone}

Test ten eliminuje kolejne 45 węzłów poniżej 13 skrzyżowań.

\begin{proposition}
    Niech $K$ będzie węzłem plastrowym.
    Wtedy $\operatorname{Arf} K = 0$.
\end{proposition}

\begin{proof}
    Ustalmy węzeł $K$, wiemy już, że jego wyznacznik jest kwadratem, a na mocy faktu \ref{cor:knot_determinant_odd} także tyle, że jest liczbą nieparzystą.
    Wynika stąd przystawanie $\det K \equiv 1 \mod 8$, które w~połączeniu z warunkiem Murasugiego (fakt \ref{prp:arf_murasugi}) daje $\operatorname{Arf} K = 0$.
\end{proof}

%%% Kawauchi 156:

Wprowadzimy teraz relację równoważności na zbiorze węzłów, która prowadzi przez fakt \ref{prp:cobordant_iff_sum_slice} do alternatywnej definicji węzłów plastrowych.

\begin{definition}[zgodność]
    Dwa węzły $K_0, K_1$ takie, że istnieje lokalnie płaski, zorientowany, właściwy pierścień $C$ taki, że $C \cap S^3 \times i = K_i \times i$, nazywamy zgodnymi.
    
    Dwa sploty $K, L \subseteq S^n$ nazywamy zgodnymi (z angielskiego \emph{concordant}), jeśli istnieje włożenie $f \colon K \times [0,1] \to S^n \times [0,1]$ spełniające dwa warunki: $f(K \times 0) = K \times 0$ oraz $f(K \times 1) = L \times 1$.
\end{definition}

Zgodność to zazwyczaj concordance, rzadziej używa się terminu cobordism.

\begin{proposition}
    \label{prp:cobordant_iff_sum_slice}
    Dwa węzły $K_1, K_2$ są zgodne wtedy i tylko wtedy, gdy suma $mr K_0 \shrap K_1$ jest plastrowa.
\end{proposition}

\begin{proof}
    Ćwiczenie 12.1.3 w \cite{kawauchi96}.
\end{proof}

\begin{definition}
    Węzeł zgodny z~niewęzłem nazywamy plastrowym.
\end{definition}

,,Bycie zgodnym'' jest relacją równoważności, słabszą od bycia izotopijnym.
% ale mocniejszą od homotopii?
% izotopia: https://encyclopediaofmath.org/wiki/Cobordism_of_knots
Klasę abstrakcji węzła $K$ oznaczamy przez $[K]$.

\begin{definition}[grupa zgodności]
    \index{grupa!zgodności}
    Niech $C^1$ oznacza iloraz zbioru wszystkich węzłów przez relację zgodności.
    Zbiór $C^1$ wyposażony w~działanie
    \begin{equation}
        [K_1] + [K_2] = [K_1 \shrap K_2]
    \end{equation}
    staje się grupą abelową, nazywaną grupą zgodności.
    Jej elementem eneutralnym jest klasa abstrakcji niewęzła.
    Elementem przeciwnym do $[K]$ jest $[mr K]$.
\end{definition}

%%% Kawauchi 157:

Niech $\Theta$ oznacza rodzinę macierzy Seiferta, kwadratowych macierzy $V$ o całkowitych wyrazach takich, że $\det (V - V^T) = 1$.
Mówimy, macierz $V \in \Theta$ jest zerowo kobordantna, jeśli istnieje całkowitoliczbowa macierz $P$ o~wyznaczniku równym $\pm 1$, że
\begin{equation}
    V = P \begin{pmatrix} 0 & V_{21} \\ V_{12} & V_{22} \end{pmatrix} P^{-1}
\end{equation}
Takie macierze nazywamy unimodularnie sprzężonymi.

\begin{proposition}
    Niech $V \in \Theta$ będzie macierzą zerowo kobordantną.
    Wtedy istnieje plastrowy węzeł $K$, którego macierzą Seiferta jest $V$.
\end{proposition}

\begin{proof}
    \cite[proposition 12.2.1]{kawauchi96}
\end{proof}

Przez analogię, o dwóch macierzach $V_1, V_2 \in \Theta$ mówimy, że są kobordantne, jeżeli $(-V_1) \oplus V_2$ jest zerowo kobordantna.
Kobordyzm jest znowu relacją równoważności, iloraz $\Theta$ przez nią oznaczamy przez $G_-$, a elementy tego ilorazu jako $[V]$.
Wyposażony w działanie $[V_1] + [V_2] = [V_1 \oplus V_2]$ staje się grupą abelową.

\begin{proposition}
    % Kawauchi 12.2.8
    Odwzorowanie $\psi \colon C^1 \to G_-$ posyłające klasę abstrakcji węzła w klasę abstrakcji jego macierzy Seiferta jest dobrze określonym epimorfizmem.
\end{proposition}

\begin{proof}
    Funkcja $\psi$ jest dobrze określona na mocy faktu \ref{prp:cobordant_to_algebraic_is_algebraic}, jest homomorfizmem jak wynika z dowodu faktu \ref{prp:signature_additive}.
    To, że jest ,,na'', jest wnioskiem z \cite[s. 62]{kawauchi96}
\end{proof}

Funkcję $\psi$ rozpatrywał Levine \cite{levine69}.
Casson, Gordon pokazali w latach 70., że jej jądro jest niepuste \cite{gordon78}.

\begin{proposition}
    $G_- \cong \Z^\infty \oplus (\Z/4\Z)^\infty \oplus (\Z/2\Z)^\infty$.
\end{proposition}

% Dowód tego faktu jest pominięty nawet w pracy Kawauchiego

% \alexander(t) = 1 => K plastowy (w TOP) ale niekoniecznie (w PL)

% Kawauchi, supplementary notes for Chapter 12: for a knot K with trivial Alexander polynomial it was shown by Freedman 1982 that K is necessarily a slice knot in TOP category but by...

% TODO: \textbf{Livingston: A SURVEY OF CLASSICAL KNOT CONCORDANCE}. The application of abelian knot invariants (those determined by the cohomology of abelian covers or, equivalently, by the Seifert form) to concordance culminated in 1969 with Levine’s classification of higher dimensional knot concordance, [62, 63], which applied in the classical dimension to give a surjective homomorphism $\varphi \colon C \to \Z^\infty \oplus \Z_2^\infty \oplus \Z_4^\infty$. In 1975 Casson and Gordon [8, 9] proved that Levine’s homomorphism is not an isomorphism, constructing nontrivial elements in the kernel, and Jiang expanded on this to show that the kernel contains a subgroup isomorphic to $\Z_2^\infty$. More significant, Freedman proved that all knots with trivial Alexander polynomial are in fact slice in the topological locally flat category.


\subsection{Węzły taśmowe}
\index{węzeł!taśmowy|(}
\begin{definition}
    Węzeł $K = f(S^1)$ będący brzegiem osobliwego dysku $f \colon D \to S^3$ posiadającego następującą własność: każda przecinająca siebie składowa jest łukiem $A \subseteq f(D^2)$, dla którego $f^{-1}(A)$ składa się z~dwóch łuków w~$D^2$ (jeden z~nich jest wewnętrzny), nazywamy taśmowym.
\end{definition}

Jak pisze Kawauchi, mamy oczywiste wynikanie:

\begin{proposition}
\index{węzeł!plastrowy}%
    Każdy węzeł taśmowy jest plastrowy.
\end{proposition}

Dawno temu Fox zapytał, czy implikacja odwrotna jest prawdziwa (\cite[problem 1.33]{kirby78}):
\index[persons]{Fox, Ralph}%

\begin{conjecture}[slice-ribbon problem]
    \index{hipoteza!plastrowo-taśmowa}
    Czy każdy węzeł plastrowy jest taśmowy?
\end{conjecture}

Wprawdzie Lisca pokazał prawdziwość hipotezy dla węzłów dwumostowych \cite{lisca07},
\index[persons]{Lisca, ?}%
% korzystając ze słynnego tw. Donaldsona: that a definite intersection form of a compact, oriented, simply connected, smooth manifold of dimension 4 is diagonalisable
\index{węzeł!dwumostowy}%
zaś Greene oraz Jabuka zrobili to dla precli o trzech pasmach w~\cite{greene11};
\index[persons]{Greene, ?}%
\index[persons]{Jabuka, ?}%
\index{precel}%
ale Gompf, Scharlemann i~Thompson zasugerowali w~\cite{gompf10} potencjalny kontrprzykład.
\index[persons]{Gompf, ?}%
\index[persons]{Scharlemann, ?}%
\index[persons]{Thompson, ?}%
\index{rozmaitość szwowa}%
Nie możemy przytoczyć tego kontrprzykładu, gdyż korzysta z~rozmaitości szwowych, opisanych w~\cite[s. 53-59]{kawauchi96}.

Teichner myśli\footnote{Patrz \url{https://mathoverflow.net/a/18154}.} o hipotezie plastrowo-taśmowej jako o~życzeniu, które uprościłoby pewne czterowymiarowe problemy, gdyby było prawdziwe.
\index[persons]{Teichner, ?}%

\index{węzeł!taśmowy|)}

% koniec podsekcji węzły taśmowe



\subsection{Węzły skręcone}
\begin{definition}
    \index{węzeł!skręcony}
    \label{def:twist_knot}
    Węzeł powstały przez $n$-krotne półskręcanie domkniętej pętli oraz splecienie końców nazywamy węzłem skręconym.
\end{definition}

Węzły skręcone to dokładnie towarzyszące niewęzłowi w~węzłach satelitarnych, tak zwane whiteheadowskie duble niewęzła.
Wszystkie są odwracalne (ale tylko niewęzeł oraz ósemka są amfichiralne) i~mają liczbę gordyjską $1$, ponieważ wystarczy rozwiązać skrzyżowanie, które plotło końce.
Każdy jest $2$-mostowy i~posiada zerową sygnaturę.
Dalsze własności węzłów skręconych zależą od $n$, ilości półskrętów.
Indeks skrzyżowaniowy wynosi $n + 2$.

\begin{proposition}
    Wielomianowymi niezmiennikami węzłów skręconych są:
    \begin{align*}
    (q+1)\jones(q) & = \begin{cases}
        1+q^{-2}+q^{-n}-q^{-n-3} & n \mbox{ nieparzyste} \\
        q^{3}+q-q^{3-n}+q^{-n} & n \mbox{ parzyste}
    \end{cases} \\
    2 \conway (z) & = \begin{cases}
        (n+1) z^{2} + 2 & n \mbox{ nieparzyste} \\
        2 - nz^2 & n \mbox{ parzyste}
    \end{cases}
    \end{align*}
\end{proposition}

\begin{proposition}
    Niewęzeł oraz węzeł dokerski $6_1$ są jedynymi skręconymi węzłami plastrowymi.
\end{proposition}

\begin{proof}
    \cite{casson86}.
\end{proof}

