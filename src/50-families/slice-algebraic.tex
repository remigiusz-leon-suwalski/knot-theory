%%% Kawauchi 157:
\subsection{Węzły algebraicznie plastrowe}
Węzeł, którego macierz Seiferta jest zerowo kobordantna, nazywamy plastrowym algebraicznie.
Lokalnie płaską, zwartą, zorientowaną, właściwą powierzchnię $S$ w $B^4$ taką, że $K = \partial S$ jest węzłem w $\partial B^4 = S^3$ nazywamy izotropową, jeżeli istnieje lokalnie płaska, zwarta, zorientowana 3-podrozmaitość $M \subseteq B^4$, gdzie $S \subseteq \partial M$ oraz $F = \operatorname{cl} \partial M \setminus S$ jest powierzchnią Seiferta dla $K$ w $S^3$, zaś $S$ jest izotropowa w $M$.

\begin{proposition}
    Węzeł $K$ w~$S^3$ jest algebraicznie plastrowy dokładnie wtedy, gdy ogranicza izotropową powierzchnię $S$ w~kuli $B^4$.
\end{proposition}

\begin{corollary}
    Niech $K$ będzie węzłem plastrowym.
    Wtedy $K$ jest węzłem algebraicznie plastrowym.
\end{corollary}

\begin{proof}
    \cite[s. 158]{kawauchi96}
\end{proof}

\begin{proposition}
    \label{prp:cobordant_to_algebraic_is_algebraic}
    Niech $K$ będzie węzłem zgodnym z węzłem algebraicznie plastrowym.
    Wtedy każda macierz Seiferta dowolnej powierzchni Seiferta $K$ jest zerowo kobordantna.
    W szczególności, $K$ jest węzłem algebraicznie plastrowym.
\end{proposition}

\begin{proof}
    \cite[s. 159]{kawauchi96}
\end{proof}

% Theorem 1.3[Long 1984].A strongly positive amphicheiral knot is algebraicallyslice.
% Theorem 1.4[Hartley and Kawauchi 1979].If K is strongly positive amphicheiral,the Alexander polynomial1Kis the square of a symmetric polynomial.
