
\subsection{Reprezentacja Burau}
Na zakończenie sekcji wspomnijmy o~macierzowej reprezentacji Burau, wprowadzonej do matematyki w latach trzydziestych zeszłego wieku \cite{burau33}.
\index[persons]{Burau, Werner}%
\index{reprezentacja!Burau}%
Wyznaczona jest ona przez obrazy generatorów:
\begin{equation}
    \varphi(\sigma_i) = I_{i-1} \oplus \begin{pmatrix}
        1-t & t \\
        1   & 0
    \end{pmatrix} \oplus I_{n-i-1}
\end{equation}
Bezpośredni rachunek dowodzi, że reprezentacja $\varphi$ jest wierna dla $n \le 2$.
Magnus, Peluso \cite{peluso69} pokazali to samo dla $n = 3$, ale ich pracę czyta się tak trudno, że polecamy sięgnąć raczej po to, co napisała Birma \cite[s. 129]{birman74} albo Kassel, Turajew \cite[s. 110]{kassel08}.
Z drugiej strony Moody \cite{moody91} odkrył używając całek po konturach i nie używając komputera, że reprezentacja nie jest wierna dla $n \ge 9$.
\index[persons]{Moody, John}%
Long, Paton \cite{paton93} zauważyli związek między technikami Moody'ego, dualnością Poincarégo i formą Squiera, co pozwoliło ulepszyć wynik do $n \ge 6$.
\index[persons]{Paton, Mark}%
\index[persons]{Long, Darren}%
% Paton = Mark https://www.genealogy.math.ndsu.nodak.edu/id.php?id=139714
Ich kontrukcja korzysta z~pewnej zamkniętej krzywej na sześciokrotnie przekłutym dysku o~pewnych cechach homologicznych.
Bigelow \cite{bigelow99} pokazał u schyłku stulecia podobnymi metodami (ale już ze wsparciem komputera), że przypadek $n = 5$ też jest niewierny: jeśli
\index[persons]{Bigelow, Stephen}%
\begin{align}
    \psi_1 & = \sigma_3^{{-1}}\sigma_2\sigma_1^2\sigma_2\sigma_4^3\sigma_3\sigma_2, \\
\psi_2 & = \sigma_4^{{-1}}\sigma_3\sigma_2\sigma_1^{{-2}}\sigma_2\sigma_1^2\sigma_2^2\sigma_1\sigma_4^5,
\end{align}
to komutator $[\psi_1^{{-1}}\sigma_4\psi_1,\psi_2^{{-1}}\sigma_4\sigma_3\sigma_2\sigma_1^2\sigma_2\sigma_3\sigma_4\psi_2]$ należy do jądra.
Czy reprezentacja Burau dla $B_4$ jest wierna?
Negatywna odpowiedź na to pytanie prawie na pewno prowadziłaby do
nietrywialnego węzła, którego wielomianem HOMFLY jest $1$,
natomiast odpowiedź pozytywna raczej nie ma aż tak dramatycznych następstw.

Część informacji pochodzi z artykułu ,,Burau representation'' \url{https://w.wiki/7Wpz}.

