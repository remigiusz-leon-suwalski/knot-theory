\subsection{Spin} % (fold)
\label{sub:writhe}
Przypomnijmy, że znak skrzyżowania na diagramie to liczba $1$ lub $-1$ (definicja \ref{sign_def}).

\begin{definition}[spin]
	\index{spin}
	Wielkość
	\begin{equation}
		w(D) = \sum_c \operatorname{sign} c,
	\end{equation}
	gdzie sumowanie przebiega po wszystkich skrzyżowaniach diagramu $D$ zorientowanego splotu lub węzła, nazywamy spinem.
\end{definition}

Co ważne, spin nie jest niezmiennikiem splotów ani węzłów.
Para Perko przedstawia ten sam węzeł z~minimalną liczbą skrzyżowań i~spinem równym siedem lub dziewięć.
Dzięki temu przez wiele lat nie została dostrzeżona.
Spin jest za to niezmiennikiem węzłów alternujących, mówi o~tym druga hipoteza Taita.

\begin{lemma}
	Spin nie zależy od orientacji.
	Tylko I ruch Reidemeistera zmienia spin: $w(\MalyreidemeisterIa) = w(\MalyreidemeisterIb )-1$, pozostałe ruchy nie mają na niego wpływu.
\end{lemma}

%\begin{proof}
	%Proste ćwiczenie.
%\end{proof}

% Koniec sekcji Spin
