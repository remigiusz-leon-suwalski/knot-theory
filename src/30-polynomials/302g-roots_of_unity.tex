
\subsection{Wartości w pierwiastkach jedności}
Niech $L$ będzie splotem mającym $n$ ogniw, zaś $\jones$ jego wielomianem Jonesa.
Wtedy wartości $\jones(\omega)$ w~niektórych pierwiastkach jedności są związane z~innymi niezmiennikami węzłów.

I tak przyjmując oznaczenie $\omega_k = \exp(2\pi i/k)$ mamy

\begin{proposition}
    \label{prp:jones_at_roots_of_unity}
    $\jones_L(\omega_3) = 1$.
\end{proposition}

\begin{proposition}
    $\jones_L(1) = (-2)^{n-1}$.
\end{proposition}

\begin{proof}
\index[persons]{Jones, Vaughan}%
    Jak wkrótce się przekonamy, to proste wnioski z~relacji kłębiastej.
    Explicite wskazał je Jones \cite[twierdzenie 14, 15]{jones85}.
\end{proof}

\begin{proposition}
    Pochodna w punkcie $t = 1$ znika: $\jones'_L(1) = 0$.
\end{proposition}

\begin{proof}
    Jones \cite[twierdzenie 16]{jones85}.
\end{proof}

\begin{proposition}
    $V_L(\omega_6) = \pm i^{n-1} \cdot (\sqrt 3i)^r$, gdzie $r$ jest rangą pierwszej grupy homologii podwójnego rozgałęzionego nakrycia $L$ nad $\Z/3\Z$.
\end{proposition}

\begin{proof}
\index[persons]{Lipson, Andrew}%
    Znak $\pm$ został wyznaczony przez Lipsona w \cite{lipson86}, praca ta zawiera też odsyłacz do pracy Milletta, Lickorisha \cite{lickorish86}, którzy wyprowadzają resztę wzoru.
\end{proof}

\begin{proposition}
    Liczba trzy-kolorowań splotu $L$ wynosi $3|\jones_L(\omega_6)|^2$.
\end{proposition}

\begin{proof}
\index[persons]{Przytycki, Józef}%
    Przytycki w \cite{przytycki98}.
\end{proof}

\begin{proposition}
    Jeśli $L$ jest właściwym splotem (indeks zaczepienia każdej składowej o~resztę splotu jest parzysty), to $\jones_L(i) = (\sqrt 2)^{n-1}(-1)^{\operatorname{Arf} L}$.
    W przeciwnym razie $\jones_L(i) = 0$.
\end{proposition}

\begin{proof}
\index[persons]{Murakami, Hitoshi}%
    Murakami \cite{murakami86}, ta praca jest wyjątkowo krótka (tylko 3 strony).
\end{proof}

\begin{proposition}
    Niech $H_1$ będzie pierwszą grupą homologii podwójnego nakrycia $S^3$ rozgałęzionego nad składowymi.
    Jeśli $H_1$ jest torsyjna, to $\jones_L(-1) = |H_1|$.
    W przeciwnym razie $\jones_L(-1) = 0$.
\end{proposition}

Nie znamy dowodu tego faktu: Ohtsuki \cite[s. 383]{ohtsuki02} pisze tylko \emph{,,It is known that $|V_L(-1)|$ is equal to the order of $H_1(M_{2,L})$''}. \hfill \texttt{:(}

Nie jest znana topologiczna interpretacja wielomianu Jonesa (którą posiada wielomian Alexandera) ani charakteryzacja poza warunkami koniecznymi z~pięciu faktów powyżej.

\begin{corollary}
    Niech $K$ będzie węzłem.
    Wtedy
    \begin{align}
        \jones(1) & = 1 \\
        \jones(-1) & = \pm \det K \\
        \jones(i) & = \begin{cases}
            1 & \text{dla } \alexander(-1) \equiv \pm 1 \mod 8 \\
            -1 & \text{w przeciwnym razie.}
        \end{cases}
    \end{align}
\end{corollary}

Poza powyżej opisanymi przypadkami, wartości wielomianu Jonesa nie można znaleźć w~czasie wielomianowym od ilości skrzyżowań na diagramie (jest to problem $\#P$-trudny).

% Koniec sekcji Relacja kłębiasta
% Koniec podsekcji Wielomian Jonesa

\index{klamra Kauffmana|)}

% Koniec podsekcji Nawias Kauffmana

