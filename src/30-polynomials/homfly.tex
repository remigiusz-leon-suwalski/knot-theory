\section{Wielomian HOMFLY} % (fold)
\label{sec:homfly}
Po tym, jak Jones przedstawił światu swój wielomian w 1984 roku, matematycy
zaczęli szukać jego uogólnienia zależnego nie od jednej, lecz dwóch zmiennych.
Pierwszym takim węzłowym niezmiennikiem okazał się wielomian (Laurenta) HOMFLY.
Jego nazwa pochodzi od sześciu odkrywców -- Hoste, Ocneanu, Millett, Freyd, Lickorish, Yetter z pracy \cite{homfly85}.

Dwa lata później i niezależnie od nich, Przytycki z Traczykiem otrzymał ten sam obiekt w \cite{przytycki87}.
Ich nazwiska czasami są pomijane ze względu na wolno działającą pocztę (!).

\begin{definition}
    \label{homflydef}
    \index{wielomian!HOMFLY}
    Wielomian HOMFLY zorientowanego splotu $L$ to niezmienniczy na izotopie wielomian Laurenta dwóch zmiennych $m, l$ taki, że $P(\LittleUnknot) = 1$ oraz
    \[
        l P(L_+) +  \frac 1l P(L_-) + mP(L_0) = 0,
    \]
    przy oznaczeniach ze stwierdzenia \ref{tracheotomia}.
\end{definition}

Relacja kłębiasta pozwala wywnioskować własności wielomianu sumy spójnej i prostej.
Potrzebować będziemy lematu:

\begin{lemma}
    \label{links_homfly}
    Wielomianem HOMFLY dla niesplotu o $n$ składowych jest
    \[
        P(U_n) = \left(-\frac{l+1/l}{m}\right)^{n-1}.
    \]
\end{lemma}

%\begin{proof}
    % Dla dowodu tej równości nie trzeba powoływać się na fakt \ref{homfly_sums}.
Można potraktować to jako proste ćwiczenie z indukcji matematycznej, pozostawiam je uwadze Czytelnika.
%\end{proof}

\begin{proposition}
    \label{homfly_sums}
    Dla splotów $L_1, L_2$ zachodzą równości:
    \begin{align*}
        P(L_1 \sqcup L_2) & = - \frac{l + 1/l}{m} \cdot P(L_1) P(L_2) \\
        P(L_1 \shrap L_2) & = P(L_1) P(L_2)
    \end{align*}
\end{proposition}

\begin{proof}
    Dowód analogiczny do ,,multiplikatywności'' wielomianu Jonesa/Alexandera.

    Każdy splot $L$ jest kombinacją liniową (o współczynnikach będących wielomianami w $m$ oraz $l$) niesplotów $U_k$ o różnej liczbie składowych $k$.
    Mamy więc
    \[
        L_1 = \sum_{k=1}^n a_k U_k, \quad
        L_2 = \sum_{k=1}^n b_k U_k.
    \]

    Skorzystamy w tym miejscu z lematu \ref{links_homfly}.
    Wynika z niego, że $P(U_i)P(U_j) = P(U_{i+j-1})$ i bezpośrednio
    \[
        P(L_1)P(L_2) = \sum_{k=1}^{2n} \sum_{i=1}^{k-1} a_i b_{k-i}P(U_{k-1}).
    \]

    Pozostało spojrzeć na sumę spójną diagramów  $L_1 \shrap L_2$.
    Jeśli usuniemy teraz wszystkie skrzyżowania z diagramu $L_1$ relacją kłębiastą, dostaniemy $L_1 \# L_2 = \sum_{k=1}^n a_k (U_{k-1} \cup L_2)$, gdyż jedna z niezawęźlonych składowych zostanie wchłonięta do diagramu diagramu $L_2$.
    Rozwijamy dalej i otrzymujemy
    \begin{align*}
        L_1 \# L_2 & = \sum_{k=1}^n a_k \left(U_{k-1} \cup \sum_{k=1}^n b_k U_k\right) \\
                   & = \sum_{k=1}^{2n} \sum_{i=1}^{k-1} a_i b_{k-i} U_{i-1} \cup U_{k-i} \\
                   & = \sum_{k=1}^{2n} \sum_{i=1}^{k-1} a_i b_{k-i} U_{k-1},
    \end{align*}
    co kończy dowód.
\end{proof}

Nie wiemy jeszcze, czy definicja \ref{homflydef} pozwala wyliczyć wielomian w skończenie wielu krokach, ani czy wynik jest jednoznaczny.
Przejdę do pokazania, że tak istotnie jest.

\begin{lemma}
    W dowolnym rzucie splotu można odwrócić pewne skrzyżowania tak, by uzyskać diagram niesplotu.
\end{lemma}

Zauważmy, że wyznaczenie nawiasu Kauffmana było prostsze, ponieważ w każdym kroku liczba skrzyżowań ulegała zmniejszeniu.

\begin{proof}
Bez straty ogólności założę, że diagram przedstawia węzeł.
Ustalmy zatem diagram węzła i wybierzmy jakiś początkowy punkt na nim, różny od skrzyżowania wraz z kierunkiem, wzdłuż którego będziemy przemierzać węzeł.
Za każdym razem, kiedy odwiedzamy nowe skrzyżowanie, zmieniamy je w razie potrzeby na takie, przez które przemieszczamy się górnym łukiem.
Skrzyżowań już odwiedzonych nie zmieniamy wcale.
Po powrocie do punktu wybranego na początku uzyskamy diagram niewęzła.

Teraz wyobraźmy sobie nasz węzeł w trójwymiarowej przestrzeni $\mathbb R^3$, przy czym oś $z$ skierowana jest z płaszczyzny, w której leży diagram, w naszą stronę.
Umieśćmy początkowy punkt tak, by jego trzecią współrzędną była $z = 1$.

Przemierzając węzeł, zmniejszamy stopniowo tę współrzędną, aż osiągniemy wartość $0$ tuż przed punktem, z którego wyruszyliśmy.
Połączmy obydwa punkty (początkowy oraz ten, w którym osiągamy współrzędną $z = 0$) pionowym odcinkiem.
Zauważmy, że kiedy patrzymy na węzeł w kierunku osi $z$, nie widzimy żadnych skrzyżowań.

Oznacza to, że dostaliśmy niewęzeł.
\end{proof}

\begin{proposition}
    Wielomian HOMFLY z definicji \ref{homflydef} dla zorientowanych splotów można wyznaczyć w skończenie wielu krokach.
\end{proposition}

\begin{proof}
    Niech $L$ będzie splotem, którego wielomian HOMFLY próbujemy wyznaczyć.
    Ustalmy jego dowolny diagram i wybierzmy jedno ze skrzyżowań, które należy odwrócić, by uzyskać niesplot.

    Początkowy diagram odpowiada $L_+$ lub $L_-$, relacja kłębiasta pozwala na uzyskanie wielomianu wyjściowego splotu zależnego od wielomianu splotu z
    diagramem, na którym jest mniej skrzyżowań oraz splotu, który jest ,,jedno skrzyżowanie bliżej'' niesplotu.

    Powtarzając tę procedurę dojdziemy w pewnym momencie do splotów trywialnych, gdzie powołujemy się na wniosek \ref{links_homfly}.
\end{proof}

Wielomian HOMFLY jest dużo mocniejszy od wielomianów Jonesa czy Alexandera -- są one jego szczególnymi przypadkami.
Warto pamiętać, że żaden z nich nie jest mocniejszy od drugiego, gdyż istnieją pary węzłów, które są rozróżnialne przez dokładnie jeden z nich.

\begin{proposition}
    Dla dowolnego zorientowanego węzła zachodzą równości:
        \begin{align*}
        V(t) & = P(l = i/t, m = i(t^{-1/2} - t^{1/2})) \\
        \Delta(t) & = P(l = i, m = i(t^{-1/2} - t^{1/2}))
    \end{align*}
\end{proposition}

Zaletą wielomianu HOMFLY jest to, że często wykrywa chiralność (węzeł chiralny nie jest równoważny swemu lustrzanemu odbiciu), ale nie odróżnia enancjomerów węzłów $9_{42}$, $10_{48}$, $10_{71}$, $10_{91}$, $10_{104}$, oraz $10_{125}$.
Wśród węzłów do dziesięciu skrzyżowań dokładnie dwa opierają się testom chiralności opartym na niezmiennikach Jonesa, HOMFLY oraz Kauffmana: $9_{42}$ oraz $10_{71}$.
Do jej wykrycia potrzebna jest na przykład $SU(2)$-teoria Cherna-Simonsa, wyjaśniona w kolejnej wersji tego skryptu.

M. B. Thistlethwaite wyznaczył wielomiany HOMFLY dla wszystkich 12966 węzłów, które mają mniej niż 14 skrzyżowań.
Wśród nich 30 posiada wielomian Conwaya $1 + 2z^2 + 2z^4$, ale pary rozróżniane wielomianem HOMFLY mają także różne wielomiany Jonesa.
Wielomian HOMFLY odróżnia $11_{388}$ od swojego odbicia, wielomiany Jonesa i Conwaya nie -- jest więc od nich istotnie mocniejszy.

Pomimo swej mocy nie jest jednak niezmiennikiem zupełnym, nie odróżnia od siebie mutantów\footnote{z definicji \ref{def:mutant}}.
Konkretne przykłady kolizji to $5_1$ oraz $10_{132}$, $8_{8}$ oraz $10_{129}$, $8_{16}$ oraz $10_{156}$ albo $10_{25}$ oraz $10_{56}$.
Co bardziej dramatyczne, Kanenobu wskazał (przy użyciu elementarnych metod!) w pracy \cite{kanenobu86} z 1986 roku przeliczalnie wiele węzłów, które są jednocześnie hiperboliczne, włókniste\footnote{fibered}, taśmowe\footnote{ribbon}, 3-mostowe, o genusie 2.
Żadnych dwóch spośród nich nie można rozróżnić wielomianem HOMFLY, ale posiadają nieizomorficzne moduły Alexandera, więc są (parami) różne.

% A POLYNOMIAL INVARIANT OF ORIENTED LINKS W. B. R. LICKORISH and KENNETH C. MILLET
% Example 16 - książka

Dla każdego splotu $L$ o $k$ składowych, $P_L(x,y) - 1$ jest krotnością $x+y-1$, zaś $P_L(x,y) + (-1)^K$ jest krotnością $x +y + 1$, zatem wielomian HOMFLY nigdy nie jest zerem.
Uwaga: to jest inna parametryzacja!

Podstawiając $x = a/z$ i $y = -1/az$ mamy $a = l$, $z = -m$
% Koniec sekcji Wielomian HOMFLY
