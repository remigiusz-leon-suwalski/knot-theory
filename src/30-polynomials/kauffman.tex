\section{Wielomian Kauffmana} % (fold)
\label{sec:kauffman_polynomial}
Mniej więcej w~tym samym czasie, gdy odkryto wielomian BLM/Ho, Kauffman zaproponował sposób, dzięki któremu ten wielomian można uogólnić do odróżniającego lustra.
% Dopuszczał stosowanie tylko drugiego i~trzeciego ruchu Reidemeistera.

\index{wielomian!Kauffmana}
Wielomianu Kauffmana nie należy mylić z~nawiasem Kauffmana!
Jest to półzorientowany wielomian dwóch zmiennych dany wzorem
\begin{equation}
    % F_L(a, z) = a^{-w(L)} \langle |L| \rangle,
    F_K(a, z) = a^{-w(K)} L(K),
\end{equation}
gdzie $w$ to spin zorientowanego diagramu splotu, zaś $L(K)$ jest wielomianem wyznaczonym jednoznacznie przez cztery własności:
\begin{enumerate}
	\item $L(\LittleUnknot) = 1$,
	\item $L(s) = a^{-1} L(s_r) = a L(s_l)$, gdzie $s$ jest pojedynczym węzłem, zaś $s_r, s_l$ to jego obrazy względem dwóch wariantów I ruchu Reidemeistera,
	\item $L$ jest niezmienniczy względem II i III ruchu Reidemeistera,
	\item $L(\LittleLeftCrossing) + L(\LittleRightCrossing) = z L(\LittleLeftSmoothing) + z L(\LittleRightSmoothing)$.
\end{enumerate}

Stanowi uogólnienie wielomianu BLM/Ho: $F(1, x) = Q(x)$ oraz wielomianu Jonesa:
\begin{equation}
    \jones(t)=F(-t^{-3/4},t^{-1/4}+t^{1/4}).
\end{equation}

Jego związki z~wielomianem HOMFLY pozostają nieznane.
% Koniec sekcji Wielomian Kauffmana
