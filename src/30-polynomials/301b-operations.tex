
\subsection{Wielomian Alexandera a operacje na węzłach}
Wielomian Alexandera nie odróżnia luster $mL$ i~rewersów $rL$ od wyjściowych splotów $L$:

\begin{proposition}
\index{lustro}%
\index{rewers}%
    Niech $L$ będzie zorientowanym splotem.
    Wtedy $\alexander_{mL}(t) = \alexander_L(1/t) = \alexander_{rL}(t)$.
\end{proposition}

\begin{proof}
    Po odbiciu diagramu względem pionowej prostej skrzyżowanie z~definicji \ref{def:colouring_equation} też się odbija.
    Równanie związane z~nim zmienia się według schematu:
    \begin{equation}
        a + tc - ta - b = 0 \rightleftharpoons a + tb - ta - c = 0
    \end{equation}
    Pierwsze równanie z~$t$ zamienionym na $1/t$ staje się drugim równaniem przemnożonym przez $-1/t$, zatem $\alexander_{mL}(t) = \alexander_L(1/t)$.
    
    Dowód drugiej równości przebiega analogicznie.
\end{proof}

\begin{proposition}
\label{prp:alexander_multiplicative}%
    Niech $K_1, K_2$ będą zorientowanymi węzłami.
    Wtedy
    \begin{equation}
        \alexander_{K_1 \shrap K_2}(t) \equiv \alexander_{K_1}(t) \alexander_{K_2}(t).
    \end{equation}
\end{proposition}

\begin{proof}
    Wybierzmy poniższe diagramy dla węzłów $K_1$ oraz $K$:
\begin{comment}
    \[\begin{tikzpicture}[baseline=-0.65ex, scale=0.07]
    %\useasboundingbox (-5, -5) rectangle (5,5);
    \begin{knot}[clip width=7, end tolerance=1pt]
        \strand[thick] (-70, -20) rectangle (-30, 20);
        \strand[thick] (30, -20) rectangle ( 70, 20);
        \strand[thick] (-10, -10) [in=right, out=left] to (-25, 10);
        \strand[thick,-latex] (-25, 10) to (-30, 10);
        \strand[thick] (-30,-10) [in=left, out=right] to (-25, -10) to (-10, 10);
        \strand[thick] (-10, 10) [in=up, out=right] to (-5, 0) [in=right, out=down] to (-10, -10);

        % prawe strzalki
        \strand[thick] (30, 10) [in=right, out=left] to (25, 10) to (10, -10);
        \strand[thick,latex-] (30, -10) [in=right, out=left] to (25, -10) to (10, 10);
        \strand[thick] (10, 10) [in=up, out=left] to (5, 0) [in=left, out=down] to (10, -10);

        \node[darkblue] at (-50,10) [below] {$x_1,\ldots,x_{m-1}$};
        \node[red] at (-50,-10) [above] {$1,\ldots,m$};

        \node[darkblue] at (50,10) [below] {$y_1,\ldots,y_{n-1}$};
        \node[red] at (50,-10) [above] {$1,\ldots,n$};

        \node[darkblue] at (-20,-10)[below] {$x_m$};
        \node[darkblue] at (-10, 10)[above] {$x_0$};
        \node[darkblue] at (25,-10)[below] {$y_n$};
        \node[darkblue] at (10,10)[above] {$y_0$};
        \node[red] at ( 20,  0)[right]{$0$};
        \node[red] at (-20,  0)[left]{$0$};
    \end{knot}
    \end{tikzpicture}\]
\end{comment}
    Niech $M_1$ oraz $M_2$ oznaczają macierze otrzymane z~wielomianowych równań kolorujących dla $K_1$ oraz $K_2$ przez skreślenie pierwszej kolumny i~pierwszego wiersza.
    Wtedy $\alexander_{K_1}(t) = \det A$ oraz $\alexander_{K_2}(t) = \det B$.
    
    Poniższy diagram przedstawia sumę $K_1 \shrap K_2$:
\begin{comment}
    \[\begin{tikzpicture}[baseline=-0.65ex, scale=0.07]
        %\useasboundingbox (-5, -5) rectangle (5,5);
        \begin{knot}[clip width=5, end tolerance=1pt]
            \strand[thick] (-70, -20) rectangle (-30, 20);
            \strand[thick] (30, -20) rectangle ( 70, 20);
            \strand[thick] (-10, -10) [in=right, out=left] to (-25, 10);
            \strand[thick,-latex] (-25, 10) to (-30, 10);
            \strand[thick] (-30,-10) [in=left, out=right] to (-25, -10) to (-10, 10);
            \strand[thick] (-10, -10) to (10, -10);
            \strand[thick] (-10, 10) to (10, 10);

            % prawe strzalki
            \strand[thick] (30, 10) [in=right, out=left] to (25, 10) to (10, -10);
            \strand[thick,latex-] (30, -10) [in=right, out=left] to (25, -10) to (10, 10);

            \node[darkblue] at (-50,10) [below] {$x_1,\ldots,x_{m-1}$};
            \node[red] at (-50,-10) [above] {$1,\ldots,m$};

            \node[darkblue] at (50,10) [below] {$y_1,\ldots,y_{n-1}$};
            \node[red] at (50,-10) [above] {$1,\ldots,n$};

            \node[darkblue] at (-20,-10)[below] {$x_m$};
            \node[darkblue] at (0, 10)[above] {$z$};
            \node[darkblue] at (25,-10)[below] {$y_n$};
            \node[darkblue] at (0, -10)[below] {$x_0 = y_0$};
            \node[red] at ( 20,  0)[right]{$\zeta$};
            \node[red] at (-20,  0)[left]{$0$};
        \end{knot}
    \end{tikzpicture}\]
\end{comment}

    Ponumerujemy teraz łuki i skrzyżowania na nowo tak, by znalezienie wyznacznika nie wymagało dużo liczenia.
    Przyjmijmy kolejność dla łuków sumy od $x_0$, $x_1, \ldots, x_m$ przez $y_1, \ldots, y_n$ do $z$; zaś skrzyżowania ustawmy od $0, 1, \ldots, m$ (pochodzące od $K_1$) przez $1, \ldots, n$ (pochodzące od $K_2$) do $\zeta$.
    Wielomianowe równanie kolorujące dla $K_1 \shrap K_2$ nad wszystkimi skrzyżowaniami poza ostatnim są takie same, jak ich odpowiedniki przed dodaniem do siebie węzłów.
    (Jeżeli to stwierdzenie jest nieoczywiste, warto wykonać rysunek dla $K_1 = K_2 = 3_1$ i przeprowadzić stosowne rachunki).
    Natomiast nad skrzyżowaniem $\zeta$ stosowne równanie grzmi $(1-t)x_0 + tz - y_n = 0$.

    Z otrzymanej dużej macierzy skreślmy ponownie pierwszą kolumnę oraz pierwszy wiersz.
    Aby zakończyć dowód, musimy obliczyć jej wyznacznik.
    Mamy:
    \begin{align*}
        \alexander_{K_1 \shrap K_2}(t) & = \det \left(\begin{array}{cccc}
            M_1 &     &        & \\
                & M_2 &        & \\
                &     & \ddots & \\
                &     & -1     & t
    \end{array}\right) = \pm t \alexander_{K_1}(t) \alexander_{K_2}(t),
    \end{align*}
    gdzie ostatnia równość wynika z rozwinięcia Laplace'a wyznacznika względem ostatniej kolumny.
\end{proof}

Można zadać sobie pytanie, czy z równości $\Delta_{K_1}(t) = \Delta_{K_2}(t) \Delta_{K_3}(t)$ wynika, że $K_1$ jest sumą $K_2 \shrap K_3$ albo chociaż czy $K_1$ jest złożony.
Odpowiedź będzie negatywna, jak pokazuje przykład:
\begin{align}
    \Delta_{8_{17}}(t) & = 3(t^2 +t^{-2}) -8(t+t^{-1}) + 11 \\
    & = (t+t^{-1} + 1) (3(t+t^{-1}) - 5) \\
    & = \Delta_{3_1}(t) \Delta_{7_2}(t),
\end{align}
ponieważ wszystkie występujące w niej węzły są pierwsze.

% koniec podsekcji Wielomian Alexandera a operacje na węzłach

