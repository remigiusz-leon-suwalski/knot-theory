\subsection{Wielomian Alexandera a operacje na węzłach}

Wielomian Alexandera nie odróżnia luster i~rewersów od wyjściowych węzłów:

\begin{proposition}
    Niech $L$ będzie zorientowanym splotem.
    Wtedy $\alexander_{mL}(t) = \alexander_L(1/t) = \alexander_{rL}(t)$.
\end{proposition}

\begin{proof}
    Po odbiciu diagramu względem pionowej prostej skrzyżowanie z~definicji \ref{def:colouring_equation} też się odbija.
    Równanie związane z~nim zmienia się według schematu:
    \begin{equation}
        a + tc - ta - b = 0 \rightleftharpoons a + tb - ta - c = 0
    \end{equation}
    Pierwsze równanie z~$t$ zamienionym na $1/t$ staje się drugim równaniem przemnożonym przez $-1/t$.
    Dowód drugiej równości przebiega analogicznie.
\end{proof}

\begin{proposition}
    \label{prp:alexander_multiplicative}
    Niech $K_1, K_2$ będą zorientowanymi węzłami.
    Wtedy
    \begin{equation}
        \alexander_{K_1 \shrap K_2}(t) \equiv \alexander_{K_1}(t) \alexander_{K_2}(t)
    \end{equation}
\end{proposition}

\begin{proof}
    Wybierzmy poniższe diagramy dla węzłów $K_1$ oraz $K$:
\begin{comment}
    \[\begin{tikzpicture}[baseline=-0.65ex, scale=0.07]
    %\useasboundingbox (-5, -5) rectangle (5,5);
    \begin{knot}[clip width=5, end tolerance=1pt]
        \strand[semithick] (-70, -10) rectangle (-30, 10);
        \strand[semithick] ( 30, -10) rectangle ( 70, 10);
        \strand[semithick,Latex-] (-30, 5) .. controls (-22, 5) and (-18, -5) .. (-10, -5);
        \strand[semithick] (-30,-5) .. controls (-22, -5) and (-18, 5) .. (-10,  5);
        \strand[semithick] (-10, 5) [in=up, out=right] to (-5, 0) [in=right, out=down] to (-10, -5);

        % prawe strzalki
        \strand[semithick,-Latex] (30, 5) .. controls (22, 5) and (18, -5) .. (10, -5);
        \strand[semithick] (30,-5) .. controls (22, -5) and (18, 5) .. (10,  5);
        \strand[semithick] (10, 5) [in=up, out=left] to (5, 0) [in=left, out=down] to (10, -5);

        \node[darkblue] at (-50,5) {$x_1,\ldots,x_{m-1}$};
        \node[red] at (-50,-5) {$1,\ldots,m$};

        \node[darkblue] at (50,5) {$y_1,\ldots,y_{n-1}$};
        \node[red] at (50,-5) {$1,\ldots,n$};

        \node[darkblue] at (-30,-5)[below right] {$x_m$};
        \node[darkblue] at (-15,-5)[below] {$x_0$};
        \node[darkblue] at (30,-5)[below left] {$y_n$};
        \node[darkblue] at (15,-5)[below] {$y_0$};
        \node[red] at ( 19.5,  1)[above]{$0$};
        \node[red] at (-19.5,  1)[above]{$0$};
    \end{knot}
    \end{tikzpicture}
\]
\end{comment}
    Niech $A$ oraz $B$ oznaczają macierze otrzymane z~wielomianowych równań kolorujących dla $K_1$ oraz $K_2$ przez skreślenie skrajnie lewej kolumny i~górnego wiersza.
    Wtedy $\alexander_{K_1}(t) = \det A$ oraz $\alexander_{K_2}(t) = \det B$.
    Poniższy diagram przedstawia sumę $K_1 \shrap K_2$:

\begin{comment}
\[\begin{tikzpicture}[baseline=-0.65ex, scale=0.07]
    %\useasboundingbox (-5, -5) rectangle (5,5);
    \begin{knot}[clip width=5, end tolerance=1pt]
        \strand[semithick] (-70, -10) rectangle (-30, 10);
        \strand[semithick] ( 30, -10) rectangle ( 70, 10);
        \strand[semithick,Latex-] (-30, 5) .. controls (-22, 5) and (-18, -5) .. (-10, -5);
        \strand[semithick] (-30,-5) .. controls (-22, -5) and (-18, 5) .. (-10,  5);

        % prawe strzalki
        \strand[semithick] (30, 5) .. controls (22, 5) and (18, -5) .. (10, -5);
        \strand[semithick] (30,-5) .. controls (22, -5) and (18, 5) .. (10,  5);
        \strand[semithick] (10, 5) to (-10, 5);
        \strand[semithick,-Latex] (10, -5) to (-10, -5);

        \node[darkblue] at (-50,5) {$x_1,\ldots,x_{m-1}$};
        \node[red] at (-50,-5) {$1,\ldots,m$};

        \node[darkblue] at (50,5) {$y_1,\ldots,y_{n-1}$};
        \node[red] at (50,-5) {$1,\ldots,n$};

        \node[darkblue] at (-30,-5)[below right] {$x_m$};
        \node[darkblue] at (0,-5)[below] {$x_0 = y_0$};
        \node[darkblue] at (0, 5)[above] {$z$};
        \node[darkblue] at (30,-5)[below left] {$y_n$};
        \node[red] at ( 19.5,  1)[above]{$\zeta$};
        \node[red] at (-19.5,  1)[above]{$0$};
    \end{knot}
    \end{tikzpicture}\]
\end{comment}

    Uporządkujmy łuki na diagramie jako $x_0 = y_0$, $x_1, \ldots, x_m$, $y_1, \ldots, y_n$, $z$; skrzyżowania: $0, 1, \ldots, m$ (z $K_1$), $1, \ldots, n$ (z $K_2$), $\zeta$.
    Wielomianowe równanie kolorujące dla $K_1 \shrap K_2$ nad skrzyżowaniami $1, \ldots, m$ ($1, \ldots, n$) są takie same, jak przed dodaniem do siebie węzłów.
    Nad skrzyżowaniem $\zeta$ równanie orzeka, że $(1-t)y_0+t z-y_n=0$.

    Wynika stąd, że $\alexander_{K_1 \shrap K_2}(t)$ jest wyznacznikiem macierzy
    \begin{align*}
        M &= \left(\begin{array}{cc|cc|c}
            & & & & \\
            \multicolumn{2}{c|}{\smash{\raisebox{.5\normalbaselineskip}{$A$}}} & & \\
            \hline \\[-\normalbaselineskip]
            & & & & \\
            & & \multicolumn{2}{c|}{\smash{\raisebox{.5\normalbaselineskip}{$B$}}}\\ \hline
            & & & -1 & t
    \end{array}\right)
    \end{align*}

    Skreśliliśmy lewą kolumnę oraz górny wiersz.
    Zatem $\alexander_{K_1 \shrap K_2}(t) = t^?\alexander_{K_1}(t) \alexander_{K_2}(t)$, jeśli nie pomyliliśmy się w~obliczeniach.
\end{proof}

% koniec podsekcji Wielomian Alexandera a operacje na węzłach