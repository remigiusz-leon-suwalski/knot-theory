
\subsection{Różniaste różności}
\paragraph{Warunki Torresa}
Istnieje odmiana wielomianu Alexandera, która liczy sobie tyle zmiennych, ile ogniw posiada splot (nie opisywaliśmy jej i~nie zamierzamy).
Klasyfikację \ref{prp:alexander_hosokawa} można częściowo uogólnić: Torres \cite{torres1953} znalazł dwie geometryczne własności, nazwane później warunkami Torresa.
\index[persons]{Torres, Guillermo}%
\index{warunek!Torresa}%
Są warunkami koniecznymi, ale nie wystarczającymi, jak odkrył ponad ćwierć wieku później Hillman \cite{hillman1981}: wielomian
\index[persons]{Hillman, Jonathan}%
\begin{equation}
    D(x,y) = \frac{(1 - x^6y^6)(x - 1 + 1/x) - 2(1 - x^5y^5)(1 - x)(1 - y)}{1-xy}
\end{equation}
spełnia warunki Torresa, ale nie jest wielomianem Alexandera żadnego splotu.

\paragraph{Unimodalne współczynniki}
Fox \cite{fox1962} podejrzewał, że
\index[persons]{Fox, Ralph}%
\index{hipoteza!trapezoidalna}%
ciąg współczynników wielomianu Alexandera węzła alternującego jest unimodalny.
Dowód podano dla węzłów algebraicznych (Murasugi \cite{murasugi1985}) oraz genusu dwa (Ozsváth i~Szabó \cite{ozsvath2003}).
\index[persons]{Murasugi, Kunio}%
\index[persons]{Ozsváth, Peter}%
\index[persons]{Szabó, Zoltán}%
Hipoteza w~ogólnym przypadku pozostaje otwarta.
Wiemy natomiast, że kolejne współczynniki wielomianu Conwaya węzła alternującego są przeciwnych znaków i niezerowe (Murasugi \cite[s. 242]{murasugi1996} odsyła do swojej wcześniejszej pracy \cite{murasugi1959}, gdzie używa archaicznej nomenklatury: \emph{Schlauchknoten} zamiast węzłów satelitarnych jak w \cite[s. 245]{schubert1953}).
\index{Schlauchknoten}%
Wynika stąd, że węzły $(p, q)$-torusowe dla $p > q > 2$ oraz pierwsze węzły satelitarne nie są alternujące.

% koniec podsekcji Różniaste różności

