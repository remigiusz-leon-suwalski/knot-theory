\subsection{Definicja algebraiczna -- algebra Temperleya-Lieba} % (fold)
\label{sub:jones_paper}
Jones otrzymał swój wielomian jako efekt uboczny badań nad algebrami operatorowymi: wziął ślad pewnej reprezentacji warkoczy w~algebrę, która miała ważne znaczenie w~mechanice statystycznej.
Dalszy opis pochodzi z Wikipedii.
Zaletą tego podejścia jest możliwość wyboru algebry, która reprezentuje grupę warkoczy.

\begin{definition}[algebra Temperleya-Lieba]
    \index{algebra!Temperleya-Lieba}
    Niech $R$ będzie przemiennym pierścieniem, w~którym ustalono element $\delta \in R$.
    Wtedy $R$-algebrę $TL_n(\delta)$ generowaną przez elementy $e_1, \ldots, e_{n-1}$, które związane są relacjami
    \begin{align}
        e_i^2 & = \delta e_i, \\
        e_i e_{i \pm 1} e_i & = e_i, \\
        e_i e_j & = e_j e_i
    \end{align}
    dla $|i-j| \ge 2$, nazywamy algebrą Temperleya-Lieba.
    % Algebra Temperleya-Lieba $A_n$ to wolna addytywna algebra na multiplikatywnych generatorach $e_1, \ldots, e_{n-1}$ traktowana jako $\C[\tau, \tau^{-1}]$-moduł.
    % Zmienna $\tau$ komutuje ze wszystkimi generatorami, generatory zaś spełniają relacje ($j$ jest różne od $i - 1, i, i+1$):
\end{definition}

$TL_n(\delta)$ można przedstawić przy użyciu diagramów: prostokątów, których przeciwległe boki zawierają po $n$ punktów połączonych w~pary tak, by uniknąć samoprzecięć.
Mnożenie elementów algebry odpowiada sklejaniu dwóch diagramów, przy czym każdą zamkniętą pętlę zamieniamy na dodatkowy czynnik $\delta$.
To w~gruncie rzeczy są warkocze.

\begin{comment}
\begin{figure}[H]
\[
    \begin{tikzpicture}[baseline=-0.65ex, scale=0.2]
        \useasboundingbox (-6, -5) rectangle (6, 4);
        \begin{knot}[clip width=5, end tolerance=1pt]
            \strand[semithick] (-3, -4) to (3, -4);
            \strand[semithick] (-3, -2) to (3, -2);
            \strand[semithick] (-3, -0) to (3, +0);
            \strand[semithick] (-3, +2) to (3, +2);
            \strand[semithick] (-3, +4) to (3, +4);
            \node[darkblue] at (0, -6) {$1$};
    \end{knot}
    \end{tikzpicture}
    \begin{tikzpicture}[baseline=-0.65ex, scale=0.2]
        \useasboundingbox (-6, -5) rectangle (6, 4);
        \begin{knot}[clip width=5, end tolerance=1pt]
            \strand[semithick] (-3, -4) [in=down, out=right] to (-1, -3) [in=right, out=up] to (-3, -2);
            \strand[semithick] (3, -4) [in=down, out=left] to (1, -3) [in=left, out=up] to (3, -2);
            \strand[semithick] (-3, -0) to (3, +0);
            \strand[semithick] (-3, +2) to (3, +2);
            \strand[semithick] (-3, +4) to (3, +4);
            \node[darkblue] at (0, -6) {$e_1$};
    \end{knot}
    \end{tikzpicture}
    \begin{tikzpicture}[baseline=-0.65ex, scale=0.2]
        \useasboundingbox (-6, -5) rectangle (6, 4);
        \begin{knot}[clip width=5, end tolerance=1pt]
            \strand[semithick] (-3, -4) to (3, -4);
            \strand[semithick] (-3, -2) [in=down, out=right] to (-1, -1) [in=right, out=up] to (-3, 0);
            \strand[semithick] (3, -2) [in=down, out=left] to (1, -1) [in=left, out=up] to (3, 0);
            \strand[semithick] (-3, +2) to (3, +2);
            \strand[semithick] (-3, +4) to (3, +4);
            \node[darkblue] at (0, -6) {$e_2$};
    \end{knot}
    \end{tikzpicture}
    \begin{tikzpicture}[baseline=-0.65ex, scale=0.2]
        \useasboundingbox (-6, -5) rectangle (6, 4);
        \begin{knot}[clip width=5, end tolerance=1pt]
            \strand[semithick] (-3, -4) to (3, -4);
            \strand[semithick] (-3, -2) to (3, -2);
            \strand[semithick] (-3, 0) [in=down, out=right] to (-1, 1) [in=right, out=up] to (-3, 2);
            \strand[semithick] (3, 0) [in=down, out=left] to (1, 1) [in=left, out=up] to (3, 2);
            \strand[semithick] (-3, +4) to (3, +4);
            \node[darkblue] at (0, -6) {$e_3$};
    \end{knot}
    \end{tikzpicture}
    \begin{tikzpicture}[baseline=-0.65ex, scale=0.2]
        \useasboundingbox (-6, -5) rectangle (6, 4);
        \begin{knot}[clip width=5, end tolerance=1pt]
            \strand[semithick] (-3, -4) to (3, -4);
            \strand[semithick] (-3, -2) to (3, -2);
            \strand[semithick] (-3, -0) to (3, +0);
            \strand[semithick] (-3, 2) [in=down, out=right] to (-1, 3) [in=right, out=up] to (-3, 4);
            \strand[semithick] (3, 2) [in=down, out=left] to (1, 3) [in=left, out=up] to (3, 4);
            \node[darkblue] at (0, -6) {$e_4$};
    \end{knot}
    \end{tikzpicture}
\]
\caption{Diagramatyczne przedstawienie elementów bazowych algebry $TL_5(\delta)$}
\end{figure}
\end{comment}

\begin{definition}[ślad Markowa]
    \index{ślad!Markowa}
    Niech $K \in TL_n(\delta)$ będzie elementem algebry Temperleya-Lieba będącym iloczynem generatorów $e_1, \ldots, e_{n-1}$, którego domknięcie rozpada się na $m$ składowych spójności.
    Śladem Markowa elementu $K$ nazywamy wielkość $\operatorname{tr} K = \delta^{m-n}$.
\end{definition}

Każdy splot $L$ jest domknięciem warkocza zaplecionego na pewnej liczbie pasm, jak głosi twierdzenie \ref{alex_thm} Alexandera.
Rozpatrzmy pierścień $R = \Z[A, 1/A]$ z wyróżnionym elementem $\delta = -A^2 - A^{-2}$ oraz związaną z nim algebrę Temperleya-Lieba.
Wzór
\begin{equation}
    \rho(\sigma_i) = A \cdot e_i + \frac{1}{A} \cdot 1
\end{equation}
zadaje reprezentację grupy warkoczy $\rho \colon B_n \to TL_n$.
Wtedy $\langle K \rangle = \delta^{n-1} \operatorname{tr} \rho (\sigma)$ jest klamrą Kauffmana.
Ponieważ sploty nie przedstawiają się jako domknięcia warkoczy jednoznacznie, trzeba jeszcze sprawdzić wpływ ruchów Reidemeistera na złożenie $\operatorname{tr} \circ \rho$.
Pozostawiamy to Czytelnikowi jako ćwiczenie.

% Koniec podsekcji Oryginalna praca Jonesa
