\section{Wielomian Alexandera} % (fold)
\label{sec:alexander}

Wielomian Alexandera został odkryty w~1923 roku i~aż do roku 1984 pozostał jedynym znanym niezmiennikiem wielomianowym węzłów zorientowanych.
Przytoczymy jego opis oparty nie o~teorię homologii, a~równania kolorujące -- głównym powodem jest elementarność takiego podejścia.
Uogólnijmy najpierw równania kolorujące.

\begin{definition}
\label{def:polynomial_colouring}
    Wielomianowe równanie kolorujące związane z~poniższym skrzyżowaniem splotu zorientowanego to $a + tc - ta - b = 0$.
    Tylko orientacja górnej wiązki ma znaczenie.
    \[\begin{tikzpicture}[baseline=-0.65ex, scale=0.12]
    \useasboundingbox (-5, -5) rectangle (5,5);
    \begin{knot}[clip width=5, end tolerance=1pt, flip crossing/.list={1}]
        \strand[semithick] (-5,5) to (5,-5);
        \strand[semithick,-Latex] (-5,-5) to (5,5);
        \node[darkblue] at (5, 5)[below right] {$a$};
        \node[darkblue] at (5, -5)[above right] {$b$};
        \node[darkblue] at (-5, 5)[below left] {$c$};
    \end{knot}
    \end{tikzpicture}\]
\end{definition}

Istotnie, wystarczy podstawić tutaj $t = -1$, by otrzymać mniej ogólną definicję \ref{def:colour_equation}.

\begin{definition}[wielomian Alexandera]
    \label{def:alexander_polynomial}
    \index{wielomian!Alexandera}
    Niech $L$ będzie zorientowanym splotem z~diagramem bez krzywych zamkniętych.
    Przypiszmy etykiety $x_0, \ldots, x_m$ do włókien oraz $0, \ldots, m$ do skrzyżowań.
    Niech $P_{ij}$ będzie współczynnikiem przy $x_j$ w~wielomianowym równaniu kolorującym nad wierzchołkiem $i$.
    Z macierzy $P=(P_{ij})$ wykreślmy jedną kolumnę i~jeden wiersz.
    Wyznacznik tak otrzymanej macierzy nazywamy wielomianem Alexandera i~oznaczamy $\Delta_L(t)$.
\end{definition}

Nasz nowy niezmiennik nie jest zwykłym wielomianem, tylko wielomianem Laurenta jednej zmiennej, czyli elementem pierścienia $\Z[t, t^{-1}]$.

\begin{proposition} \label{alexander_invariance}
    Wielomian Alexandera z~dokładnością do relacji równoważności
    \begin{equation}
        f(t) \equiv g(t) \iff f(t) = \pm t^m g(t) \mbox{ dla pewnego } m \in \Z
    \end{equation}
    jest niezmiennikiem zorientowanych splotów.
\end{proposition}

W dowodzie niezmienniczości wyznacznika węzła skorzystaliśmy z~relacji między nim a~grupą kolorującą.
Poprzednie wydanie książki zawierało sugestię, że elementarny (czyli taki, który nie korzysta z~teorii modułów) dowód niezmienniczości wielomianu Alexandera nie istnieje.
Sugestia ta była błędna, wystarczy użyć alternatywnej definicji.

\begin{proof}[Szkic dowodu]
    Ustalmy diagram o~$k$ skrzyżowaniach, który rozcina płaszczyznę na $k+2$ obszarów i~utwórzmy macierz o~wymiarach $k \times k$, której kolumny odpowiadają obszarom, wiersze zaś skrzyżowaniom -- pomijając przy tym dwa sąsiadujące ze sobą obszary -- o~wyrazach ze współczynników równań kolorujących.
    Jej wyznacznik jest wielomianem Alexandera.

    Sąsiadującym ze sobą obszarom przypiszmy kolejne liczby całkowite tak, by obszar leżący po prawej stronie włókna miał niższy indeks.
    Pokażemy najpierw, że skasowanie kolumny indeksu $n$ oraz $n+1$ sprawia, że wyznacznik zmienia się co najwyżej o~czynnik $\pm t^m$ dla pewnego $m$.
    Niech $S_n$ oznacza sumę kolumn indeksu $n$.
    Każdy wiersz macierzy zawiera cztery niezerowe wyrazy: $\pm 1, \pm t$, zatem $\sum_n S_n = 0$.
    Równość ta zachodzi nawet po przemnożeniu kolumny indeksu $n$ przez $t^{-n}$: $\sum_n t^{-n}S_n = 0$, co prowadzi do relacji $\sum_n (t^{-n}-1) S_n = 0$.
    Jeśli więc indeks kolumny $v_j$ wynosi $n$, to $(t^{-n}-1)v_j$ jest kombinacją liniową innych kolumn niezerowego indeksu (ponieważ $t^0 - 1 = 0$).

    Rozpatrzmy macierze $M_{0,j}, M_{0,k}$, gdzie indeksy $j$-tej i~$k$-tej kolumny to odpowiednio $p$ i~$q$.
    Z powyższych rozważań wynika, że $(t^{-q}-1) \Delta_{0,j} = \pm (t^{-p}-1)\Delta_{0,k}$, ale indeksy obszarów są wyznaczone z~dokładnością do stałej addytywnej.
    Biorąc $i$-tą oraz $l$-tą kolumnę, indeksów $r$ oraz $s$, dostaniemy zależności
    \begin{align}
        (t^{r-q}-1) \Delta_{l,j} & = \pm (t^{r-p} - 1)\Delta_{l,i} \\
        (t^{q-s}-1) \Delta_{k,l} & = \pm (t^{q-r} - 1)\Delta_{k,i}
    \end{align}
    co prowadzi do
    \begin{equation}
        \Delta_{l,j} = \pm \frac{t^{q-r}(t^{r-p}-1)}{t^{q-s}-1} \Delta_{k,i}
    \end{equation}
    Położenie $p = r +1$, $s =q+1$ pokazuje, że różny wybór kolumn do skreślenia zmienia wyznacznik macierzy co najwyżej o~czynnik $\pm t^m$.

    Wprowadźmy jeszcze jedną techniczną definicję.
    Dwie kwadratowe macierze będą dla nas równoważne, jeśli można przejść od jednej do drugiej przy użyciu pięciu operacji:
    \begin{enumerate}[leftmargin=*]
    \itemsep0em
        \item przemnożenie wiersza lub kolumny przez $-1$;
        \item zamiana dwóch wierszy lub kolumn miejscami;
        \item dodanie jednego wiersza do innego (lub kolumny do innej);
        \item przemnożenie lub podzielenie kolumny przez $t$;
        \item rozszerzenie lub zmniejszenie macierzy o~$1$ na przekątnej i~zera w~innych miejscach.
    \end{enumerate}

    Ruchy Reidemeistera prowadzą do macierzy równoważnych wyjściowym.
    Każda z~tych operacji zmienia wyznacznik macierzy o~czynnik $\pm t^{-m}$, co kończy dowód.
\end{proof}

Dla wygody można przeprowadzić normalizację wielomianu przez wzięcie reprezentanta, który jest symetryczny w~zmiennych $t$ oraz $t^{-1}$ i~przyjmuje w~punkcie $1$ wartość $\Delta(1) = 1$.
Odwrotnie, każdy wielomian Laurenta z~całkowitymi współczynnikami o~takich własnościach jest wielomianem Alexandera pewnego węzła.
% Kawauchi 1996

Wielomian nie odróżnia luster i~odwrotności od wyjściowych węzłów:

\begin{proposition}
    \label{prp:alexander_mirror_reverse}
    Niech $L$ będzie zorientowanym splotem.
    Wtedy $\Delta_{mL}(t) = \Delta_L(1/t) = \Delta_{rL}(t)$.
\end{proposition}

\begin{proof}
    Po odbiciu diagramu względem pionowej prostej skrzyżowanie z~definicji \ref{def:colour_equation} również się odbija.
    Równanie związane z~nim zmienia się według schematu:
    \begin{equation}
        a + tc - ta - b = 0 \rightleftharpoons a + tb - ta - c = 0
    \end{equation}
    Pierwsze równanie z~$t$ zamienionym na $1/t$ staje się drugim równaniem przemnożonym przez $-1/t$.
    Dowód drugiej równości przebiega analogicznie.
\end{proof}

Nasz niezmiennik nie wykrywa niewęzła.
Na przykład $11_{471} = 11n_{34}$, $11_{473} = 11n_{42}$ albo $(-3, 5, 7)$-precel posiadają trywialny wielomian Alexandera, zjawisko to nie występuje wśród nietrywialnych węzłów o co najwyżej 10 skrzyżowaniach.

\begin{proposition}
    \label{prp:alexander_det}
    Niech $L$ będzie zorientowanym splotem.
    Wtedy $\Delta_L(-1) = \pm \det L$.
\end{proposition}

\begin{proof}
    Wystarczy porównać definicję dla $\Delta_L$ (\ref{def:alexander_polynomial}) oraz $\det L$ (\ref{def:determinant}).
\end{proof}

\begin{proposition}
    Jeśli $J, K$ są zorientowanymi węzłami, to $\Delta_{J \shrap K}(t) \equiv \Delta_J(t) \Delta_K(t)$.
\end{proposition}

\begin{proof}
    Wybierzmy poniższe diagramy dla węzłów $J$ oraz $K$:
    \[\begin{tikzpicture}[baseline=-0.65ex, scale=0.07]
    %\useasboundingbox (-5, -5) rectangle (5,5);
    \begin{knot}[clip width=5, end tolerance=1pt]
        \strand[semithick] (-70, -10) rectangle (-30, 10);
        \strand[semithick] ( 30, -10) rectangle ( 70, 10);
        \strand[semithick,Latex-] (-30, 5) .. controls (-22, 5) and (-18, -5) .. (-10, -5);
        \strand[semithick] (-30,-5) .. controls (-22, -5) and (-18, 5) .. (-10,  5);
        \strand[semithick] (-10, 5) [in=up, out=right] to (-5, 0) [in=right, out=down] to (-10, -5);

        % prawe strzalki
        \strand[semithick,-Latex] (30, 5) .. controls (22, 5) and (18, -5) .. (10, -5);
        \strand[semithick] (30,-5) .. controls (22, -5) and (18, 5) .. (10,  5);
        \strand[semithick] (10, 5) [in=up, out=left] to (5, 0) [in=left, out=down] to (10, -5);

        \node[darkblue] at (-50,5) {$x_1,\ldots,x_{m-1}$};
        \node[red] at (-50,-5) {$1,\ldots,m$};

        \node[darkblue] at (50,5) {$y_1,\ldots,y_{n-1}$};
        \node[red] at (50,-5) {$1,\ldots,n$};

        \node[darkblue] at (-30,-5)[below right] {$x_m$};
        \node[darkblue] at (-15,-5)[below] {$x_0$};
        \node[darkblue] at (30,-5)[below left] {$y_n$};
        \node[darkblue] at (15,-5)[below] {$y_0$};
        \node[red] at ( 19.5,  1)[above]{$0$};
        \node[red] at (-19.5,  1)[above]{$0$};
    \end{knot}
    \end{tikzpicture}
\]
    Niech $A$ oraz $B$ oznaczają macierze otrzymane z~wielomianowych równań kolorujących dla $J$ oraz $K$ przez skreślenie skrajnie lewej kolumny i~górnego wiersza.
    Wtedy $\Delta_J(t) = \det A$ oraz $\Delta_K(t) = \det B$.
    Poniższy diagram przedstawia sumę $J \shrap K$:

\[\begin{tikzpicture}[baseline=-0.65ex, scale=0.07]
    %\useasboundingbox (-5, -5) rectangle (5,5);
    \begin{knot}[clip width=5, end tolerance=1pt]
        \strand[semithick] (-70, -10) rectangle (-30, 10);
        \strand[semithick] ( 30, -10) rectangle ( 70, 10);
        \strand[semithick,Latex-] (-30, 5) .. controls (-22, 5) and (-18, -5) .. (-10, -5);
        \strand[semithick] (-30,-5) .. controls (-22, -5) and (-18, 5) .. (-10,  5);

        % prawe strzalki
        \strand[semithick] (30, 5) .. controls (22, 5) and (18, -5) .. (10, -5);
        \strand[semithick] (30,-5) .. controls (22, -5) and (18, 5) .. (10,  5);
        \strand[semithick] (10, 5) to (-10, 5);
        \strand[semithick,-Latex] (10, -5) to (-10, -5);

        \node[darkblue] at (-50,5) {$x_1,\ldots,x_{m-1}$};
        \node[red] at (-50,-5) {$1,\ldots,m$};

        \node[darkblue] at (50,5) {$y_1,\ldots,y_{n-1}$};
        \node[red] at (50,-5) {$1,\ldots,n$};

        \node[darkblue] at (-30,-5)[below right] {$x_m$};
        \node[darkblue] at (0,-5)[below] {$x_0 = y_0$};
        \node[darkblue] at (0, 5)[above] {$z$};
        \node[darkblue] at (30,-5)[below left] {$y_n$};
        \node[red] at ( 19.5,  1)[above]{$\zeta$};
        \node[red] at (-19.5,  1)[above]{$0$};
    \end{knot}
    \end{tikzpicture}\]

    Uporządkujmy łuki na diagramie jako $x_0 = y_0$, $x_1, \ldots, x_m$, $y_1, \ldots, y_n$, $z$; skrzyżowania: $0, 1, \ldots, m$ (z $J$), $1, \ldots, n$ (z $K$), $\zeta$.
    Wielomianowe równanie kolorujące dla $J \shrap K$ nad skrzyżowaniami $1, \ldots, m$ ($1, \ldots, n$) są takie same, jak przed dodaniem do siebie węzłów.
    Nad skrzyżowaniem $\zeta$ równanie orzeka, że $(1-t)y_0+t z-y_n=0$.

    Wynika stąd, że $\Delta_{J \shrap K}(t)$ jest wyznacznikiem macierzy
    \begin{align*}
        M &= \left(\begin{array}{cc|cc|c}
            & & & & \\
            \multicolumn{2}{c|}{\smash{\raisebox{.5\normalbaselineskip}{$A$}}} & & \\
            \hline \\[-\normalbaselineskip]
            & & & & \\
            & & \multicolumn{2}{c|}{\smash{\raisebox{.5\normalbaselineskip}{$B$}}}\\ \hline
            & & & -1 & t
    \end{array}\right)
    \end{align*}

    Skreśliliśmy lewą kolumnę oraz górny wiersz.
    Zatem $\Delta_{J \shrap K}(t) = t^?\Delta_J(t) \Delta_K(t)$, jeśli nie pomyliliśmy się w~obliczeniach.
\end{proof}

\begin{proposition}
    Wielomian Alexandera zadaje ograniczenie na indeks skrzyżowaniowy $c$:
    \begin{equation}
        \deg \Delta_K(t) < c(K).
    \end{equation}
\end{proposition}

\begin{proposition}
    Tylko skończenie wiele węzłów alternujących może mieć ten sam wielomian Alexandera.
\end{proposition}

\begin{proof}
    Załóżmy nie wprost, że istnieje nieskończony ciąg $K_n$ węzłów alternujących o~tym samym wielomianie Alexandera $\Delta_K(t)$.
    Wszystkie jego wyrazy mają ten sam wyznacznik, ponieważ $\det K_n = |\Delta_K(-1)|$.
    Z faktu \ref{prp:bankwitz} wynika, że indeks skrzyżowaniowy węzłów $K_n$ jest wspólnie ograniczony: $c_k \le \det K_n = \det K$.
    To prowadzi do sprzeczności: węzłów o~danym indeksie skrzyżowaniowym jest tylko skończenie wiele.
\end{proof}

Przedstawimy teraz kilka innych definicji, które prowadzą do tego samego niezmiennika.

\begin{definition}
    Wielomian Alexandera $\Delta_L(t) \in \Z[t^{-1/2}, t^{1/2}]$ zorientowanego splotu $L$ to taki wielomian Laurenta, który spełnia relację kłębiastą (przy oznaczeniach z~twierdzenia \ref{tracheotomia}):
    \begin{enumerate}
        \item $\Delta_{\LittleUnknot}(t) = 1$
        \item $\Delta_{L_+}(t) - \Delta_{L_-}(t) - (t^{1/2} - t^{-1/2}) \Delta_{L_0}(t) = 0$,
    \end{enumerate}

    gdzie $L_+$, $L_-$, $L_0$ to zorientowane sploty, które różnią się jedynie na małym obszarze:
    \[
        \skeinplus \quad\quad\quad\quad
        \skeinminus \quad\quad\quad\quad
        \skeinzero
    \]
\end{definition}

Relacja kłębiasta wystarcza do wyznaczenia $\Delta_L$ dla każdego splotu na mocy następującego lematu:

\begin{lemma}
    Dowolny diagram węzła można przekształcić do diagramu niewęzła przez odwrócenie pewnej liczby skrzyżowań.
\end{lemma}

\begin{proof}
    Wystarczy ograniczyć się do diagramu węzła.
    Wybierzmy jakiś początkowy punkt na nim, różny od skrzyżowania wraz z~kierunkiem, wzdłuż którego będziemy przemierzać węzeł.
    Za każdym razem, kiedy odwiedzamy nowe skrzyżowanie, zmieniamy je w~razie potrzeby na takie, przez które przemieszczamy się górnym łukiem.
    Skrzyżowań już odwiedzonych nie zmieniamy wcale. Po powrocie do punktu wybranego na początku uzyskamy diagram niewęzła.
\end{proof}

Conway chcąc szybko liczyć wielomian Alexandera zaproponował, by reparametryzować go wzorem $\Delta(x^2) = \nabla(x - 1/x)$.
Wzór ten był znany Alexanderowi, nie wiedzieć jednak czemu nie zyskał uwagi.
Spełniona jest wtedy inna relacja kłębiasta:
\begin{equation}
    \nabla_{L_+}(x)- \nabla_{L_-}(x) = x \nabla_{L_0}(x).
\end{equation}

Wielomian Jonesa (zdefiniowany w~kolejnej sekcji) spełnia podobną równość.
Ich istnienie może nasunąć przypuszczenie, że dają się wspólnie uogólnić.
Tak jest w~rzeczywistości -- mocniejszym niezmiennikiem okazał się wielomian HOMFLY-PT.

Niech $\nabla_k$ będzie współczynnikiem przy $z^k$ w~wielomianie Conwaya.
Jest to przykład niezmiennika Wasiljewa rzędu $k$, poznamy je w~sekcji \ref{sec:vassiliev}.
Lin oraz Wang w~1994 roku na podstawie niezmienników małych rzędów, to jest $v_2$ oraz $v_3$, wysunęli następującą hipotezę: istnieje uniwersalna stała $C$ taka, że
\begin{equation}
    |v_k(K)| \le C (\operatorname{cr} K)^k.
\end{equation}

Hipotezę wkrótce udowodniono, najpierw dla węzłów (Bar-Natan, \cite{barnatan95}), nieco później także dla splotów (Stojmenow, \cite{stoimenow_01}).
Wartość stałej $C$ trudno obliczyć, dlatego Stojmenow zaproponował ograniczenie się do przypadku $v_k = \nabla_k$.

\begin{conjecture}
    Niech $L$ będzie splotem.
    Wtedy
    \begin{equation}
        |\nabla_k(L)| \le \frac{1}{2^kk!} \cdot c^k.
    \end{equation}
\end{conjecture}

Nierówność jest nietrywialna tylko dla splotu $L$ z~$k+1, k-1, \ldots$ składowymi; trywialna dla $k = 0$, łatwa dla $k=1$ (wtedy $\nabla_1$ jest indeksem zaczepienia splotów o~dwóch składowych) oraz udowodniona dla węzłów i~$k=2$ przez Polyaka, Viro w~2001 (\cite{polyak01}).
% The Casson knot invariant (to be distinguished from the better-known Casson invariant) is defined to be the Vassiliev knot invariant v_2, which turns out to be \Delta_k''(1) / 2 , where \Delta_k is the Alexander polynomial of k. It can be characterized as the unique Vassiliev invariant of degree 2 that takes value 0 on the trivial knot and value 1 on the trefoil knot.

\begin{proposition}
    Wielomianem Alexandera splotu rozszczepialnego jest $1$.
\end{proposition}

Za to wielomian Jonesa, zgodnie z~faktem \ref{prp:jones_trivial_link}, rozróżnia sploty trywialne.

Macierz Seiferta -- wymaga definicji.
\begin{proposition}
    Spełniona jest relacja $\Delta(t)=\det(V^T-tV)$, gdzie $V$ to macierz Seiferta.
\end{proposition}

Można scharakteryzować możliwe postacie wielomianu Alexandera.

\begin{proposition}[Hosokowa, 1958]
    Każdy wielomian Laurenta $p(t)$ o~całkowitych współczynnikach taki, że $p(1/t) = p(t)$ i~$p(1) = \pm 1$ może być wielomianem Alexandera jakiegoś węzła.
\end{proposition}

Murasugi podejrzewał, że w~przypadku węzłów alternujących ciąg współczynników jest unimodalny.
Dowód podano dla węzłów algebraicznych (Murasugi \cite{murasugi85}) oraz genusu dwa (Ozsvath i~Szabo w~\cite{ozsvath03}).
Hipoteza w~ogólnym przypadku pozostaje otwarta.

% Remark (M. Hutchings) There does exist a~categorification of the Alexander polynomial, or more precisely of ∆K(t)/(1 − t)2, where ∆K(t) denotes the (symmetrized) Alexander polynomial of the knot K. It is a~kind of Seiberg-Witten Floer homology of the three-manifold obtained by zero surgery on K.
% One can regard it as Z×Z/2Z graded, although in fact the column whose Euler characteristic gives the coefficient of tk is relatively Z/2kZ graded.

Na sam koniec pozostawiliśmy najstarszą definicję wielomianu Alexandera.
Niech $K$ będzie węzłem w~3-sferze, zaś $X$ nieskończonym nakryciem cyklicznym jego dopełnienia.
Można je otrzymać rozcinając dopełnienie wzdłuż powierzchni Seiferta.
Na $X$, a~przez to także na grupie homologii $H_1(X)$, działa automorfizm $t$, który czyni z~niej moduł nad pierścieniem $\Z[t, t^{-1}]$, i~to skończenie prezentowalny.
Jeśli posiada przedstawienie z~$r$ generatorami i~$s$ relacjami, gdzie $r \le s$, rozpatrzmy ideał generowany przez minory $r \times r$ macierzy prezentacji (jeśli nie, weźmy ideał zerowy).
Alexander pokazał, że ideał ten zawsze jest niezerowy i~główny.

% Koniec sekcji Wielomian Alexandera
