
\subsection{Wielomian Alexandera a niezmienniki numeryczne}
\begin{proposition}
\label{prp:alexander_determinant}%
    Niech $L$ będzie zorientowanym splotem.
    Wtedy $|\alexander_L(-1)| = \det L$.
\end{proposition}

\begin{proof}
    Wystarczy porównać definicję dla $\alexander_L$ (\ref{def:alexander_polynomial}) oraz $\det L$ (\ref{def:determinant}).
\end{proof}

\begin{proposition}
    Wielomian Alexandera zadaje ograniczenie na indeks skrzyżowaniowy:
    \begin{equation}
        \operatorname{span} \alexander_K(t) < \crossing K.
    \end{equation}
\end{proposition}

Być może istnieje bezpośredni dowód tej nierówności, ale jedyne uzasadnienie, jakie znam, opiera się na fakcie \ref{prp:alexander_genus} oraz wniosku \ref{cor:crossing_genus} -- czyli własnościach genusu.
\index{genus}%

\begin{proposition}
\index{węzeł!alternujący}%
    Tylko skończenie wiele węzłów alternujących może mieć ten sam wielomian Alexandera.
\end{proposition}

\begin{proof}
    Załóżmy nie wprost, że istnieje nieskończony ciąg $K_n$ węzłów alternujących o~tym samym wielomianie Alexandera $\alexander_K(t)$.
    Wszystkie jego wyrazy mają ten sam wyznacznik, ponieważ $\det K_n = |\alexander_K(-1)|$.
    Z faktu \ref{prp:bankwitz} wynika, że indeks skrzyżowaniowy węzłów $K_n$ jest wspólnie ograniczony: $c_k \le \det K_n = \det K$.
    To prowadzi do sprzeczności: węzłów o~danym indeksie skrzyżowaniowym jest tylko skończenie wiele.
\end{proof}

Wielomian Alexandera nie nakłada żadnych ograniczeń na liczbę gordyjską, wspominamy o~tym w dowodzie faktu \ref{balanced_iff_four_conditions}.
Mamy też:

\begin{proposition}
\index{liczba mostowa}%
\label{no_relation_bridge_alexander}%
    Nie istnieje związek między liczbą mostową oraz stopniem wielomianu Alexandera.
\end{proposition}

\begin{proof}
    Suma spójna węzłów opisanych po fakcie \ref{prp:alexander_determinant} ma trywialny wielomian Alexandera, ale dowolnie dużą liczbę mostową.

    Węzły $(2,n)$-torusowe są dwumostowe, ale stopień ich wielomianów Alexandera jest nieograniczony.
\end{proof}

% koniec podsekcji Wielomian Alexandera a niezmienniki numeryczne

