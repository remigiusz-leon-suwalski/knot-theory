\subsection{Definicja kombinatoryczna -- klamra Kauffmana} % (fold)
\label{sub:kauffman_bracket}
Klamra Kauffmana to wielomian Laurenta jednej zmiennej zdefiniowany w pracy \cite{kauffman87} z 1987 roku, oparty na ruchach Reidemeistera.
Dzięki swojej prostocie mógł być odkryty na początku XX wieku, nim jeszcze maszyneria teorii węzłów została rozwinięta.

Poszukujemy niezmiennika dla splotów o~kilku prostych własnościach.
Przede wszystkim żądamy, by niewęzłowi przypisany był wielomian $1$: $\bracket{\LittleUnknot} = 1$.
Po drugie chcemy wyznaczać nawiasy znając je dla prostszych splotów, co zapiszemy symbolicznie $\bracket{\LittleRightCrossing} = A \bracket{\LittleRightSmoothing} + B \bracket{\LittleLeftSmoothing}$.
Zależy nam wreszcie na tym, by móc dodać do splotu trywialną składową: $\langle L \cup \LittleUnknot \rangle = C \langle L \rangle$.
Prosty rachunek pokazuje wpływ drugiego ruchu Reidemeistera na klamrę:
\begin{equation}
    \bracket{\reidemeisterIIaa}
    = (A^2 + ABC + B^2) \bracket{\LittleLeftSmoothing} + BA \bracket{\LittleRightSmoothing}
    \stackrel{?}{=} \bracket{\LittleRightSmoothing}.
\end{equation}

Aby zachodziła ostatnia równość wystarczy przyjąć $B = A^{-1}$, co wymusza na nas wartość trzeciego parametru: $C = -A^2 - A^{-2}$.
W ten sposób odkryliśmy definicję.

\begin{definition}[klamra Kauffmana]
    \index{klamra!Kauffmana}
    \label{def:kauffman_bracket}
    Wielomian Laurenta $\bracket{D}$ dla diagramu splotu $D$ zmiennej $A$,
    który jest niezmienniczy ze względu na gładkie deformacje diagramu,
    a~przy tym spełnia trzy poniższe aksjomaty:
    \begin{align}
        \bracket{\LittleUnknot} & = 1 \\
        \bracket{D \sqcup \LittleUnknot} & = (-A^{-2} - A^2) \bracket{D} \\
        \bracket{\LittleRightCrossing} & = A \bracket{\LittleRightSmoothing} + A^{-1} \bracket{\LittleLeftSmoothing}
    \end{align}
    nazywamy klamrą Kauffmana.
\end{definition}

Tutaj $\LittleUnknot$ oznacza standardowy diagram dla niewęzła,
zaś trzy symbole $\LittleRightCrossing$, $\LittleRightSmoothing$ oraz $\LittleLeftSmoothing$ odnoszą się do diagramów,
które są identyczne wszędzie poza małym obszarem (tak jak w~relacji kłębiastej).
\textbf{Duplikat wyjaśnienia, czym jest relacja kłębiasta?}
Diagramy $\LittleRightSmoothing$ oraz $\LittleLeftSmoothing$ nazywa się odpowiednio
dodatnim (prawym) i~ujemnym (lewym) wygładzeniem $\LittleRightCrossing$.

\begin{lemma}
    Klamra Kauffmana każdego diagramu wyznacza się w~skończenie wielu krokach.
\end{lemma}

\begin{proof}
    Jeżeli diagram $D$ ma $n$ skrzyżowań, to nieustanne stosowanie aksjomatu trzeciego pozwala na zapisanie $\bracket{D}$ jako sumy $2^n$ składników,
    z~których każdy jest po prostu zamkniętą krzywą i~ma trywialną klamrę ($\bracket{\LittleUnknot} = 1$).
    Klamrę sumy wyznacza się korzystając z~drugiego aksjomatu.
\end{proof}

Przedstawimy teraz wpływ ruchów Reidemeistera na nasz nowy wielomian.

\begin{lemma}
    Drugi i~trzeci ruch Reidemeistera nie ma wpływu na klamrę Kauffmana,
    pierwszy ruch zmienia ją zgodnie z~regułą:
    \begin{equation}
        \bracket{\reidemeisterIa} = -A^{-3} \bracket{\,\reidemeisterIb\,}.
    \end{equation}
\end{lemma}

\begin{proof}
Pierwszy ruch Reidemeistera:
\begin{comment}
\begin{align*}
    \bracket{\reidemeisterIa} & \stackrel{K3}{=} A \bracket{
    \begin{tikzpicture}[baseline=-0.65ex,scale=0.07]
    \useasboundingbox (-4, -5) rectangle (3, 5);
    \begin{knot}[clip width=5, end tolerance=1pt]
        \strand[semithick]
            (-3, 5) [in=left, out=down] to (-1,1) [in=left, out=right]
                                        to (1,3)
                                        to [in=up, out=right] (3,0);
        \strand[semithick]
            (-3, -5) [in=left, out=up] to (-1,-1) [in=left, out=right]
                                       to (1, -3)
                                       to [in=down, out=right] (3,0);
    \end{knot}
    \end{tikzpicture}}
    + A^{-1} \bracket{\,
    \begin{tikzpicture}[baseline=-0.65ex,scale=0.07]
    \begin{knot}[clip width=5]
        \strand[semithick] (0,-5) [in=down, out=up] to (1, -2) to (1, 2) to (0, 5);
        \strand[semithick] (4,0) circle (1.5);
    \end{knot}
    \end{tikzpicture}} \\
    & \stackrel{K2}{=} A \bracket{\,\reidemeisterIb\,} + A^{-1}(-A^{-2}-A^2) \bracket{\,\reidemeisterIb\,}
    = -A^{-3}\bracket{\,\reidemeisterIb\,}
\end{align*}
\end{comment}

Dla drugiego ruchu:
\begin{comment}
\begin{align*}
    \bracket{\reidemeisterIIa} &\stackrel{K3}{=} A
    \bracket{\reidemeisterIab}
    + A^{-1} \bracket{\begin{tikzpicture}[baseline=-0.65ex,scale=0.07]
    \useasboundingbox (-5, -6) rectangle (5, 6);
    \begin{knot}[clip width=5, end tolerance=1pt]
        \strand[semithick] (4,-5) .. controls (4,-2) and (-4,-2) .. (-4,0);
        \strand[semithick] (4,5) to (4,0);
        \strand[semithick] (-4,-5) .. controls (-4,-2) and (4,-2) .. (4,0);
        \strand[semithick] (-4,5) to (-4,0);
    \end{knot}
    \end{tikzpicture}}
    \stackrel{K1}{=} -A^{-2} \bracket{\LittleLeftSmoothing} + A^{-1}
    \bracket{\begin{tikzpicture}[baseline=-0.65ex,scale=0.07]
    \useasboundingbox (-5, -6) rectangle (5, 6);
    \begin{knot}[clip width=5, end tolerance=1pt]
        \strand[semithick] (4,-5) .. controls (4,-2) and (-4,-2) .. (-4,0);
        \strand[semithick] (4,5) to (4,0);
        \strand[semithick] (-4,-5) .. controls (-4,-2) and (4,-2) .. (4,0);
        \strand[semithick] (-4,5) to (-4,0);
    \end{knot}
    \end{tikzpicture}}
    \\ & \stackrel{K3}{=} -A^{-2} \bracket{\LittleLeftSmoothing}
    + A^{-1}A \bracket{\LittleRightSmoothing} + A^{-1}A^{-1} \bracket{\LittleLeftSmoothing}
    = \bracket{\LittleRightSmoothing}
\end{align*}
\end{comment}

Dla trzeciego ruchu:
\begin{comment}
\begin{align*}
\bracket{\,\reidemeisterIIIa\,} &\stackrel{K3}{=} A
\bracket{\,\begin{tikzpicture}[baseline=-0.65ex,yscale=0.07, xscale=0.1]
    \useasboundingbox (-5, -6) rectangle (5, 6);
    \begin{knot}[clip width=5, flip crossing/.list={1,2,3}, end tolerance=1pt]
        \strand[semithick] (-5, 5) [in=-135, out=-45] to (5,5);
        \strand[semithick] (-5, -5) [in=135, out=45] to (5,-5);
        \strand[semithick] (-5, 0) .. controls (-2, 0) and (-2,5) .. (0,5) .. controls (2, 5) and (2, 0) .. (5, 0);
    \end{knot}
    \end{tikzpicture}\,}
+A^{-1} \bracket{\RightCrossSmoothing} \\
%\stackrel{R2}{=} A \bracket{\,\LeftCrossSmoothing\,} +A^{-1} \bracket{\RightCrossSmoothing} \\
& \stackrel{R2}{=} A \bracket{\,\LeftCrossSmoothing\,} +A^{-1} \bracket{\RightCrossSmoothing}
\stackrel{K3}{=} \bracket{\,\reidemeisterIIIb\,}
\end{align*}
korzystaliśmy tu z~własności drugiego ruchu.
\end{comment}
\end{proof}

\begin{corollary}
    Rozpiętość klamry Kauffmana jest niezmiennikiem węzłów.
\end{corollary}

Klamra Kauffmana nie jest niezmiennikiem węzłów ze względu na I ruch Reidemeistera.
Jeżeli przypomnimy sobie, że na mocy lematu \ref{writhe_not_invariant} spin także nie jest niezmiennikiem węzłów, odkryjemy ,,trik Kauffmana'': niedoskonałości tych dwóch obiektów znoszą się wzajemnie.

\begin{definition}
    \index{wielomian!Jonesa}
    \label{def:jones_polynomial}
    Niech $L$ będzie zorientowanym splotem.
    Wielomian Laurenta $\jones(L) \in \Z[t^{\pm 1/2}]$ określony przez
    \begin{equation}
        \jones(L)=\left[(-A)^{-3w(D)} \bracket{D}\right]_{t^{1/2}=A^{-2}},
    \end{equation}
    gdzie $D$ to dowolny diagram dla $L$, nazywamy wielomianem Jonesa.
\end{definition}

Sama klamra odegrała ważną rolę podczas unifikacji wielomianu Jonesa oraz innych niezmienników kwantowych.
W szczególności pozwoliła na uogólnienie go do niezmiennika 3-rozmaitości.

\begin{proposition}
    Wielomian Jonesa jest niezmiennikiem zorientowanych splotów.
\end{proposition}

\begin{proof}
    %Skorzystamy z~tego, że indeks zaczepienia jest niezmiennikiem.
    Wystarczy pokazać niezmienniczość $(-A)^{-3w(D)}\langle D\rangle$ na ruchy Reidemeistera.
    Ale
    \begin{equation}
        (-A)^{-3 w\left(\MalyreidemeisterIa\right)} \bracket{\MalyreidemeisterIa} =
        (-A)^{-3 w\left(\ \MalyreidemeisterIb\ \right)+3} (-A)^{-3}\bracket{\ \MalyreidemeisterIb\ } =
        (-A)^{-3 w\left(\ \MalyreidemeisterIb\ \right)}    \bracket{\,\MalyreidemeisterIb\,}. \qedhere
    \end{equation}
\end{proof}

Zazwyczaj, ale nie zawsze, wielomian Jonesa lepiej radzi sobie z odróżnianiem od siebie splotów.
Zaczniemy od wyznaczenia bezpośrednio z definicji, jakie są wielomiany Jonesa niesplotów.
Dla porównania, wielomian Alexandera wszystkich splotów rozszczepialnych jest taki sam (stwierdzenie \ref{prp:alexander_unlinks}).

\begin{proposition}
\label{prp:jones_trivial_link}
    Wielomianem Jonesa splotu trywialnego o $n$ ogniwach jest
    \begin{equation}
        \jones(K_n) = \left(-\sqrt{t} - \frac{1}{\sqrt {t}}\right)^{n-1}.
    \end{equation}
\end{proposition}

Co więcej, wielomian Jonesa odróżnia od siebie dowolne dwa węzły pierwsze o~co najwyżej 9 skrzyżowaniach.
Dalej występują już kolizje, oto pełna ich lista do 10 skrzyżowań:
$5_{1}$ -- $10_{132}$,
$8_{8}$ -- $10_{129}$,
$8_{16}$ -- $10_{156}$,
$10_{22}$ -- $10_{35}$,
$10_{25}$ -- $10_{56}$,
$10_{40}$ -- $10_{103}$,
$10_{41}$ -- $10_{94}$,
$10_{43}$ -- $10_{91}$,
$10_{59}$ -- $10_{106}$,
$10_{60}$ -- $10_{86}$,
$10_{71}$ -- $10_{104}$,
$10_{73}$ -- $10_{83}$,
$10_{81}$ -- $10_{109}$,
$10_{137}$ -- $10_{155}$.
Jones wiedział, że wielomianowe niezmienniki nie radzą sobie z~odróżnianiem od siebie mutantów, dlatego zapytał w~2000 roku, czy jego wielomian wykrywa niewęzły.
Pozostaje to otwartym problemem do dziś.

\begin{conjecture} \label{jones_conjecture}
    Niech $K$ będzie węzłem.
    Jeśli $\jones_K(t) \equiv 1$, to $K$ jest niewęzłem.
\end{conjecture}

Hipotezę zweryfikowano komputerowo dla węzłów o~małej liczbie skrzyżowań.
W latach dziewięćdziesiątych Hoste, Thistlethwaite, Weeks zrobili to dla węzłów spełniających $\operatorname{cr} \le 16$.
Wynik poprawiano: Dasbach, Hougardy w~1997 do $\operatorname{cr} = 17$; Yamada w~2000 do $\operatorname{cr} = 18$; wreszcie Tuzun, Sikora w~2016 do $\operatorname{cr} \le 22$.

Argumentem przemawiającym za prawdziwością hipotezy jest twierdzenie udowodnione przez Jørgena Andersena.
\textbf{NIE Pokazał on, że rodzina okablowanych wielomianów Jonesa wykrywa niewęzeł.}
Tutaj $n$-okablowanie węzła $K$ to $n$-komponentowy splot $K^n$, który powstaje z~$K$ po zamianie pojedynczej ,,żyły'' na $n$ równoległych żył.

Istnieją sploty o~trywialnym wielomianie Jonesa, jest ich nawet nieskończenie wiele, jak Eliahou, Kauffman i~Thistlethwaite pokazali w~pracy \cite{eliahou03}.

\begin{proposition}
    Niech $k \ge 2$ będzie liczbą naturalną.
    Istnieje nieskończenie wiele splotów pierwszych z $k$ ogniwami, których wielomian Jonesa nie odróżnia od niesplotu z $k$ ogniwami.
    Co więcej, można wymagać, by wszystkie te sploty były satelitami splotu Hopfa.s
\end{proposition}

Niech $\jones$ będzie wielomianem Jonesa splotu $K$ o~$n$ składowych spójności.
Jego wartości w~niektórych pierwiastkach jedności są związane z~innymi niezmiennikami węzłów.
I tak przyjmując oznaczenie $\omega_k = \exp(2\pi i/k)$ mamy

\begin{proposition} \label{jones_sharp_p_hard}
    $\jones(\omega_3) = 1$.
\end{proposition}

\begin{proposition}
    $\jones(1) = (-2)^{n-1}$.
\end{proposition}

\begin{proof}
    Proste wnioski z~relacji kłębiastej.
    Explicite wskazał je Jones w \cite{jones85} (jako theorem 14, 15).
\end{proof}

\begin{proposition}
    Liczba trzy-kolorowań splotu wynosi $3|\jones(\omega_6)|^2$.
\end{proposition}

\begin{proof}
    Dowód zawiera praca ,,3-coloring and other elementary invariants of knots'' (Przytycki, 1998).
\end{proof}

\begin{proposition}
    Jeśli $K$ jest właściwym splotem (indeks zaczepienia każdej składowej o~resztę splotu jest parzysty), to $\jones(i) = (-\sqrt 2)^{n-1}(-1)^{\operatorname{Arf} K}$.
    W przeciwnym razie $\jones(i) = 0$.
\end{proposition}

\begin{proof}
    Równość 4. pokazał Murakami w~1986 roku (\cite{murakami86}).
\end{proof}

\begin{proposition}
    Niech $G$ będzie pierwszą grupą homologii podwójnego nakrycia $S^3$ rozgałęzionego nad składowymi.
    Jeśli $G$ jest torsyjna, to $\jones(-1) = |G|$.
    W przeciwnym razie $\jones(-1) = 0$.
\end{proposition}

\begin{proof}
    ????
\end{proof}

Nie jest znana topologiczna interpretacja wielomianu Jonesa (którą posiada wielomian Alexandera) ani charakteryzacja poza warunkami koniecznymi z~pięciu faktów powyżej.

\begin{corollary}
    Niech $K$ będzie węzłem.
    Wtedy
    \begin{align}
        \jones(1) & = 1 \\
        \jones(-1) & = \pm \det K \\
        \jones(i) & = \begin{cases}
            1 & \text{dla } \alexander(-1) \equiv \pm 1 \mod 8 \\
            -1 & \text{w przeciwnym razie.}
        \end{cases}
    \end{align}
\end{corollary}

Poza powyżej opisanymi przypadkami, wartości wielomianu Jonesa nie można znaleźć w~czasie wielomianowym od ilości skrzyżowań na diagramie (jest to problem $\#P$-trudny).

Czemu wielomian Jonesa jest wielomianem?
Odpowiedniki wielomianu Jonesa dla węzłów w~3-rozmaitościach innych niż sfera $S^3$ nie są wielomianami, ale funkcjami z~pierwiastków jedności w~zbiór elementów całkowitch\footnote{algebraic integers} (jak podaje J. Roberts).

Dotychczas wyznaczyliśmy wielomian Jonesa jedynie dla splotów trywialnych (fakt \ref{prp:jones_trivial_link}).
Dlaczego?
Chociaż klamra Kauffmana to użyteczne narzędzie podczas dowodzenia różnych teoretycznych własności, niezbyt nadaje się do obliczeń, szczególnie ręcznych.
Na szczęście wtedy z pomocą przychodzi:

\begin{theorem}[relacja kłębiasta]
    \label{tracheotomia}
    \index{relacja kłębiasta}
    Wielomian Jonesa spełnia zależność rekurencyjną
    \begin{equation}
        t^{-1} \jones(L_+) - t\jones(L_-) + (t^{-1/2} - t^{1/2}) \jones(L_0) = 0
    \end{equation}
    z warunkiem brzegowym $\jones(\LittleUnknot) = 1$.
\end{theorem}

Termin ,,skein'' (kłąb) wprowadził Conway około roku 1970, kontynuując tradycję używania słów, które kojarzą się ze sznurkami.

\begin{proof}
Wyraźmy wielomian Jonesa przez nawias Kauffmana i~spin.
Chcemy pokazać, że
\begin{equation}
    A^{4}(-A)^{-3w(L_+)}\bracket{\LittleRightCrossing}
    -A^{-4}(-A)^{-3w(L_-)}\bracket{\LittleLeftCrossing}
    +(A^2-A^{-2})(-A)^{-3w(L_0)}\bracket{\LittleRightSmoothing} = 0.
\end{equation}

Ale $w(L_\pm)=w(L_0)\pm 1$, zatem to jest równoważne z
\[
    -A\bracket{\LittleRightCrossing} +A^{-1}\bracket{\LittleLeftCrossing} +(A^2-A^{-2})\bracket{\LittleRightSmoothing} =0.
\]
Z definicji nawiasu Kauffmana wnioskujemy, że
$\bracket{\LittleRightCrossing} = A\bracket{\LittleRightSmoothing}+A^{-1}\bracket{\LittleLeftSmoothing}$ i
$\bracket{\LittleLeftCrossing} = A\bracket{\LittleLeftSmoothing}+A^{-1}\bracket{\LittleRightSmoothing}$.
Pierwsze równanie przemnóżmy przez $A$, drugie przez $A^{-1}$, a~następnie dodajmy je do siebie.
Wtedy otrzymamy $A\bracket{\LittleRightCrossing}-A^{-1}\bracket{\LittleLeftCrossing} =
A^2\bracket{\LittleRightSmoothing} - A^{-2}\bracket{\LittleRightSmoothing}$.
\end{proof}

% \subsection{Odwrotności, lustra i~sumy}
Wielomian Jonesa nie wykrywa orientacji splotu.

\begin{proposition}
    Niech $L$ będzie zorientowanym splotem.
    Wtedy $\jones(rL)=\jones(L)$.
\end{proposition}

\begin{proof}
    Aby obliczyć wielomian rewersu, wykorzystujemy te same diagramy kłębiaste,
    jak dla zwykłego, a~przy tym nie zmieniamy znaku żadnego skrzyżowania.
\end{proof}

Ale czasami potrafi odróżnić splot od jego lustra:

\begin{proposition}
    Niech $L$ będzie zorientowanym splotem.
    Wtedy $\jones(mL)(t)=\jones(L)(t^{-1})$.
\end{proposition}

\begin{proof}
    Zauważmy, że diagramy $L_-$ oraz $L_+$ są wzajemnymi lustrami.
    Dlatego każda relacja kłębiasta dla splotu postaci
    \begin{equation}
        t^{-1} \jones(L_+)(t) - t\jones(L_-)(t) + (t^{-1/2} - t^{1/2}) \jones(L_0)(t) = 0
    \end{equation}
    odpowiada pewnej relacji dla lustra splotu:
    \begin{equation}
        -t\jones(L_+)(t) + t^{-1} \jones(L_-)(t) + (t^{-1/2} - t^{1/2}) \jones(L_0)(t) = 0,
    \end{equation}
    co po zamianie zmiennych $t \mapsto t^{-1}$ i przemnożeniu przez $-1$ daje
    \begin{equation}
        -t^{-1} \jones(L_+)(t^{-1}) + t \jones(L_-)(t^{-1}) + (t^{1/2} - t^{-1/2}) \jones(L_0)(t^{-1}) = 0.
    \end{equation}

    Patrz też: Florian Gellert, Kombinatorische Invarianten, strona 12.
\end{proof}

\begin{corollary}
    \label{amphicheiral_implies_jones}
    Jeśli $K$ jest węzłem zwierciadlanym, to wielomian $\jones_K$ jest symetryczny.
\end{corollary}

Implikacja odwrotna nie zachodzi na mocy wniosku \ref{crl:acheiral_signature}: węzeł $9_{42}$ ma symetryczny wielomian Jonesa, ale niezerową sygnaturę.
Poniżej trzynastu skrzyżowań taka sytuacja ma miejsce dla dokładnie czternastu węzłów pierwszych.
% 9_42, 10_125, 11n_19, 11n_24, 11n_82, 12a_0669, 12a_1171, 12a_1179, 12a_1205, 12n_0362, 12n_0506, 12n_0562, 12n_0571, 12n_0821

Równość $\jones(mL)(t)=\jones(L)(t^{-1})$ nie jest spełniona dla trójlistnika, zatem ten nie jest równoważny ze swoim lustrem.
Wcześniej pokazał to z~dużo większym wysiłkiem Dehn, patrz przykład \ref{trefoil_is_chiral}.

\begin{corollary}
    Wielomian Jonesa nie zależy od orientacji węzła.
    Nie jest to prawdą dla splotów.
\end{corollary}

\begin{proof}
    Każdy węzeł ma tylko dwie orientacje, splot może mieć ich $2^n$, gdzie $n$ to liczba składowych.
\end{proof}

\begin{proposition} \label{prp:jones_multiplicative_1}
    Niech $L, M$ będą zorientowanymi splotami, wtedy
    \begin{equation}
        \jones(L \sqcup M) = (-t^{1/2} - t^{-1/2}) \jones(L) \jones(M).
    \end{equation}
\end{proposition}

\begin{proof}
    Wybierzmy diagramy $D, E$ dla (odpowiednio) $L, M$.
    Po podstawieniu $t^{1/2}=A^{-2}$ widzimy, że chcemy pokazać
    \begin{equation}
        (-A)^{-3w(D\sqcup E)} \langle D\sqcup E\rangle = (-A^2-A^{-2})(-A)^{-3(w(D)+w(E))} \langle D \rangle \langle E \rangle.
    \end{equation}

    Oczywiście $w(D\sqcup E)=w(D)+w(E)$, więc wystarczy udowodnić, że
    \begin{equation}
        \langle D\sqcup E\rangle = (-A^2-A^{-2})\langle D\rangle\langle E\rangle.
    \end{equation}

    Oznaczmy przez $f_1(D)$, $f_2(D)$ lewą i~prawą stronę ostatniego równania.
    Są to wielomiany Laurenta, które zależą tylko od $D$.
    Aksjomaty Kauffmana pozwalają na pokazanie, że obie funkcje mają następujące własności:
    \begin{align*}
        f_i(\LittleUnknot)        & = (-A^2-A^{-2}) \langle E \rangle \\
        f_i(D\sqcup\LittleUnknot) & = (-A^2-A^{-2}) f_i(D) \\
        f_i(\LittleRightCrossing) & = Af_i(\LittleRightSmoothing) + A^{-1}f_i(\LittleLeftSmoothing).
    \end{align*}
    To wyznacza ich wartości dla dowolnego diagramu $D$, zatem $f_1 \equiv f_2$, co kończy dowód.
\end{proof}

\begin{proposition} \label{prp:jones_multiplicative_2}
    Niech $J, K$ będą zorientowanymi węzłami, wtedy
    \begin{equation}
        \jones(J \# K) = \jones(J) \jones(K).
    \end{equation}
\end{proposition}

\begin{proof}
    Rozpatrzmy sploty
\begin{comment}
    \[
        \begin{tikzpicture}[baseline=-0.65ex,scale=0.07]
        \begin{knot}[clip width=5, flip crossing/.list={1}]
            \strand[semithick] (-22, -5) rectangle (-12, 5);
            \strand[semithick] (22, -5) rectangle (12, 5);

            \strand[semithick,Latex-] (-12, 3) [in=left, out=right] to (12, -3);
            \strand[semithick,Latex-] (12, 3) [in=right, out=left] to (-12, -3);

            \node at (-17, 0) {$J$};
            \node at (17, 0) {$K$};
        \end{knot}
        \end{tikzpicture}
        \quad\quad
        \begin{tikzpicture}[baseline=-0.65ex,scale=0.07]
        \begin{knot}[clip width=5]
            \strand[semithick] (-22, -5) rectangle (-12, 5);
            \strand[semithick] (22, -5) rectangle (12, 5);

            \strand[semithick,Latex-] (-12, 3) [in=left, out=right] to (12, -3);
            \strand[semithick,Latex-] (12, 3) [in=right, out=left] to (-12, -3);

            \node at (-17, 0) {$J$};
            \node at (17, 0) {$K$};
        \end{knot}
        \end{tikzpicture}
        \quad\quad
        \begin{tikzpicture}[baseline=-0.65ex,scale=0.07]
        \begin{knot}[clip width=5]
            \strand[semithick] (-22, -5) rectangle (-12, 5);
            \strand[semithick] (-12, -3) [in=down, out=right] to (-2, 0);
            \strand[semithick,Latex-] (-12, 3) [in=up, out=right] to (-2, 0);

            \strand[semithick] (22, -5) rectangle (12, 5);
            \strand[semithick] (12, -3) [in=down, out=left] to (2, 0);
            \strand[semithick,Latex-] (12, 3) [in=up, out=left] to (2, 0);

            \node at (-17, 0) {$J$};
            \node at (17, 0) {$K$};
        \end{knot}
        \end{tikzpicture}
    \]
\end{comment}
    Relacja kłębiasta może zostać użyta do pokazania, że
    \begin{equation}
    t^{-1}\jones(J\#K)-t\jones(J\#K)+(t^{-1/2}-t^{1/2})\jones(J\sqcup K)=0.
    \end{equation}
    Ale $\jones(J\sqcup K)=(-t^{1/2}-t^{-1/2})\jones(J)\jones(K)$, co upraszcza się do $\jones(J\#K)=\jones(J)\jones(K)$ i~kończy dowód.
\end{proof}

% Koniec sekcji Relacja kłębiasta
% Koniec podsekcji Wielomian Jonesa
% Koniec podsekcji Nawias Kauffmana