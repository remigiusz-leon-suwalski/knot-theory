
\section{Wielomian Kauffmana}
\index{wielomian!Kauffmana|(}%
Mniej więcej w~tym samym czasie, gdy odkryto wielomian BLM/Ho, Kauffman \cite{kauffman1990} opisał sposób, jak uogólnić ten niezmiennik do odróżniającego lustra.
\index[persons]{Kauffman, Louis}%
Wielomianu $F$ Kauffmana nie należy mylić z~klamrą Kauffmana $\langle \ldots \rangle$!

\begin{definition}[wielomian Kauffmana]
    Niech $L$ będzie zorientowanym splotem, zaś $D$ ustalonym diagramem o~spinie $\writhe D$.
    Istnieje wielomian $\Lambda(L)$ wyznaczony przez relację kłębiastą (z oznaczeniami jak na stronie \pageref{unoriented_diagrams_infty})
    \begin{equation}
\begin{comment}
        \Lambda (L_+) +
        \Lambda (L_-) =
        z \cdot \left(
        \Lambda (L_0) +
        \Lambda (L_\infty)
        \right)
\end{comment}
        ,
    \end{equation}
    który jest niezmienniczy względem II i III ruchu Reidemeistera, spełnia równości:
    \begin{equation}
\begin{comment}
        \frac 1 a \Lambda \left(\MediumThinReidemeisterOneRight\right) =
        \Lambda \left(\MediumThinReidemeisterOneStraight\right)
\end{comment}
        =
\begin{comment}
        a \Lambda \left(\MediumThinReidemeisterOneLeft\right)
\end{comment}
    \end{equation}
    oraz warunek brzegowy $\Lambda(\SmallUnknot) = 1$.
    Wtedy wielomian dwóch zmiennych
    \begin{equation}
        F_L(a, z) = a^{-\writhe D} \Lambda(D),
    \end{equation}
    nazywamy wielomianem Kauffmana.
    Jest niezmiennikiem splotów.
\end{definition}

Jego związki z~wielomianem HOMFLY pozostają nieznane.
Wiemy natomiast, że

\begin{proposition}
\index{wielomian!BLM/Ho}%
    Wielomian BLM/Ho jest szczególnym przypadkiem wielomianu Kauffmana, mamy równość
    \begin{equation}
        Q(x) = F(1, x).
    \end{equation}
\end{proposition}

\begin{proposition}
\index{wielomian!Jonesa}%
    Wielomian Jonesa jest szczególnym przypadkiem wielomianu Kauffmana, mamy równość
    \begin{equation}
        \jones(t)=F(-t^{-3/4},t^{-1/4}+t^{1/4}).
    \end{equation}
\end{proposition}

Rozpatrzmy relację kłębiastą $\Lambda_+ - \Lambda_- = x(\Lambda_0 - \Lambda_\infty)$.
Prowadzi ona do wielomianu ,,z~Dubrownika'': Kauffman na pocztówce napisanej do Lickorisha z Dubrownika w~1985 roku opisał ten wielomian sądząc, że jest to nowy, niezależny od $F$ niezmiennik.
\index[persons]{Lickorish, William}%
\index{wielomian!z Dubrownika}%
\index{wielomian!Kauffmana|)}%

% Koniec sekcji Wielomian Kauffmana

