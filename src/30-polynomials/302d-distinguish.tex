
\subsection{Odróżnianie węzłów i splotów wielomianem Jonesa}
Wielomian Jonesa często (chociaż nie zawsze) odróżnia od siebie sploty lepiej niż wielomian Alexandera.
Na przykład: wielomian Alexandera wszystkich splotów rozszczepialnych jest taki sam (stwierdzenie \ref{prp:alexander_unlinks}), więc nie odróżnia wcale niesplotów.
Dla porównania, wielomian Jonesa odróżnia je wszystkie:

\begin{proposition}
\label{prp:jones_trivial_link}%
    Wielomianem Jonesa splotu trywialnego o $n$ ogniwach jest
    \begin{equation}
        \jones(K_n) = \left(-\sqrt{t} - \frac{1}{\sqrt {t}}\right)^{n-1}.
    \end{equation}
\end{proposition}

Co więcej, wielomian Jonesa odróżnia od siebie dowolne dwa węzły pierwsze o~co najwyżej 9 skrzyżowaniach.
Dalej występują już kolizje, oto pełna ich lista do 10 skrzyżowań:

\renewcommand*{\arraystretch}{1.4}
\footnotesize
\begin{longtable}{lcccccccccccccc}
    $K_1$ & \rotatebox{90}{$5_{1}$} & \rotatebox{90}{$8_{8}$} & \rotatebox{90}{$8_{16}$} & \rotatebox{90}{$10_{22}$} & \rotatebox{90}{$10_{25}$} & \rotatebox{90}{$10_{40}$}  & \rotatebox{90}{$10_{41}$}  & \rotatebox{90}{$10_{43}$} & \rotatebox{90}{$10_{59}$} & \rotatebox{90}{$10_{60}$} & \rotatebox{90}{$10_{71}$}  & \rotatebox{90}{$10_{73}$}  & \rotatebox{90}{$10_{81}$} & \rotatebox{90}{$10_{137}$} \\
    $K_2$ & \rotatebox{90}{$10_{132}$} & \rotatebox{90}{$10_{129}$} & \rotatebox{90}{$10_{156}$} & \rotatebox{90}{$10_{35}$} & 
\rotatebox{90}{$10_{56}$} & \rotatebox{90}{$10_{103}$} & \rotatebox{90}{$10_{94}$} & \rotatebox{90}{$10_{91}$} & \rotatebox{90}{$10_{106}$} & \rotatebox{90}{$10_{86}$} & \rotatebox{90}{$10_{104}$\,\,} & \rotatebox{90}{$10_{83}$} & \rotatebox{90}{$10_{109}$} & \rotatebox{90}{$10_{155}$}  \\
    \hline
\end{longtable}
% ZWERYFIKOWANO: funkcja jones_collisions
\normalsize

Jones wiedział, że wielomianowe niezmienniki nie radzą sobie z~odróżnianiem od siebie mutantów, dlatego zapytał, czy jego wielomian wykrywa niewęzły.
Pozostaje to otwartym problemem do dziś.

\begin{conjecture}
\index{hipoteza!o wielomianie Jonesa i niewęźle}%
\label{con:jones}%
    Niech $K$ będzie węzłem.
    Jeśli $\jones_K(t) \equiv 1$, to $K$ jest niewęzłem.
\end{conjecture}

Hipotezę zweryfikowano komputerowo dla węzłów o~małej liczbie skrzyżowań.
W latach dziewięćdziesiątych Hoste, Thistlethwaite, Weeks \cite{thistlethwaite98} zrobili to podczas tablicowania węzłów spełniających $\crossing K \le 16$.
\index[persons]{Hoste, Jim}%
\index[persons]{Thistlethwaite, Morwen}%
\index[persons]{Weeks, Jeff}%
Wynik poprawiano:
Dasbach, Hougardy \cite{hougardy97} w~1997 do $\crossing K \le 17$; 
\index[persons]{Dasbach, Oliver}%
\index[persons]{Hougardy, Stefan}%
Yamada \cite{yamada00} w~2000 do $\crossing K \le 18$;
\index[persons]{Yamada, Shuji}%
wreszcie Tuzun, Sikora \cite{tuzun18} w~2016 do $\crossing K \le 22$,
\index[persons]{Sikora, Adam}%
\index[persons]{Tuzun, Robert}%
potem (w \cite{tuzun21}) w~2020 do $\crossing K \le 24$.
Ośmiordzeniowe procesory Intel Xeon L5520 z Uniwersytetu w~Buffalo potrzebowały na to łącznie 41,8 lat pracy.
Patrz też \cite[s. 381]{ohtsuki02}.

Jeżeli chodzi o ogólne wyniki, to Lickorish i~Thistlethwaite \cite{lickorish88} pokazali wcześniej, że adekwatne węzły mają nietrywialny wielomian Jonesa.
% Knots with adequate diagrams have nontrivial Jones polynomial. - MSN
\index[persons]{Thistlethwaite, Morwen}%
\index[persons]{Lickorish, William}%

Istnieją sploty o~trywialnym wielomianie Jonesa.
Thistlethwaite wskazał dwa z~dwoma oraz jeden z~trzema ogniwami w~\cite{thistlethwaite01}.
\index[persons]{Thistlethwaite, Morwen}%
Jest ich nawet nieskończenie wiele, jak Eliahou, Kauffman i~Thistlethwaite pokazali w~pracy \cite{eliahou03}.
\index[persons]{Eliahou, Shalom}%
\index[persons]{Kauffman, Louis}%

\begin{proposition}
\index{splot!Hopfa}%
\index{węzeł!satelitarny}%
    Niech $k \ge 2$ będzie liczbą naturalną.
    Istnieje nieskończenie wiele splotów pierwszych z $k$ ogniwami, których wielomian Jonesa nie odróżnia od niesplotu z $k$ ogniwami.

    Co więcej, można wymagać, by wszystkie te sploty były satelitami splotu Hopfa.
\end{proposition}

