
\subsection{Genus}
\index{genus|(}%
Termin genus został wymyślony przez Clebscha około 1864 roku (w~,,Über die Anwendung der Abelschen Functionen in der Geometrie'', jak można przeczytać w~przypisie do pracy ,,Development of the concept of homotopy'' Rii Vanden Eynde) i początkowo oznaczał po prostu liczbę dziur w~powierzchni.
Jak piszą Burde, Zieschang, Heusener \cite[s. 20]{burde2014}, \emph{,,the notion of the genus was first introduced by H. Seifert in \cite{seifert1935}, it holds a central position in knot theory''}; zbadamy więc teraz bliżej ten geometryczny niezmiennik węzłów.
(Nie pamiętamy, czemu ta podsekcja znajduje się akurat tutaj, ale chyba nie jest to najgorsze miejsce dla niej.)

Zaczniemy od starego twierdzenia, które klasyfikuje powierzchnie domknięte.

\begin{proposition}
    Niech $S$ będzie powierzchnią domkniętą.
    Wtedy dokładnie jedno zdanie z~poniższych jest prawdziwe:
    \begin{enumerate}[leftmargin=*]
        \itemsep0em
        \item $S$ jest orientowalna, sumą spójną $g \ge 0$ torusów
        \item $S$ jest nieorientowalna, sumą spójną $k \ge 1$ rzeczywistych płaszczyzn rzutowych.
    \end{enumerate}
    Liczby $g$ lub $k$ są wyznaczone jednoznacznie. % , to znaczy jeśli coś jest sumą spójną $g_1$ lub $g_2$ torusów, to $g_1 = g_2$ i podobnie dla płaszczyzn rzutowych).
\end{proposition}

Sferę traktujemy dla wygody jako sumę spójną $g = 0$ torusów.
Wtedy sumę spójną $g$ torusów możemy wyobrazić sobie jako sferę, do której doklejono $g$ rączek.
(Suma spójna torusa i~płaszczyzny rzutowej jest tym samym, co suma spójna trzech płaszczyzn rzutowych -- wykazanie tego jest jednym z kroków w klasyfikacji powierzchni.)

\begin{definition}[genus powierzchni]
    Niech $S$ będzie orientowalną powierzchnią będącą sumą spójną $\genus$ torusów.
    Liczbę $\genus$ nazywamy genusem powierzchni $S$.
\end{definition}

Podobna charakteryzacja istnieje dla powierzchni z~brzegiem.
Każdy taki obiekt jest homeomorficzny z~sumą spójną $g$ torusów, w~których wydrążono pewną liczbę otworów: tyle, ile składowych spójności ma brzeg powierzchni.
W~przypadku powierzchni Seiferta mamy do czynienia z jednym otworem.

Definiujemy jeszcze jeden klasyczny niezmiennik powierzchni: charakterystykę Eulera-Poincarégo, odkrytą najpierw dla wielościanów platońskich w 1537 roku przez Francesco Maurolico i wiele lat później przez Eulera dla wielościanów wypukłych.
(Obecnie wyprowadza się ją z homologii/algebry homologicznej, ale darujemy to sobie).

\begin{definition}[charakterystyka Eulera]
\index{charakterystyka Eulera}%
    Niech $S$ będzie domkniętą powierzchnią orientowalną.
    Po striangulowaniu, składa się z $k_0$ wierzchołków, $k_1$ krawędzi oraz $k_2$ ścian.
    Wielkość
    \begin{equation}
        \chi = k_0 - k_1 + k_2
    \end{equation}
    jest niezmiennikiem powierzchni, zwanym charakterystyką Eulera.
\end{definition}

Definicja ta nie jest wygodna podczas ręcznych obliczeń.
Mamy za to:

\begin{proposition}
    Charakterystykę Eulera powierzchni jednoznacznie wyznaczają cztery reguły, gdzie stale $S_1, S_2$ są pewnymi powierzchniami, zaś $S$ jest dyskiem:
    \begin{itemize}
        \item $\chi(S) = 1$,
        \item $\chi(S_1 \sqcup S_2) = \chi(S_1) + \chi(S_2)$,
        \item Jeśli $S_2$ powstaje z $S_1$ przez dołączenie paska, to $\chi(S_2) = \chi(S_1) - 1$,
        \item Jeśli $S_2$ powstaje z $S_1$ przez dołączenie dysku do całej składowej spójności brzegu, to $\chi(S_2) = \chi(S_1) + 1$.
    \end{itemize}
\end{proposition}

Genus oraz charakterystyka Eulera są ze sobą związane:

\begin{proposition}
    Niech $S$ będzie powierzchnią o genusie $\genus$ i $\mu$ składowych spójności brzegu.
    Wtedy zachodzi równość
    \begin{equation}
        \chi = 2 - \mu - 2\genus.
    \end{equation}
\end{proposition}

Nas interesują głównie powierzchnie Seiferta węzłów:
\index{powierzchnia Seiferta}%

\begin{proposition}
\label{prp:seifert_euler_characteristics}%
    Niech $D$ będzie diagramem o $b$ skrzyżowaniach pewnego węzła $K$, z którego algorytm Seiferta produkuje $d$ okręgów.
    Wtedy $\chi(M_D) = d - b$.
\end{proposition}

Można przeczytać o tym w \cite[s. 82]{murasugi1996}.

\begin{proof}
    W~dowodzie faktu \ref{prp:seifert_exists} widzieliśmy, że liczba skrzyżowań $b$ jest jednocześnie liczbą pasków doklejonych do dysków.
    Bezpośredni rachunek pokazuje, że wtedy $k_0 = 4b$, $k_1 = 6b$ oraz $k_2 = b+d$.
    Wynika stąd, że $\chi = 4b - 6b + b + d = d - b$.
\end{proof}

Reszta tej podsekcji nie jest wymagana do zrozumienia macierzy Seiferta, przyjrzymy się genusowi jako obiektowi ciekawemu samemu w sobie.

\begin{definition}[3-genus]
    Niech $K$ będzie węzłem.
    Wśród wszystkich powierzchni Seiferta węzła $K$ istnieje co najmniej jedna o minimalnym genusie, jej genus nazywamy 3-genusem węzła $K$ i oznaczamy także przez $\genus$.
\end{definition}

Jeżeli nie powoduje to nieporozumień, zamiast 3-genus można pisać po prostu genus.

\begin{proposition}
\label{prp:genus_detects_unknot}%
    Węzeł $K$ jest niewęzłem wtedy i tylko wtedy, gdy $\genus K = 0$.
\end{proposition}

,,Genus wykrywa niewęzły''.

\begin{proof}
    Niech $K$ będzie węzłem o genusie $0$.
    Z~charakteryzacji powierzchni wynika, że jego powierzchnia Seiferta to suma spójna $0$ torusów, to znaczy kula z tyloma otworami, ile $K$ ma ogniw.
    Innymi słowy, powierzchnią Seiferta węzła $K$ jest dysk, którego brzeg stanowi niewęzeł.
    To pokazuje, że implikacja w lewo jest prawdziwa.

    Implikacja w prawo jest oczywista.
\end{proof}

\input{40-topology/402ca-genus_lower_bound}

\input{40-topology/402cb-genus_upper_bound}

\input{40-topology/402cc-genus_decomposition}

\input{40-topology/402cd-genus_free_canonical}

\index{genus|)}%

% Koniec podsekcji Genus

