
Powoli kończymy definiować niezmienniki węzłów, po tym rozdziale brakować będzie tylko liczby warkoczowej.
Zaczniemy opisu grupy podstawowej dopełnienia węzła oraz prezentacji tej grupy znalezionej przez Wirtingera.
Następnie pokażemy, że każdy węzeł jest brzegiem pewnej zorientowanej powierzchni (zwanej powierzchnią Seiferta) i odkryjemy tak źródło kolejnych niezmienników: genusu, wyznacznika, sygnatury.
Nie do końca wiadomo dlaczego, ale wspomnimy krótko o~niezmienniku Arfa.
Na koniec spróbujemy przekonać czytelnika, że najciekawszą teorią homologii dla węzłów są homologie Chowanowa.

% koniec wstępu do rozdziału 4: topologia

