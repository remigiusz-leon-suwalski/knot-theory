\subsection{Sygnatura} % (fold)
\label{sub:signature}
\index{sygnatura}
Sygnatura jest jednym z wielu niezmienników, do zdefiniowania których wystarczy znać macierz Seiferta.
Pochodzi prawdopodobnie z lat sześdziesiątych (Trotter 1962, Murasugi 1965) i~pojawia się jeszcze na przykład w~fakcie \ref{slice_signature}.

\begin{definition}[sygnatura]
    Niech $M$ będzie macierzą Seiferta zorientowanego splotu $L$.
    Wielkość
    \begin{equation}
        \sigma_L := \operatorname{sign} (M + M^t)
    \end{equation}
    nazywamy sygnaturą splotu $L$.
\end{definition}

\begin{proposition} \label{prop_sigma_additive}
    Sygnatura jest addytywna: $\sigma(K_1 \shrap \ldots \shrap K_n) = \sum_{k=1}^n \sigma(K_k)$.
\end{proposition}

\begin{proof}
    Bez straty ogólności ograniczmy się do przypadku $n = 2$ i~ustalmy powierzchnie Seiferta $F_1, F_2$ dla węzłów $K_1, K_2$ z~macierzami Seiferta $M_1, M_2$.
    Powierzchnia dla ich sumy spójnej $K_1 \shrap K_2$ powstaje przez sklejenie $F_1$ oraz $F_2$ paskiem.
    W języku macierzy oznacza to, że macierz Seiferta węzła $K_1 \shrap K_2$ ma postać $M = M_1 \oplus M_2$.
    Zatem:
    \begin{align}
        \sigma(K_1 \shrap K_2) & = \sigma(M + M^t) \\
                               & = \sigma(M_1 + M_1^t) + \sigma(M_2 + M_2^t) = \sigma(K_1) + \sigma(K_2),
    \end{align}
    co kończy dowód.
\end{proof}

\begin{proposition} \label{prop_sigma_inverse}
    Niech $L$ będzie splotem.
    Wtedy $\sigma(mL) = -\sigma(L)$ oraz $\sigma(rL) = \sigma(L)$.
\end{proposition}

\begin{proof}
    Wynika to z podobnych faktów dla macierzy Seiferta.
    Równoważność $M_{mL} \simeq - M_L^t$ wynika z tego, że zamiana nad- i podskrzyżowań odwraca wzajemne położenie krzywych, których indeksu zaczepienia szukamy.

    Podobnie pokazuje się, że $M_{rL} \simeq M_L^t$.
\end{proof}

\begin{corollary}
    \label{crl:acheiral_signature}
    Jeśli $K$ jest węzłem achiralnym, to $\sigma(K) = 0$.
\end{corollary}

\begin{proposition}
    Niech $L$ będzie węzłem.
    Jeśli $\alexander_K(t) \equiv 1$, to $\sigma (K) = 0$.
\end{proposition}

\begin{proof}
    Patrz \cite{milnor68}.
\end{proof}

Węzły achiralne mają zerową sygnaturę, zatem trójlistnik nie jest achiralny.
Z faktów \ref{prop_sigma_additive} oraz \ref{prop_sigma_inverse} wynika, że suma tak samo skręconych trójlistników nie jest achiralna ($\sigma = \pm 4$), natomiast węzeł prosty (suma różnie skręconych) ma zerową sygnaturę i jak można przekonać się ze standardowego diagramu, jest achiralny.

Nie istnieje bezpośredni związek między sygnaturą i~liczbą mostową.
Węzeł torusowy $T_{2,n}$ jest dwumostowy, jego sygnatura wynosi $n - 1$.
Suma spójna węzłów prostych ma zerową sygnaturę, ale na mocy faktu \ref{bridge_additive} jej liczba mostowa jest nieograniczona.

Istnieje równoważna definicja, która nie wymaga czasochłonnego wyznaczania macierzy Seiferta.
Wyjaśnia to trzecie spostrzeżenie z~\cite{giller82} oparte o~twierdzenie 5.6 z~\cite{murasugi65}.

\begin{definition}
    Sygnatura to niezmiennik topologiczny zadany kłębiastą relacją rekurencyjną:
    \begin{itemize}[leftmargin=*]
    \itemsep0em
        \item $\sigma (\LittleUnknot) = 0$,
        \item $\sigma (K_+) - \sigma (K_-) \in \{0, 2\}$,
        \item $4 \mid \sigma (K)$ wtedy i~tylko wtedy, gdy $\conway(2i) > 0$ (wielomian Conwaya).
    \end{itemize}
\end{definition}

Sygnatura pozwala uzyskać proste oszacowanie liczby gordyjskiej od dołu:

\begin{proposition}
    Mamy $2 u(K) \ge |\sigma(K)|$.
\end{proposition}

Liczba gordyjska 87 z~801 węzłów pierwszych o mniej niż dwunastu skrzyżowaniach nie jest jeszcze znana.
Dla 311 spośród pozostałych mamy równość $2u = |\sigma|$.

\begin{proof}
    Ustalmy diagram $D$ dla węzła $K$.
    Odwrócenie dowolnego skrzyżowania polega na przejściu z~diagramu $D_+$ do $D_-$ lub z~$D_-$ do $D_+$.
    Zgodnie z relacją kłębiastą, sygnatura pozostaje taka sama lub zmienia wartość o $2$.
    Po wykonaniu $u$ odwróceń otrzymujemy diagram niewęzła o~sygnaturze zero, zatem sygnatura wyjściowego węzła nie mogła przekraczać $2u$.
    To kończy dowód.
\end{proof}

Jeszcze w latach sześćdziesiątych sygnatura została uogólniona.
Sygnatura Levine'a-Tristrama to funkcja $S^1 \to \Z$, niezmiennik zorientowanych splotów.
Klasyczna sygnatura to wartość tej funkcji w punkcie $-1$.
Większość podręczników, a także prace Levine'a \textbf{Jerome Levine. Knot cobordism groups in codimension two. Comment. Math. Helv., 44:229– 244, 1969.} oraz Tristrama (\textbf{Andrew Tristram. Some cobordism invariants for links. Proc. Cambridge Philos. Soc., 66:251–264, 1969.}) wprowadza ją przy użyciu macierzy Seiferta, my postąpimy dokładnie tak samo.
Dużo pożytecznych informacji można znaleźć w pracy przeglądowej \cite{conway19}.

\begin{definition}[sygnatura Levine'a-Tristrama]
    Niech $M$ będzie macierzą Seiferta zorientowanego splotu $L$.
    Funkcję $\sigma_L \colon S^1 \to \Z$ daną wzorem
    \begin{equation}
        \sigma_L(\omega) := \operatorname{sign} [(1-\omega) M + (1 - \overline{\omega})M^t]
    \end{equation}
    nazywamy sygnaturą Levine'a-Tristrama splotu $L$.
\end{definition}

Funkcja $\sigma_L$ jest kawałkami stała.
Conway pisze w \cite{conway19}, że wynika to ze wzoru na wielomian Alexandera $\Delta_L(t) = \det(tM - M^t)$.
Jedynymi punktami nieciągłości są zera wielomianu $(t-1)\Delta^{tor}_L(t)$ (\textbf{Patrick M. Gilmer and Charles Livingston. Signature jumps and Alexander polynomials for links. Proc. Amer. Math. Soc., 144(12):5407–5417, 2016.}).

Livingston scharakteryzował w pełni funkcje $S^1 \to \Z$, które są realizowane jako sygnatura Levine'a-Tristrama pewnego węzła (\textbf{Charles Livingston. Signature functions of knots. September 2017.
https://arxiv.org/pdf/1709.00732.pdf}), analogiczny problem dla splotów wydaje się być wciąż otwarty.

Wreszcie:

\begin{proposition}
    Niech $L$ będzie splotem.
    Wtedy albo wielomian Alexandera $\Delta_L(t)$ jest tożsamościowo zerem, albo posiada co najmniej $|\sigma_L|$ zer, liczonych z krotnościami, na okręgu jednostkowym.
\end{proposition}

\begin{proof}
    \textbf{Livio Liechti. Signature, positive Hopf plumbing and the Coxeter transformation. Osaka J.
Math., 53(1):251–266, 2016. With an appendix by Peter Feller and Liechti.}
\end{proof}

Czas na raczej niezbyt użyteczne ciekawostki.
Czy istnieje węzeł o~sygnaturze $4$ i~wyznaczniku postaci $n = 4k + 1$ dla $k$ całkowitego dodatniego?
Stojmenow twierdzi, że jeśli tak jest, to wszystkie pierwsze dzielniki $n$ dają resztę $1$ z~dzielenia przez $24$ i~są większe od $2857$.

% In a somewhat different direction, the Levine- Tristram signature is also a lower bound on the algebraic unknotting number [5, 6, 7, 27, 74].

W~\cite{shinohara71} Shinohara pokazał, że dla każdej pary nieujemnych liczb całkowitych $m, n$ istnieje węzeł $K$ o wyznaczniku $4m+1$ ($8m+5$, $4m+3$) oraz sygnaturze bez znaku $8n$ ($8n+4$, $4n+2$).
Ponadto, jeśli $m$ nie dzieli się przez $3$, istnieje węzeł o wyznaczniku $8m+1$ i sygnaturze bez znaku $8n+4$.

% Koniec podsekcji Sygnatura
