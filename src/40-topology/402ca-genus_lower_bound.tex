
\subsubsection{Ograniczanie genusu z dołu}
Znalezienie 3-genusu dowolnego węzła sprawia te same trudności, co wyznaczenie jego liczby gordyjskiej.
Dowolna powierzchnia Seiferta zadaje ograniczenie z góry.
Z dołu 3-genus można szacować przy użyciu wielomianu Alexandera:
\index{wielomian!Alexandera}%

\begin{proposition}
\label{prp:alexander_genus}%
    Niech $K$ będzie węzłem.
    Wtedy $\operatorname{span} \alexander_K(t) \le 2\genus(K)$.
\end{proposition}

To dolne ograniczenie jest realizowane przez pewną powierzchnię Seiferta dla każdego pierwszego węzła o~co najwyżej 11 skrzyżowaniach poza siedmioma wyjątkami: $11n_{42}$, $11n_{67}$, $11n_{97}$ ($g = 2$), $11n_{34}$, $11n_{45}$, $11n_{73}$ oraz $11n_{152}$ ($g = 3$).
% ZWERYFIKOWANO, funkcja alexander_genus
Fakt znaleźliśmy z dowodem w~podręczniku Murasugiego \cite[s. 121]{murasugi96}, a~potem jako ćwiczenie w~\cite[s. 137]{burde14}.

\begin{proof}
    Załóżmy, że $F$ jest powierzchnią Seiferta węzła $K$ o genusie $g$.
    Wtedy macierz Seiferta powstała z $F$ jest stopnia $2g$, więc żaden ze składników jej wyznacznika nie może mieć stopnia (jako wielomian) większego niż $2g$.
\end{proof}

\begin{proposition}
    Niech $K$ będzie węzłem, zaś $M$ jego macierzą Seiferta.
    Równość $\operatorname{span} \alexander_K(t) = 2\genus(K)$ zachodzi wtedy i tylko wtedy, gdy wyznacznik $\det M \neq 0$ jest niezerowy.
\end{proposition}

Floer zdefiniował w~\cite{floer90} przestrzeń wektorową nazywaną teraz homologią Floera, jest ona wyposażona w~endomorfizm parzystego stopnia, który powstaje z 2-wymiarowej klasy homologii reprezentowanej przez powierzchnię Seiferta.
%~kanoniczną gradację modulo $2$ oraz
\index{homologia!Floera}
Ta homologia rozkłada się na sumę prostą przestrzeni własnych wyróżnionego endomorfizmu, ich charakterystyki Eulera są współczynnikami wielomianu Alexandera.
Pozwala to na dokładniejsze szacowanie genusu węzła, patrz prace Ozsvátha, Szabó \cite{szabo03} i Ghigginiego \cite{ghiggini08}.
\index[persons]{Ghiggini, Paolo}%
\index[persons]{Ozsváth, Peter}%
\index[persons]{Szabó, Zoltán}%

