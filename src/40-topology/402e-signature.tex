
\subsection{Sygnatura}
\label{sub:signature}%
\index{sygnatura|(}%
Sygnatura jest kolejnym niezmiennikiem, do zdefiniowania których wystarczy znać macierz Seiferta.
Pochodzi prawdopodobnie z lat sześćdziesiątych (Trotter \cite{trotter1962} dla węzłów, Murasugi \cite{murasugi1965} dla splotów).
\index[persons]{Trotter, Hale}%
\index[persons]{Murasugi, Kunio}%
% z recenzji do 275415: Shinohara, Yaichi - On the signature of knots and links.
Ktoś powiedział nam kiedyś, że ujednolicenia różnych podejść do form kwadratowych związanych z węzłami dokonali Gordon, Litherland, Murasugi w~pracy \cite{litherland1981}, ale potem trafiliśmy na artykuł Przytyckiego \cite{przytycki2011} pełen historycznych ciekawostek oraz dwóch kolejnych odnośników: do wspomnianej przed chwilą pracy \cite{murasugi1965}, ale też \cite{gordon1978} samych Gordona, Litherlanda, którzy sięgają zenitu i wiążą dziełą Goeritza, Trottera, Murasugiego et alli.
Może po prostu najlepiej przeczytać wszystko?

% DICTIONARY;signature;sygnatura;-
\begin{definition}[sygnatura]
\label{def:signature}%
    Niech $M$ będzie macierzą Seiferta zorientowanego splotu $L$.
    Wielkość
    \begin{equation}
        \sigma_L := \operatorname{\sigma} (M + M^t),
    \end{equation}
    sygnaturę macierzy $M + M^t$, nazywamy sygnaturą splotu $L$.
\end{definition}

\begin{proposition}
\label{prp:signature_additive}%
    Sygnatura jest addytywna: $\sigma(K_1 \shrap \ldots \shrap K_n) = \sum_{k=1}^n \sigma(K_k)$.
\end{proposition}

Wiemy o tym od Murasugiego \cite[s. 127]{murasugi1996}.

\begin{proof}
    Bez straty ogólności ograniczmy się do przypadku $n = 2$ i~ustalmy powierzchnie Seiferta $F_1, F_2$ dla węzłów $K_1, K_2$ z~macierzami Seiferta $M_1, M_2$.
    Powierzchnia dla ich sumy spójnej $K_1 \shrap K_2$ powstaje przez sklejenie $F_1$ oraz $F_2$ paskiem.
    W języku macierzy oznacza to, że macierz Seiferta węzła $K_1 \shrap K_2$ ma postać $M = M_1 \oplus M_2$.
    Zatem:
    \begin{align}
        \sigma(K_1 \shrap K_2) & = \sigma(M + M^t) \\
                               & = \sigma(M_1 + M_1^t) + \sigma(M_2 + M_2^t) \\
                               & = \sigma(K_1) + \sigma(K_2),
    \end{align}
    co kończy dowód.
\end{proof}

\begin{corollary}
\index{liczba mostowa}%
\label{no_relation_signature_bridge}%
    Nie istnieje bezpośredni związek między sygnaturą i~liczbą mostową.
\end{corollary}

Wiemy o tym z książki Livingstona \cite[s. 145]{livingston1993}.

\begin{proof}
    Węzeł torusowy $T_{2,n}$ jest dwumostowy, jego sygnatura wynosi $n - 1$.
    Suma spójna węzłów prostych (sumy przeciwnie zorientowanych trójlistników) ma zerową sygnaturę, ale na mocy faktu~\ref{prp:bridge_additive} jej liczba mostowa jest nieograniczona.
\end{proof}

\begin{proposition}
\index{lustro}%
\index{rewers}%
\label{prp:signature_mirror_reverse}%
    Niech $L$ będzie splotem.
    Wtedy $\sigma(mL) = -\sigma(L)$ oraz $\sigma(rL) = \sigma(L)$.
\end{proposition}

O tym także wiemy od Murasugiego \cite[s. 127]{murasugi1996}.

\begin{proof}
    Wynika to z podobnych faktów dla macierzy Seiferta.
    Równoważność $M_{mL} \simeq - M_L^t$ wynika z tego, że zamiana nad- i podskrzyżowań odwraca wzajemne położenie krzywych, których indeksu zaczepienia szukamy.

    Podobnie pokazuje się, że $M_{rL} \simeq M_L^t$.
\end{proof}

\begin{corollary}
\index{węzeł!achiralny}%
\label{cor:acheiral_signature}%
    Jeśli $K$ jest węzłem achiralnym, to $\sigma(K) = 0$.
\end{corollary}

Węzły achiralne mają zerową sygnaturę, zatem trójlistnik nie jest achiralny.
Z faktów~\ref{prp:signature_additive} oraz~\ref{prp:signature_mirror_reverse} wynika, że suma tak samo zorientowanych trójlistników nie jest achiralna ($\sigma = \pm 4$).
Jak można przekonać się ze standardowego diagramu węzła prostego, ten jest achiralny.
Pisaliśmy coś o tym na stronie \pageref{two_sums_of_two_trefoils}.

\begin{proposition}
\label{trivial_alexander_polynomial}%
\index{wielomian!Alexandera}%
    Niech $K$ będzie węzłem.
    Jeśli $\alexander_K(t) \equiv 1$, to $\sigma (K) = 0$.
\end{proposition}

Założenie $\alexander_K(t) \equiv 1$ jest spełnione przez cztery węzły pierwsze do 12 skrzyżowań, są to $11n_{34}, 11n_{42}, 12n_{313}$ oraz $12n_{430}$, patrz wzmianka po fakcie~\ref{alexander_no_detects_unknot}.
% ZWERYFIKOWANO: funkcja trivial_alexander

\begin{proof}
\index[persons]{Milnor, John}%
    Murasugi twierdzi, że zostało to udowodnione przez Milnora w \cite{milnor1968}, nie jesteśmy jednak pewni, gdzie dokładnie, ale raczej poza sekcją piątą.
\end{proof}

Istnieje równoważna definicja, która nie wymaga czasochłonnego wyznaczania macierzy Seiferta.

\begin{proposition}
\index{relacja kłębiasta}%
    Sygnatura to niezmiennik topologiczny zadany kłębiastą relacją rekurencyjną:
    \begin{itemize}[leftmargin=*]
    \itemsep0em
        \item $\sigma (\SmallUnknot) = 0$,
        \item $\sigma (K_+) - \sigma (K_-) \in \{0, 2\}$,
        \item $4 \mid \sigma (K)$ wtedy i~tylko wtedy, gdy $\conway(2i) > 0$, gdzie $\conway$ to wielomian Conwaya.
    \end{itemize}
\end{proposition}

Symbole $K_+, K_-$ objaśnione są przy definicji~\ref{skein_symbols}.

\begin{proof}
    Wystarczy pokazać, że sygnatura węzła spełnia trzy powyższe aksjomaty, a~następnie zauważyć, że korzystając z~nich jesteśmy w~stanie wyznaczyć jednoznacznie sygnaturę dla dowolnego węzła.
    Wynika to z~faktu, że każdy węzeł można zmienić w~niewęzeł odwracając pewne skrzyżowania.
    Pomysł opisał dokładnie Giller \cite[trzecie spostrzeżenie]{giller1982}, sam oparł się o~\cite[twierdzenie 5.6]{murasugi1965}.
\index[persons]{Giller, Cole}%
\end{proof}

Sygnatura pozwala uzyskać proste oszacowanie liczby gordyjskiej od dołu:
\index{liczba gordyjska}%

\begin{proposition}
    Mamy $2 u(K) \ge |\sigma(K)|$.
\end{proposition}

Liczba gordyjska 83 z~801 węzłów pierwszych o mniej niż dwunastu skrzyżowaniach nie jest jeszcze znana.
Dla 272 spośród pozostałych mamy równość $2u = |\sigma|$.
% ZWERYFIKOWANO: funkcja unknotting_sigma 

\begin{proof}
    Ustalmy diagram $D$ dla węzła $K$.
    Odwrócenie dowolnego skrzyżowania polega na przejściu z~diagramu $D_+$ do $D_-$ lub z~$D_-$ do $D_+$.
    Zgodnie z relacją kłębiastą, sygnatura pozostaje taka sama lub zmienia wartość o $2$.
    Po wykonaniu $u$ odwróceń otrzymujemy diagram niewęzła o~sygnaturze zero, zatem sygnatura wyjściowego węzła nie mogła przekraczać $2u$.
    To kończy dowód.
\end{proof}

W~\cite{shinohara1971} Shinohara pokazał, że dla każdej pary nieujemnych liczb całkowitych $m, n$ istnieje węzeł $K$ o wyznaczniku $4m+1$ ($8m+5$, $4m+3$) oraz sygnaturze bez znaku $8n$ ($8n+4$, $4n+2$).
\index[persons]{Shinohara, Yaichi}%
\index{wyznacznik}%
Ponadto, jeśli $m$ nie dzieli się przez $3$, istnieje węzeł o wyznaczniku $8m+1$ i sygnaturze bez znaku $8n+4$.
% skąd to? Ohtsuki?

Czas na raczej niezbyt użyteczną ciekawostkę.

\begin{conjecture}
    Czy istnieje węzeł o~sygnaturze $4$ i~wyznaczniku postaci $n = 4k + 1$?
\end{conjecture}

Stojmenow twierdzi, że jeśli tak jest, to wszystkie pierwsze dzielniki $n$ muszą dawać resztę $1$ z~dzielenia przez $24$ i~są większe od $2857$.
\index[persons]{Stojmenow, Aleksander}%
Patrz \cite[s. 540]{ohtsuki2002}.

Czytając przeglądową pracę Conwaya \cite{conway2019} dowiedzieliśmy się, że w~latach sześćdziesiątych sygnatura została uogólniona do funkcji $\sigma_L \colon S^1 \to \Z$.
\index[persons]{Conway, John}%
Większość podręczników, a także prace Levine'a \cite{levine1969} oraz Tristrama \cite{tristram1969}, wprowadza ją przy użyciu macierzy Seiferta, więc my postąpimy dokładnie tak samo.
\index[persons]{Levine, Jerome}%
\index[persons]{Tristram, Andrew}%

\begin{definition}[sygnatura Levine'a-Tristrama]
\index{sygnatura!Levine'a-Tristrama}%
    Niech $M$ będzie macierzą Seiferta zorientowanego splotu $L$.
    Funkcję $\sigma_L \colon S^1 \to \Z$ daną wzorem
    \begin{equation}
        \sigma_L(\omega) := \operatorname{\sigma} [(1-\omega) M + (1 - \overline{\omega})M^t]
    \end{equation}
    nazywamy sygnaturą Levine'a-Tristrama splotu $L$.
    Jest niezmiennikiem splotów.
\end{definition}

Funkcja $\sigma_L$ jest kawałkami stała.
Conway pisze w \cite{conway2019}, że wynika to ze wzoru na wielomian Alexandera $\Delta_L(t) = \det(tM - M^t)$.
\index[persons]{Conway, John}%
\index{wielomian!Alexandera}%
Jedynymi punktami nieciągłości są zera wielomianu $(t-1)\Delta_L(t)$, to świeży wynik Gilmera, Livingstona z~\cite{gilmer2016}.
\index[persons]{Gilmer, Patrick}%
\index[persons]{Livingston, Charles}%

Mówimy, że funkcja zdefiniowana na okręgu jest zbalansowana, jeżeli w każdym punkcie nieciągłości przyjmuje wartość równą średniej z~lewo- oraz prawostronnej granicy w tym punkcie.
Livingston podał pełną charakteryzację zbilansowanych sygnatur Levine'a-Tristrama dla węzłów, analogiczny problem dla splotów wydaje się być wciąż otwarty.

\begin{proposition}
\label{balanced_iff_four_conditions}%
    Funkcja zbalansowana $\sigma \colon S^1 \to \Z$ jest realizowana jako sygnatura pewnego węzła wtedy i tylko wtedy, gdy:
    \begin{enumerate}
        \item dla każdego $\omega \in S^1$ mamy $\sigma(\omega) = \sigma(\overline{\omega})$,
        \item $\sigma(1) = 0$,
        \item każdy punkt nieciągłości funkcji $\sigma$ jest miejscem zerowym wielomianu Alexandera węzła,
        \item jeżeli argumenty $\omega_1, \omega_2$ są sprzężone w sensie Galois, to $\sigma(\omega_1) \equiv \sigma(\omega_2)$ modulo $2$.
    \end{enumerate}
\end{proposition}

\begin{proof}
    Livingston pisze w \cite{livingston2018}, że dowód w prawą stronę jest dość dobrze znany, natomiast w lewo korzysta z~wyników Kondo \cite{kondo1979} i Sakaiego \cite{sakai1977}, że każdy wielomian Alexandera węzła jest realizowany przez węzeł 1-gordyjski oraz zachowania zbalansowanej sygnatury podczas odwracania skrzyżowania.
\end{proof}

Kawauchi \cite[s. 151]{kawauchi1996} wspomina nierówność $\unknotting K \ge \log_3 |Q(-1)|$, ale nie podaje żadnych przykładów.
Poprawimy ten stan rzeczy.
Węzeł $8_{18}$ jest 2-gordyjski i jego wielomian BLM/Ho wynosi
\begin{equation}
    Q (8_{18}; x) = 6x^7 + 24x^6 + 14x^5 - 36x^4 - 26x^3 + 12x^2 + 2x + 5,
\end{equation}
więc $Q (8_{18}, -1) = 9$ i żaden 1-gordyjski węzeł nie może mieć tego samego wielomianu BLM/Ho (ani Kauffmana) jak $8_{18}$.
(Dokładnie to samo rozumowanie działa dla $9_{35}$, $9_{37}$, $9_{46}$, $9_{47}$, $9_{48}$, $10_{74}$, $10_{75}$, $10_{98}$, $10_{99}$, 31 węzłów pierwszych o 11 skrzyżowaniach i 123 węzłów pierwszych o 12 skrzyżowaniach).
% TODO: tu? czy bliżej 1-gordyjskich?
% TODO: dopisać pythonowy program
% 8_18 9_35 9_37 9_46 9_47 9_48 10_74 10_75 10_98 10_99 11a_43 11a_44 11a_47 11a_57 11a_123 11a_135 11a_155 11a_173 11a_181 11a_231 11a_249 11a_263 11a_277 11a_291 11a_293 11a_314 11a_332 11a_352 11a_366 11n_71 11n_72 11n_73 11n_74 11n_75 11n_76 11n_77 11n_78 11n_81 11n_126 11n_164 11n_167 12a_119 12a_164 12a_166 12a_167 12a_177 12a_244 12a_245 12a_265 12a_270 12a_295 12a_297 12a_298 12a_311 12a_332 12a_386 12a_396 12a_413 12a_427 12a_433 12a_435 12a_493 12a_503 12a_554 12a_563 12a_569 12a_574 12a_576 12a_594 12a_615 12a_634 12a_647 12a_679 12a_683 12a_692 12a_701 12a_712 12a_725 12a_742 12a_750 12a_769 12a_787 12a_801 12a_810 12a_873 12a_886 12a_895 12a_905 12a_973 12a_987 12a_990 12a_1022 12a_1092 12a_1093 12a_1123 12a_1142 12a_1181 12a_1225 12a_1260 12a_1283 12a_1286 12a_1288 12n_268 12n_269 12n_270 12n_332 12n_333 12n_334 12n_379 12n_380 12n_386 12n_387 12n_388 12n_389 12n_402 12n_403 12n_420 12n_440 12n_460 12n_480 12n_494 12n_495 12n_496 12n_505 12n_508 12n_518 12n_546 12n_549 12n_553 12n_554 12n_555 12n_556 12n_565 12n_567 12n_570 12n_571 12n_574 12n_581 12n_582 12n_583 12n_598 12n_600 12n_601 12n_602 12n_604 12n_605 12n_622 12n_626 12n_636 12n_637 12n_642 12n_654 12n_666 12n_669 12n_701 12n_737 12n_756 12n_806 12n_813 12n_846 12n_869 12n_876 12n_883 12n_888

\begin{proposition}
\index{węzeł!satelitarny}%
    Niech węzeł $S$ będzie satelitą z towarzyszem $C$, wzorcem $P$ oraz indeksem zaczepenia $n$.
    Wtedy
    \begin{equation}
        \sigma_S(\omega) = \sigma_P(\omega) + \sigma_C(\omega^n).
    \end{equation}
\end{proposition}

\begin{proof}
    Szczególny przypadek $\omega = -1$ rozpatrywał wcześniej Shinohara \cite{shinohara1971}.
\index[persons]{Shinohara, Yaichi}%
    Pełny dowód znajduje się w artykule Litherlanda \cite{litherland1979}.
\index[persons]{Litherland, Richard}%
\end{proof}

Wreszcie:

\begin{proposition}
    Niech $L$ będzie splotem.
    Wtedy albo wielomian Alexandera $\Delta_L(t)$ jest zerem tożsamościowo, albo posiada co najmniej $|\sigma_L|$ zer, liczonych z krotnościami, na okręgu jednostkowym.
\end{proposition}

\begin{proof}
    Aneks w książce Liechtiego \cite{liechti2016}, która nie wygląda na związaną z~teorią węzłów.
\end{proof}

\index{sygnatura|)}%

% Koniec podsekcji Sygnatura

