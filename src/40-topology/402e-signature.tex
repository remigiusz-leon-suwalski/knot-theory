
\subsection{Sygnatura}
\index{sygnatura|(}%
Sygnatura jest kolejnym niezmiennikiem, do zdefiniowania których wystarczy znać macierz Seiferta.
Pochodzi prawdopodobnie z lat sześćdziesiątych (Trotter \cite{trotter62} dla węzłów, Murasugi \cite{murasugi65} dla splotów).
\index[persons]{Trotter, Hale}%
\index[persons]{Murasugi, Kunio}%
% z recenzji do 275415

% DICTIONARY;signature;sygnatura;-
\begin{definition}[sygnatura]
\label{def:signature}%
    Niech $M$ będzie macierzą Seiferta zorientowanego splotu $L$.
    Wielkość
    \begin{equation}
        \sigma_L := \operatorname{\sigma} (M + M^t),
    \end{equation}
    sygnaturę macierzy $M + M^t$, nazywamy sygnaturą splotu $L$.
\end{definition}

\begin{proposition}
\label{prp:signature_additive}%
    Sygnatura jest addytywna: $\sigma(K_1 \shrap \ldots \shrap K_n) = \sum_{k=1}^n \sigma(K_k)$.
\end{proposition}

Wiem o tym z \cite[s. 127]{murasugi96}.

\begin{proof}
    Bez straty ogólności ograniczmy się do przypadku $n = 2$ i~ustalmy powierzchnie Seiferta $F_1, F_2$ dla węzłów $K_1, K_2$ z~macierzami Seiferta $M_1, M_2$.
    Powierzchnia dla ich sumy spójnej $K_1 \shrap K_2$ powstaje przez sklejenie $F_1$ oraz $F_2$ paskiem.
    W języku macierzy oznacza to, że macierz Seiferta węzła $K_1 \shrap K_2$ ma postać $M = M_1 \oplus M_2$.
    Zatem:
    \begin{align}
        \sigma(K_1 \shrap K_2) & = \sigma(M + M^t) \\
                               & = \sigma(M_1 + M_1^t) + \sigma(M_2 + M_2^t) \\
                               & = \sigma(K_1) + \sigma(K_2),
    \end{align}
    co kończy dowód.
\end{proof}

\begin{corollary}
\index{liczba mostowa}%
\label{no_relation_signature_bridge}%
    Nie istnieje bezpośredni związek między sygnaturą i~liczbą mostową.
\end{corollary}

\begin{proof}
    Węzeł torusowy $T_{2,n}$ jest dwumostowy, jego sygnatura wynosi $n - 1$.
    Suma spójna węzłów prostych (sumy przeciwnie zorientowanych trójlistników) ma zerową sygnaturę, ale na mocy faktu \ref{prp:bridge_additive} jej liczba mostowa jest nieograniczona.
\end{proof}

\begin{proposition}
\index{lustro}%
\index{rewers}%
\label{prp:signature_mirror_reverse}%
    Niech $L$ będzie splotem.
    Wtedy $\sigma(mL) = -\sigma(L)$ oraz $\sigma(rL) = \sigma(L)$.
\end{proposition}

O tym także wiem z \cite[s. 127]{murasugi96}.

\begin{proof}
    Wynika to z podobnych faktów dla macierzy Seiferta.
    Równoważność $M_{mL} \simeq - M_L^t$ wynika z tego, że zamiana nad- i podskrzyżowań odwraca wzajemne położenie krzywych, których indeksu zaczepienia szukamy.

    Podobnie pokazuje się, że $M_{rL} \simeq M_L^t$.
\end{proof}

\begin{corollary}
\index{węzeł!achiralny}%
\label{cor:acheiral_signature}%
    Jeśli $K$ jest węzłem achiralnym, to $\sigma(K) = 0$.
\end{corollary}

Węzły achiralne mają zerową sygnaturę, zatem trójlistnik nie jest achiralny.
Z faktów \ref{prp:signature_additive} oraz \ref{prp:signature_mirror_reverse} wynika, że suma tak samo zorientowanych trójlistników nie jest achiralna ($\sigma = \pm 4$).
Jak można przekonać się ze standardowego diagramu węzła prostego, ten jest achiralny.
Pisaliśmy coś o tym na stronie \pageref{two_sums_of_two_trefoils}.

\begin{proposition}
\index{wielomian!Alexandera}%
    Niech $L$ będzie węzłem.
    Jeśli $\alexander_K(t) \equiv 1$, to $\sigma (K) = 0$.
\end{proposition}

Założenie $\alexander_K(t) \equiv 1$ jest spełnione przez cztery węzły pierwsze do 12 skrzyżowań, są to $11n34, 11n42, 12n313$ oraz $12n430$.

\begin{proof}
\index[persons]{Milnor, John}%
    Murasugi twierdzi, że zostało to udowodnione przez Milnora w \cite{milnor68}, nie jesteśmy jednak pewni, gdzie dokładnie, ale raczej poza sekcją piątą.
\end{proof}

Istnieje równoważna definicja, która nie wymaga czasochłonnego wyznaczania macierzy Seiferta.

\begin{proposition}
\index{relacja kłębiasta}%
    Sygnatura to niezmiennik topologiczny zadany kłębiastą relacją rekurencyjną:
    \begin{itemize}[leftmargin=*]
    \itemsep0em
        \item $\sigma (\SmallUnknot) = 0$,
        \item $\sigma (K_+) - \sigma (K_-) \in \{0, 2\}$,
        \item $4 \mid \sigma (K)$ wtedy i~tylko wtedy, gdy $\conway(2i) > 0$ (wielomian Conwaya).
    \end{itemize}
\end{proposition}

\begin{proof}
    Wystarczy pokazać, że sygnatura węzła spełnia trzy powyższe aksjomaty, a~następnie zauważyć, że korzystając z~nich jesteśmy w~stanie wyznaczyć jednoznacznie sygnaturę dla dowolnego węzła.
    Wynika to z~faktu, że każdy węzeł można zmienić w~niewęzeł odwracając pewne skrzyżowania.
    Pomysł opisał dokładnie Giller \cite[trzecie spostrzeżenie]{giller82}, sam oparł się o~\cite[twierdzenie 5.6]{murasugi65}.
\end{proof}

Sygnatura pozwala uzyskać proste oszacowanie liczby gordyjskiej od dołu:
\index{liczba gordyjska}%

\begin{proposition}
    Mamy $2 u(K) \ge |\sigma(K)|$.
\end{proposition}

Liczba gordyjska 87 z~801 węzłów pierwszych o mniej niż dwunastu skrzyżowaniach nie jest jeszcze znana.
Dla 311 spośród pozostałych mamy równość $2u = |\sigma|$.

\begin{proof}
    Ustalmy diagram $D$ dla węzła $K$.
    Odwrócenie dowolnego skrzyżowania polega na przejściu z~diagramu $D_+$ do $D_-$ lub z~$D_-$ do $D_+$.
    Zgodnie z relacją kłębiastą, sygnatura pozostaje taka sama lub zmienia wartość o $2$.
    Po wykonaniu $u$ odwróceń otrzymujemy diagram niewęzła o~sygnaturze zero, zatem sygnatura wyjściowego węzła nie mogła przekraczać $2u$.
    To kończy dowód.
\end{proof}

W~\cite{shinohara71} Shinohara pokazał, że dla każdej pary nieujemnych liczb całkowitych $m, n$ istnieje węzeł $K$ o wyznaczniku $4m+1$ ($8m+5$, $4m+3$) oraz sygnaturze bez znaku $8n$ ($8n+4$, $4n+2$).
\index{wyznacznik}%
Ponadto, jeśli $m$ nie dzieli się przez $3$, istnieje węzeł o wyznaczniku $8m+1$ i sygnaturze bez znaku $8n+4$.
% skąd to? Ohtsuki?

Czas na raczej niezbyt użyteczną ciekawostkę.

\begin{conjecture}
    Czy istnieje węzeł o~sygnaturze $4$ i~wyznaczniku postaci $n = 4k + 1$?
\end{conjecture}

Stojmenow twierdzi, że jeśli tak jest, to wszystkie pierwsze dzielniki $n$ dają resztę $1$ z~dzielenia przez $24$ i~są większe od $2857$.
Patrz \cite[s. 540]{ohtsuki02}.

Czytając przeglądową pracę Conwaya \cite{conway19} dowiedzieliśmy się, że jeszcze w~latach sześćdziesiątych sygnatura została uogólniona do funkcji $\sigma_L \colon S^1 \to \Z$.
\index[persons]{Conway, John}%
Większość podręczników, a także prace Levine'a \cite{levine69} oraz Tristrama \cite{tristram69}, wprowadza ją przy użyciu macierzy Seiferta, więc my postąpimy dokładnie tak samo.
\index[persons]{Levine, Jerome}%
\index[persons]{Tristram, Andrew}%

\begin{definition}[sygnatura Levine'a-Tristrama]
\index{sygnatura!Levine'a-Tristrama}%
    Niech $M$ będzie macierzą Seiferta zorientowanego splotu $L$.
    Funkcję $\sigma_L \colon S^1 \to \Z$ daną wzorem
    \begin{equation}
        \sigma_L(\omega) := \operatorname{\sigma} [(1-\omega) M + (1 - \overline{\omega})M^t]
    \end{equation}
    nazywamy sygnaturą Levine'a-Tristrama splotu $L$.
    Jest niezmiennikiem splotów.
\end{definition}

Funkcja $\sigma_L$ jest kawałkami stała.
Conway pisze w \cite{conway19}, że wynika to ze wzoru na wielomian Alexandera $\Delta_L(t) = \det(tM - M^t)$.
\index[persons]{Conway, John}%
\index{wielomian!Alexandera}%
Jedynymi punktami nieciągłości są zera wielomianu $(t-1)\Delta_L(t)$, to świeży wynik Gilmera, Livingstona z~\cite{gilmer16}.
\index[persons]{Gilmer, Patrick}%
\index[persons]{Livingston, Charles}%

Mówimy, że funkcja zdefiniowana na okręgu jest zbalansowana, jeżeli w każdym punkcie nieciągłości przyjmuje wartość równą średniej z~lewo- oraz prawostronnej granicy w tym punkcie.
Livingston podał pełną charakteryzację zbilansowanych sygnatur Levine'a-Tristrama dla węzłów, analogiczny problem dla splotów wydaje się być wciąż otwarty.

\begin{proposition}
\label{balanced_iff_four_conditions}%
    Funkcja zbalansowana $\sigma \colon S^1 \to \Z$ jest realizowana jako sygnatura pewnego węzła wtedy i tylko wtedy, gdy:
    \begin{enumerate}
        \item dla każdego $\omega \in S^1$ mamy $\sigma(\omega) = \sigma(\overline{\omega})$
        \item $\sigma(1) = 0$
        \item każda nieciągłość funkcji $\sigma$ jest miejscem zerowym wielomianu Alexandera węzła
        \item jeżeli argumenty $\omega_1, \omega_2$ są sprzężone w sensie Galois, to $\sigma(\omega_1) \equiv \sigma(\omega_2)$ modulo $2$.
    \end{enumerate}
\end{proposition}

\begin{proof}
    Livingston pisze w \cite{livingston18}, że dowód w prawą stronę jest dość dobrze znany, natomiast w lewo korzysta z~wyników Kondo \cite{kondo79} i Sakaiego \cite{sakai77}, że każdy wielomian Alexandera węzła jest realizowany przez węzeł 1-gordyjski oraz zachowania zbalansowanej sygnatury podczas odwracania skrzyżowania.
\end{proof}

(Nie każdy wielomian Kauffmana/HOMFLY jest realizowany przez węzły 1-gordyjskie, Kawauchi \cite[s. 151]{kawauchi96} wspomina na przykład, że $\unknotting K \ge \log_3 |Q(-1)|$.)

\begin{proposition}
\index{węzeł!satelitarny}%
    Niech $S$ będzie satelitą z towarzyszem $C$, wzorcem $P$ oraz indeksem zaczepenia $n$.
    Wtedy
    \begin{equation}
        \sigma_S(\omega) = \sigma_P(\omega) + \sigma_C(\omega^n).
    \end{equation}
\end{proposition}

\begin{proof}
    Szczególny przypadek $\omega = -1$ rozpatrywał wcześniej Shinohara w~\cite{shinohara71}.
    Pełny dowód znajduje się w artykule \cite{litherland79} Litherlanda.
\end{proof}

Wreszcie:

\begin{proposition}
    Niech $L$ będzie splotem.
    Wtedy albo wielomian Alexandera $\Delta_L(t)$ jest tożsamościowo zerem, albo posiada co najmniej $|\sigma_L|$ zer, liczonych z krotnościami, na okręgu jednostkowym.
\end{proposition}

\begin{proof}
    Aneks w książce Liechtiego \cite{liechti16}, która nie wygląda na związaną z~teorią węzłów.
\end{proof}

\index{sygnatura|)}

% Koniec podsekcji Sygnatura

