\section{Wraki i kwandle} % (fold)
\label{sec:quandle}
Sekcja ta powstała częściowo w oparciu o notatki autorstwa Andrew Bergera, Chrisa Geriga
\footnote{dostępne pod adresem \url{https://math.berkeley.edu/~cgerig/notes}}
oraz Andrew Bergera, Brandona Flannery'ego i Chrisa Sumnichta
\footnote{dostępne pod adresem \url{https://github.com/thyrgle/191_Final_Project/blob/master/paper.pdf}}.
Kwandle, z angielskiego \emph{quandle}, są strukturami algebraicznymi przypominającymi grupy,
jednak których aksjomaty nie odzwierciedlają własności symetrii, tylko ruchów Reidemeistera.

\begin{definition}[wrak]
    Zbiór $R$ z binarnym działaniem $\triangleleft$, które spełnia relację
    \[
        a \triangleleft (b \triangleleft c) = (a \triangleleft b) \triangleleft (a \triangleleft c)
    \]
    dla wszystkich $a, b, c \in R$, nazywamy wrakiem,
    jeśli każdej parze elementów $a, b \in R$ odpowiada dokładnie jeden $c \in R$ taki,
    że $a \triangleleft c = b$. \index{Wrak}
\end{definition}

Ta definicja, chociaż zwięzła i powszechnie używana, nie jest optymalna.
Pozbędziemy się kwantyfikatora egzystencjalnego za cenę wprowadzenia drugiego działania.

Jedyny $c \in R$ dla pary $(a,b)$ oznaczmy przez $b \triangleright a$.
To oznacza, że $a \triangleleft c = b$, wtedy i tylko wtedy gdy $c = b \triangleright a$,
a zatem $a \triangleleft (b \triangleright a) = b$ oraz $(a \triangleleft b) \triangleright a = b$.
Korzystając z tego pomysłu możemy podać alternatywną definicję.

\begin{definition}[wrak]
    Zbiór $R$ z dwoma binarnymi działaniami,
    $\triangleleft$ oraz $\triangleright$,
    które spełniają następujące aksjomaty:
    \begin{enumerate}
        \item $a \triangleleft (b \triangleleft c) = (a \triangleleft b) \triangleleft (a \triangleleft c)$
        oraz $(c \triangleright b) \triangleright a = (c \triangleright a) \triangleright (b \triangleright a)$
        \item $(a \triangleleft b) \triangleright a = b$ oraz $a \triangleleft (b \triangleright a) = b$
    \end{enumerate}
    nazywamy wrakiem.
\end{definition}

\begin{definition}[kwandel]
    Wrak $Q$, którego każdy element $a$ spełnia relację $a \triangleleft a = a$, nazywamy kwandlem. \index{Kwandel}
\end{definition}

Wiele znanych z algebry obiektów można zmienić w kwandle.
\begin{itemize}
\item grupa abelowa $G$: %\todo[inline]{z działaniem $x \triangleright y = 2y - x$}
\item grupa $G$: %\todo[inline]{z działaniem  $x \triangleright y = y^{-n}xy^n$ ($n \in \Z$ dowolne)}
\item Moduł nad pierścieniem $\Z[t^{\pm 1}]$ wielomianów Laurenta: %\todo[inline]{$x \triangleright y = tx + (1-t)y$}
\item Przestrzeń liniowa z dwuliniową formą antysymetryczną: %\todo[inline]{$x \triangleright y = x + \langle x \mid y \rangle \cdot y$}
\end{itemize}

%Pokażemy, że kwandle uogólniają kolorowania.
%Niech $X$ będzie zbiorem kolorów z operacją $\triangleright$, które spełnia aksjomaty z definicji kwandli.
%Wtedy przy każdym skrzyżowaniu występują trzy kolory: $x$, $y$ oraz $x \triangleright y$.

\begin{tikzpicture}[scale=0.18, baseline=0]
    \path[TIKZ_ARCH,Latex-] (-4,0) -- (4,8);
    \path[TIKZ_ARCH] (4,0) -- (1,3);
    \path[TIKZ_ARCH] (-1,5) -- (-4,8);
    \node[darkblue] at (-4, 0)[above left] {$y$};
    \node[darkblue] at (4, 0)[above right] {$x \triangleright y$};
    \node[darkblue] at (-4, 8)[below left] {$x$};
\end{tikzpicture}

%Przypomnijmy, że 3-kolorowanie diagramu polegało na przypisaniu każdemu włóknu pewnego koloru (z trzech) tak, by każdy został użyty, a żadne skrzyżowanie nie stało się dwubarwne.
%Ogólniej, jeśli kolorami były liczby $0, \ldots, n - 1$, żądaliśmy od skrzyżowań, by kolor $y$ po przejściu pod kolorem $x$ stawał się $z$, gdzie $z \equiv 2x - y$ modulo $n$.
%Można to uogólnić jeszcze bardziej, właśnie do quandli: $\Z/n$-kolorowanie węzła to quandle związany z pierścieniem $\Z/n$ operacją $x \triangleright y =  2y - x$}

Z każdym węzłem związany jest specjalny kwandel.

\begin{definition}
%Wspomnieć o związku z prezentacją Wirtingera (i zajrzeć do hiszpańskiego skryptu).
%   Niech $K \subseteq S^3$ będzie węzłem, natomiast $X_K = S^3 \setminus \operatorname{int} N(K)$ dopełnieniem jego regularnego otoczenia.
%   Wybierzmy $x_0 \in \partial X_K$.
%   Zdefiniujmy $Q(K) = \pi_1(C(K), \partial C(K), x_0)$ z dobrze określonym działaniem
%   \[
%       [\gamma_1] \triangleright [\gamma_2] = [\gamma_2 \circ m \circ \gamma_2^{-1} \circ \gamma_1],
%   \]
%   gdzie $m$ to południk wzdłuż $\gamma_2(1)$.
    Zbiór $Q(K)$ nazywamy kwandlem fundamentalnym (podstawowym?).
\end{definition}

Muszę w tym miejscu wtrącić uwagę językową.
Conway nazwał wraki wrakami (\emph{wracks}), by częściowo zażartować z nazwiska jego kolegi Gavina Wraitha,
a częściowo by zaznaczyć, że są one tym, co zostaje z grupy, w której zapomniano o mnożeniu, ale nie sprzęganiu.
Obecnie w języku angielskim dominuje określenie \emph{racks}.

% Historia:
% \begin{itemize}
    %\item 1943 Takasaki: keis
    %\item 1959 Conway, Wraith: racks
    %\item 1980s Joyce, niezależnie. Matveev: niekoniecznie skończone kwandle fundamentalne, zainspirowani (przynajmniej Joyce) przez prezentację Wirtingera.
%\end{itemize}
% Koniec sekcji Wraki i kwandle
