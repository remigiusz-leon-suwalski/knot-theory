
\section{Etykietowanie}
%\index{etykietowanie|(}%

Etykietowalność to niezmiennik, który uogólnia kolorowalność.
Poznaliśmy go dzięki książce Livingstona \cite[s. 89-99]{livingston1993}, ale poza panem Charlesem mało kto o nim wspomina!

\begin{definition}[etykietowanie]
    Niech $G$ będzie grupą generowaną przez elementy $g_1, \ldots, g_n$, zaś  $D$ będzie diagramem zorientowanego węzła $K$.
    Surjekcję ze zbioru włókien diagramu $D$ na zbiór generatorów grupy $G$ taką, że przy każdym skrzyżowaniu z oznaczeniami jak na rysunku (z~włóknem $g_U$ biegnącym górą oraz włóknami $g_L$, $g_R$ po lewej i prawej stronie podskrzyżowania) spełnione jest równanie $g_Ug_L = g_Rg_U$, nazywamy etykietowaniem diagramu $D$ grupą $G$.
\begin{comment}
        \[
            \LargePlusCrossingLabel
        \]
\end{comment}
    Węzeł, który posiada diagram etykietowalny grupą $G$ też nazywamy etykietowalnym grupą $G$.
\end{definition}

Równanie $g_R = g_Ug_Lg_U^{-1}$ mówi, że etykiety włókien wchodzących oraz wychodzących są sprzężone.
Wynika stąd, że wszystkie etykiety pochodzą z~jednej klasy sprzężoności.
Muszą jednocześnie generować całą grupę, dlatego $G$ musi być grupą nieprzemienną lub trywialną.
Etykietowalność jest niezmiennikiem węzłów i~nie zależy od orientacji węzła: jeżeli elementy $g_1, \ldots, g_n$ generują grupę, to ich odwrotności także.

Etykietowania są naturalnym uogólnieniem kolorowań.
Dlaczego?
% Pokazuję i objaśniam - Małgorzata K.
Niech $p \ge 3$ będzie liczbą pierwszą, natomiast $D_p = \langle r, s \mid r^p = s^2 = e, rsr = s \rangle$ grupą diedralną rzędu $2p$.
Elementy tej grupy to $1, r, r^2, \ldots, r^{p-1}, s, sr, \ldots, sr^{p-1}$.
,,Obrót'' $r^k$ jest sprzężony tylko ze swoją odwrotnością, ale ,,symetrie osiowe'' $sr^k$ tworzą jedną klasę sprzężoności.
Łatwo widać, że dowolne dwie z~nich generują całą grupę $D_p$.

\begin{proposition}
    Węzeł $K$ jest $p$-kolorowalny wtedy i~tylko wtedy, gdy jest $D_p$-etykietowalny.
\end{proposition}

\begin{proof}
    Załóżmy, że $K$ ma $n$ włókien.
    Wiemy już, że każde $D_p$-etykietowanie wykorzystuje tylko elementy $sr^{a_1}, \ldots, sr^{a_n}$ dla $1 \le a_i \le p$.
    Jest ono prawidłowe dokładnie wtedy, gdy analogiczne kolorowanie liczbami $a_1, \ldots, a_n$ jest prawidłowe.
\end{proof}

Pozostało wskazać parę węzłów, których nie odróżniają kolorowania, ale odróżnia pewne etykietowanie.
Taką parę stanowią na przykład węzły $6_1$ oraz $9_{46}$.
Spróbujmy etykietować je transpozycjami z~grupy $S_4$.
Wybranie dwóch etykiet przy jednym skrzyżowaniu $6_1$ wymusza etykiety dla wszystkich włókien.
Dwie transpozycje nie mogą generować grupy $S_4$, natomiast włókna węzła $9_{46}$ dają się etykietować samymi transpozycjami.
Węzły te są więc różne, choć mają te same własności kolorujące (ponieważ mają ten sam wyznacznik, ponieważ mają ten sam wielomian Alexandera).
Dużo później odkryliśmy tę parę jeszcze raz w~\cite[s. 138]{burde2014}, gdzie pokazano, że mają różne ideały elementarne: $E_2(6_1, t) = \Z(t)$; $E_2(9_{46}, t) = (t-2, 2t-1)$.
Niestety nie wiemy, kto pierwszy wpadł na ten pomysł.
\index{wielomian Alexandera}%

Etykietowanie jest mocnym narzędziem odróżniającym węzły.
Thistlethwaite w 1985 roku korzystając z niego klasyfikował węzły o~co najwyżej 13 skrzyżowaniach (jest ich, jak ostatecznie się okazało, 12965).
Mają one tylko 5639 różnych wielomianów Alexandera, ale etykietowania trzynastoma różnymi grupami pozwoliły zmniejszyć liczbę nierozpoznanych węzłów do około tysiąca.
Wśród nich 30 posiada wielomian Conwaya $1 + 2z^2 + 2z^4$, ale pary rozróżniane wielomianem HOMFLY mają też różne wielomiany Jonesa.
Wielomiany opisujemy w~rozdziale trzecim.

Kolorowania definiowano kiedyś jako surjekcje $\rho \colon \pi \to D_{2n}$ z~grupy podstawowej.
Jak mówi prezentacja Wirtingera, grupa splotu generowana jest przez ścieżki z~punktu bazowego w~$S^3$ do brzegu rurowego otoczenia splotu, wokół południka i~znowu do bazowego punktu.
\index{prezentacja Wirtingera}%
Fox zauważył, że z~surjektywności $\rho$ wynika, iż generatory mapują się na symetrie osiowe $sr^k$.
Ponieważ istnieje wzajemnie jednoznaczna odpowiedniość między generatorami grupy splotu oraz łukami diagramu, każdemu możemy przypisać liczbę całkowitą $k$.
Etykietowania są więc uogólnieniem kolorowań.
Rozumowanie, które przedstawiliśmy, prowadzi do prostej klasyfikacji grup, których można użyć do etykietowania.

\begin{proposition}
    Niech $K$ będzie węzłem, $\pi$ grupą podstawową jego dopełnienia, zaś $G$ dowolną grupą.
    Następujące warunki są równoważne: $K$ jest $G$-etykietowalny; istnieje surjekcja $\pi_1 \to G$.
\end{proposition}

Historycznie, prezentacja Wirtingera była pierwsza, zaś etykietowania odkryto później.
% TODO: ustalić, kiedy później

\begin{proposition}[Perko]
    Niech $K$ będzie węzłem.
    Następujące warunki są równoważne: $K$ jest etykietowalny grupą $S_3$, $K$ jest etykietowalny grupą $S_4$.
\end{proposition}
% To jest w https://faculty.etsu.edu/gardnerr/Knot-Theory/Notes-Livingston/Livingston-Knot-5-2.pdf, ale nie książce Livingston a livingston1993

\begin{proof}
    Dowód zajmuje tylko kilka stron i składa się z dwóch części.
    Po pierwsze $S_4$ ma podgrupę czwórkową Kleina, więc istnieje epimorfizm $S_4 \to S_3$; pozwala to opuszczać epimorfizmy $G \to S_4$ do epimorfizmów $G \to S_3$.
    Po drugie, Perko \cite{perko1975} pokazuje jak podnosić epimorfizmy $G \to S_3$ do epimorfizmów $G \to S_4$.
\end{proof}

Nie znamy innych nietrywialnych faktów dotyczących etykietowań.

\index{etykietowanie|)}%

