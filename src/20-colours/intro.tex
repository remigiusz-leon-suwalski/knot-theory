Zamknęliśmy już pierwszy rozdział,
Wciąż nie pokazaliśmy jednak pełnego dowodu, że dwa konkretne węzły (na przykład niewęzeł i~trójlistnik) są od siebie różne.
Dlatego teraz podamy proste narzędzie odróżniające węzły: \emph{trójkolorowalność}, która przypisuje włóknom diagramu różne kolory.
Następnie rozszerzymy paletę do dowolnie wielu kolorów i~zastąpimy ją grupą skończoną.

Nawet ten wzmocniony wariant nie jest idealnym narzędziem klasyfikującym.
Istnieją węzły, których nie odróżnia.
Problem ten dotyka wielu późniejszych niezmienników, dość ważna hipoteza Jonesa \ref{jones_conjecture} pyta, czy wielomian Jonesa jest zupełny (tak nazywamy niezmienniki, które różnym węzłom przypisują różne wartości).
Pierwszy niezmiennik zupełny poznamy w~rozdziale czwartym.
Nie stanowi to wielkiego powodu do radości ze względu na wysoką złożoność obliczeniową.

\section{Kolorowanie splotów} % (fold)
\label{sec:colour_links}
Oto mniej mętny opis trójkolorowalności, czyli jak nietrudno się domyślić, kolorowalności trzema kolorami.
Diagram $D$ splotu $K$ jest trójkolorowalny, jeśli każdemu włóknu można przypisać jeden z~trzech kolorów tak, by co najmniej dwa zostały użyte.
Wymagamy przy tym, by przy żadnym skrzyżowaniu nie spotykały się dokładnie dwa kolory.

Dla własnej wygody jako kolorów używać będziemy kolejnych liczb naturalnych $0, 1, 2, \ldots$.
Pozwala to zapisać warunek kolorowalności równaniem algebraicznym.

\begin{definition}[kolorowanie] \label{def:colour_equation}
    Niech $L$ będzie splotem, zaś $n$ liczbą naturalną.
    Mówimy, że splot $L$ jest kolorowalny modulo $n$, jeśli posiada diagram, którego włóknom można przypisać liczby całkowite $0, \ldots, n - 1$ tak, by równanie $a + b \equiv 2c$ modulo $n$ było spełnione przy każdym skrzyżowaniu oraz istniały dwa włókna różnych kolorów.
    Takie przyporządkowanie nazywamy (nietrywialnym) kolorowaniem.
    \[
        \begin{tikzpicture}[baseline=-0.65ex, scale=0.12]
            \useasboundingbox (-5, -5) rectangle (5,5);
            \begin{knot}[clip width=5, end tolerance=1pt, flip crossing/.list={1}]
                \strand[semithick] (-5,5) to (5,-5);
                \strand[semithick] (-5,-5) to (5,5);
                \node[darkblue] at (5, 5)[below right] {$c$};
                \node[darkblue] at (5, -5)[above right] {$b$};
                \node[darkblue] at (-5, 5)[below left] {$a$};
            \end{knot}
        \end{tikzpicture}
    \]
\end{definition}

Metoda ta została odkryta razem z~uogólnieniem do $n$ kolorów przez Ralpha Foxa w~1956, kiedy próbował uczynić teorię węzłów bardziej przystępną\footnote{
    Niech $L$ będzie splotem, zaś $\pi$ grupą podstawową jego dopełnienia.
    Reprezentację $\rho$ dla $\pi$ na $D_{2n}$, grupę diedralną, nazywamy $n$-kolorowaniem Foxa.
    Grupa splotu generowana jest przez ścieżki z~punktu bazowego w~$S^3$ do brzegu rurowego otoczenia splotu, wokół południka i~znowu do bazowego punktu.
    Obrazami tych generatorów są izometrie $n$-kąta foremnego, które zapisujemy jako $ts^i$ ($t$ to odbicie, zaś $s$ jest najmniejszym obrotem).
    Każdy generator odpowiada pewnemu łukowi, przypiszmy mu liczbę $i \in \Z/n\Z$.
    Dostaliśmy zwykłe kolorowanie, przystępne dla studentów.
}
dla studentów.
Opierając się tylko na definicji kolorowania oraz ruchach Reidemeistera jesteśmy w~stanie wykazać pierwsze własności niezmiennika.

\begin{proposition}
    Żaden węzeł nie koloruje się modulo dwa.
\end{proposition}

\begin{proof}
    Załóżmy nie wprost, że istnieje nietrywialne kolorowanie.
    Analiza czterech możliwych skrzyżowań pokazuje,
    że włókna wychodzące z~tunelu muszą mieć ten sam kolor.
    Przechodząc wzdłuż węzła widzimy jeden kolor, wbrew założeniu nie wprost.
\end{proof}

Sploty o~wielu składowych są dwukolorowalne.
Wystarczy pomalować jedną składową zerem, a~pozostałe jedynkami.
Sploty rozszczepialne są $n$-kolorowalne dla każdego $n \ge 2$, można skorzystać z~tego samego schematu kolorowania.

\begin{proposition} \label{color_invariant}
    ,,Bycie $n$-kolorowalnym'' jest niezmiennikiem węzłów.
\end{proposition}

\begin{proof}
    Wystarczy sprawdzić, jak ruchy Reidemeistera zmieniają kolory.
    Pierwszy i~drugi:
    \[
        \fbox{
        \begin{tikzpicture}[baseline=-0.65ex,scale=0.07]
        \begin{knot}[clip width=5]
            \strand[semithick] (-10,10) .. controls (-10,2) and (-10,2) .. (-6,-2);
            \strand[semithick] (-10,-10) .. controls (-10,-2) and (-10,-1) .. (-9,0);

            \strand[semithick] (-7,1) -- (-6,2);
            \strand[semithick] (-6,2) .. controls (2,9) and (2,-9) .. (-6,-2);
            \node[darkblue] at (-10, 10)[below left] {$a$};
            \node[darkblue] at (-10, -10)[above left] {$b \equiv a$};
        \end{knot}
        \end{tikzpicture}
        $\stackrel{R_1}{\cong} \,\,$
        \begin{tikzpicture}[baseline=-0.65ex,scale=0.07]
        \begin{knot}[clip width=5]
            \strand[semithick] (0,10) -- (0,-10);
            \node[darkblue] at (0, 0)[left] {$a$};
        \end{knot}
        \end{tikzpicture}}
        %%%
        \quad \fbox{
        \begin{tikzpicture}[baseline=-0.65ex,scale=0.07]
        \begin{knot}[clip width=5]
            \strand[semithick] (4,-10) .. controls (4,-4) and (-4,-4) .. (-4,0);
            \node[darkblue] at (-4, -10)[above left] {$d \equiv b$};
            \strand[semithick] (4,10) .. controls (4, 4) and (-4, 4) .. (-4,0);
            \node[darkblue] at (4, 10)[below right] {$a$};
            \strand[semithick] (-4,-10) .. controls (-4,-4) and (4,-4) .. (4,0);
            \node[darkblue] at (4, 0) [right] {$c \equiv 2a-b$};
            \strand[semithick] (-4, 10) .. controls (-4, 4) and (4,4) .. (4,0);
            \node[darkblue] at (-4, 10) [below left] {$b$};
        \end{knot}
        \end{tikzpicture}
        $\stackrel{R_2}{\cong} \,\,$
        \begin{tikzpicture}[baseline=-0.65ex,scale=0.07]
        \begin{knot}[clip width=5]
            \strand[semithick] (4,-10) .. controls (4,-4) and (1,-4) .. (1,0);
            \strand[semithick] (4,10) .. controls (4, 4) and (1, 4) .. (1,0);
            \strand[semithick] (-4,-10) .. controls (-4,-4) and (-1,-4) .. (-1,0);
            \strand[semithick] (-4,10) .. controls (-4, 4) and (-1,4) .. (-1,0);
        \end{knot}
        \end{tikzpicture}}
    \]
    Trzeci ruch także nie wymaga skomplikowanych rachunków.
    Najkrótszy łuk na diagramach ma kolor $2a-c$ po lewej oraz $2b-c$ po prawej stronie.
    \[
     \fbox{
        \begin{tikzpicture}[baseline=-0.65ex,scale=0.07]
        \begin{knot}[clip width=5, flip crossing/.list={1,2,3}]
            \node[darkblue] at (-10, 10) [above] {$b$};
            \node[darkblue] at (10, 10) [above] {$c$};
            \node[darkblue] at (-10, -10) [below] {$2a-2b+c$};
            \node[darkblue] at (10, -10) [below] {$2a-b$};
            \node[darkblue] at (-10, -2) [left] {$a$};
            \strand[semithick] (-10,-10) -- (10,10);
            \strand[semithick] (-10,10) -- (10,-10);
            \strand[semithick] (-10,-2) .. controls (-4, -2) and (-4,8) .. (0,8);
            \strand[semithick] (10,-2) .. controls (4, -2) and (4,8) .. (0,8);
        \end{knot}
        \end{tikzpicture}
        $\stackrel{R_3}{\cong} \,\,$
        \begin{tikzpicture}[baseline=-0.65ex,scale=0.07]
        \begin{knot}[clip width=5, flip crossing/.list={1,2,3}]
            \node[darkblue] at (-10, 10) [above] {$b$};
            \node[darkblue] at (10, 10) [above] {$c$};
            \node[darkblue] at (-10, -10) [below] {$2a-2b+c$};
            \node[darkblue] at (10, -10) [below] {$2a-b$};
            \node[darkblue] at (10, 2) [right] {$a$};
            \strand[semithick] (-10,-10) -- (10,10);
            \strand[semithick] (-10,10) -- (10,-10);
            \strand[semithick] (-10,2) .. controls (-4, 2) and (-4,-8) .. (0,-8);
            \strand[semithick] (10,2) .. controls (4, 2) and (4,-8) .. (0,-8);
        \end{knot}
        \end{tikzpicture}}
    \]
\end{proof}

Trójlistnik koloruje się dokładnie modulo krotności trójki, ósemka zaś -- piątki.
Pierścienie Boromeuszy nie kolorują się modulo trzy, nie są zatem rozszczepialne.
Sama kolorowalność nie mówi wiele, splot jest kolorowalny lub nie.
Dowód faktu \ref{color_invariant} pokazuje coś więcej: liczba kolorowań, być może trywialnych, także jest niezmiennikiem (mocniejszym).
Węzły, których żadne kolorowanie nie odróżnia od niewęzła, nazywa się czasem niewidzialnymi.

Niech $\tau_3(K)$ oznacza liczbę trójkolorowań węzła $K$.

\begin{proposition}
    Jeśli $K, L$ są węzłami, to $3\tau_3(K \shrap L) = \tau_3(K)\tau_3(L)$.
\end{proposition}

\begin{corollary}
    Istnieje nieskończenie wiele węzłów.
\end{corollary}

\begin{proof}
    Suma spójna $n$ trójlistników ma $3^n$ (być może trywialnych) $3$-kolorowań.
\end{proof}

\begin{proposition}
    Niech $n$ będzie liczbą pierwszą.
    Węzeł $p, q, r$-preclowy jest $n$-kolorowalny wtedy i~tylko wtedy, gdy $n$ dzieli $|pq+qr+pr|$.
    Jeśli przynajmniej jedna z~liczba $p, q, r$ nie jest wielokrotnością $n$, kolorowanie jest jedyne.
    W przeciwnym przypadku istnieją cztery kolorowania (z dokładnością do permutacji!).
\end{proposition}

\begin{proof}
    Dowód zawiera praca \emph{Counting $m$-coloring classes of knots and links} autorstwa trzech amerykanek (K. Brownell, K. O'Neil oraz L. Taalman).
    Podano tam także ogólny wzór na liczbę $n$-kolorowań dowolnego węzła.
\end{proof}

Jako kolorów użyjemy teraz elementów $g_1, \ldots, g_n$ pewnej skończonej grupy $G$.

\begin{definition}[etykietowanie]
    Mówimy, że zorientowany węzeł $K$ jest etykietowalny grupą $G$ generowaną przez elementy $g_1, \ldots, g_n$, jeśli posiada diagram, którego włóknom przypisano elementy $g_1, \ldots, g_n$ tak, by równanie $gk=hg$ było spełnione przy każdym skrzyżowaniu ($g$: włókno biegnące górą, $k$: bo jego lewej stronie, $h$: po prawej).
    \[
        \begin{tikzpicture}[baseline=-0.65ex, scale=0.12]
            \useasboundingbox (-5, -5) rectangle (5,5);
            \begin{knot}[clip width=5, end tolerance=1pt, flip crossing/.list={1}]
                \strand[semithick] (-5,5) to (5,-5);
                \strand[semithick, -Latex] (-5,-5) to (5,5);
                \node[darkblue] at (5, 5)[below right] {$g$};
                \node[darkblue] at (5, -5)[above right] {$h$};
                \node[darkblue] at (-5, 5)[below left] {$k$};
            \end{knot}
        \end{tikzpicture}
    \]
\end{definition}

Równanie $gkg^{-1}=h$ mówi, że etykiety włókien wchodzących oraz wychodzących są sprzężone.
Wynika stąd, że wszystkie etykiety pochodzą z~jednej klasy sprzężoności.
Muszą jednocześnie generować całą grupę, dlatego $G$ musi być grupą nieprzemienną lub trywialną.
Etykietowalność jest niezmiennikiem węzłów i~nie zależy od orientacji węzła:
jeżeli elementy $g_1, \ldots, g_n$ generują grupę, to ich odwrotności także.

Rozpatrzmy węzły $6_1$ oraz $9_{46}$ i~spróbujmy etykietować je transpozycjami z~grupy $S_4$.
Wybranie dwóch etykiet przy jednym skrzyżowaniu $6_1$ wymusza etykiety dla wszystkich włókien.
Dwie transpozycje nie mogą generować grupy $S_4$, natomiast włókna węzła $9_{46}$ dają się etykietować samymi transpozycjami.
Węzły te są więc różne.
Nie odróżniają ich ani kolorowania, ani wielomian Alexandera.

Livingston pisze w~swojej książce,\footnote{Rozdział 5.2: Węzły i~grupy} że Thistlethwaite tworząc tablice węzłów pierwszych znalazł 12965 węzłów o~13 skrzyżowaniach o~5639 różnych wielomianach Alexandera.
Używając etykietowań podgrupami grupy $S_5$ zmniejszył liczbę nierozwiązanych przypadków do mniej niż tysiąca.
Perko pokazał zaś, że każdy $S_3$-kolorowalny węzeł jest $S_4$-kolorowalny.

Niech $p \ge 3$ będzie liczbą pierwszą, natomiast $D_p = \langle r, s \mid r^p = s^2 = e, rsr = s \rangle$ grupą diedralną rzędu $2p$.
Elementy tej grupy to $1, r, r^2, \ldots, r^{p-1}, s, sr, \ldots, sr^{p-1}$.
,,Obrót'' $r^k$ jest sprzężony tylko ze swoją odwrotnością, ale ,,symetrie osiowe'' $sr^k$ tworzą jedną klasę sprzężoności.
Łatwo widać, że dowolne dwie z~nich generują całą grupę $D_p$.

\begin{proposition}
    Węzeł $K$ jest $p$-kolorowalny wtedy i~tylko wtedy, gdy jest $D_p$-etykietowalny.
\end{proposition}

\begin{proof}
    Załóżmy, że $K$ ma $n$ włókien.
    Wiemy już, że każde $D_p$-etykietowanie wykorzystuje tylko elementy $sr^{a_1}, \ldots, sr^{a_n}$ dla $1 \le a_i \le p$.
    Jest ono prawidłowe dokładnie wtedy, gdy analogiczne kolorowanie liczbami $a_1, \ldots, a_n$ jest prawidłowe.
\end{proof}

Etykietowania są więc uogólnieniem kolorowań.
Istnieje prosta klasyfikacja grup, których można użyć do etykietowania.

\begin{proposition}
    Niech $K$ będzie węzłem, $\pi_1$ grupą podstawową jego dopełnienia, zaś $G$ dowolną grupą.
    Następujące warunki są równoważne: $K$ jest $G$-etykietowalny; istnieje surjekcja $\pi_1 \to G$.
\end{proposition}



% Koniec sekcji Kolorowanie splotów
