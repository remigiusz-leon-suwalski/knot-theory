
\section{Grupa kolorująca}
\index{grupa!kolorująca|(}%

\begin{definition}
\label{def:colouring_group}%
    Niech $L$ będzie splotem.
    Abelową grupę generowaną przez łuki diagramu przedstawiająceo splot $L$ z równaniami skrzyżowań tegoż diagramu jako relacjami oraz dodatkową relacją ,,$a = 0$'', gdzie $a$ jest dowolnie ustalonym łukiem, nazywamy grupą kolorującą splot $L$ i oznaczamy przez $\ColoringGroup (L)$.
\end{definition}

\begin{proposition}
\label{prp:colourings_are_morphisms}%
    Splot $L$ koloruje się modulo $n$, wtedy i~tylko wtedy gdy istnieje nietrywialny homomorfizm $\ColoringGroup(L) \to \Z/n\Z$.
\end{proposition}

\begin{proof}
    Niech $a$ będzie ustalonym łukiem na diagramie splotu $L$ z definicji \ref{def:colouring_group}.
    Funkcja $\varphi \colon \ColoringGroup(L) \to \Z/n\Z$ jest nietrywialnym morfizmem, jeśli przyjmuje choć raz wartość różną od zera, spełnia warunek $\varphi(a) = 0$ i~dla każdego skrzyżowania prawdziwe jest równanie
    \begin{equation}
        \varphi(x_j) + \varphi(x_k) - 2\varphi(x_i) = 0
    \end{equation}
    przy oznaczeniach z definicji \ref{def:colouring_equation}.
    To pokazuje, że niezerowe morfizmy $\ColoringGroup(L) \to \Z/n\Z$ to dokładnie kolorowania modulo $n$, z kolorem $\varphi(l)$ na łuku $l$.
\end{proof}

Homomorfizm $\ColoringGroup (L) \to \Z/n\Z$ nie musi być surjekcją, zatem kolorowalność modulo $n$ pociąga za sobą kolorowalność modulo $mn$ (dla każdego naturalnego $m$).

\begin{proposition}
    Grupa kolorująca jest z dokładnością do izomorfizmu niezmiennikiem węzłów.
\end{proposition}

\begin{proof}[Szkic dowodu]
    Do wyznaczenia grupy kolorującej potrzebujemy diagramu $D$ z wybranym łukiem $a$.
    Musimy zatem sprawdzić, że ustalenie innego diagramu lub łuku prowadzi do grupy izomorficznej z wyjściową.
    Dla diagramów wystarczy sprawdzić, co dzieje się podczas ruchów Reidemeistera jak w dowodzie faktu \ref{prp:colouring_invariance}.
    W przypadku łuku rozumowanie przebiega analogicznie do dowodu lematu \ref{lem:colouring_arc}.
\end{proof}

Oto metoda pozwalająca na znalezienie grupy kolorującej.
Wybierzmy diagram splotu $L$ bez zamkniętych krzywych, etykietowanie $x_0, \ldots, x_m$ dla łuków oraz etykietowanie $0, \ldots, m$ dla skrzyżowań.
Utwórzmy macierz kolorującą $A$.
Bezpośrednio z~definicji macierzy $A$ wynika, że grupy abelowe $\ColoringGroup(L)$ oraz $\Z^m / A^t \Z^m$ są izomorficzne.
Następnie znajdźmy macierz diagonalną $D = \operatorname{diag}(d_1, \ldots, d_m)$, taką że $D = RAC$, gdzie całkowite macierze $R, C$ mają wyznacznik $1$.
Z~algebry liniowej wiemy, że to zawsze się uda: macierz $D$ nazywamy postacią normalną Smitha.
Wtedy funkcja
\begin{equation}
    f(x) \colon \frac{\Z^m}{A^t \Z^m} \to \frac{\Z^m}{D\Z^m}, \quad f(x) = C^t x
\end{equation}
stanowi izomorfizm, a~skoro $D$ jest macierzą diagonalną, to
\begin{equation}
    \frac{\Z^m}{D\Z^m} \cong \bigoplus_{k=1}^m \frac{\Z}{|d_k| \Z}.
\end{equation}

Dokładnie to samo można uczynić z macierzą Goeritza $G$ zamiast macierzy kolorującej $A$.
Podsumujmy.

\begin{proposition}
\label{prp:colouring_group_summands}%
    Niech $L$ będzie splotem z~macierzą kolorującą $A$, zaś $D = \operatorname{diag}(d_1, \ldots, d_n)$ postacią normalną Smitha macierzy $A$.
    Wtedy grupy 
    \begin{equation}
        \ColoringGroup(L) \cong \bigoplus_{k=1}^n \frac{\Z}{|d_k| \Z}
    \end{equation}
    są izomorficzne.
\end{proposition}

\begin{corollary}
    Grupa kolorująca splotu $L$ jest skończona wtedy i tylko wtedy, gdy wyznacznik tego splotu, $\det L$, jest niezerowy.
\end{corollary}

\begin{proof}
    Przy oznaczeniach z faktu \ref{prp:colouring_group_summands}, $\det(L) = |d_1| \cdot \ldots \cdot |d_n|$.
\end{proof}

\begin{corollary}
\index{wyznacznik}%
\label{cor:determinant_invariant}%
    Wyznacznik jest niezmiennikiem splotów.
\end{corollary}

Wyznacznik jest słabszym niezmiennikiem od grupy kolorującej.
Na przykład węzły $6_1$ oraz $3_1 \shrap 3_1$ mają ten sam wyznacznik, $9$, ale różne grupy kolorujące: odpowiednio $\Z/9$ oraz ~$(\Z/3)^2$.
Jedna jest cykliczna, druga niezbyt.

Jednakowoż wyznacznik jest na tyle mocnym niezmiennikiem, że pozwala elementarnie udowodnić, że węzłów pierwszych jest nieskończenie wiele.
Trzeba nam lematu:

\begin{lemma}
\label{lem:det_multiplicativ}%
    Niech $K_1, K_2$ będą węzłami.
    Wtedy $\det(K_1 \shrap K_2) = \det(K_1) \det(K_2)$.
\end{lemma}

\begin{proof}
    Dla oszczędności papieru zamiast podać pełen dowód napiszemy tylko, że wynika to z faktów \ref{prp:alexander_multiplicative} oraz \ref{prp:alexander_determinant}.
    (Podczas czytania dowodu faktu \ref{prp:alexander_multiplicative} należy zastąpić każde wystąpienie $t$ przez $-1$).
\end{proof}

\begin{proposition}
\label{prop:infinite_prime_knots_1}%
    Istnieje nieskończenie wiele węzłów pierwszych.
\end{proposition}

Po poznaniu genusu oraz węzłów skręconych poznamy jeszcze jeden dowód (patrz fakt \ref{prp:infinitely_many_prime_knots}).

\begin{proof}
    Tak jak Cygan dawno temu, rozpatrzmy rodzinę węzłów $K_k$ dla $k \ge 4$:
\index[persons]{Cygan, Szymon}%

\begin{figure}[H]
    \centering
    \begin{comment}
    \begin{tikzpicture}[baseline=-0.65ex, scale=0.1]
    \useasboundingbox (-20, -13) rectangle (40, 13);
    \begin{knot}[clip width=5, end tolerance=1pt, flip crossing/.list={1,2,3,6}]
        \strand[ultra thick] (-10, +3) .. controls (-4, +3) and (-4, -3) .. (0, -3);
        \strand[ultra thick] (-10, -3) .. controls (-4, -3) and (-4, +3) .. (0, +3);
        \node at (5, 0) {$\ldots$};
        \strand[ultra thick] (10+10, +3) .. controls (10+ 4, +3) and (10+ 4, -3) .. (10+0, -3);
        \strand[ultra thick] (10+ 10, -3) .. controls (10+ 4, -3) and (10+4, +3) .. (10+0, +3);
        \strand[ultra thick] (20+10, +3) .. controls (20+ 4, +3) and (20+ 4, -3) .. (20+0, -3);
        \strand[ultra thick] (20+ 10, -3) .. controls (20+ 4, -3) and (20+4, +3) .. (20+0, +3);
        \strand[ultra thick] (30, 3) [in=up, out=right] to (35, -3);
        \strand[ultra thick] (30, -3) [in=down, out=right] to (35, 3);
        \strand[ultra thick] (35, 3) [in=right, out=up] to (0, 10);
        \strand[ultra thick] (35, -3) [in=right, out=down] to (0, -10);
        \strand[ultra thick] (-10, -3) [in=down, out=left] to (-20, 0) to [in=left, out=up] (-10, 3);
        \strand[ultra thick] (-15, 5) [in=left, out=up] to (0, 10);
        \strand[ultra thick] (-15, 5) to (-15, -5) [in=left, out=down] to (0, -10);
        \node at (-5, -5) {$c_3$};
        \node at (15, -5) {$c_{k-2}$};
        \node at (25, -5) {$c_{k-1}$};
    \end{knot}
    \end{tikzpicture}
\end{comment}
    \caption[caption-cygan]{Węzeł $K_k$}
\end{figure}

    Niech $W_n$ będzie macierzą $n \times n$, na przekątnej której znajdują się $2$, zaś bezpośrednio nad i~pod nią -- wyrazy $-1$.
    W macierzy kolorującej (ze skreślonym wierszem ,,$c_k$'' oraz kolumną ,,$a_k$'') zamieńmy miejscami dwie pierwsze kolumny, dodajmy do drugiej dwa razy pierwszą, zaś do drugiego wiersza -- dwa razy pierwszy wiersz.
    Otrzymamy macierz
    \begin{equation}
        \begin{pmatrix}
            -1 & 0 & 0 \\
            0 & 3 & -1 \\
            0 & -1 & W_{k-3}
        \end{pmatrix}
    \end{equation}

    Powtarzając operacje: zamiana miejscami skrajnie lewych kolumn, dodanie do drugiej $2m+1$ razy pierwszej, odjęcie pierwszego wiersza od trzeciego, czyli sprowadzając naszą macierz do postaci normalnej Smitha przekonamy się, że na jej przekątnej znajdują się wyrazy $-1, -1, \ldots, -1, 2k-3$.
    To oznacza, że $\det K_k = 2k-3$.

    I to już prawie koniec!
    Wskażemy injekcję ze zbioru liczb pierwszych w zbiór węzłów pierwszych.
    Niech $2k-3$ będzie liczbą pierwszą.
    Wtedy któryś składnik w rozkładzie węzła $K_k$ na węzły pierwsze (tak jak w fakcie \ref{prp:knots_decompose_into_primes}) musi mieć wyznacznik $2k-3$.
    Jest jasne (po wniosku \ref{cor:determinant_invariant}), dlaczego te składniki są parami różne.
\end{proof}

\begin{proposition}
    Niech $k \ge 9$ będzie takie, że wyznacznik węzła $K_k$ jest pewną potęgą $3$.
    Wtedy $K_k$ nie jest splotem trójlistników.
\end{proposition}

\begin{proof}
    Macierz diagonalna otrzymana po wniosku \ref{cor:determinant_invariant} także jest niezmiennikiem węzłów.
    (Czemu?).
    Macierzą dla splotu trójlistników $(3_1)^{\# n}$ trójlistników jest $\operatorname{diag}(1, \ldots, 1, 3, \ldots, 3)$, zaś dla węzła $K_k$: $\operatorname{diag} (1, \ldots, 1, 3^n)$.
\end{proof}

\index{grupa!kolorująca|)}%

% koniec sekcji grupa

