
\section{Macierz kolorująca}
\index{macierz!kolorująca|(}%
Zajmiemy się dwoma niezmiennikami pochodzącymi od macierzy kolorującej: defektem oraz wyznacznikiem.
Podamy też opis macierzy Goeritza, która prowadzi do tych samych niezmienników, ale pozwala na uniknięcie części rachunków.

\begin{definition}[macierz kolorująca]
    Ustalmy diagram bez zamkniętych krzywych dla splotu $L$ z~łukami $x_0, \ldots, x_m$ oraz skrzyżowaniami $0, \ldots, m$.
    Definiujemy macierz $A_+$, której wyraz $a_{lj}$ jest współczynnikiem przy $x_j$ w~$l$-tym równaniu kolorującym:
\begin{center}
\begin{comment}
    \LargePlusCrossingMatrix
\end{comment}
\end{center}
    Macierz kolorująca $A$ powstaje z~macierzy $A_+$ przez skreślenie dowolnego wiersza oraz dowolnej kolumny.
\end{definition}

Zauważmy, że pierwszy ruch Reidemeistera usuwa ,,zamknięte krzywe'', czyli pojedyncze łuki bez skrzyżowań.
Diagram bez takich krzywych ma tyle samo skrzyżowań, co łuków.
Z~każdego skrzyżowania wychodzą (tunelem) dwa włókna mające dwa końce i otrzymana macierz jest kwadratowa.

Wykreślenie wiersza i~kolumny jest konieczne.
Gdybyśmy tego zaniechali, otrzymana macierz nie byłaby odwracalna, bowiem wiersze sumują się do zera (patrz fakt \ref{prp:colouring_sum_zero}).
Dla alternujących diagramów możemy żądać, by górą $i$-tego skrzyżowania biegło $i$-te włókno, wtedy na diagonali macierzy $A$ znajdą się same minus dwójki.

\input{40-colours/401a-determinant}


\subsection{Defekt}
\index{defekt|(}%
\begin{definition}[defekt]
    Niech $K$ będzie węzłem, zaś $A$ jego macierzą kolorującą modulo $p$.
    Wymiar jądra $\ker A$ nazywamy defektem węzła.
\end{definition}

Prawdopodobnie jest to ostatnia strona, gdzie wprowadzamy do teorii węzłów nowy termin z algebry liniowej.

\begin{proposition}
\label{no_relation_defects}%
    Defekty modulo różne liczby pierwsze są niezależne od siebie.
\end{proposition}

Wspomina o tym Livingston \cite[s. 145]{livingston1993}.

\begin{proof}[Niedowód]
    Na przykład suma spójna $k$ trójlistników i~$j$ węzłów $T_{2,5}$ posiada defekt $k$ modulo $3$ oraz $j$ modulo $5$.
    Podobne przykłady istnieją dla innych zbiorów liczb pierwszych.
\end{proof}

Defekt także jest niezmiennikiem, choć rzadziej używanym (nie pojawia się na żadnej późniejszej stronie w tej książce).
Węzeł o~defekcie $n$ modulo $p$ posiada $p(p^n-1)$ kolorowań $p$ kolorami, jest \cite[twierdzenie 2]{taalman2005}.
Węzły $8_{18}$ oraz $9_{24}$ mają ten sam wyznacznik, $45$.
Ich defekty modulo $3$ to $1$ i~$2$, zatem są różne.
\index{defekt|)}%



\input{40-colours/401c-goeritz}

\index{macierz!kolorująca|)}

% Koniec sekcji Macierz i~wyznacznik

